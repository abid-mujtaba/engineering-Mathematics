\documentclass{standalone}
\usepackage{fontspec}
\usepackage{commath}   %differential symbol   \dif
\usepackage{tikz}
\usepackage{pgfplots}
\pgfplotsset{compat=1.9}
\usepackage[europeanresistors]{circuitikz}
\usetikzlibrary{intersections,decorations.markings,calc,patterns}
\usepackage{polyglossia}                %is loaded the last
\usepackage{siunitx}
\usepackage{amsmath}
\newcommand*{\kvec}[1]{{\ensuremath{{\boldsymbol{#1}}}}}
\newcommand*{\kvecsub}[2]{{\ensuremath{{\boldsymbol{#1}}}_{\textup{#2}}}}

\newcommand*{\ax}{\ensuremath{{\boldsymbol{a}}_{\textup{x}}}}
\newcommand*{\ay}{\ensuremath{{\boldsymbol{a}}_{\textup{y}}}}
\newcommand*{\az}{\ensuremath{{\boldsymbol{a}}_{\textup{z}}}}
%
\newcommand*{\arho}{\ensuremath{{\boldsymbol{a}}_{\rho}}}
\newcommand*{\aphi}{\ensuremath{{\boldsymbol{a}}_{\phi}}}
%
\newcommand*{\ar}{\ensuremath{{\boldsymbol{a}}_{\textup{r}}}}
\newcommand*{\atheta}{\ensuremath{{\boldsymbol{a}}_{\theta}}}

\newcommand*{\aN}{\ensuremath{{\boldsymbol{a}}_N}}
\newcommand*{\aR}{\ensuremath{{\boldsymbol{a}}_{\textup{R}}}}
\newcommand*{\aL}{\ensuremath{{\boldsymbol{a}}_{\textup{L}}}}

\newcommand*{\au}{\ensuremath{{\boldsymbol{a}}_u}}
\newcommand*{\av}{\ensuremath{{\boldsymbol{a}}_v}}
\newcommand*{\aw}{\ensuremath{{\boldsymbol{a}}_w}}

\newcommand*{\Ex}{\ensuremath{{\boldsymbol{E}}_x}}
\newcommand*{\Ey}{\ensuremath{{\boldsymbol{E}}_y}}
\newcommand*{\Ez}{\ensuremath{{\boldsymbol{E}}_z}}
%
\newcommand*{\Erho}{\ensuremath{{\boldsymbol{E}}_{\rho}}}
\newcommand*{\Ephi}{\ensuremath{{\boldsymbol{E}}_{\phi}}}
%
\newcommand*{\Er}{\ensuremath{{\boldsymbol{E}}_r}}
\newcommand*{\Etheta}{\ensuremath{{\boldsymbol{E}}_{\theta}}}

\newcommand*{\TE}[1]{\ensuremath{\textup{TE}_{#1}}}
\newcommand*{\TM}[1]{\ensuremath{\textup{TM}_{#1}}}
\newcommand*{\TEM}{\ensuremath{\textup{TEM}}}
%===========================
\newcommand{\RightAngle}[4][5pt]{\draw[gray] ($#3!#1!#2$)--($ #3!2!($($#3!#1!#2$)!.5!($#3!#1!#4$)$) $) --($#3!#1!#4$) ;        }


\setmainlanguage[numerals=maghrib]{arabic}     %for english numbers use numerals=maghrib, for arabic numerals=arabicdigits
\setotherlanguages{english}


\newfontfamily\arabicfont[Scale=1.0,Script=Arabic]{Jameel Noori Nastaleeq}
%\newfontfamily\urdufont[Scale=1.0,Script=Arabic]{XB Tabriz}
\newfontfamily\urdufont[Scale=1.25,WordSpace=6.0,,Script=Arabic]{Jameel Noori Nastaleeq}

\pgfmathsetmacro{\A}{0.75}
\pgfmathdeclarefunction{Ey}{2}{\pgfmathparse{\A*cos(#1)*sin(#2)}}

\pgfplotsset{select coords between index/.style 2 args={
    x filter/.code={
        \ifnum\coordindex<#1\def\pgfmathresult{}\fi
        \ifnum\coordindex>#2\def\pgfmathresult{}\fi
    }
}}


\begin{document}
\begin{urdufont}
\def\ringa{(0,-1) circle (1) (0,-1/2) circle (1/2)}
\def\ringb{(-1/2,0) circle (1/2) (-1/4,0) circle (1/4)}
\def\ringc{(0,1) circle (1) (0,1/2) circle (1/2)}
\def\ringd{(1,0) circle (1) (1/2,0) circle (1/2)}
\begin{tikzpicture}%[z={(-0.5cm,-0.5cm)},x={(1cm,0cm)},y={(0cm,1cm)}]

  \begin{scope}[even odd rule]
        % Define a clipping path. All paths outside ringa will
        % be cut because the even odd rule is set. 
        \clip \ringa;
        % Fill ringb. Since the even odd rule is set, only the
        % ring will be filled, not the hole in the middle.  
        \fill[fill=black] \ringb;
    \end{scope}
   \begin{scope}[even odd rule]
        % Define a clipping path. All paths outside ringa will
        % be cut because the even odd rule is set. 
        \clip \ringc;
        % Fill ringb. Since the even odd rule is set, only the
        % ring will be filled, not the hole in the middle.  
        \fill[fill=gray] \ringd;
    \end{scope}
%
\draw(-3,0)--(3,0)node[below]{$u$};
\draw(0,-3)--(0,3)node[left]{$v$};
%
\draw (1,0) circle (1); %x=-1/2
\draw (-1,0) circle (1); %x=1/2
\draw (1/2,0) circle (1/2); %x=-1
\draw (-1/2,0) circle (1/2); %x=1
\draw (1/4,0) circle (1/4); %x=-2
\draw (-1/4,0) circle (1/4); %x=2
%
\draw (0,1) circle (1); %y=-1/2
\draw (0,-1) circle (1); %y=1/2
\draw (0,1/2) circle (1/2); %y=-1
\draw (0,-1/2) circle (1/2); %y=1
%
\draw[stealth-](1,0)++(45:1) to [out=60,in=180]++(0.3,0.3)node[right]{$x=\tfrac{1}{2}$};
\draw[stealth-](-1,0)++(135:1) to [out=120,in=-20]++(-0.3,0.3)node[left]{$x=-\tfrac{1}{2}$};
\draw[stealth-](0,1)++(45:1) to [out=60,in=180]++(0.3,0.3)node[right]{$y=-\tfrac{1}{2}$};
\draw[stealth-](0,-1)++(-135:1) to [out=-120,in=20]++(-0.3,-0.3)node[left]{$y=\tfrac{1}{2}$};
%
\draw(0,1)node[shift={(-0.2,0.2)}]{$1$};
\draw(0,-1)node[shift={(-0.3,-0.2)}]{$-1$};
\draw(1,0)node[shift={(0.2,-0.2)}]{$1$};
\draw(-1,0)node[shift={(-0.3,-0.2)}]{$-1$};
%
\draw(0,2)node[shift={(-0.2,0.2)}]{$2$};
\draw(0,-2)node[shift={(-0.3,-0.2)}]{$-2$};
\draw(2,0)node[shift={(0.2,-0.2)}]{$2$};
\draw(-2,0)node[shift={(-0.3,-0.2)}]{$-2$};
%text
\draw[stealth-](0,-2.5) to [out=20,in=180]++(0.3,0.1)node[right]{$x=0$};
\draw[stealth-](-2.5,0) to [out=120,in=-45]++(-0.3,0.3)node[left]{$y=0$};
\end{tikzpicture}%
\end{urdufont}
\end{document} 
