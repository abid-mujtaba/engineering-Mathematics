\باب{لاپلاس تبادلہ}
لاپلاس بدل کی ترکیب سے ابتدائی قیمت (سرحدی قیمت) تفرقی مساوات حل کیے جاتے ہیں۔یہ ترکیب تین قدم پر مشتمل ہے۔
\begin{itemize}
\item
پہلا قدم: ابتدائی قیمت (سرحدی قیمت) تفرقی مساوات کا لاپلاس بدل لیتے ہوئے سادہ \اصطلاح{ضمنی مساوات} حاصل کی جاتی ہے۔
\item
دوسرا قدم:  ضمنی مساوات کو خالصاً الجبرائی طور پر حل کیا جاتا ہے۔
\item
تیسرا قدم: ضمنی مساوات کے حل کا الٹ لاپلاس بدل لیتے ہوئے اصل حل حاصل کیا جاتا ہے۔
\end{itemize}  

یوں لاپلاس بدل تفرقی مساوات کے مسئلے کو سادہ الجبرائی مسئلہ میں تبدیل کرتا ہے۔تیسرے قدم پر الٹ لاپلاس بدل حاصل کرتے ہوئے عموماً ایسی جدول کا سہارہ لیا جاتا ہے جس  میں تفاعل اور تفاعل کے الٹ لاپلاس بدل درج ہوں۔ 

انجینری میں لاپلاس بدل کی ترکیب اہم کردار ادا کرتی ہے، بالخصوص ان مسائل میں جہاں جبری تفاعل غیر استمراری ہو، مثلاً جب جبری تفاعل کچھ وقفے کے لئے کار آمد ہو یا جبری تفاعل غیر سائن نما دہراتا تفاعل ہو۔

اب تک غیر متجانس مساوات کا عمومی حل حاصل کرتے ہوئے پہلے مطابقتی متجانس مساوات کا حل اور پھر غیر متجانس مساوات کا مخصوص حل حاصل کیا جاتا رہا۔لاپلاس بدل کی ترکیب میں عمومی حل ایک ہی بار میں حاصل ہوتا ہے۔اسی طرح لاپلاس بدل استعمال کرتے ہوئے ابتدائی قیمت (سرحدی قیمت) مسائل کے حل میں عمومی حل حاصل کرنے کے بعد ابتدائی (سرحدی) شرائط پر کرنے کی ضرورت پیش نہیں آتی چونکہ حل یہ شرائط شامل ہوتے ہیں۔
%=========================

\حصہ{لاپلاس بدل۔ الٹ لاپلاس بدل۔ خطیت}
فرض کریں کہ تفاعل \عددی{f(t)} تمام \عددی{t \ge 0} پر معین ہے۔ ہم \عددی{f(t)} کو \عددی{e^{-st}} سے ضرب دیتے ہوئے،  \عددی{t} کے ساتھ، \عددی{0} تا \عددی{\infty}، تکمل لیتے ہیں۔ اگر ایسا تکمل موجود ہو تو یہ \عددی{s} پر منحسر ہو گا لہٰذا اس کو \عددی{F(s)} لکھا جا سکتا ہے۔
\begin{align}
F(s)=\int_{0}^{\infty}e^{-st}f(t)\dif t
\end{align} 
تفاعل \عددی{F(s)} کو تفاعل \عددی{f(t)} کا \اصطلاح{لاپلاس بدل}\فرہنگ{لاپلاس بدل}\حاشیہب{ٔLaplace transform}\فرہنگ{Laplace transform} کہا جاتا ہے اور اس کو
 \عددی{\Laplace(f)} سے ظاہر کیا جاتا ہے۔
\begin{align}\label{مساوات_لاپلاس_بدل_الف}
F(s)=\Laplace(f)=\int_{0}^{\infty} e^{-st} f(t)\dif t
\end{align}
\عددی{f(t)} سے \عددی{F(s)} کے حصول کو \اصطلاح{لاپلاس تبادلہ}\فرہنگ{لاپلاس تبادلہ}\حاشیہب{Laplace transformation}\فرہنگ{Laplace transformation} کہتے ہیں۔

اسی طرح \عددی{f(t)} کو \عددی{F(s)} کا \اصطلاح{الٹ لاپلاس بدل}\فرہنگ{لاپلاس!الٹ بدل}\حاشیہب{inverse Laplace transform}\فرہنگ{Laplace!inverse transform} کہتے ہیں جسے \عددی{\Laplace^{-1}(F)} سے ظاہر کیا جاتا ہے۔
\begin{align}
f(t)=\Laplace^{-1}(F)
\end{align}

\موٹا{علامت نویسی}\\
اصل تفاعل کو چھوٹے لاطینی حرف تہجی سے ظاہر کیا جاتا ہے جبکہ لاپلاس بدل کو اسی حرف تہجی کی بڑی صورت سے ظاہر کیا جاتا ہے۔ یوں \عددی{f(t)} کا بدل \عددی{F(s)} ہو گا اور \عددی{g(t)} کا لاپلاس بدل \عددی{G(s)} ہو گا۔

%================
\ابتدا{مثال}
تفاعل \عددی{f(t)=1}، جہاں \عددی{t \ge 0} ہے، کا لاپلاس بدل مساوات \حوالہ{مساوات_لاپلاس_بدل_الف} سے بذریعہ تکمل حاصل کرتے ہیں۔
\begin{align*}
\Laplace(f)=\Laplace(1)=\int_{0}^{\infty} e^{-st} \dif t=\left. -\frac{1}{s}e^{-st}\right|_{0}^{\infty}
\end{align*}
ہو گا جو \عددی{s >0} کی صورت میں درج ذیل ہو گا۔
\begin{align*}
\Laplace(1)=\frac{1}{s}
\end{align*}
تکمل \حوالہ{مساوات_لاپلاس_بدل_الف} کی علامت پر آسائش ضرور ہے لیکن اس پر مزید غور کی ضرورت ہے۔اس تکمل کا وقفہ لامتناہی ہے۔ایسے تکمل کو \اصطلاح{غیر مناسب تکمل}\فرہنگ{غیر مناسب تکمل}\حاشیہب{improper integral}\فرہنگ{improper integral} کہتے ہیں اور حزب  تعریف، اس کی قیمت درج ذیل اصول کے تحت حاصل کی جاتی ہے۔ 
\begin{align*}
\int_{0}^{\infty} e^{-st}f(t)\dif t=\lim_{T\to \infty} \int_{0}^{T}e^{-st}f(t)\dif t
\end{align*}
یوں اس مثال میں اس آسائش علامت کا مطلب درج ذیل ہے۔
\begin{align*}
\int_{0}^{\infty}e^{-st}\dif t=\lim_{T\to \infty}\int_{0}^{T}e^{-st}\dif t=\lim_{T\to \infty}\left[-\frac{1}{s}e^{-sT}+\frac{1}{s}e^{0}\right]=\frac{1}{s}, \quad (s>0)
\end{align*}
اس پورے باب میں تکمل کی یہی علامت استعمال کی جائے گی۔
\انتہا{مثال}
%=================
\ابتدا{مثال}\شناخت{مثال_لاپلاس_قوت_نمائی_الف}
تفاعل \عددی{f(t)=e^{at}} جہاں \عددی{t \ge 0} اور \عددی{a} مستقل ہے کا لاپلاس بدل \عددی{\Laplace(f)} دریافت کریں۔

حل:مساوات \حوالہ{مساوات_لاپلاس_بدل_الف} سے
\begin{align*}
\Laplace(e^{at})=\int_{0}^{\infty} e^{-st}e^{at}\dif t=\left.\frac{1}{a-s}e^{-(s-a)t} \right|_{0}^{\infty}
\end{align*}
ملتا ہے۔اب اگر \عددی{s-a >0} ہو (یعنی \عددی{s} کی قیمت \عددی{a} سے زیادہ چننی گئی ہو۔)  تب درج ذیل حاصل ہو گا۔
\begin{align*}
\Laplace(e^{at})=\frac{1}{s-a}
\end{align*}
\انتہا{مثال}
%=================

اگرچہ ہم بالکل اسی طرز پر دیگر تفاعل کے لاپلاس بدل بذریعہ تکمل حاصل کر سکتے ہیں، حقیقت میں لاپلاس تبادلہ کے ایسی کئی خواص ہیں جنہیں استعمال کرتے ہوئے دیگر لاپلاس بدل نہایت عمدگی کے ساتھ حاصل کیے جا سکتے ہیں۔لاپلاس تبادلہ کی ایک خاصیت  خطیت ہے جس سے مراد درج ذیل ہے۔
%================

\ابتدا{مسئلہ}\شناخت{مسئلہ_لاپلاس_خطیت}\quad لاپلاس تبادلہ کی خطیت\\
لاپلاس تبادلہ خطی عمل ہے۔یوں ایسے تفاعل \عددی{f(t)} اور \عددی{g(t)}، جن کے لاپلاس بدل موجود ہوں، کے عمومی مجموعے کا لاپلاس بدل درج ذیل ہو گا جہاں \عددی{a} اور \عددی{b} مستقل ہیں۔
\begin{align*}
\Laplace[af(t)+bg(t)]=a\Laplace[f(t)]+b\Laplace[g(t)]
\end{align*}
\انتہا{مسئلہ}
%=================
\ابتدا{ثبوت}
لاپلاس تبدلہ کی تعریف سے درج ذیل لکھتے ہیں۔
\begin{align*}
\Laplace[af(t)+bg(t)]&=\int_{0}^{\infty}e^{-st}[af(t)+bg(t)]\dif t\\
&=a\int_{0}^{\infty}e^{-st}f(t)\dif t+b\int_{0}^{\infty}e^{-st}g(t) \dif t\\
&=a\Laplace[f(t)]+b\Laplace[g(t)]
\end{align*}
\انتہا{ثبوت}
%=====================

\ابتدا{مثال}
فرض کریں کہ \عددی{f(t)=\cosh at=\tfrac{e^{at}+e^{-at}}{2}} ہے۔اس کا لاپلاس بدل مسئلہ \حوالہ{مسئلہ_لاپلاس_خطیت} اور مثال \حوالہ{مثال_لاپلاس_قوت_نمائی_الف} کی مدد سے لکھتے ہیں جہاں \عددی{s>a\ge 0} چننا گیا ہے۔
\begin{align*}
\Laplace(\cosh at)=\frac{1}{2}\Laplace(e^{at})+\frac{1}{2}\Laplace(e^{-at})=\frac{1}{2}\left(\frac{1}{s-a}+\frac{1}{s+a}\right)=\frac{s}{s^2-a^2}
\end{align*}
\انتہا{مثال}
%=======================

جدول \حوالہ{جدول_لاپلاس_بدل_الف} میں چند اہم بنیادی تفاعل اور ان کے لاپلاس بدل دیے گئے ہیں۔اس جدول میں دیے لاپلاس بدل جاننے کے بعد ہم تقریباً ان تمام تفاعل کے بدل، لاپلاسی خواص سے حاصل کر پائیں گے،  جو ہمیں درکار ہوں گے۔
\begin{table}
\caption{چند بنیادی تفاعل \عددی{f(t)} اور ان کے لاپلاس بدل \عددی{\Laplace(f)}}
\label{جدول_لاپلاس_بدل_الف}
\centering
\begin{tabular}{ccc| ccc}
 شمار& $f(t)$& $\Laplace(f)$& شمار& $f(t)$& $\Laplace(f)$\\[0.5ex]
\hline
$1$&$1$&$\frac{1}{s}$&$6$& $e^{at}$&$\frac{1}{s-a}$\Tstrut\\[1ex]
$2$&$t$&$\frac{1}{s^2}$&$7$&$\cos \omega t$&$\frac{s}{s^2+\omega^2}$\\[1ex]
$3$&$t^2$&$\frac{2!}{s^3}$&$8$&$\sin \omega t$&$\frac{\omega}{s^2+\omega^2}$\\[1ex]
$4$&\shortstack{$t^n$\\ ($n=1,2,\cdots$)}&$\frac{n!}{s^{n+1}}$&$9$&$\cosh at$&$\frac{s}{s^2-a^2}$\\[1ex]
$5$&\shortstack{$t^a$\\($ a>0$)}&$\frac{\Gamma(a+1)}{s^{a+1}}$&$10$&$\sinh at$&$\frac{a}{s^2-a^2}$
\end{tabular}
\end{table}

جدول \حوالہ{جدول_لاپلاس_بدل_الف} میں پہلا، دوسرا اور تیسرا کلیہ چوتھے کلیے سے اخذ کیے جا سکتے ہیں جبکہ چوتھا کلیہ از خود پانچویں کلیہ میں مساوات \حوالہ{مساوات_بیسل_گیما_بطور_عدد_ضربیہ} استعمال کرتے ہوئے \عددی{\Gamma(n+1)=n!} لکھ کر حاصل کیا جا سکتا ہے، جہاں \عددی{n} غیر منفی \عددی{n \ge 0} عدد صحیح ہے۔ پانچواں کلیہ، لاپلاس بدل کی تعریف مساوات \حوالہ{مساوات_لاپلاس_بدل_الف}
\begin{align*}
\Laplace(t^a)=\int_{0}^{\infty}e^{-st}t^a\dif t
\end{align*}
میں \عددی{st=x} پر کرتے ہوئے مساوات \حوالہ{مساوات_بیسل_گیما_الف} کے استمعمال سے حاصل کرتے ہیں۔
\begin{align*}
\Laplace(t^a)=\int_{0}^{\infty}e^{-x}\left(\frac{x}{s}\right)^a \frac{\dif x}{s}=\frac{1}{s^{a+1}}\int_{0}^{\infty}e^{-x}x^a\dif x=\frac{\Gamma(a+1)}{s+1},\quad \quad (s>0)
\end{align*}
