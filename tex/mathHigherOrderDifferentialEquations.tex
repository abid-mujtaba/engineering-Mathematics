\باب{بلند درجی خطی سادہ تفرقی مساوات}
دو درجی خطی سادہ تفرقی مساوات کو حل کرنے کے طریقے بلند درجی خطی سادہ ترفقی مساوات کے لئ بھی قابل استعمال ہیں۔ہم دیکھیں گے کہ بلند درجی صورت میں مساوات زیادہ پیچیدہ ہوں گے،  امتیازی مساوات کے جذر بھی تعداد میں زیادہ اور حصول میں نسبتاً مشکل ہوں گے اور ورونسکی زیادہ اہم کردار ادا کرے گا۔ 

\حصہ{متجانس خطی سادہ تفرقی مساوات}
\عددی{n} درجی سادہ تفرقی مساوات سے مراد ایسی مساوات ہے جس میں نا معلوم متغیرہ \عددی{y(x)} کا \عددی{y^n=\tfrac{\dif^{\, n} y}{\dif x^n}} سب سے بلند درجی تفرق ہو۔ایسی سادہ تفرقی مساوات کو
\begin{align*}
F(x,y,y',\cdots, y^{(n)})=0
\end{align*}
لکھا جا سکتا ہے جس میں \عددی{y} اور کم درجی تفرق موجود یا غیر موجود ہو سکتے ہیں۔ایسی مساوات کو \اصطلاح{خطی}\فرہنگ{خطی}\فرہنگ{linear} کہتے ہیں اگر اس کو 
\begin{align}\label{مساوات_سادہ_بلند_خطی_الف}
y^{(n)}+p_{n-1}(x)y^{(n-1)}+\cdots+p_1(x)y'+p_0(x)y=r(x)
\end{align}
لکھنا ممکن ہو۔صفحہ \حوالہصفحہ{مساوات_سادہ_دو_درجی_تعریف} پر دو درجی خطی سادہ تفرقی مساوات کی بات کی گئی۔موجودہ مساوات میں \عددی{n=2}، \عددی{p_1=p} اور \عددی{p_0=q} پر کرنے سے دو درجی مساوات حاصل ہو گی۔عددی سر \عددی{p_0(x)} تا \عددی{p_n(x)}  اور جبری تفاعل \عددی{r(x)} غیر تابع متغیرہ  \عددی{x} کے کوئی بھی تفاعل ہو سکتے ہیں جبکہ \عددی{y(x)} نا معلوم متغیرہ ہے۔خطی مساوات کو معیاری صورت میں لکھا گیا ہے جہاں \عددی{y^{(n)}} کا عددی سر اکائی \عددی{1} ہے۔ تفرقی مساوات میں  \عددی{p_n(x)y^{(n)}} موجود ہونے کی صورت میں پوری مساوات کو \عددی{p_n(x)} سے تقسیم کرتے ہوئے معیاری صورت حاصل کریں۔جو تفرقی مساوات درج بالا صورت میں لکھنا ممکن نہ ہو \اصطلاح{غیر خطی}\فرہنگ{غیر خطی}\فرہنگ{non linear} کہلاتی ہے۔

کسی کھلے وقفے \عددی{I} پر \عددی{r(x)} \اصطلاح{مکمل صفر}\فرہنگ{مکمل صفر} \عددی{r \equiv 0} ہونے کی صورت میں  مساوات \حوالہ{مساوات_سادہ_بلند_خطی_الف} سے \اصطلاح{متجانس مساوات}\فرہنگ{متجانس مساوات}\فرہنگ{homogeneous}
\begin{align}\label{مساوات_سادہ_بلند_خطی_ب}
y^{(n)}+p_{n-1}(x)y^{(n-1)}+\cdots+p_1(x)y'+p_0(x)y=0
\end{align}
حاصل ہوتی ہے۔کھلے وقفے پر \عددی{r(x)} کے مکمل صفر ہونے سے مراد یہ ہے کہ اس وقفے پر ہر \عددی{x} کے لئے \عددی{r(x)} کی قیمت صفر کے برابر ہے۔دو درجی تفرقی مساوات کی طرح  اگر \عددی{r(x)} مکمل صفر نہ ہو تب مساوات \اصطلاح{غیر متجانس}\فرہنگ{غیر متجانس}\فرہنگ{nonhomogeneous} کہلائے گی۔

کھلے وقفہ \عددی{I} پر \عددی{n} درجی خطی یا غیر خطی سادہ تفرقی مساوات کے حل \عددی{y=h(x)} سے مراد ایسا تفاعل ہے جو \عددی{I} پر معین ہو،  کھلے وقفے پر اس کا \عددی{n} درجی تفرق موجود ہو اور تفرقی مساوات میں \عددی{y} اور اس کے تفرقات کی جگہ \عددی{h} اور اس کے تفرقات پر کرنے سے مساوات کے دونوں اطراف بالکل یکساں حاصل ہوں۔ 
%=======================

\جزوحصہء{متجانس خطی سادہ تفرقی مساوات:خطی میل اور عمومی حل}
\اصطلاح{خطی میل}\فرہنگ{خطی میل} یا \اصطلاح{اصول خطیت}\فرہنگ{اصول خطیت} جس کا ذکر صفحہ \حوالہصفحہ{مسئلہ_دو_درجی_خطی_میل} مسئلہ \حوالہ{مسئلہ_دو_درجی_خطی_میل} میں کیا گیا بلند درجی خطی متجانس سادہ تفرقی مساوات کے لئے بھی درست ہے۔
%===============

\ابتدا{مسئلہ}\quad بنیادی مسئلہ برائے متجانس خطی سادہ بلند درجی تفرقی مساوات\فرہنگ{مسئلہ!بنیادی۔متجانس خطی}\\
کھلے وقفہ \عددیء{I} پر متجانس خطی بلند درجی تفرقی مساوات \حوالہ{مساوات_سادہ_بلند_خطی_ب} کے حل کا خطی میل بھی \عددیء{I} پر اس مساوات کا حل ہو گا۔بالخصوص ان حل کو مستقل مقدار سے ضرب دینے سے بھی مساوات کے حل حاصل ہوتے ہیں۔(یہ اصول غیر خطی اور  غیر متجانس مساوات پر لاگو نہیں ہوتا۔)
\انتہا{مسئلہ}
%==========================

اس کا ثبوت گزشتہ باب میں دئے گئے ثبوت کی طرح ہے جس کو یہاں پیش نہیں کیا جائے گا۔

ہماری بقایا گفتگو ہو بہو دو درجی تفرقی مساوات کی طرح ہو گی لہٰذا یہاں بلند درجی خطی متجانس مساوات کی عمومی حل کی بات کرتے ہیں۔ایسا کرنے کی خاطر \عددی{n} عدد تفاعل کی  \اصطلاح{خطی طور غیر تابع}\فرہنگ{غیر تابع!خطی طور}\فرہنگ{خطی طور!غیر تابع} ہونے کی تصور کو وسعت دیتے ہیں۔

%====================
\ابتدا{تعریف}\quad عمومی حل، اساس اور مخصوص حل\\
کھلے وقفے \عددی{I} پر مساوات \حوالہ{مساوات_سادہ_بلند_خطی_ب} کا \اصطلاح{عمومی حل}\فرہنگ{عمومی حل}\فرہنگ{general solution}
\begin{align}
y(x)=c_1y_1(x)+c_2y_2(x)+\cdots +c_ny_n(x)
\end{align}
ہے جہاں \عددی{y_1(x)} تا \عددی{y_n(x)} حل کی اساس اور \عددی{c_1} تا \عددی{c_2} اختیاری مستقل ہیں۔یوں \عددی{y_1} تا \عددی{y_n} کھلے وقفے پر خطی طور غیر تابع ہیں۔ 

عمومی حل کے مستقل کی قیمتیں مقرر کرنے سے \اصطلاح{مخصوص حل}\فرہنگ{مخصوص حل}\فرہنگ{particular solution} حاصل ہو گا۔
\انتہا{تعریف}
%=========================

\ابتدا{تعریف}\quad خطی طور تابع اور خطی طور غیر تابع\\
تصور کریں کہ کھلے وقفے \عددی{I} پر  \عددی{n} عدد تفاعل \عددی{y_1(x)} تا \عددی{y_n(x)} معین ہیں۔

 وقفہ \عددی{I} پر معین \عددی{y_1} تا \عددی{y_n}،  وقفہ \عددی{I}  پر، اس صورت \اصطلاح{خطی طور غیر تابع}\فرہنگ{خطی طور! غیر تابع}\حاشیہب{linearly independent}\فرہنگ{linearly independent} کہلاتے ہیں جب پورے وقفے پر
\begin{align}\label{مساوات_سادہ_بلند_خطی_طور_غیر_تابع_الف}
k_1 y_1(x)+k_2 y_2(x)+\cdots+k_ny_n(x)=0
\end{align}
سے مراد 
\begin{align}
k_1=k_2= \cdots =k_n=0 
\end{align}
ہو۔ 
\انتہا{تعریف}
%==========================
