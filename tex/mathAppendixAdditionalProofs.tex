\باب{اضافی ثبوت}\شناخت{ضمیمہ_اضافی_ثبوت}
صفحہ \حوالہصفحہ{مسئلہ_سادہ_دو_درجی_یکتا_مخصوص_حل} پر مسئلہ \حوالہ{مسئلہ_سادہ_دو_درجی_یکتا_مخصوص_حل} بیان کیا گیا جس کا ثبوت یہاں پیش کرتے ہیں۔

\ابتدا{ثبوت}\quad یکتائی (مسئلہ \حوالہ{مسئلہ_سادہ_دو_درجی_یکتا_مخصوص_حل})\\
تصور کریں کہ کھلے وقفے \عددی{I} پر ابتدائی قیمت مسئلہ
\begin{align}\label{مساوات_ضمیمہ_خطی_متجانس_تفرقی_الف}
y''+p(x)y'+q(x)y=0, \quad y(x_0)=K_0,\quad y'(x_0)=K_1
\end{align}
کے دو عدد حل \عددی{y_1(x)} اور \عددی{y_2(x)} پائے جاتے ہیں۔ہم ثابت کرتے ہیں کہ \عددی{I} پر ان کا فرق
\begin{align*}
y(x)=y_1(x)-y_2(x)
\end{align*}
مکمل صفر کے برابر ہے۔یوں \عددی{y_1(x) \equiv y_2(x)} ہو گا جو یکتائی کا ثبوت ہے۔

چونکہ مساوات \حوالہ{مساوات_ضمیمہ_خطی_متجانس_تفرقی_الف} خطی اور متجانس ہے  لہٰذا \عددی{I} پر \عددی{y(x)} بھی اس کا حل ہو گا اور چونکہ \عددی{y_1} اور \عددی{y_2} دونوں یکساں ابتدائی معلومات پر پورا اترتے ہیں لہٰذا \عددی{y}  درج ذیل ابتدائی معلومات پر پورا اترے گا۔
\begin{align}\label{مساوات_ضمیمہ_ضمنی_ابتدائی}
y(x_0)=0, \quad y'(x_0)=0
\end{align}
ہم تفاٰعل
\begin{align}\label{مساوات_ضمیمہ_ضمنی_تفاعل}
z=y^2+y'^2
\end{align}
اور اس کے تفرق
\begin{align}
z'=2yy'+2y'y''
\end{align}
پر غور کرتے ہیں۔تفرقی مساوات \حوالہ{مساوات_ضمیمہ_خطی_متجانس_تفرقی_الف} کو
\begin{align*}
y''=-py'-qy
\end{align*}
لکھتے ہوئے اس کو \عددی{z'} میں پر کرتے ہیں۔
\begin{align}\label{مساوات_ضمیمہ_ضمنی_تفاعل_ب}
z'=2yy'+2y'(-py'-qy)=2yy'-2py'^2-2qyy'
\end{align}
اب چونکہ \عددی{y} اور \عددی{y'} حقیقی تفاعل ہیں لہٰذا ہم 
\begin{align}
(y \mp y')^2=y^2\mp 2yy'+y'^2 \ge 0
\end{align}
یعنی
\begin{align}\label{مساوات_ضمیمہ_ضمنی_حدود_الف}
\text{(الف)}\quad 2yy' \le y^2+y'^2=z, \quad \text{(ب)}\quad -2yy' \le y^2+y'^2=z,
\end{align}
لکھ سکتے ہیں جہاں مساوات \حوالہ{مساوات_ضمیمہ_ضمنی_تفاعل} کا استعمال کیا گیا ہے۔مساوات \حوالہ{مساوات_ضمیمہ_ضمنی_حدود_الف}-ب کو \عددی{2yy' \ge -z} لکھتے ہوئے مساوات \حوالہ{مساوات_ضمیمہ_ضمنی_حدود_الف} کے دونوں حصوں کو \عددی{\abs{2yy'} \le z} لکھا جا سکتا ہے۔یوں مساوات \حوالہ{مساوات_ضمیمہ_ضمنی_تفاعل_ب} کے آخری جزو کے لئے
\begin{align*}
-2qyy'\le \abs{-2qyy'}=\abs{q}\abs{2yy'} \le \abs{q}z
\end{align*}
لکھا جا سکتا ہے۔اس نتیجے کے ساتھ ساتھ \عددی{-p \le \abs{p}} استعمال کرتے ہوئے اور  مساوات \حوالہ{مساوات_ضمیمہ_ضمنی_حدود_الف}-الف  کو  مساوات \حوالہ{مساوات_ضمیمہ_ضمنی_تفاعل_ب} کے \عددی{2yy'} جزو میں استعمال کرتے ہوئے 
\begin{align*}
z' \le z+2\abs{p}y'^2+\abs{q}z
\end{align*}
ملتا ہے۔اب چونکہ \عددی{y'^2 \le y^2+y'^2=z} ہے لہٰذا اس سے
\begin{align*}
z' \le (1+\abs{p}+\abs{q})z
\end{align*}
ملتا ہے۔اس میں \عددی{1+\abs{q}+\abs{p}=h} لکھتے ہوئے
\begin{align}\label{مساوات_ضمیمہ_شرط_الف}
z' \le hz\quad \quad \text{\RL{$I$ پر تمام $x$}}
\end{align}
حاصل ہوتا  ہے۔اسی طرح مساوات \حوالہ{مساوات_ضمیمہ_ضمنی_تفاعل_ب} اور مساوات \حوالہ{مساوات_ضمیمہ_ضمنی_حدود_الف} سے درج ذیل بھی حاصل ہوتا ہے۔
\begin{gather}
\begin{aligned}\label{مساوات_ضمیمہ_شرط_ب}
-z'&=-2yy'+2py'^2+2qyy'\\
& \le z+2\abs{p}z+\abs{q}z=hz
\end{aligned}
\end{gather}
مساوات \حوالہ{مساوات_ضمیمہ_شرط_الف} اور مساوات \حوالہ{مساوات_ضمیمہ_شرط_ب} کے غیر مساوات  درج ذیل غیر مساوات کے مترادف ہیں
\begin{align}\label{مساوات_ضمیمہ_شرط_پ}
z'-hz \le 0, \quad z'+hz\ge 0
\end{align}
جن کے بائیں ہاتھ کے جزو تکمل درج ذیل ہیں۔
\begin{align*}
F_1=e^{-\int h(x)\dif x}, \quad \quad F_2=e^{\int h(x)\dif x}
\end{align*}
چونکہ \عددی{h(x)} استمراری ہے لہٰذا اس کا تکمل پایا جاتا ہے۔چونکہ \عددی{F_1} اور \عددی{F_2} مثبت ہیں لہٰذا انہیں مساوات \حوالہ{مساوات_ضمیمہ_شرط_پ} کے ساتھ ضرب کرنے سے
\begin{align*}
(z'-hz)F_1=(zF_1)' \le 0, \quad  (z'+hz)F_2=(zF_2)' \ge 0
\end{align*}
حاصل ہوتا ہے۔اس کا مطلب ہے کہ \عددی{I} پر \عددی{zF_1} بڑھ نہیں رہا اور \عددی{zF_2} گھٹ نہیں رہا۔مساوات \حوالہ{مساوات_ضمیمہ_ضمنی_ابتدائی} کے تحت \عددی{z(x_0)=0} ہے  لہٰذا \عددی{x \le x_0} کی صورت میں
\begin{align}
zF_1 \ge (zF_1)_{x_0}=0, \quad \quad zF_2\le (zF_2)_{x_0}
\end{align}
ہو گا اور اسی طرح \عددی{x \ge x_0} کی صورت میں
\begin{align}
zF_1\le 0, \quad \quad zF_2 \ge 0
\end{align}
ہو گا۔اب انہیں مثبت قیمتوں \عددی{F_1} اور \عددی{F_2} سے تقسیم کرتے ہوئے
\begin{align}
z\le 0, \quad \quad z \ge 0 \quad \quad \text{\RL{$I$ پر تمام $x$ کے لئے}}
\end{align}
ملتا ہے جس کا مطلب ہے کہ \عددی{I} پر \عددی{z=y^2+y'^2 \equiv 0} ہے۔یوں \عددی{I} پر \عددی{y \equiv 0}  یعنی \عددی{y_1 \equiv y_2} ہے جو درکار ثبوت ہے۔
\انتہا{ثبوت}
%=============================
