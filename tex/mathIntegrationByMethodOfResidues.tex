\باب{تکمل بذریعہ ترکیب بقیہ}
چونکہ مساوات \حوالہ{مساوات_ٹیلر_لوغوں_تسلسل_ب} کے تکمل استعمال کیے بغیر لوغوں تسلسل (مساوات \حوالہ{مساوات_ٹیلر_لوغوں_تسلسل_الف}) کے عددی سر حاصل کرنے کے کئی تراکیب پائے جاتے ہیں لہٰذا ہم \عددی{c_1} کا کلیہ استعمال کرتے ہوئے مخلوط تکمل کی قیمت کو با آسانی اور  نفاست کے ساتھ حاصل کر سکتے ہیں۔ \عددی{c_1} کو \عددی{z=a} پر \عددی{f(z)} کا \اصطلاح{بقیہ} کہا جائے گا۔جیسا ہم حصہ میں دیکھیں گے، اس طاقتور ترکیب کو استعمال کرتے ہوئے کئی اہم حقیقی تکمل بھی حل کیے جاتے ہیں۔

\حصہ{بقیہ}
تفاعل \عددی{f(z)} جو نقطہ \عددی{z=0} کی پڑوس میں تحلیلی ہو کے لئے کوشی مسئلہ تکمل سے اس پڑوس میں کسی بھی خط ارتفاع  پر 
\begin{align}\label{مساوات_بقیہ_تکمل_الف}
\int_C f(z)\dif z=0
\end{align}
ہو گا۔البتہ اگر \عددی{C} کے اندر نقطہ  \عددی{z=a} پر \عددی{f(z)} کا تنہا ندرت پایا جاتا ہو تب  مساوات \حوالہ{مساوات_بقیہ_تکمل_الف} میں دیا گیا تکمل عموماً غیر صفر ہو گا۔ایسی صورت میں \عددی{f(z)} کو لوغوں تسلسل
\begin{align}\label{مساوات_بقیہ_تکمل_ب}
f(z)=\sum\limits_{n=0}^{\infty} b_n(z-a)^n+\frac{c_1}{z-a}+\frac{c_2}{(z-a)^2}+\cdots
\end{align}
سے ظاہر کیا جا سکتا ہے جو دائرہ کار \عددی{0<\abs{z-a}<R} میں مرتکز ہو گا جہاں \عددی{a} سے \عددی{f(z)} کی قریب ترین ندرت کا فاصلہ \عددی{R} ہے۔مساوات \حوالہ{مساوات_ٹیلر_لوغوں_تسلسل_ب} سے ہم دیکھتے ہیں کہ عددی سر \عددی{c_1} درج ذیل ہو گا
\begin{align*}
c_1=\frac{1}{i2\pi}\int_C f(z)\dif z
\end{align*}
لہٰذا
\begin{align}\label{مساوات_بقیہ_تکمل_پ}
\int_C f(z)\dif z=i2\pi c_1
\end{align}
لکھا جا سکتا ہے جہاں تکمل کو گھڑی کے الٹ رخ، دائرہ کار \عددی{0<\abs{z-a}<R} میں سادہ بند راہ \عددی{C}  پر حاصل کیا جاتا ہے۔مساوات \حوالہ{مساوات_بقیہ_تکمل_ب} میں \عددی{c_1} کو نقطہ \عددی{z=a} پر \عددی{f(z)} کا \اصطلاح{بقیہ}\فرہنگ{بقیہ}\حاشیہب{residue}\فرہنگ{residue} کہتے ہیں جس کو ہم درج ذیل لکھ کر ظاہر کرتے ہیں۔
\begin{align}
c_1=\underset{z=a\hfill}{\Res f(z)}
\end{align}

ہم دیکھ چکے ہیں کہ لوغوں تسلسل کے عددی سر کو، عددی سر کی تکمل کلیات کو استعمال کیے بغیر، مختلف تراکیب سے حاصل کیا جا سکتا ہے۔ان میں سے  کسی ایک ترکیب سے \عددی{c_1} حاصل کرتے ہوئے \اصطلاح{ارتفاعی تکمل}\فرہنگ{تکمل!ارتفاعی}\حاشیہب{contour integral}\فرہنگ{integral!contour} کی قیمت حاصل کی جا سکتی ہے۔  

%========================
\ابتدا{مثال}\quad \موٹا{تکمل کی قیمت کا حصول بذریعہ بقیہ}\\
تفاعل \عددی{f(z)=z^{-4}\sin z} کا اکائی دائرے پر گھڑی کی رخ تکمل حاسل کریں۔\\
مساوات \حوالہ{مساوات_ٹیلر_تکونیاتی_تفاعل} سے ہم لوغوں تسلسل
\begin{align*}
f(z)=\frac{\sin z}{z^4}=\frac{1}{z^3}-\frac{1}{3!z}+\frac{z}{5!}-\frac{z^3}{7!+-\cdots}
\end{align*}
حاصل کرت ہیں۔ہم دیکھتے ہیں کہ \عددی{z=0} پر \عددی{f(z)} کا تین درجی قطب پایا جاتا ہے جس کا مطابقتی بقیہ \عددی{c_1=-\tfrac{1}{3!}} ہے لہٰذا مساوات \حوالہ{مساوات_بقیہ_تکمل_پ} سے درج ذیل حاصل ہو گا۔
\begin{align*}
\int_C \frac{\sin z}{z^4}\dif z=i2\pi c_1=-\frac{i\pi}{3}
\end{align*}
\انتہا{مثال}
%===========================
آگے بڑھنے سے پہلے قطب کی صورت مین بقیہ دریافت کرنے کا ایک منظم طریقہ سیکھتے ہیں۔

اگر نقطہ \عددی{z=a} پر \عددی{f(z)} کا \موٹا{سادہ قطب} پایا جاتا ہو تب تفاعل کا مطابقتی لوغوں تسلسل (مساوات \حوالہ{مساوات_بقیہ_تکمل_ب}) 
\begin{align*}
f(z)=\frac{c_1}{z-a}+b_0+b_1(z-a)+b_2(z-a)^2+\cdots\quad \quad (0<\abs{z-a}<R)
\end{align*}
ہو گا جہاں \عددی{c_1\ne 0} ہے۔دونوں اطراف کو \عددی{z-a} سے ضرب دیتے ہیں۔
\begin{align}
(z-a)f(z)=c_1+(z-a)[b_0+b_1(z-a)+\cdots]
\end{align}
اب \عددی{z\to 0} کرنے سے دایاں ہاتھ \عددی{c_1} تک پہنچتا ہے لہٰذا ہمیں درج ذیل حاصل ہو گا۔
\begin{align}\label{مساوات_بقیہ_پہلا_کلیہ}
\underset{z=a\hfill}{\Res f(z)}=c_1=\lim_{z\to a} (z-a)f(z)
\end{align}  
یہ پہلا درکار نتیجہ ہے جو سادہ قطب کی صورت میں بقیہ دیتا ہے۔

سادہ قطب کی صورت میں بقیہ کا دوسرا کلیہ حاصل کرت ہیں۔اگر \عددی{f(z)} کا نقطہ \عددی{z=a} پر سادہ قطب ہو تب ہم
\begin{align*}
 f(z)=\tfrac{p(z)}{q(z)}
\end{align*}
 لکھتے ہیں جہاں \عددی{p(z)} اور \عددی{q(z)} نقطہ \عددی{z=a} پر تحلیلی ہیں، \عددی{p(a)\ne 0} ہے اور \عددی{q(z)} کا نقطہ \عددی{z=a} پر سادہ صفر پایا جائے گا۔نتیجتاً \عددی{q(z)} کو ٹیلر تسلسل
\begin{align*}
q(z)=(z-a)q'(a)+\frac{(z-a)^2}{2!}q''(a)+\cdots
\end{align*}
کی صورت میں لکھا جا سکتا ہے۔یوں مساوات \حوالہ{مساوات_بقیہ_پہلا_کلیہ}  سے 
\begin{align*}
\underset{z=a\hfill}{\Res f(z)}=\lim_{z\to a} (z-a)\frac{p(z)}{q(q)}=\lim_{z\to a}\frac{(z-a)p(z)}{(z-a)[q'(a)+\tfrac{1}{2}(z-a)q''(a)+\cdots]}
\end{align*}
یعنی 
\begin{align}\label{مساوات_بقیہ_دوسرا_کلیہ}
\underset{z=a\hfill}{\Res f(z)}=\underset{z=a\hfill}{\Res} \frac{p(z)}{q(z)}=\frac{p(a)}{q'(a)}
\end{align}
حاصل ہو گا جو سادہ قطب کی صورت میں بقیہ حاصل کرنے کا دوسرا کلیہ ہے۔

%==============
\ابتدا{مثال}\شناخت{مثال_بقیہ_دو_ندرت}\quad \موٹا{سادہ قطب کی صورت میں بقیہ}\\
تفاعل \عددی{f(z)=\tfrac{4-3z}{z^2-z}} کا \عددی{z=0} اور \عددی{z=1} پر سادہ قطب پائے جاتے ہیں۔مساوات \حوالہ{مساوات_بقیہ_دوسرا_کلیہ}کی مدد سے درج ذیل حاصل ہوتا ہے۔
\begin{align*}
\underset{z=0\hfill}{\Res} f(z)=\big[\frac{4-3z}{2z-1}\big]_{z=0}=-4,\quad \underset{z=1\hfill}{\Res} f(z) =\big[\frac{4-3z}{2z-1}\big]_{z=1}=1
\end{align*}
\انتہا{مثال}
%===================

آئیں اب \اصطلاح{بلند درجی قطبین}\فرہنگ{قطب!بلند درجی} کی بات کرتے ہیں۔اگر نقطہ \عددی{z=a} پر \عددی{f(z)} کے قطب کا درجہ \عددی{m>1} ہو تب  تفاعل کا لوغوں تسلسل
\begin{align*}
f(z)=\frac{c_m}{(z-a)^m}+\frac{c_{m-1}}{(z-a)^{m-1}}+\cdots+\frac{c_2}{(z-a)^2}+\frac{c_1}{z-a}
+b_0+b_1(z-a)+\cdots
\end{align*}
ہو گا جہاں \عددی{c_m \ne 0} ہے اور نقطہ \عددی{z=a} کی پڑوس میں، ماسوائے نقطہ \عددی{z=a} پر، تسلسل مرتکز ہو گا۔ دونوں اطراف کو \عددی{(z-a)^m} سے ضرب دیتے ہوئے 
\begin{multline*}
(z-a)^m f(z)=c_m+c_{m-1}(z-a)+\cdots+c_2(z-a)^{m-2}+c_1(z-a)^{m-1}\\
+b_0(z-a)^m+b_1(z-a)^{m+1}+\cdots
\end{multline*}
ملتا ہے۔یوں نقطہ \عددی{z=a} پر \عددی{f(z)} کا بقیہ \عددی{c_1} اب تفاعل \عددی{g(z)=(z-a)^mf(z)} کا \عددی{z=a} کے گرد  ٹیلر تسلسل میں \عددی{(z-a)^{m-1}} کا عددی سر ہے۔یوں مسئلہ ٹیلر (مسئلہ \حوالہ{مسئلہ_ٹیلر_مسئلہ_ٹیلر}) کے تحت درج ذیل ہو گا۔
\begin{align*}
c_1=\frac{g^{(m-1)}(a)}{(m-1)!}
\end{align*}
یوں اگر نقطہ \عددی{z=a} پر \عددی{f(z)} کے قطب کا درجہ \عددی{m} ہو تب بقیہ درج ذیل (تیسرا) کلیہ دے گا۔
\begin{align}\label{مساوات_بقیہ_تیسرا_کلیہ}
\underset{z=a\hfill}{\Res} f(z)=\frac{1}{(m-1)!}\lim_{z\to a}\big\{\frac{\dif^{\, m-1}}{\dif z^{m-1}}[(z-a)^mf(z)]\big\}
\end{align}

%===================
\ابتدا{مثال}\quad \موٹا{بلند درجہ قطب پر بقیہ}\\
تفاعل
\begin{align*}
f(z)=\frac{2z}{(z+4)(z-1)^2}
\end{align*}
کا \عددی{z=1} پر دو درجی قطب پایا جاتا ہے۔یوں مساوات \حوالہ{مساوات_بقیہ_تیسرا_کلیہ} درج ذیل بقیہ دے گا۔
\begin{align*}
\underset{z=1\hfill}{\Res} f(z)=\lim_{z=1}\frac{\dif}{\dif z}[(z-1)^2 f(z)]=\lim_{z=1}\frac{\dif}{\dif z}\big(\frac{2z}{z+4}\big)=\frac{8}{25}
\end{align*}
\انتہا{مثال}
%=======================
ظاہر ہے کہ ناطق تفاعل \عددی{f(z)} کی صورت میں بقیہ کو \عددی{f(z)} کی جزوی کسری پھیلاو سے بھی حاصل کیا جا سکتا ہے۔

%==================
\ابتدا{مثال}\quad
\begin{align*}
f(z)=\frac{7z^4-13z^3+z^2+4z-1}{(z^3+z^2)(z-1)^2}=\frac{3}{z}-\frac{1}{z^2}+\frac{4}{z+1}-\frac{1}{(z-1)^2}
\end{align*}
لکھتے ہوئے درج ذیل بقیہ حاصل ہوں گے۔
\begin{align*}
\underset{z=0\hfill}{\Res} f(z)=3,\quad \underset{z=-1\hfill}{\Res} f(z)=4,\quad \underset{z=1\hfill}{\Res} f(z)=0
\end{align*}
\انتہا{مثال}
%======================= 

\حصہء{سوالات}
سوال \حوالہ{سوال_بقیہ_ندرت_پر_تلاش_الف} تا سوال \حوالہ{سوال_بقیہ_ندرت_پر_تلاش_ب} میں دیے تفاعل کا ندرت پر بقیہ تلاش کریں۔

%==================
\ابتدا{سوال}\شناخت{سوال_بقیہ_ندرت_پر_تلاش_الف}\quad
$\tfrac{1}{1-z}$\\
جواب:\quad
نقطہ \عددی{z=1} پر بقیہ \عددی{-1} ہے۔
\انتہا{سوال}
%==================
\ابتدا{سوال}\quad
$\tfrac{z-3}{z+1}$\\
جواب:\quad
نقطہ \عددی{z=-1} پر بقیہ \عددی{-4} ہے۔
\انتہا{سوال}
%==================
\ابتدا{سوال}\quad
$\tfrac{1}{z^2}$\\
جواب:\quad
نقطہ \عددی{z=0} پر بقیہ \عددی{0} ہے۔
\انتہا{سوال}
%==================
\ابتدا{سوال}\quad
$\tfrac{z}{z^2-1}$\\
جواب:\quad
نقطہ \عددی{z=1} اور \عددی{z=-1} پر بقیہ بالترتیب  \عددی{\tfrac{1}{2}} اور \عددی{\tfrac{1}{2}} ہیں۔
\انتہا{سوال}
%==================
\ابتدا{سوال}\quad
$\tfrac{1}{z^2+1}$\\
جواب:\quad
نقطہ \عددی{z=-i} اور \عددی{z=i} پر بقیہ بالترتیب  \عددی{\tfrac{i}{2}} اور \عددی{\tfrac{-}{2}-} ہیں۔
\انتہا{سوال}
%==================
\ابتدا{سوال}\quad
$\tfrac{1}{(z^2+1)^2}$\\
جواب:\quad
نقطہ \عددی{z=-i} اور \عددی{z=i} پر بقیہ بالترتیب  \عددی{\tfrac{i}{4}} اور \عددی{\tfrac{i}{4}-} ہیں۔
\انتہا{سوال}
%==================
\ابتدا{سوال}\quad
$\tfrac{1}{(z^2-1)^2}$\\
جواب:\quad
نقطہ \عددی{z=-1} اور \عددی{z=1} پر بقیہ بالترتیب  \عددی{\tfrac{1}{4}} اور \عددی{\tfrac{1}{4}-} ہیں۔
\انتہا{سوال}
%==================
\ابتدا{سوال}\quad
$\tfrac{z}{z^4-1}$\\
جواب:\quad
نقطہ \عددی{z=-1,1,-i,i} پر بقیہ اسی ترتیب سے  \عددی{\tfrac{1}{4}, \tfrac{1}{4},-\tfrac{1}{4},-\tfrac{1}{4}} ہیں۔
\انتہا{سوال}
%==================
\ابتدا{سوال}\quad
$\tfrac{1}{z^4-1}$\\
جواب:\quad
نقطہ \عددی{z=-1,1,-i,i} پر بقیہ اسی ترتیب سے  \عددی{-\tfrac{1}{4}, \tfrac{1}{4},-\tfrac{i}{4},\tfrac{i}{4}} ہیں۔
\انتہا{سوال}
%==================
\ابتدا{سوال}\quad
$\tfrac{1}{1-e^z}$\\
جواب:\quad
نقطہ \عددی{z=\mp i2n\pi} پر بقیہ \عددی{-1} ہے۔
\انتہا{سوال}
%==================
\ابتدا{سوال}\quad
$\sec z$\\
جواب:\quad 
نقطہ \عددی{z=\tfrac{\pi}{2}+2n\pi} اور \عددی{z=-\tfrac{\pi}{2}-2n\pi} پر بقیہ بالترتیب \عددی{-1} اور \عددی{1} ہے جہاں \عددی{n=0,1,2\cdots} ہے۔
\انتہا{سوال}
%=====================
\ابتدا{سوال}\quad
$\tan z$\\
جواب:\quad 
نقطہ \عددی{z=\tfrac{\pi}{2}+n\pi} پر بقیہ \عددی{-1}  ہے جہاں \عددی{n=\mp 1,\mp 2,\cdots} ہے۔
\انتہا{سوال}
%=====================
\ابتدا{سوال}\شناخت{سوال_بقیہ_ندرت_پر_تلاش_ب}\quad
$\cot z$\\
جواب:\quad 
نقطہ \عددی{z=\mp n\pi} پر بقیہ \عددی{1}  ہے۔
\انتہا{سوال}
%=====================
سوال \حوالہ{سوال_بقیہ_دائرہ_الف} تا سوال \حوالہ{سوال_بقیہ_دائرہ_ب} میں دائرہ \عددی{\abs{z}=1.5} کے اندر ندرت پر تفاعل کا بقیہ تلاش کریں۔

%===================
\ابتدا{سوال}\شناخت{سوال_بقیہ_دائرہ_الف}\quad
$\tfrac{3z^2}{1-z^4}$\\
جواب:
نقطہ \عددی{z=-1,1,-i,i} پر بقیہ اسی ترتیب سے  \عددی{\tfrac{3}{4}, i\tfrac{3}{4}} ہیں۔
\انتہا{سوال}
%=====================
\ابتدا{سوال}\quad
$\tfrac{z-\tfrac{3}{4}}{z^2-3z+2}$\\
جواب:
نقطہ \عددی{z=1} پر بقیہ  \عددی{-\tfrac{1}{4}} ہے۔
\انتہا{سوال}
%=====================
\ابتدا{سوال}\quad
$\tfrac{6z+1}{z^2-3z}$\\
جواب:
نقطہ \عددی{z=0} پر بقیہ  \عددی{-\tfrac{1}{3}} ہے۔
\انتہا{سوال}
%=====================
\ابتدا{سوال}\quad
$\tfrac{z-1}{(z+1)(z^2+16)}$\\
جواب:
نقطہ \عددی{z=-1} پر بقیہ  \عددی{-\tfrac{2}{17}} ہے۔
\انتہا{سوال}
%=====================
\ابتدا{سوال}\شناخت{سوال_بقیہ_دائرہ_ب}\quad
$\tfrac{4+3z}{z^3-3z^2+2z}$\\
جواب:
نقطہ \عددی{z=0,1} پر اسی ترتیب سے بقیہ  \عددی{2,-7} ہیں۔
\انتہا{سوال}
%=====================
سوال \حوالہ{سوال_بقیہ_تکمل_اکائی_دائرہ_الف} تا سوال \حوالہ{سوال_بقیہ_تکمل_اکائی_دائرہ_ب} میں اکائی دائرے پر گھڑی کی الٹ رخ تکمل کی قیمت تلاش کریں۔

%===============
\ابتدا{سوال}\شناخت{سوال_بقیہ_تکمل_اکائی_دائرہ_الف}\quad
$\int_C e^{\tfrac{1}{z}}\dif z$\\
جواب:\quad
$i2\pi$

\انتہا{سوال}
%======================
\ابتدا{سوال}\quad
$\int_C ze^{\tfrac{1}{z}}\dif z$
\انتہا{سوال}
%======================
\ابتدا{سوال}\quad
$\int_C \cot z\dif z$\\
جواب:\quad
$i2\pi$
\انتہا{سوال}
%======================
\ابتدا{سوال}\quad
$\int_C \tan z\dif z$
\انتہا{سوال}
%======================
\ابتدا{سوال}\quad
$\int_C \tfrac{\dif z}{\sin z}$\\
جواب:\quad
$i2\pi$
\انتہا{سوال}
%======================
\ابتدا{سوال}\quad
$\int_C \tfrac{z}{2z+i}\dif z$
\انتہا{سوال}
%======================
\ابتدا{سوال}\quad
$\int_C \tfrac{\dif z}{\cosh z}$\\
جواب:\quad
$0$
\انتہا{سوال}
%======================
\ابتدا{سوال}\quad
$\int_C \tfrac{z^2-4}{(z-2)^4}\dif z$
\انتہا{سوال}
%======================
\ابتدا{سوال}\quad
$\int_C \tfrac{z^2+1}{z^2-2z}\dif z$\\
جواب:\quad
$-i\pi$
\انتہا{سوال}
%======================
\ابتدا{سوال}\quad
$\int_C \tfrac{\sin \pi z}{z^4}\dif z$$-i\pi$
\انتہا{سوال}
%======================
\ابتدا{سوال}\quad
$\int_C \tfrac{\dif z}{1-e^z}\dif z$\\
جواب:\quad
$-i2\pi$
\انتہا{سوال}
%======================
\ابتدا{سوال}\شناخت{سوال_بقیہ_تکمل_اکائی_دائرہ_ب}\quad
$\int_C \tfrac{z^2+1}{e^z\sin z}\dif z$
\انتہا{سوال}
%======================

\حصہ{مسئلہ بقیہ}
گزشتہ حصے میں ہم نے ایسا ارتفاعی تکمل جس کے متکمل کا خط ارتفاع میں بند صرف ایک عدد ندرت پایا جاتا ہو کو حل کرنا سیکھا۔ہم اب دیکھیں گے کہ اسی ترکیب کو وسعت دے کر ان تکمل کو بھی حل کیا جا سکتا ہے جن کے متکمل کا خط ارتفاع میں بند  ایک سے زیادہ تنہا ندرت پائے جاتے ہوں۔

%==================
\ابتدا{مسئلہ}\شناخت{مسئلہ_بقیہ}\quad \موٹا{مسئلہ بقیہ}\\
فرض کریں کہ تفاعل \عددی{f(z)} سادہ بند راہ \عددی{ C} پر اور \عددی{C} کے اندر  تحلیلی ہے ماسوائے محدود تعداد کے نقطوں \عددی{a_1,a_2,\cdots,a_m} پر جہاں \عددی{f(z)} کے ندرت پائے جاتے ہیں۔تب درج ذیل ہو گا جہاں \عددی{C} پر تکمل گھڑی کی الٹ رخ  حاصل کیا جائے گا۔
\begin{align}\label{مساوات_بقیہ_مسئلہ_بقیہ_الف}
\int_C f(z)\dif z=i2\pi\sum\limits_{j=1}^{m} \,\underset{z=a_j}{\Res} f(z)
\end{align} 
\انتہا{مسئلہ}
%=======================  
\ابتدا{ثبوت}\quad
ہم ہر ندرت \عددی{a_j} کو انفرادی دائرہ \عددی{C_j} میں بند کرتے ہیں جس کا رداس اتنا چھوٹا رکھا جاتا ہے کہ تمام \عددی{m} عدد دائرے اور \عددی{C} ایک دوسرے کو نہ چھوئے (شکل \حوالہ{شکل_بقیہ_مسئلہ_بقیہ})۔تب مضرب تعلق دائرہ کار \عددی{D} جس کے حدود \عددی{C} اور \عددی{C_1} تا \عددی{C_m} ہوں پر اور \عددی{D} کی تمام سرحد پر \عددی{f(z)} تحلیلی ہو گا۔کوشی مسئلہ تکمل سے 
\begin{align}\label{مساوات_بقیہ_مسئلہ_بقیہ_ب}
\int\limits_C f(z)\dif z+\int\limits_{C_1} f(z)\dif z+\int\limits_{C_2} f(z)\dif z+\cdots+\int\limits_{C_m} f(z)\dif z=0
\end{align}
لکھا جا سکتا ہے جہاں تکمل کو \عددی{C} پر گھڑی کی الٹ رخ اور \عددی{C_1} تا \عددی{C_m} پر تکمل کو گھڑی کی رخ حاصل کیا جاتا ہے (حصہ \حوالہ{حصہ_مخلوط_تکمل_کوشی_مسئلہ_تکمل})۔ ہم اب \عددی{C_1} تا \عددی{C_m} پر تکمل کا رخ الٹ کرتے ہیں جس سے ان تکمل کی قیمتوں کی علامت تبدیل ہو جائے گی لہٰذا مساوات \حوالہ{مساوات_بقیہ_مسئلہ_بقیہ_الف} سے 
\begin{align}\label{مساوات_بقیہ_مسئلہ_بقیہ_پ}
\int\limits_C f(z)\dif z=\int\limits_{C_1} f(z)\dif z+\int\limits_{C_2} f(z)\dif z+\cdots+\int\limits_{C_m} f(z)\dif z
\end{align}
حاصل ہو گا جہاں تمام تکمل گھڑی کی الٹ رخ حاصل کیے جائیں گے۔ اب چونکہ مساوات \حوالہ{مساوات_بقیہ_تکمل_پ} کے تحت
\begin{align*}
\int\limits_{C_j} f(z)\dif z=i2\pi \,\underset{z=a_j\hfill}{\Res} f(z)
\end{align*}
ہو گا لہٰذا مساوات \حوالہ{مساوات_بقیہ_مسئلہ_بقیہ_پ} سے مساوات \حوالہ{مساوات_بقیہ_مسئلہ_بقیہ_الف} حاصل ہو گا۔ یوں مسئلے کا ثبوت مکمل ہوتا ہے۔
\begin{figure}
\centering
\begin{tikzpicture}
\draw(0,0)node[ocirc]{}node[below]{$a_1$} circle (0.6);
\draw(1.5,0.5)node[ocirc]{}node[below]{$a_2$} circle (0.6);
\draw(3,0)node[ocirc]{}node[below]{$a_3$} circle (0.6);
\draw[->-=0.5](-1,0) to [out=-90,in=180] (0,-1) to [out=0,in=180](1.5,-0.5)to [out=0,in=180](3,-1) to [out=0,in=-90] (4,0) to [out=90,in=0] (1.5,1.5) to [out=180,in=90] (-1,0);
\draw(4,0)node[right]{$C$};
\end{tikzpicture}
\caption{مسئلہ بقیہ}
\label{شکل_بقیہ_مسئلہ_بقیہ}
\end{figure}
\انتہا{ثبوت}
%========================

اس اہم مسئلے کی مختلف مخلوط اور حقیقی تکملات  میں ضرورت پیش آتی ہے۔ہم چند مخلوط تکملات کی مثالیں پیش کرتے ہیں۔

%================
\ابتدا{مثال}\quad \موٹا{تکمل بذریعہ مسئلہ بقیہ}\\
تفاعل \عددی{\tfrac{4-3z}{z^2-z}} تحلیلی ہے  ماسوائے نقطہ \عددی{0} اور \عددی{1} کے جہاں تفاعل کے سادہ قطب پائے جاتے ہیں جن کے بقیہ بالترتیب \عددی{-4} اور \عددی{1} ہیں (مثال \حوالہ{مثال_بقیہ_دو_ندرت})۔یوں ہر اس راہ \عددی{C} کے لئے جو نقطہ \عددی{0} اور \عددی{1} دونوں کو گھیرتی ہے پر 
\begin{align*}
\int_C \frac{4-3z}{z^2-z}\dif z=i2\pi(-4+1)=-i6\pi
\end{align*} 
ہو گا جہاں تکمل گھڑی کی الٹ رخ حاصل کیا جائے گا۔اسی طرح ہر اس راہ \عددی{C} پر جس کے اندر نقطہ \عددی{z=0} پایا جاتا ہو جبکہ نقطہ \عددی{z=1} اس کے باہر پایا جاتا ہو کے لئے
\begin{align*}
\int_C \frac{4-3z}{z^2-z}\dif z=i2\pi(-4)=-i8\pi
\end{align*} 
ہو گا جہاں تکمل گھڑی کی الٹ رخ حاصل کیا جائے گا۔
\انتہا{مثال}
%=============================
\ابتدا{مثال}\quad \موٹا{متکمل کے بلند درجی قطبین پائے جاتے ہیں}\\
دائرہ \عددی{\abs{z-a}=1} پر گھڑی کی الٹ رخ تفاعل \عددی{\tfrac{1}{(z^3-1)^2}} کا تکمل تلاش کریں۔اس تفاعل کے نقطہ \عددی{z=1}، \عددی{z=e^{i\tfrac{2\pi}{2}}} اور \عددی{z=e^{-i\tfrac{2\pi}{3}}} پر دو درجی قطب پائے جاتے ہیں۔صرف نقطہ \عددی{z=1} پر قطب دائرے کے اندر ہے۔یوں
\begin{align*}
\int_C \frac{\dif z}{(z^3-1)^2}=i2\pi \,\underset{z=1\hfill}{\Res} \frac{1}{(z^3-1)^2}=i2\pi\big(-\frac{2}{9}\big)=-\frac{i4\pi}{9}
\end{align*}
ہو گا جہاں بقیہ کو مساوات \حوالہ{مساوات_بقیہ_تیسرا_کلیہ} کی مدد سے حاصل کیا گیا ہے۔
\انتہا{مثال}
%==========================
\ابتدا{مثال}\quad \موٹا{پہلے حاصل کردہ نتیجے کی تصدیق}\\
ہم تفاعل \عددی{\tfrac{1}{(z-a)^m}} جہاں \عددی{m} مثبت عدد صحیح ہے کا گھڑی کی الٹ رخ تکمل ایسی سادہ بند راہ  \عددی{C} پر حاصل کرتے ہیں جو نقطہ \عددی{z=a} کو گھیرتی ہو۔ پہلے بقیہ تلاش کرتے ہیں۔
\begin{align*}
\underset{z=a\hfill}{\Res} \frac{1}{z-a}=1,\quad \underset{z=a\hfill}{\Res} \frac{1}{(z-a)^m}=0 \quad (m=2,3,\cdots)
\end{align*}
یوں نتیجہ  عین مثال \حوالہ{مثال_مخلوط_تکمل_طاقت} کی طرح درج ذیل ہو گا۔
\begin{align*}
\int_C \frac{\dif z}{(z-a)^m}=
\begin{cases}
i2\pi \quad (m=1)\\
0&\quad (m=2,3,\cdots)
\end{cases}
\end{align*}

\انتہا{مثال}
%===========================

\حصہء{سوالات}
%=================
سوال \حوالہ{سوال_بقیہ_تلاش_تکمل_الف} تا سوال \حوالہ{سوال_بقیہ_تلاش_تکمل_ب} میں تفاعل \عددی{\tfrac{3z^2+2z-4}{z^3-4z}} کا تکمل گھڑی کی الٹ رخ دی گئی راہ \عددی{C} پر تلاش کریں۔

%===================
\ابتدا{سوال}\شناخت{سوال_بقیہ_تلاش_تکمل_الف}\quad
$\abs{z}=1$\\
جواب:\quad
$i2\pi$
\انتہا{سوال}
%==================
\ابتدا{سوال}\quad
$\abs{z}=3$\\
جواب:\quad
$i6\pi$
\انتہا{سوال}
%==================
\ابتدا{سوال}\شناخت{سوال_بقیہ_تلاش_تکمل_ب}\quad
$\abs{z-4}=1$\\
جواب:\quad
$0$
\انتہا{سوال}
%==================
سوال \حوالہ{سوال_بقیہ_تلاش_تکمل_گھڑی_رخ_الف} تا سوال \حوالہ{سوال_بقیہ_تلاش_تکمل_گھڑی_رخ_ب} میں تفاعل \عددی{\tfrac{z+1}{z(z-1)(z-2)}} کا تکمل گھڑی کی الٹ رخ دی گئی راہ \عددی{C} پر تلاش کریں۔

%====================
\ابتدا{سوال}\شناخت{سوال_بقیہ_تلاش_تکمل_گھڑی_رخ_الف}\quad
$\abs{z-2}=\tfrac{1}{2}$\\
جواب:\quad
$-i3\pi$
\انتہا{سوال}
%=======================
\ابتدا{سوال}\quad
$\abs{z}=\tfrac{3}{2}$\\
جواب:\quad
$i3\pi$
\انتہا{سوال}
%=======================
\ابتدا{سوال}\شناخت{سوال_بقیہ_تلاش_تکمل_گھڑی_رخ_ب}\quad
$\abs{z-\tfrac{1}{2}}=\tfrac{1}{4}$\\
جواب:\quad
$0$
\انتہا{سوال}
%=======================
سوال \حوالہ{سوال_بقیہ_اکائی_دائرہ_تکمل_الف} تا سوال \حوالہ{سوال_بقیہ_اکائی_دائرہ_تکمل_ب} کا تکمل اکائی دائرہ \عددی{C} پر گھڑی کی الٹ رخ حاصل کریں۔

%====================
\ابتدا{سوال}\شناخت{سوال_بقیہ_اکائی_دائرہ_تکمل_الف}\quad
$\int_C \,\tfrac{3z}{3z-1}\dif z$\\
جواب:\quad
$\tfrac{i2\pi}{3}$
\انتہا{سوال}
%==========================
\ابتدا{سوال}\quad
$\int_C \,\tfrac{z}{4z^2-1}\dif z$
\انتہا{سوال}
%==========================
\ابتدا{سوال}\quad
$\int_C \,\tfrac{\dif z}{z^2-2z}$\\
جواب:\quad
$-i\pi$
\انتہا{سوال}
%==========================
\ابتدا{سوال}\quad
$\int_C \,\tfrac{\dif z}{z^2+4}$
\انتہا{سوال}
%==========================
\ابتدا{سوال}\quad
$\int_C \,\tfrac{z+1}{4z^3-z}\dif z$\\
جواب:\quad
$0$
\انتہا{سوال}
%==========================
\ابتدا{سوال}\quad
$\int_C \,\tfrac{z^5-3z^3+1}{(2z+1)(z^2+4)}\dif z$
\انتہا{سوال}
%==========================
\ابتدا{سوال}\quad
$\int_C \,\tfrac{z}{1+9z^2}\dif z$\\
جواب:\quad
$\tfrac{i2\pi}{9}$
\انتہا{سوال}
%==========================
\ابتدا{سوال}\quad
$\int_C \,\tfrac{z+1}{z^4-2z^3}\dif z$
\انتہا{سوال}
%==========================

\ابتدا{سوال}\quad
$\int_C \,\tfrac{(z+4)^3}{z^4+5z^3+6z^2}\dif z$\\
جواب:\quad
$-\tfrac{i16\pi}{9}$
\انتہا{سوال}
%==========================
\ابتدا{سوال}\quad
$\int_C \,\tan z\dif z$
\انتہا{سوال}
%==========================
\ابتدا{سوال}\quad
$\int_C \,\tan \pi z\dif z$\\
جواب:\quad
$-i4$
\انتہا{سوال}
%==========================
\ابتدا{سوال}\quad
$\int_C \,\tfrac{6z^2-4z+1}{(z-2)(1+4z^2)}\dif z$
\انتہا{سوال}
%==========================
\ابتدا{سوال}\quad
$\int_C \,\tan 2\pi z\dif z$\\
جواب:\quad
$-i4$
\انتہا{سوال}
%==========================
\ابتدا{سوال}\quad
$\int_C \,\tfrac{\tan \pi z}{z^3}\dif z$
\انتہا{سوال}
%==========================
\ابتدا{سوال}\quad
$\int_C \,\tfrac{e}{z^2-5z}\dif z$\\
جواب:\quad
$-\tfrac{i2\pi}{5}$
\انتہا{سوال}
%==========================
\ابتدا{سوال}\quad
$\int_C \,\tfrac{e^z}{\sin z}\dif z$
\انتہا{سوال}
%==========================
\ابتدا{سوال}\quad
$\int_C \,\tfrac{e^z}{\cos z}\dif z$\\
جواب:\quad
$0$
\انتہا{سوال}
%==========================
\ابتدا{سوال}\quad
$\int_C \,\tfrac{e^z}{\cos \pi z}\dif z$
\انتہا{سوال}
%==========================
\ابتدا{سوال}\quad
$\int_C \,\tfrac{\cosh z}{z^2-i3z}\dif z$\\
جواب:\quad
$-\tfrac{i2\pi}{3}$
\انتہا{سوال}
%==========================
\ابتدا{سوال}\quad
$\int_C \,\coth z\dif z$
\انتہا{سوال}
%==========================
\ابتدا{سوال}\quad
$\int_C \,\tfrac{\sinh z}{2z-i}\dif z$\\
جواب:\quad
$-\pi\sin \tfrac{1}{2}$
\انتہا{سوال}
%==========================
\ابتدا{سوال}\quad
$\int_C \,\cot z\dif z$
\انتہا{سوال}
%==========================
\ابتدا{سوال}\quad
$\int_C \,\tfrac{\cot z}{z}\dif z$\\
جواب:\quad
$0$
\انتہا{سوال}
%==========================
\ابتدا{سوال}\شناخت{سوال_بقیہ_اکائی_دائرہ_تکمل_ب}\quad
$\int_C \,\tfrac{e^{z^2}}{\cos \pi z}\dif z$
\انتہا{سوال}
%==========================

\حصہ{حقیقی تکمل بذریعہ مسئلہ بقیہ}
کئی پیچیدہ قسم کے حقیقی تکمل کو نہایت نفاست کے ساتھ مسئلہ بقیہ کی  مدد سے حل کیا جا سکتا ہے۔

\جزوحصہء{\عددی{\cos\theta} اور \عددی{\sin\theta} کے ناطق تفاعل کے تکمل}
ہم سب سے پہلے درج ذیل قسم کے تکمل پر غور کرتے ہیں
\begin{align}\label{مساوات_بقیہ_حقیقی_تکمل_الف}
I=\int\limits_{0}^{2\pi} R(\cos \theta,\sin\theta)\dif \theta
\end{align}
جہاں \عددی{R} وقفہ \عددی{0\le \theta\le 2\pi} پر متناہی حقیقی ناطق تفاعل  ہے جس کے متغیرات \عددی{\cos\theta} اور \عددی{\sin\theta} ہیں۔ہم \عددی{e^{i\theta}=z} لے کر
\begin{align*}
\cos\theta&=\frac{1}{2}(e^{i\theta}+e^{-i\theta})=\frac{1}{2}(z+\frac{1}{z})\\
\sin\theta&=\frac{1}{2}(e^{i\theta}-e^{-i\theta})=\frac{1}{i2}(z-\frac{1}{z})
\end{align*}
لکھتے ہوئے دیکھتے ہیں کہ متکمل، \عددی{z} کا ناطق تفاعل مثلاً \عددی{f(z)} بنتا ہے۔\عددیء{\theta} کو  \عددی{0} تا \عددی{2\pi} کرنے سے \عددی{z} اکائی دائرہ \عددی{\abs{z}=1} پر گھڑی کی الٹ رخ ایک چکر کاٹتا ہے۔چونکہ \عددی{\tfrac{\dif z}{\dif \theta}=ie^{i\theta}} ہے لہٰذا \عددی{\dif \theta=\tfrac{\dif z}{iz}} ہو گا اور یوں تکمل درج ذیل صورت اختیار کرتا ہے
\begin{align}\label{مساوات_بقیہ_حقیقی_تکمل_ب}
I=\int_C\,f(z)\frac{\dif z}{iz}
\end{align}
جہاں اکائی دائرے پر گھڑی کی الٹ رخ تکمل حاصل کیا جاتا ہے۔

%==================
\ابتدا{مثال}\quad \موٹا{حقیقی تکمل (قسم مساوات \حوالہ{مساوات_بقیہ_حقیقی_تکمل_ب})}\\
فرض کریں کہ \عددی{p} وقفہ \عددی{0<p<1} میں کوئی مقررہ عدد ہے۔ہم درج ذیل پر غور کرتے ہیں۔
\begin{align*}
\int\limits_0^{2\pi}\frac{\dif \theta}{1-2p\cos\theta+p^2}=\int\limits_C\frac{\frac{\dif z}{iz}}{1-2p\frac{1}{2}\big(z+\frac{1}{z}\big)+p^2}=\int\limits_C \frac{\dif z}{i(1-pz)(z-p)}
\end{align*}
متکمل کے \عددی{z=\tfrac{1}{p}>1} اور \عددی{z=p<1} پر سادہ قطبین پائے جاتے ہیں۔صرف \عددی{z=p} پر قطب اکائی دائرہ \عددی{C} کے اندر پایا جاتا ہے جس کا بقیہ
\begin{align*}
\underset{z=p\hfill}{\Res} \frac{1}{i(1-pz)(z-p)}=\big[\frac{1}{i(1-pz)}\big]_{z=p}=\frac{1}{i(1-p^2)}
\end{align*}
ہے۔ یوں مسئلہ بقیہ کے تحت تکمل کی قیمت درج ذیل ہو گی۔
\begin{align*}
\int\limits_0^{2\pi}\frac{\dif \theta}{1-2p\cos\theta+p^2}=i2\pi\frac{1}{i(1-p^2)}=\frac{2\pi}{1-p^2}\quad \quad (0<p<1)
\end{align*}
\انتہا{مثال}
%========================

\جزوحصہء{ناطق تفاعل کے غیر مناسب تکمل}
ہم اب درج ذیل قسم کے حقیقی تکمل پر غور کرتے ہیں۔
\begin{align}\label{مساوات_بقیہ_غیر_مناسب_تکمل_الف}
\int\limits_{-\infty}^{\infty} f(x)\dif x
\end{align} 
اس قسم کا تکمل جس میں تکمل کے حدود غیر متناہی ہوں کو \اصطلاح{غیر مناسب تکمل}\فرہنگ{غیر مناسب!تکمل}\فرہنگ{تکمل!غیر مناسب}\حاشیہب{improper integral}\فرہنگ{improper!integral} کہتے ہیں اور اس سے مراد درج ذیل ہے۔
\begin{align}\label{مساوات_بقیہ_غیر_متناہی_تکمل_حد}
\int\limits_{-\infty}^{\infty} f(x)\dif x=\lim_{a\to-\infty}\int\limits_{a}^{0} f(x)\dif x+\lim_{b\to\infty}\int\limits_{0}^{b} f(x)\dif x
\end{align}
اگر دونوں حد موجود ہوں تب دونوں راہ کو ایک ساتھ ملا کر ہم درج ذیل لکھتے ہیں
\حاشیہد{مساوات \حوالہ{مساوات_بقیہ_کوشی_صدر_قیمت} کا دایاں ہاتھ تکمل کی \موٹا{کوشی صدر قیمت} کہلاتی ہے؛ جو مساوات \حوالہ{مساوات_بقیہ_غیر_متناہی_تکمل_حد} کے حد کی غیر موجودگی میں بھی موجود ہو سکتی ہے۔مثال کے طور پر\\
$\lim\limits_{b\to\infty}\int_0^bx\dif x=\infty\quad \text{لیکن}\quad \lim\limits_{r\to\infty}\int_{-r}^r x\dif x=\lim\limits_{r\to\infty} (\tfrac{r^2}{2}-\tfrac{r^2}{2})=0$
}۔

\begin{align}\label{مساوات_بقیہ_کوشی_صدر_قیمت}
\int\limits_{-\infty}^{\infty}f(x)\dif x=\lim_{r=\infty}\int\limits_{-r}^{r} f(x)\dif x
\end{align} 

ہم فرض کرتے ہیں کہ مساوات \حوالہ{مساوات_بقیہ_غیر_مناسب_تکمل_الف} میں تفاعل \عددی{f(x)} حقیقی ناطق تفاعل ہے جس کا نسب نما تمام حقیقی \عددی{x} کے لئے غیر صفر ہے اور جس کا درجہ شمار کنندہ  سے کم از کم \عددی{2} زیادہ ہے ۔تب  مساوات \حوالہ{مساوات_بقیہ_غیر_متناہی_تکمل_حد} کے حد موجود ہوں گے لہٰذا ہم  مساوات \حوالہ{مساوات_بقیہ_کوشی_صدر_قیمت} استعمال کر سکتے ہیں۔ہم مطابقتی ارتفاعی تکمل
\begin{align}\label{مساوات_بقیہ_تکمل_ارتفاعی}
\int_C\, f(z)\dif z
\end{align}
پر غور کرتے ہیں جس کی راہ \عددی{C} کو شکل \حوالہ{شکل_مساوات_بقیہ_تکمل_ارتفاعی} میں دکھایا گیا ہے۔چونکہ \عددی{f(x)} ناطق ہے، بالائی نصف مستوی میں \عددی{f(z)} کے قطبین کی تعداد متناہی ہے اور اگر ہم \عددی{r} کو کافی بڑا منتخب کریں تب \عددی{C} ان تمام قطبین کو گھیرے گی۔تب مسئلہ بقیہ کے تحت
\begin{align*}
\int\limits_C f(z)\dif z=\int\limits_S f(z)\dif z+\int\limits_{-r}^{r} f(x)\dif x=i2\pi\sum \Res f(z)
\end{align*}
ہو گا جہاں مجموعہ، بالائی نصف مستوی میں ان تمام نقطوں پر \عددی{f(z)} کے  بقیہ پر مشتمل  ہے جہاں \عددی{f(z)} کا قطب پایا جاتا ہو۔اس سے ہم درج ذیل لکھ سکتے ہیں۔
\begin{align}\label{مساوات_بقیہ_ناطق_اور_مجموعہ_بقیہ}
\int\limits_{-r}^{r}f(x)\dif x=i2\pi\sum \Res f(z)-\int\limits_S f(z)\dif z
\end{align}
%
\begin{figure}
\centering
\begin{tikzpicture}
\draw(-1.5,0)--(2,0)node[right]{$x$};
\draw(0,-0.25)--(0,1.5)node[left]{$y$};
\draw[thick,->-=0.25](-1,-0)node[below,xshift={(-0.1cm)}]{$-r$}--(1,0)node[below]{$r$};
\draw[thick,->-=0.25] ([shift={(0:1)}]0,0) arc (0:180:1);
\draw(45:1.1)node[right]{$S$};
\end{tikzpicture}
\caption{ارتفاعی تکمل (مساوات \حوالہ{مساوات_بقیہ_تکمل_ارتفاعی}) کی راہ}
\label{شکل_مساوات_بقیہ_تکمل_ارتفاعی}
\end{figure}

ہم اب ثابت کرتے ہیں کہ \عددی{r\to \infty} کرنے سے نصف دائرہ \عددی{S} پر تکمل کی قیمت صفر تک پہنچتی ہے۔اگر ہم \عددی{z=re^{i\theta}} لیں تب ہم \عددی{S} کو \عددی{r=\text{مستقل}} سے ظاہر کریں گے اور جیسے جیسے \عددی{z} نصف دائرہ \عددی{S} پر چلتا ہے ویسے ویسے  متغیرہ \عددی{\theta} کی قیمت \عددی{0} سے \عددی{2\pi} تک پہنچتی ہے۔چونکہ نسب نما کا درجہ شمار کنندہ کے درجہ سے کم از کم \عددی{2} زیادہ ہے لہٰذا کافی بڑے مستقل \عددی{k} اور \عددی{r_0} کے لئے درج ذیل ہو گا۔
\begin{align*}
\abs{f(z)}<\frac{k}{\abs{z}^2}\quad \quad \quad (\abs{z}=r>r_0)
\end{align*}
مساوات \حوالہ{مساوات_مخلوط_تکمل_حتمی_قیمت_تخمینہ} کی اطلاق سے 
\begin{align*}
\abs{\int_S f(z)\dif z}<\frac{k}{r^2}\pi r=\frac{k\pi}{r}\quad \quad \quad (r>r_0)
\end{align*}
حاصل ہوتا ہے۔یوں  جیسے جیسے \عددی{r} لامتناہی تک پہنچتا ہے ویسے ویسے \عددی{S} پر تکمل کی قیمت  صفر تک پہنچتی ہے  لہٰذا مساوات \حوالہ{مساوات_بقیہ_کوشی_صدر_قیمت} اور  مساوات \حوالہ{مساوات_بقیہ_ناطق_اور_مجموعہ_بقیہ} سے درج ذیل لکھا جا سکتا ہے
\begin{align}\label{مساوات_بقیہ_غیر-مناسب_تکمل}
\int\limits_{-\infty}^{\infty} f(x)\dif x=i2\pi\sum\Res f(z)
\end{align}
جہاں بالائی نصف مستوی میں \عددی{f(z)} کی تمام قطبین کے مطابقتی بقیہ کو مجموعہ میں شامل کیا جائے گا۔

%======================
\ابتدا{مثال}\شناخت{مثال_بقیہ_غیر_مناسب}\quad \موٹا{\عددی{0} تا \عددی{\infty} ایک غیر مناسب تکمل}\\
مساوات \حوالہ{مساوات_بقیہ_غیر-مناسب_تکمل} استعمال کرتے ہوئے ہم درج ذیل دکھانا چاہتے ہیں۔
\begin{align*}
\int_0^{\infty} \frac{\dif x}{1+x^4}=\frac{\pi}{2\sqrt{2}}
\end{align*}
تفاعل \عددی{\tfrac{1}{1+z^4}} کے چار عدد قطبین درج ذیل نقطوں پر پائے جاتے ہیں۔
\begin{align*}
z_1=e^{i\tfrac{\pi}{4}}, \quad z_2=e^{i\tfrac{3\pi}{4}},\quad z_3=e^{-i\tfrac{3\pi}{4}},\quad z_4=e^{-i\tfrac{\pi}{4}}
\end{align*}
ان میں سے \عددی{z_1} اور \عددی{z_2} پر قطبین بالائی نصف مستوی میں پائے جاتے ہیں (شکل \حوالہ{شکل_مثال_بقیہ_غیر_مناسب})۔ مساوات \حوالہ{مساوات_بقیہ_دوسرا_کلیہ} کی درج ذیل حاصل ہو گا۔
\begin{figure}
\centering
\begin{tikzpicture}
\draw(-1.5,0)--(2,0)node[right]{$x$};
\draw(0,-1.2)--(0,1.5)node[left]{$y$};
\draw(0,0) circle (1);
\draw(-45:1)node[ocirc]{}node[shift={(-45:0.3)}]{$z_4$};
\draw(45:1)node[ocirc]{}node[shift={(45:0.3)}]{$z_1$};
\draw(135:1)node[ocirc]{}node[shift={(135:0.3)}]{$z_2$};
\draw(-135:1)node[ocirc]{}node[shift={(-135:0.3)}]{$z_3$};
\end{tikzpicture}
\caption{شکل برائے مثال \حوالہ{مثال_بقیہ_غیر_مناسب}}
\label{شکل_مثال_بقیہ_غیر_مناسب}
\end{figure}
%
\begin{align*}
\underset{z=z_1\hfill}{\Res} f(z)&=\big[\frac{1}{(1+z^4)'}\big]_{z=z_1}=\big[\frac{1}{4z^3}\big]_{z=z_1}=\frac{1}{4}e^{-i\frac{3\pi}{4}}=-\frac{1}{4}e^{i\frac{\pi}{4}}\\
\underset{z=z_2\hfill}{\Res} f(z)&=\big[\frac{1}{(1+z^4)'}\big]_{z=z_2}=\big[\frac{1}{4z^3}\big]_{z=z_2}=\frac{1}{4}e^{-i\frac{9\pi}{4}}=\frac{1}{4}e^{-i\frac{\pi}{4}}
\end{align*} 
یوں مساوات \حوالہ{مساوات_مخلوط_ہذلولی_تعریف_الف} اور  مساوات \حوالہ{مساوات_بقیہ_غیر-مناسب_تکمل} سے  
\begin{align}\label{مساوات_بقیہ_مثال_جواب_الف}
\int\limits_{-\infty}^{\infty}\frac{\dif x}{1+x^4}=\frac{i2\pi}{4}(-e^{-\frac{\pi}{4}}+e^{-i\frac{\pi}{4}})=\pi\sin\frac{\pi}{4}=\frac{\pi}{\sqrt{2}}
\end{align}
لکھا جا سکتا ہے۔چونکہ \عددی{\tfrac{1}{1+x^4}} جفت تفاعل ہے لہٰذا 
\begin{align*}
\int\limits_{0}^{\infty}\frac{\dif x}{1+x^4}=\frac{1}{2}\int\limits_{-\infty}^{\infty}\frac{\dif x}{1+x^4}
\end{align*}
ہو گا۔اس سے اور مساوات \حوالہ{مساوات_بقیہ_مثال_جواب_الف} سے درکار نتیجہ حاصل ہوتا ہے۔
\انتہا{مثال}
%======================

\حصہء{سوالات}
سوال \حوالہ{سوال_بقیہ_سائن_الف} تا سوال \حوالہ{سوال_بقیہ_سائن_ب} میں تکمل حل کریں۔یہ تکمل \عددی{\cos\theta} اور \عددی{\sin\theta} پر مبنی ہیں۔ 

%===============
\ابتدا{سوال}\شناخت{سوال_بقیہ_سائن_الف}\quad
$\int\limits_{0}^{2\pi} \frac{\dif \theta}{2+\cos\theta}$\\
جواب:\quad
$\tfrac{2\pi}{\sqrt{3}}$

\انتہا{سوال}
%======================
\ابتدا{سوال}\quad
$\int\limits_{0}^{\pi} \frac{\dif \theta}{1+\frac{1}{3}\cos\theta}$

\انتہا{سوال}
%======================
\ابتدا{سوال}\quad
$\int\limits_{0}^{\pi} \frac{\dif \theta}{k+\cos\theta}\quad (k>1)$\\
جواب:\quad
$\tfrac{\pi}{\sqrt{k^2-1}}$

\انتہا{سوال}
%======================
\ابتدا{سوال}\quad
$\int\limits_{0}^{2\pi} \frac{\dif \theta}{25-24\cos\theta}$

\انتہا{سوال}
%======================
\ابتدا{سوال}\quad
$\int\limits_{0}^{2\pi} \frac{\dif \theta}{5-3\cos\theta}$\\
جواب:\quad
$\tfrac{\pi}{2}$

\انتہا{سوال}
%======================
\ابتدا{سوال}\quad
$\int\limits_{0}^{2\pi} \frac{\cos\theta}{17-8\cos\theta}\dif \theta$

\انتہا{سوال}
%======================
\ابتدا{سوال}\quad
$\int\limits_{0}^{2\pi} \frac{\cos \theta}{3+\sin\theta}\dif \theta$\\
جواب:\quad
$0$

\انتہا{سوال}
%======================
\ابتدا{سوال}\quad
$\int\limits_{0}^{2\pi} \frac{\cos\theta}{13-12\cos\theta}\dif \theta$

\انتہا{سوال}
%======================
\ابتدا{سوال}\quad
$\int\limits_{0}^{2\pi} \frac{\dif \theta}{\frac{5}{4}-\sin\theta}$\\
جواب:\quad
$\tfrac{8\pi}{3}$

\انتہا{سوال}
%======================
\ابتدا{سوال}\quad
$\int\limits_{0}^{2\pi} \frac{\sin^2\theta}{5-4\cos\theta}\dif \theta$

\انتہا{سوال}
%======================
\ابتدا{سوال}\quad
$\int\limits_{0}^{2\pi} \frac{\cos^2\theta}{26-10\cos2\theta}\dif \theta$\\
جواب:\quad
\begin{align*}
\cos 2\theta&=\tfrac{1}{2}(z^2+\tfrac{1}{z^2})\\
 \int_0^{2\pi} \tfrac{\cos^2\theta}{26-10\cos2\theta}\dif\theta&=-\tfrac{1}{i20}\int_C \tfrac{(z^2+1)^2}{z(z^2-\tfrac{1}{5})(z^5-5)}\dif z=\tfrac{\pi}{20}
\end{align*}

\انتہا{سوال}
%======================
\ابتدا{سوال}\شناخت{سوال_بقیہ_سائن_ب}\quad
$\int\limits_{0}^{2\pi} \frac{\cos^2 3\theta}{5-4\cos2\theta}\dif \theta$

\انتہا{سوال}
%======================

سوال \حوالہ{سوال_بقیہ_غیر_مناسب_الف} تا سوال \حوالہ{سوال_بقیہ_غیر_مناسب_ب} کے غیر مناسب تکمل حاصل کریں۔

%===================
\ابتدا{سوال}\شناخت{سوال_بقیہ_غیر_مناسب_الف}\quad
$\int\limits_{-\infty}^{\infty}\tfrac{\dif x}{1+x^2}$\\
جواب:\quad
$\pi$
\انتہا{سوال}
%=====================
\ابتدا{سوال}\quad
$\int\limits_{-\infty}^{\infty}\tfrac{\dif x}{(1+x^2)^2}$
\انتہا{سوال}
%=====================
\ابتدا{سوال}\quad
$\int\limits_{-\infty}^{\infty}\tfrac{\dif x}{1+x^6}$\\
جواب:\quad
$\tfrac{2\pi}{3}$
\انتہا{سوال}
%=====================
\ابتدا{سوال}\quad
$\int\limits_{-\infty}^{\infty}\tfrac{\dif x}{x^4+16}$
\انتہا{سوال}
%=====================
\ابتدا{سوال}\quad
$\int\limits_{-\infty}^{\infty}\tfrac{\dif x}{(1+x^2)^3}$\\
جواب:\quad
$\tfrac{3\pi}{8}$
\انتہا{سوال}
%=====================
\ابتدا{سوال}\شناخت{سوال_بقیہ_غیر_مناسب_پ}\quad
$\int\limits_{-\infty}^{\infty}\tfrac{x}{(4+x^2)^2}\dif x$
\انتہا{سوال}
%=====================
\ابتدا{سوال}\شناخت{سوال_بقیہ_غیر_مناسب_ت}\quad
$\int\limits_{-\infty}^{\infty}\tfrac{x^3}{1+x^8}\dif x$\\
جواب:\quad
$0$
\انتہا{سوال}
%=====================
\ابتدا{سوال}\quad
$\int\limits_{0}^{\infty}\tfrac{1+x^2}{1+x^4}\dif x$
\انتہا{سوال}
%=====================
\ابتدا{سوال}\quad
$\int\limits_{-\infty}^{\infty}\tfrac{\dif x}{(x^2+1)(x^2+9)}$\\
جواب:\quad
$\tfrac{\pi}{12}$
\انتہا{سوال}
%=====================
\ابتدا{سوال}\quad
$\int\limits_{-\infty}^{\infty}\tfrac{\dif x}{(x^2+1)(x^2+4)^2}$
\انتہا{سوال}
%=====================
\ابتدا{سوال}\quad
$\int\limits_{-\infty}^{\infty}\tfrac{x}{(x^2-2x+2)^2}\dif x$\\
جواب:\quad
$\tfrac{\pi}{2}$
\انتہا{سوال}
%=====================
\ابتدا{سوال}\شناخت{سوال_بقیہ_غیر_مناسب_ب}\quad
$\int\limits_{-\infty}^{\infty}\tfrac{x^2}{(x^2+1)(x^2+4)}\dif x$
\انتہا{سوال}
%=====================
\ابتدا{سوال}\quad
سوال \حوالہ{سوال_بقیہ_غیر_مناسب_ب}، سوال \حوالہ{سوال_بقیہ_غیر_مناسب_پ} اور سوال \حوالہ{سوال_بقیہ_غیر_مناسب_ت} کو بنیادی طریقہ سے حل کریں۔
\انتہا{سوال}
%=====================

\حصہ{حقیقی تکمل کے دیگر اقسام}

