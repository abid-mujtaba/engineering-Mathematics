
\حصہ{غیر مقدار معلوم پرکھ}\شناخت{حصہ_شماریات_پرکھ_برائے_غیر_مقدار_معلوم}
حصہ \حوالہ{حصہ_شماریات_قیاس_کی_پرکھ_فیصلے} کے پرکھ عمومی آبادی کے لئے تھے۔کئی بار آبادی کی تقسیم غیر عمومی یا نا معلوم تقسیم رکھتی ہے۔ایسی صورت  میں ہم \اصطلاح{غیر مقدار معلوم پرکھ}\فرہنگ{پرکھ!غیر مقدار معلوم}\حاشیہب{nonparametric test}\فرہنگ{test!nonparametric} یا \اصطلاح{تقسیم پاک پرکھ}\فرہنگ{پرکھ!تقسیم پاک}\حاشیہب{distribution-free test}\فرہنگ{test!distribution-free} استعمال کر سکتے ہیں جس کی بنیاد  \ترچھا{شماریات رجحان}\فرہنگ{شماریات!رجحان}\حاشیہب{order statistics}\فرہنگ{statistics!order} ہے لہٰذا اس کو کسی بھی \ترچھا{استمراری تقسیم} کے لئے استعمال کیا جا سکتا ہے۔ البتہ عمومی تقسیم کے لئے حصہ \حوالہ{حصہ_شماریات_قیاس_کی_پرکھ_فیصلے}
 کے پرکھ بہتر نتائج دیتے ہیں۔تقسیم پاک پرکھ کو سمجھنے کی خاطر ایک مثال پر غور کرتے ہیں۔

%====================
\ابتدا{مثال}\quad \موٹا{پرکھ برائے علامت وسطانیہ}\\
مساوات \عددی{F(x)=0.5} کے حل \عددی{x=\tilde{\mu}} کو وسطانیہ کہتے ہیں، جہاں \عددی{F} تفاعل تقسیم ہے۔ مثال \حوالہ{مثال_دو_عمومی_تقسیمات_موازنہ_تغیریت} کا نمونی فرق، یعنی،
\begin{align*}
\begin{array}{cccccccc}
16&16&2&6&0&0&13&8
\end{array}
\end{align*}
استعمال کرتے ہوئے ہم قیاس \عددی{\tilde{\mu}=0} کو پرکھتے ہیں جو کہتا ہے کہ   کام کرنے کے دو مختلف حالات میں مزدور کی کارکردگی تقریباً ایک جیسی ہے۔\\
حل:\quad
ہم متبادل \عددی{\tilde{\mu}>0} اور معنی خیز سطح \عددی{\alpha=\SI{5}{\percent}} منتخب کرتے ہوئے۔اگر قیاس درست ہو تب مثبت فرق کا احتمال \عددی{p} اور منفی فرق کا احتمال ایک جیسے ہوں گے۔ یوں \عددی{p=0.5} ہو گا اور بلا منصوبہ متغیر
\begin{align*}
X=\text{\RL{$n$ قیمتوں میں مثبت قیمتوں کا مجموعہ}}
\end{align*}
کا تقسیم ثنائی ہو گا جس کا \عددی{p=0.5} ہو گا۔ہمارے نمونے میں \عددی{8} قیمتیں ہیں۔ہم \عددی{0} قیمتوں کو خارج کرتے ہیں چونکہ ان کا فیصلہ پر کوئی اثر نہیں پایا جاتا ہے۔تب \عددی{6} قیمتیں رہ جاتی ہیں۔یہ تمام قیمتیں مثبت ہیں۔۔چونکہ
\begin{align*}
P(X=6)=\binom{6}{6} (0.5)^6(0.5)^0=0.0156=\SI{1.56}{\percent}<\alpha
\end{align*}
ہے لہٰذا ہم قیاس نا منظور کرتے ہیں۔

اگر ان \عددی{6} قیمتوں میں صرف \عددی{1} قیمت منفی ہوتی تب
\begin{align*}
P(X\ge 5)=\binom{6}{5}(0.5)^5\cdot 0.5+\binom{6}{6}(0.5)^6=\SI{10.9}{\percent}
\end{align*}
ہوتا اور ہم قیاس کو نا منظور نہ کرتے۔
\انتہا{مثال}
%===========================
\ابتدا{مثال}\شناخت{مثال_شماریات_بلا_منصوبہ_رجحان}\quad \موٹا{بلا منصوبہ رجحان کے لئے پرکھ}\\
تار کو کاٹنے کے لئے ایک مشین استعمال کی جاتی ہے۔لگاتار کٹی لمبائیاں درج ذیل ہیں۔
\begin{align*}
\begin{array}{ccccc}
29&31&28&30&32
\end{array}
\end{align*} 
اس نمونہ کو استعمال کرتے ہوئے اس قیاس کو پرکھیں کہ مشین تار کو \ترچھا{بغیر کسی رجحان} کاٹتی ہے، یعنی مشین مسلسل بڑھتی یا مسلسل گھٹتی لمبائی کی تار نہیں کاٹتی ہے۔فرض کریں کہ مشین کی قسم سے ایسا ظاہر ہوتا ہے کہ یہ مسلسل بڑھتی لمبائی کی تار کاٹے گی (مثبت رجحان)۔\\
حل:\quad
جتنی بار کوئی بڑی قیمت کسی چھوٹی قیمت سے پہلے رونما ہو، ہم ان \ترچھا{تبدیلیوں} کی تعداد گنتے ہیں۔\\
\centerline{
\عددی{29} قیمت \عددی{28} قیمت سے پہلے آتی ہے:\quad  ($1$ تبدیلی)
}
\centerline{
\عددی{31} کی قیمت \عددی{28} اور \عددی{30} سے پہلے آتی ہے:\quad  ($2$ تبدیلیاں)
}
باقی تین قیمتیں بڑھتی رجحان رکھتی ہیں۔یوں نمونہ میں \عددی{1+2=3} تبدیلیاں پائی جاتی ہیں۔ہم اب بلا منصوبہ متغیر\\
\centerline{
\عددی{=T}تعداد تبدیلیاں
}
پر غور کرتے ہیں۔اگر قیاس درست ہو (غیر رجحانی)، تب پانچ اجزاء \عددی{1\,\,2\,\,3\,\,4\,\,5} کے \عددی{5!=120} ترتیبی اجتماعات میں ہر ایک کا احتمال
 \عددی{\tfrac{1}{120}} ہو گا۔ ہم ان ترتیبی اجتماعات کو ان کی تبدیلیوں کے لحاظ سے لکھتے ہیں: 
\begin{align*}
\text{وغیرہ}\quad
\begin{array}{ccccc}
\multicolumn{5}{c}{T=3}\\
1&2&5&4&3\\
1&3&4&5&2\\
1&3&5&2&4\\
1&4&2&5&3\\
1&4&3&2&5\\
1&5&2&3&4\\
2&1&4&5&3\\
2&1&5&3&4\\
2&3&1&5&4\\
2&3&4&1&5\\
2&4&1&3&5\\
3&1&2&5&4\\
3&1&4&2&5\\
3&2&1&4&5\\
4&1&2&3&5
\end{array}\quad
\begin{array}{ccccc}
\multicolumn{5}{c}{T=2}\\
1&2&4&5&3\\
1&2&5&3&4\\
1&3&2&5&4\\
1&3&4&2&5\\
1&4&2&3&5\\
2&1&3&5&4\\
2&1&4&3&5\\
2&3&1&4&5\\
3&1&2&4&5
\end{array}\quad
\begin{array}{ccccc}
\multicolumn{5}{c}{T=1}\\
1&2&3&5&4\\
1&2&4&3&5\\
1&3&2&4&5\\
2&1&3&4&5
\end{array}\quad
\begin{array}{ccccc}
\multicolumn{5}{c}{T=0}\\
1&2&3&4&5
\end{array}
\end{align*}
ان سے ہم درج ذیل حاصل کرتے ہیں
\begin{align*}
P(T\le 3)=\frac{1}{120}+\frac{4}{120}+\frac{9}{120}+\frac{15}{120}=\frac{29}{120}=\SI{24}{\percent}
\end{align*}
لہٰذا ہم قیاس کو نا منظور نہیں کرتے ہیں۔

ضمیمہ \حوالہ{ضمیمہ_جدول} کی جدول \حوالہ{ضمیمہ_بلا_رجحان} میں بلا رجحان صورت میں بلا منصوبہ متغیر \عددی{T} کی تقسیم دی گئی ہے۔  ہمارے تراکیب اور اس  جدول کی قیمتیں استمراری تقسیمات کے کئے ہیں۔یوں ہم توقع کرتے ہیں کہ نمونہ کی تمام قیمتیں ایک دوسرے سے مختلف ہوں گی۔پور و پور کی بنا عملاً چند نمونی قیمتیں ایک جیسی ہو سکتی ہیں۔اگر \عددی{m} قیمتیں ایک جیسی ہوں تب \عددی{\tfrac{m(m-1)}{4}} (\عددی{m} اجزاء کی ترتیبی اجتماعات میں تبدیلیوں کے تعداد کی اوسط) جمع کریں، یعنی، ایک جیسی قیمتوں کے ہر جوڑی کے لئے \عددی{\tfrac{1}{2}}، ایک جیسی تین قیمتوں کے لئے \عددی{\tfrac{3}{2}}، وغیرہ۔
\انتہا{مثال}
%==================
\حصہء{سوالات}
%=============
\ابتدا{سوال}\quad
\عددی{10} کوششوں میں سے \عددی{7} کوششوں میں قسم الف ہوئی چھلنی نے قسم ب ہوائی چھلنی سے زیادہ صاف ہوا پیدا کی، \عددی{1} کوشش میں چھلنی ب نے زیادہ صاف ہوا پیدا کی جبکہ \عددی{2} کوششوں میں دونوں کے نتائج ایک جیسے تھے۔کیا چھلنی الف زیادہ بہتر ہے؟\\
جواب:\quad
قیاس: الف اور ب ایک جیسی معیار رکھتی ہیں۔تب \عددی{8} کوششوں میں \عددی{7} یا \عددی{8} بار الف کے حق میں وقوعہ کا احتمال \عددی{\SI{3.5}{\percent}} ہے۔قیاس کو نا منظور کریں۔ 
\انتہا{سوال}
%=======================
\ابتدا{سوال}\quad
کن صورتوں میں ہم پرکھ علامت کو استمراری تقسیم کی اوسط پرکھنے کے لئے استعمال کر سکتے ہیں۔
\انتہا{سوال}
%======================
\ابتدا{سوال}\quad
پرکھ علامت کو سوال \حوالہ{سوال_شماریات_ٹلسٹار}  کے نمونہ پر لاگو کریں۔\\
جواب:\quad
$P(X\le 2)=0.5^6(1+6+15)=\SI{34}{\percent}$
قیاس \عددی{\tilde{\mu}=0} کو نا منظور نہ کریں۔
\انتہا{سوال}
%======================
\ابتدا{سوال}\quad
اگر \عددی{\tilde{\mu}=0} کی بجائے قیاس \عددی{\tilde{\mu}=\tilde{\mu}_0} ہو تب آپ پرکھ علامت  کو کس طرح استعمال کریں گے۔ (\عددی{\mu_0} کوئی بھی عدد ہو سکتا ہے۔)
\انتہا{سوال}
%==========================
\ابتدا{سوال}\quad
\عددی{16} جسامت کے نمونہ میں \عددی{10} مثبت، \عددی{4} منفی اور \عددی{2} قیمتیں صفر ہیں۔(ضمیمہ \حوالہ{ضمیمہ_جدول} کی جدول \حوالہ{ضمیمہ_ثنائی_تقسیم} میں درکار قیمتیں نہیں دی گئی ہیں۔آپ کو یہ قیمتیں حاصل کرنی ہوں گی۔)\\
جواب:\quad
اگر \عددی{\tilde{\mu}=0} ہو، \عددی{14} میں سے \عددی{4} یا \عددی{4} سے کم عدد قیمتیں منفی ہونے کا احتمال \عددی{\SI{9}{\percent}} ہے۔قیاس \عددی{\tilde{\mu}=0} کو نا منظور نہ کریں۔
\انتہا{سوال}
%===========================
\ابتدا{سوال}\شناخت{سوال_شماریات_سلاخ_لمبائی}\quad
\عددی{\tilde{\mu}=5} میٹر لمبائی سلاخ پیدا کرنے  کے عمل کے ایک نمونہ میں \عددی{4} سلاخوں کی لمبائی ٹھیک ہے، \عددی{15} کی لمبائی کم اور \عددی{3} کی لمبائی زیادہ ہے۔ کیا اس عمل کو درست کرنے کی ضرورت ہے؟ (عمومی تقسیم کو ثنائی تقسیم کا تخمین لیں۔ حصہ \حوالہ{حصہ_شماریات_عمومی_تقسیم})
\انتہا{سوال}
%=======================
\ابتدا{سوال}\quad
مسئلہ \حوالہ{مسئلہ_شماریات_ڈی_موے_لاپلاس_تحدیدی} استعمال کیے بغیر سوال  \حوالہ{سوال_شماریات_سلاخ_لمبائی} کو حل کریں۔\\
جواب:\quad
\عددی{3} یا اس سے کم  سلاخوں کی لمبائی \عددی{5} میٹر سے زیادہ ہونے کا ٹھیک  احتمال \عددی{\SI{0.38}{\percent}} ہے۔یہ سوال \حوالہ{سوال_شماریات_سلاخ_لمبائی} میں حاصل تخمینی احتمال سے کچھ کم ہے۔
\انتہا{سوال}
%=====================
\ابتدا{سوال}\شناخت{سوال_شماریات_عمومی_پرکھ_استعمال}\quad
\عددی{10} مریضوں میں سے ہر ایک کو دو مختلف نیند کی دوائیاں دی گئی۔درج ذیل جدول ان کے اثرات (سونے کے دورانیے میں گھنٹوں میں اضافہ) پیش کرتا ہے۔پرکھ علامت کی مدد سے دیکھیں کہ آیا ان میں فرق معنی خیز ہے۔\\
\begin{align*}
\begin{array}{c|rrrrrrrrrr}
A&1.9&0.8&1.1&0.1&-0.1&4.4&5.5&1.6&4.6&3.4\\
B&0.7&-1.6&-0.2&-1.2&-0.1&3.4&3.7&0.8&0.0&2.0\\
\hline
\text{فرق} &1.2&2.4&1.3&1.3&0.0&1.0&1.8&0.8&4.6&1.4
\end{array}
\end{align*}
\انتہا{سوال}
%=========================
\ابتدا{سوال}\quad
مثال \حوالہ{مثال_شماریات_نا_معلوم_تغیریت_عمومی_تقسیم_اوسط_کا_پرکھ} میں سمجھائے گیے پرکھ کو  سوال \حوالہ{سوال_شماریات_عمومی_پرکھ_استعمال} پر لاگو کریں ۔(سوال میں دیے گیے  نمونہ کی آبادی کو عمومی تصور کریں۔)  \\
جواب:\quad
قیاس \عددی{\mu=0}؛ متبادل \عددی{\mu>0}، \عددی{\overline{x}=1.58}،\\
 $t=\sqrt{10}\cdot \tfrac{1.58}{1.23}=4.06>c=1.83\,(\alpha=\SI{5}{\percent})$
؛قیاس نا منظور۔
\انتہا{سوال}
%=========================
\ابتدا{سوال}\quad
نچلی چوتھائی \عددی{q_{25}} (جس کی تعریف \عددی{F(q_{25})=0.25} ہے) کے لئے پرکھ علامت بنائیں۔ 
\انتہا{سوال}
%=====================
\ابتدا{سوال}\quad
\عددی{8} قیمتوں کا نمونہ جس میں \عددی{7} کی قیمت \عددی{\SI{20}{\celsius}} سے کم اور \عددی{1} کی قیمت \عددی{\SI{20}{\celsius}} سے زیادہ ہو استعمال کرتے ہوئے  خود کار حراری سوئچ ٹھیک \عددی{\SI{20}{\celsius}} پر مقرر ہونے کے قیاس کو  بالمقابل کہ سوئچ کم درجہ حرارت پر مقرر ہے، پرکھیں۔\\
جواب:\quad
$P(X\ge1)=0.5^8(1+8)=\SI{3.5}{\percent}<\alpha=\SI{5}{\percent}$
؛ اس قیاس کو نا منظور کریں کہ سوئچ ٹھیک درجہ حرارت پر مقرر ہے۔
\انتہا{سوال}
%======================
\ابتدا{سوال}\quad
وولٹ پیما کی پیمائش درجہ حرارت \عددی{T[\si{\celsius}]} سے آزاد ہے کے قیاس کو بالمقابل کہ اس کی پیمائش بڑھتے \عددی{T} کے ساتھ بڑھتی ہے پرکھیں۔مستقل برقی دباو مہیا کرتے ہوئے حاصل درج ذیل پیمائشوں کا نمونہ استعمال کریں۔
\begin{align*}
\begin{array}{c|ccccc}
\text{\RL{درجہ حرارت}}\, T[\si{\celsius}]&10&20&30&40&50\\
\hline
\text{پیمائش}\,V[\si{\volt}]&99.8&101.0&100.4&100.8&101.5
\end{array}
\end{align*}
\انتہا{سوال}
%======================
\ابتدا{سوال}\quad
\عددی{n=4} لیتے ہوئے مثال \حوالہ{مثال_شماریات_بلا_منصوبہ_رجحان} میں دی گئی جدول کی طرح جدول بنائیں۔
\انتہا{سوال}
%============================
\ابتدا{سوال}\quad
کیا کھاد سے گندم کی استعمال سے  پیداوار \عددی{X\,[\si{\kilo\gram/\text{رقبہ}}]} بڑھتی ہے؟کھاد کی بڑھتی مقدار کے لحاظ سے مرتب درج ذیل نمونہ استعمال کریں۔
\begin{align*}
\begin{array}{cccccccc}
15.2&16.8&13.2&16.6&17.2&17.5&17.3&18.1
\end{array}
\end{align*}
\انتہا{سوال}
%========================
\ابتدا{سوال}\شناخت{سوال_شماریات_غیر_مقدار_معلوم_ڑ}\quad
 مثال \حوالہ{مثال_شماریات_بلا_منصوبہ_رجحان} کے پرکھ کو درج ذیل نمونہ پر لاگو کریں۔(اون میں ڈائی سلفائڈ کی مقدار  \عددی{x} جس کو کیمیائی عمل سے نا گزاری گئی اوون میں مقدار کے فی صد میں ناپا گیا ہے۔اون میں پانی کی فی صد مقدار \عددی{y} ہے۔)
\begin{align*}
\begin{array}{r|rrrrrrrr}
x&10&15&30&40&50&55&80&100\\
\hline
y&50&46&43&42&36&39&37&33
\end{array}
\end{align*}
\انتہا{سوال}
%===========================

\حصہ{پیمائشوں کی جوڑیاں۔ سیدھے خطوط کو موافق بنانا}
ہم اب ایسی تجربات پر غور کرتے ہیں جن میں ہم جوڑی مقدار ناپتے یا ان کا مشاہدہ کرتے ہیں۔ہم تجربات کو درج ذیل دو اقسام میں تقسیم کر سکتے ہیں۔  
\begin{itemize}
\item
\اصطلاح{تجزیہ باہمی رشتہ}\فرہنگ{تجزیہ!باہمی رشتہ}\حاشیہب{correlation analysis}\فرہنگ{correlation analysis} میں دونوں متغیرات بلا منصوبہ ہوں گے اور ہم ان کے درمیان رشتہ میں دلچسپی رکھتے ہیں۔(اس کتاب میں شماریات کی اس شاخ پر غور نہیں کی جائے گی۔)
\item
\اصطلاح{رجعی تجزیہ}\فرہنگ{رجعی تجزیہ}\حاشیہب{regression analysis}\فرہنگ{regression analysis} میں دو میں سے ایک متغیر، مثلاً \عددی{x}، کو عام متغیر تصور کیا جاتا ہے، یعنی، اس کی ناپ میں خاطر خواہ خلل نہیں پایا جاتا ہے۔دوسرا متغیر، \عددی{Y}، بلا منصوبہ متغیر ہے۔\عددی{x} کو غیر تابع متغیر کہتے ہیں اور ہم جاننا چاہتے ہیں کہ \عددی{Y}،  متغیر \عددی{x} کا کتنا تابع ہے؟اس کی ایک اچھی مثال فشار خون \عددی{Y} ہے جو انسان کے  عمر \عددی{x} کی تابع ہے، جس کو ہم اب سے \عددی{x} پر \عددی{Y} کی رجعت کہیں گے۔
\end{itemize}

تجربہ کرنے والا پہلے \عددی{x} کی \عددی{n} قیمتیں \عددی{x_1,\cdots,x_n} منتخب کرتا ہے اور اس کے بعد ان \عددی{x} پر \عددی{Y} کی قیمتیں مشاہدے سے حاصل کرتا ہے۔یوں اس کو درج ذیل صورت کا نمونہ ملتا ہے۔
\begin{align*}
(x_1,y_1), (x_2,y_2),\cdots,(x_n,y_n)
\end{align*} 
رجعی تجزیہ میں فرض کیا جاتا ہے کہ \عددی{Y} کی اوسط \عددی{\mu}، متغیر \عددی{x} کے تابع ہے، یعنی، ان کے مابین عام تعلق \عددی{\mu=\mu(x)} پایا جاتا ہے۔\عددی{\mu(x)} کی منحنی کو \ترچھا{\عددی{Y} کی \عددی{x} پر رجعی منحنی} کہتے ہیں۔اس حصہ میں ہم سادہ ترین صورت پر غور کرتے ہیں جہاں \عددی{\mu(x)} خطی تفاعل \عددی{\mu(x)=\alpha+\beta x} ہے۔ہم نمونی قیمتوں کو \عددی{xY} مستوی پر ترسیم کر کے، ان پر سیدھی خط بٹھا کر، اس خط کو استعمال کرتے ہوئے کسی بھی \عددی{x} کے لحاظ سے \عددی{\mu(x)} کی اندازاً قیمت حاصل کرنا چاہیں گے تا کہ کسی بھی \عددی{x} سے حاصل \عددی{Y} کی متوقع قیمت ہم جان سکیں۔اگر نقطے بکھرے ہوں تب، خط کو آنکھ کی مدد سے ٹھیک بٹھانا غیر یقینی ہو گا لہٰذا ہمیں حسابی طریقہ درکار ہو گا جو صرف نقطوں پر منحصر یکتا نتیجہ دے۔ایک بہت زیادہ استعمال ہونے والی ترکیب، جس کو گاوس نے بنایا،  \اصطلاح{کمتر مربعوں کی ترکیب}\فرہنگ{کمتر مربع!ترکیب}\حاشیہب{method of least squares}\فرہنگ{least squares!method} کہلاتی ہے۔ہمارے موجودہ ضرورت کو مد نظر رکھتے ہوئے اس کو درج ذیل بیان کیا جا سکتا ہے۔

\ابتدا{قانون}
نقطوں پر سیدھا خط یوں بٹھایا جائے کہ نقطوں کا سیدھی لکیر سے فاصلوں کا مربع کم سے کم ہو، جہاں نقطہ اور سیدھی لکیر کے مابین فاصلہ انتصابی رخ (\عددی{y} محور کے متوازی) ناپا جاتا ہے۔ 
\انتہا{قانون}

%============================
\ابتدا{مفروضہ} (الف)\\
نمونہ \عددی{(x_1,y_1),\cdots,(x_n,y_n)} میں تمام \عددی{x} قیمتیں \عددی{x_1,\cdots,x_n} ایک جیسی نہیں ہیں۔
\انتہا{مفروضہ}
%======================================

\begin{figure}
\centering
\begin{tikzpicture}
\draw(0,2)node[left]{$y$}--(0,0)--(4,0)node[right]{$x$};
\draw(0.5,0.25)--(3,1)coordinate[pos=0.4](kA)node[right]{$y=a+bx$};
\draw[thick](kA)--++(0,1)node[circ]{}node[right]{$y=y_j$};
\draw[dashed](kA)node[ocirc,solid]{}--($(0,0)!(kA)!(4,0)$)node[below]{$x_j$};
\end{tikzpicture}
\caption{نقطہ $(x_j,y_j)$ سے سیدھے خط $y=a+bx$ کا انتصابی فاصلہ}
\label{شکل_شماریات_خط_بٹھانا_انتصابی_فاصلہ}
\end{figure}

جسامت \عددی{n} کے نمونہ \عددی{(x_1,y_1),\cdots,(x_n,y_n)} پر غور کریں۔نمونی قیمت \عددی{(x_j,y_j)} کی سیدھی لکیر \عددی{y=a+bx} سے انتصابی رخ فاصلہ (\عددی{y} محور کے متوازی ناپا گیا فاصلہ) \عددی{\abs{y_j-a-bx_j}} ہو گا (شکل \حوالہ{شکل_شماریات_خط_بٹھانا_انتصابی_فاصلہ})۔یوں ان فاصلوں کے مربع کا مجموعہ
\begin{align}\label{مساوات_شماریات_سیدھا_خط_بٹھانا_الف}
q=\sum_{j=1}^{n}(y_j-a-bx_j)^2
\end{align}
ہو گا۔کمتر مربعوں کی ترکیب میں ہم \عددی{a} اور \عددی{b} یوں منتخب کرتے ہیں کہ \عددی{q} کی قیمت کم سے کم حاصل ہو۔\عددی{q} کی قیمت \عددی{a} اور \عددی{b} پر منحصر ہے اور اس کی کم سے کم قیمت درج ذیل لازمی شرائط سے حاصل ہو گی۔
\begin{align}\label{مساوات_شماریات_سیدھا_خط_بٹھانا_ب}
\frac{\partial q}{\partial a}=0 \quad \text{اور}\quad \frac{\partial q}{\partial b}=0
\end{align}
ہم دیکھیں گے کہ ان شرائط سے درج ذیل کلیہ حاصل ہوتا ہے
\begin{align}\label{مساوات_شماریات_سیدھا_خط_بٹھانا_پ}
y-\overline{y}=b(x-\overline{x})
\end{align}
جہاں
\begin{align}\label{مساوات_شماریات_سیدھا_خط_بٹھانا_ت}
\overline{x}=\frac{1}{n}(x_1+\cdots+x_n)\quad \text{اور}\quad \overline{y}=\frac{1}{n}(y_1+\cdots+y_n)
\end{align}
ہیں۔مساوات \حوالہ{مساوات_شماریات_سیدھا_خط_بٹھانا_ب} کو نمونے کی \عددی{y} قیمتوں کا نمونے کی \عددی{x} قیمتوں پر  \اصطلاح{رجعی خط}\فرہنگ{رجعی خط}\حاشیہب{regression line}\فرہنگ{regression line} کہتے ہیں۔اس کی ڈھلوان \عددی{b} کو \عددی{x} پر \عددی{y} کا \اصطلاح{تجزی عددی سر}\فرہنگ{تجزی!عددی سر}\حاشیہب{regression coefficient}\فرہنگ{regression!coefficient} کہتے ہیں۔ہم دیکھیں گے کہ 
\begin{align}\label{مساوات_شماریات_سیدھا_خط_بٹھانا_ٹ}
b=\frac{s_{xy}}{s_1^2}
\end{align}
ہو گا جہاں 
\begin{align}\label{مساوات_شماریات_سیدھا_خط_بٹھانا_ث}
s_1^2=\frac{1}{n-1}\sum_{j=1}^{n}(x_j-\overline{x})^2=\frac{1}{n-1}\big[\sum_{j=1}^{n}x_j^2-\frac{1}{n}\big(\sum_{j=1}^{n}x_j\big)^2\big]
\end{align}
اور
\begin{align}\label{مساوات_شماریات_سیدھا_خط_بٹھانا_ج}
s_{xy}=\frac{1}{n-1}\sum_{j=1}^{n}(x_j-\overline{x})(y_j-\overline{y})=\frac{1}{n-1}\big[\sum_{j=1}^{n}x_jy_j-\frac{1}{n}\big(\sum_{j=1}^{n}x_j\big)\big(\sum_{j=1}^{n}y_j\big)\big]
\end{align}
ہوں گے۔\عددی{s_{xy}} کو نمونے کی \اصطلاح{باہمی تغیریت}\فرہنگ{تغیریت!باہمی}\حاشیہب{covariance}\فرہنگ{covariance} کہتے ہیں۔ظاہر ہے کہ مساوات \حوالہ{مساوات_شماریات_سیدھا_خط_بٹھانا_پ} میں دیا گیا رجعی خط نقطہ \عددی{(\overline{x},\overline{y})} سے گزرے گا۔ 

مساوات \حوالہ{مساوات_شماریات_سیدھا_خط_بٹھانا_پ} کو حاصل کرنے کی خاطر ہم مساوات \حوالہ{مساوات_شماریات_سیدھا_خط_بٹھانا_الف} اور مساوات \حوالہ{مساوات_شماریات_سیدھا_خط_بٹھانا_ب} استعمال کرتے ہوئے 
\begin{align*}
\frac{\partial q}{\partial a}&=-2\sum(y_j-a-bx_j)=0\\
\frac{\partial q}{\partial b}&=-2\sum x_j(y_j-a-bx_j)=0
\end{align*}
لکھتے ہوئے (جہاں \عددی{j} پر \عددی{1} تا \عددی{n} مجموعے لیے جاتے ہیں)۔یوں
\begin{align*}
na+b\sum x_j&=\sum y_j\\
a\sum x_j+b\sum x_j^2&=\sum x_jy_j
\end{align*}
حاصل ہو گا۔مفروضہ-الف کے تحت خطی مساوات کے نظام (مساوات \حوالہ{مساوات_شماریات_سیدھا_خط_بٹھانا_ث})
\begin{align*}
n\sum x_j^2-\big(\sum x_j\big)^2=n(n-1)s_1^2
\end{align*}
کا مقطع غیر صفر ہو گا اور اس نظام کا یکتا حل (مساوات \حوالہ{مساوات_شماریات_سیدھا_خط_بٹھانا_ت}، مساوات \حوالہ{مساوات_شماریات_سیدھا_خط_بٹھانا_ث}، مساوات \حوالہ{مساوات_شماریات_سیدھا_خط_بٹھانا_ج})
\begin{align}
a=\overline{y}-b\overline{x},\quad b=\frac{n\sum x_jy_j-\sum x_j\sum y_j}{n(n-1)s_1^2}
\end{align}
 پایا جائے گا۔اس سے مساوات \حوالہ{مساوات_شماریات_سیدھا_خط_بٹھانا_پ} حاصل ہوتا ہے جس میں \عددی{b} کی قیمت مساوات \حوالہ{مساوات_شماریات_سیدھا_خط_بٹھانا_ٹ} تا مساوات \حوالہ{مساوات_شماریات_سیدھا_خط_بٹھانا_ج} دیتے ہیں۔(\عددی{s_1^2} کے دو تعلقات کا ایک جیسا ہونے کو آپ ثابت کر سکتے ہیں (سوال \حوالہ{سوال_شماریات_رجعی_عددی_سر_پ})؛اسی طرح \عددی{s_{xy}} کے لئے بھی آپ کر سکتے ہیں)

ہاتھ سے نتائج حاصل کرنے کو آسان بنانے کی خاطر ہم 
\begin{align}
x_j=c_1x_j^*+l_1,\quad y_j=c_2y_j^*+l_2
\end{align}
استعمال کرتے ہیں جن میں \عددی{c_1}، \عددی{c_2}، \عددی{l_1}، \عددی{l_2} یوں منتخب کیے جاتے ہیں کہ متبادل قیمتیں \عددی{x_j^*}، \عددی{y_j^*} سادہ ترین ہوں۔ہم متبادل قیمتیں استعمال کرتے ہوئے  \عددی{\overline{x}^*}، \عددی{\overline{y}^*}، \عددی{x_1^{*2}}، \عددی{s_{xy}^*} بذریعہ حساب تلاش کرنے کے بعد درج ذیل تلاش کرتے ہیں۔
\begin{gather}
\begin{aligned}
\overline{x}&=c_1\overline{x}^*+l_1,\\
s_1^2&=c_1^2s_1^{*2},
\end{aligned}
\quad 
\begin{aligned}
\overline{y}&=c_2\overline{y}^*+l_2\\
s_{xy}&=c_1c_2s_{xy}^*
\end{aligned}
\end{gather}

%====================
\ابتدا{مثال}\شناخت{مثال_شماریات_رجعی_خط}\quad \موٹا{رجعی خط}\\
ایک مخصوص چمڑے کی حجم میں فی صد کمی \عددی{y} بالمقابل مقررہ دباو \عددی{x} ناپے  گیے۔کرہ ہوائی کے دباو کو دباو کی اکائی لی گئی ہے۔نتائج جدول \حوالہ{جدول_شماریات_چمڑا_رجعت} میں پیش کیے گئے ہیں۔\عددی{y} کا \عددی{x} پر رجعی خط تلاش کریں۔
\begin{table}
\caption{چمڑے کی حجم میں کمی $\,y [\si{\percent}]\,$ کا دباو $\,x\,$ پر رجعت}
\label{جدول_شماریات_چمڑا_رجعت}
\centering
\begin{otherlanguage}{english}
\begin{tabular}{CC||CC}
\multicolumn{2}{C||}{\text{\RL{\urdufont{دی گئی قیمتیں}}}} &\multicolumn{2}{C}{\text{\RL{\urdufont{معاون قیمتیں}}}}\\
x_j&y_j&x_j^2&x_jy_j\\
\hline
\num{4000}&2.3&\num{16000000}&\num{9200}\\
\num{6000}&4.1&\num{36000000}&\num{24600}\\
\num{8000}&5.7&\num{64000000}&\num{45600}\\
\num{10000}&6.9&\num{100000000}&\num{69000}\\
\hline
\num{28000}&19.0&\num{216000000}&\num{148400}\\
\hline
\end{tabular}
\end{otherlanguage}
\end{table}

حل:\quad
ہم دیکھتے ہیں کہ \عددی{n=4} ہے اور \عددی{\overline{x}=\tfrac{28000}{4}=7000}، \عددی{\overline{y}=\tfrac{19.0}{4}=4.75}،
\begin{align*}
s_1^2&=\frac{1}{3}\big(\num{216000000}-\frac{\num{28000}^2}{4}\big)=\frac{\num{20000000}}{3}\\
s_{xy}&=\frac{1}{3}\big(\num{148400}-\frac{\num{28000}\cdot 19}{4}\big)=\frac{\num{15400}}{3}
\end{align*}
حاصل کرتے ہیں۔یوں \عددی{b=\tfrac{\num{15400}}{\num{20000000}}=\num{0.000777}} ہو گا  اور رجعی خط درج ذیل ہو گا۔
\begin{align*}
y-4.75=\num{0.00077}(x-7000) \quad \implies \quad y=\num{0.00077}x-0.64
\end{align*}
\انتہا{مثال}
%==============================

ہم درج ذیل دو مفروضے فرض کرتے ہیں۔

%============================
\ابتدا{مفروضہ} (ب)\\
ہر مقررہ \عددی{x} کے لئے بلا منصوبہ متغیر \عددی{Y} عمومی ہے جس کی اوسط
\begin{align}\label{مساوات_شماریات_سیدھا_خط_بٹھانا_د}
\mu(x)=\alpha+\beta x
\end{align}
اور تغیریت \عددی{\sigma^2} ہے جہاں تغیریت \عددی{x} کا تابع نہیں ہے۔
\انتہا{مفروضہ}
%===================================
\ابتدا{مفروضہ} (پ)\\
نمونہ \عددی{(x_1,y_1),\cdots,(x_n,y_n)} لینے کے لئے \عددی{n} مرتبہ تجربات غیر تابع طریقے سے سرانجام دیے گئے۔
\انتہا{مفروضہ}
%=============================

زیر مفروضہ الف تا پ دکھایا جا سکتا ہے کہ \عددی{\beta} کا زیادہ سے زیادہ امکانی اندازہ  مساوات \حوالہ{مساوات_شماریات_سیدھا_خط_بٹھانا_ٹ}  میں دیا گیا رجعی عددی سر \عددی{b} ہو گا۔اسی لئے \عددی{\beta} کو  \ترچھا{آبادی کا} \اصطلاح{رجعی عددی سر}\فرہنگ{رجعی!سر}\حاشیہب{regression coefficient}\فرہنگ{regression!coefficient} کہتے ہیں۔

زیر مفروضہ الف تا پ، جیسا جدول \حوالہ{جدول_شماریات_رجعی_عددی_سر} میں دکھایا گیا ہے، ہم \عددی{\beta} کا وقفہ اعتماد حاصل کر سکتے ہیں۔
\begin{table}
\caption{زیر مفروضہ الف تا پ مساوات \حوالہ{مساوات_شماریات_سیدھا_خط_بٹھانا_د} میں دیے گئے $\beta$ کا وقفہ اعتماد}
\label{جدول_شماریات_رجعی_عددی_سر}
\centering
\fbox{
\begin{minipage}{0.95\textwidth}
\موٹا{پہلا قدم:}\quad
سطح اعتماد \عددی{\gamma} (\SI{95}{\percent}، \SI{99}{\percent}، وغیرہ) منتخب کریں۔\\
\موٹا{دوسرا قدم:}\quad
\عددی{n-2} درجہ آزادی کے لئے  ضمیمہ \حوالہ{ضمیمہ_جدول} کی جدول \حوالہ{ضمیمہ_ٹی_تقسیم} سے درج ذیل مساوات کا حل \عددی{c} تلاش کریں۔ (نمونی جسامت=\عددی{n})
\begin{align}\label{مساوات_شماریات_سیدھا_خط_بٹھانا_ڈ}
F(c)=\frac{1}{2}(1+\gamma)
\end{align}
\موٹا{تیسرا قدم:}\quad
نمونہ \عددی{(x_1,y_1),\cdots,(x_n,y_n)} استعمال کرتے ہوئے مساوات \حوالہ{مساوات_شماریات_سیدھا_خط_بٹھانا_ث} سے \عددی{(n-1)s_1^2}، مساوات \حوالہ{مساوات_شماریات_سیدھا_خط_بٹھانا_ج} سے \عددی{(n-1)s_{xy}}، مساوات \حوالہ{مساوات_شماریات_سیدھا_خط_بٹھانا_ٹ} سے \عددی{b}،
\begin{align}\label{مساوات_شماریات_سیدھا_خط_بٹھانا_ذ}
(n-1)s_2^2=\sum_{j=1}^{n}y_j^2-\frac{1}{n}\big(\sum_{j=1}^{n}y_j\big)^2
\end{align}
اور
\begin{align}\label{مساوات_شماریات_سیدھا_خط_بٹھانا_ر}
q_0=(n-1)(s_2^2-b^2s_1^2)
\end{align}
حاصل کریں۔\\
\موٹا{چوتھا قدم:}\quad
\عددی{k=c\sqrt{\tfrac{q_0}{(n-2)(n-1)s_1^2}}} کو بذریعہ حساب حاصل کریں۔وقفہ اعتماد درج ذیل ہو گا۔
\begin{align}\label{مساوات_شماریات_سیدھا_خط_بٹھانا_ڑ}
\text{اعتماد}\{b-k\le \beta\le b+k \}
\end{align}
\end{minipage}
}
\end{table}

%=====================
\ابتدا{مثال}\quad \موٹا{رجعی عددی سر کا وقفہ اعتماد}\\
جدول \حوالہ{جدول_شماریات_چمڑا_رجعت} میں دی گئی نمونی قیمتیں استعمال کرتے ہوئے جدول \حوالہ{جدول_شماریات_رجعی_عددی_سر} میں دی گئی ترکیب سے \عددی{\beta} کا وقفہ اعتماد تلاش کریں۔\\
حل:\quad \موٹا{پہلا قدم:}\quad ہم \عددی{\gamma=0.95} منتخب کرتے ہیں۔\\
\موٹا{دوسرا قدم:}\quad مساوات \حوالہ{مساوات_شماریات_سیدھا_خط_بٹھانا_ڈ} کو \عددی{F(c)=0.975} لکھ سکتے ہیں۔ ضمیمہ \حوالہ{ضمیمہ_جدول} کی جدول \حوالہ{ضمیمہ_ٹی_تقسیم} سے \عددی{n-2=2} درجہ آزادی کے لئے \عددی{c=4.30} حاصل ہوتا ہے۔\\
\موٹا{تیسرا قدم:}\quad مثال \حوالہ{مثال_شماریات_رجعی_خط} ہمیں \عددی{3s_1^2=\num{20000000}} اور \عددی{b=\num{0.00077}}  دیتی ہے۔ جدول \حوالہ{جدول_شماریات_چمڑا_رجعت} سے ہم درج ذیل بذریعہ حساب حاصل کرتے ہیں۔
\begin{align*}
3s_2^2=102.2-\frac{19^2}{4}=11.95, \quad q_0=11.95-\num{20000000}\cdot \num{0.00077}^2=0.092
\end{align*}
\موٹا{چوتھا قدم:}\quad یوں 
$k=4.30\sqrt{\frac{0.092}{2\cdot \num{20000000}}}=\num{0.000206}$
حاصل ہو گا لہٰذا وقفہ اعتماد درج ذیل ہو گا۔
\begin{align*}
\text{} \{\num{0.00056}\le \beta\le \num{0.00098}\}
\end{align*}
\انتہا{مثال}
%===========================

\حصہء{سوالات}
%==================
\ابتدا{سوال}\quad
آنکھ سے سیدھا خط تلاش کریں۔ایک گاڑی \عددی{\SI{35}{\kilo\meter\per\hour}} کی رفتار سے چل رہی ہے جبکہ گاڑی کی (کلو میٹر فی گھنٹہ) رفتار \عددی{x} بالمقابل (میٹروں میں) رکنے کے لئے درکار فاصلہ \عددی{y} درج ذیل ہے۔
\begin{align*}
\begin{array}{c|cccc}
x & 20&30&40&50\\
\hline
y&50&95&150&210
\end{array}
\end{align*}
جواب:\quad
تقریباً \عددی{\SI{120}{\meter}}
\انتہا{سوال}
%====================
\ابتدا{سوال}\quad
\عددی{x_j=2000x_j^*+4000} اور \عددی{y_j=0.1y_j^*+5} لیتے ہوئے مثال \حوالہ{مثال_شماریات_رجعی_خط} کے نتائج حاصل کریں۔
\انتہا{سوال}
%=======================
\ابتدا{سوال}\quad
ایسا نمونہ حاصل کریں جس کے لئے \عددی{b=0} ہو۔
\انتہا{سوال}
%========================

سوال \حوالہ{سوال_شماریات_رجعی_خط_ترسیم_الف} تا سوال \حوالہ{سوال_شماریات_رجعی_خط_ترسیم_ب}  میں \عددی{x} پر \عددی{y} کی نمونی رجعی خط ترسیم کریں۔

%===============
\ابتدا{سوال}\شناخت{سوال_شماریات_رجعی_خط_ترسیم_الف}\quad
سوال \حوالہ{سوال_شماریات_غیر_مقدار_معلوم_ڑ} کا نمونہ استعمال کریں۔
\انتہا{سوال}
%======================
\ابتدا{سوال}\quad
$(1,1),(2,1.7),(3,3)$\\
جواب:\quad
$y=x-0.1$
\انتہا{سوال}
%================
\ابتدا{سوال}\شناخت{سوال_شماریات_رجعی_درکار_الف}\quad
ڈیزل انجن کی درج ذیل زاویائی رفتار \عددی{x} (فی منٹ چکر) بالمقابل طاقت \عددی{y} (کلو واٹ)
\begin{align*}
\begin{array}{c|ccccc}
x&400&500&600&700&750\\
\hline
y&580&1030&1420&1880&2100
\end{array}
\end{align*}
\انتہا{سوال}
%====================
\ابتدا{سوال}\شناخت{سوال_شماریات_رجعی_درکار_ب}\quad
ایک مخصوص فولاد کی بد شکلی \عددی{x\,[\si{\milli\meter}]} اور \اصطلاح{برینل سختی}\فرہنگ{سختی!برینل}\حاشیہب{Brinell hardness}\فرہنگ{hardness!Brinell} \عددی{y\,[\si{\kilo\gram\per\milli\meter\squared}]} 
\begin{align*}
\begin{array}{c|ccccccccc}
x&6&9&11&13&22&26&28&33&35\\
\hline
y&68&67&65&53&44&40&37&34&32
\end{array}
\end{align*}
جواب:\quad
$y-48.89=-1.32(x-20.33)$
\انتہا{سوال}
%========================
\ابتدا{سوال}\شناخت{سوال_شماریات_رجعی_خط_ترسیم_ب}\quad
کلورانیفتھالین کا گاڑھاپن \عددی{x\,[\si{\percent}]} اور دیکم کی اموات \عددی{y\,[\si{\percent}]}
\begin{align*}
\begin{array}{c|ccccc}
x&0.04&0.15&0.30&1.00&2.00\\
\hline
y&3&16&13&70&90
\end{array}
\end{align*}
\انتہا{سوال}
%=====================

 زیر  مفروضہ ب اور پ، سوال \حوالہ{سوال_شماریات_رجعی_عددی_سر_الف} تا سوال \حوالہ{سوال_شماریات_رجعی_عددی_سر_ب} میں  دیا گیا نمونہ استعمال کرتے ہوئے، رجعی عددی سر \عددی{\beta} کا \عددی{\SI{95}{\percent}} وقفہ اعتماد تلاش کریں۔

%===================
\ابتدا{سوال}\شناخت{سوال_شماریات_رجعی_عددی_سر_الف}\quad
$(1,1),(2,2+a),(3,3)$
جہاں \عددی{a} مستقل ہے۔\\
جواب:\quad
$2s_1^2=2, 2s_{xy}=2, b=1, 2s_2^2=2+\tfrac{2}{3}p^2,q_0=\tfrac{2}{3}p^2,$\\
$k=\tfrac{12.7a}{\sqrt{3}}=7.3a\, (\gamma=\SI{95}{\percent}), \text{اعتماد}\{1-7.3a\le \beta\le 1+7.3a\}$
\انتہا{سوال}
%======================
\ابتدا{سوال}\quad
سوال \حوالہ{سوال_شماریات_رجعی_درکار_الف} کا نمونہ۔
\انتہا{سوال}
%====================
\ابتدا{سوال}\quad
سوال \حوالہ{سوال_شماریات_رجعی_درکار_ب} کا نمونہ۔\\
جواب:\quad
$q_0=76,k=2.37\sqrt{\tfrac{76}{7\cdot 944}}=0.254, \text{اعتماد}=\{-1.58\le \beta\le -1.06\}$
\انتہا{سوال}
%====================
\ابتدا{سوال}\quad
ہوا میں نمی کا تناسب \عددی{x\,[\si{\percent}]} بالمقابل جیلی نما مادہ  کا پھیل \عددی{y\,[\si{\percent}]}
\begin{align*}
\begin{array}{c|cccc}
x&10&20&30&40\\
\hline
y&0.8&1.6&2.3&2.8
\end{array}
\end{align*}
\انتہا{سوال}
%====================
\ابتدا{سوال}\شناخت{سوال_شماریات_رجعی_عددی_سر_پ}\quad
مساوات \حوالہ{مساوات_شماریات_سیدھا_خط_بٹھانا_ث} میں ایک ہاتھ سے دوسرا ہاتھ حاصل کریں۔ اشارہ۔ مربع لے کر \عددی{\overline{x}} کی تعریف پر کرتے ہوئے سادہ صورت حاصل کریں۔
\انتہا{سوال}
%======================
\ابتدا{سوال}\شناخت{سوال_شماریات_رجعی_عددی_سر_ب}\quad
مساوات \حوالہ{مساوات_شماریات_سیدھا_خط_بٹھانا_ج} میں  دائیں ہاتھ کو بائیں ہاتھ سے حاصل کریں۔
\انتہا{سوال}
%=====================
