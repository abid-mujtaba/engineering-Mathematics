
\حصہ{غیر مقدار معلوم پرکھ}\شناخت{حصہ_شماریات_پرکھ_برائے_غیر_مقدار_معلوم}
حصہ \حوالہ{حصہ_شماریات_قیاس_کی_پرکھ_فیصلے} کے پرکھ عمومی آبادی کے لئے تھے۔کئی بار آبادی کی تقسیم غیر عمومی یا نا معلوم تقسیم رکھتی ہے۔ایسی صورت  میں ہم \اصطلاح{غیر مقدار معلوم پرکھ}\فرہنگ{پرکھ!غیر مقدار معلوم}\حاشیہب{nonparametric test}\فرہنگ{test!nonparametric} یا \اصطلاح{تقسیم پاک پرکھ}\فرہنگ{پرکھ!تقسیم پاک}\حاشیہب{distribution-free test}\فرہنگ{test!distribution-free} استعمال کر سکتے ہیں جس کی بنیاد  \ترچھا{شماریات رجحان}\فرہنگ{شماریات!رجحان}\حاشیہب{order statistics}\فرہنگ{statistics!order} ہے لہٰذا اس کو کسی بھی \ترچھا{استمراری تقسیم} کے لئے استعمال کیا جا سکتا ہے۔ البتہ عمومی تقسیم کے لئے حصہ \حوالہ{حصہ_شماریات_قیاس_کی_پرکھ_فیصلے}
 کے پرکھ بہتر نتائج دیتے ہیں۔تقسیم پاک پرکھ کو سمجھنے کی خاطر ایک مثال پر غور کرتے ہیں۔

%====================
\ابتدا{مثال}\quad \موٹا{پرکھ برائے علامت وسطانیہ}\\
مساوات \عددی{F(x)=0.5} کے حل \عددی{x=\tilde{\mu}} کو وسطانیہ کہتے ہیں، جہاں \عددی{F} تفاعل تقسیم ہے۔ مثال \حوالہ{مثال_دو_عمومی_تقسیمات_موازنہ_تغیریت} کا نمونی فرق، یعنی،
\begin{align*}
\begin{array}{cccccccc}
16&16&2&6&0&0&13&8
\end{array}
\end{align*}
استعمال کرتے ہوئے ہم قیاس \عددی{\tilde{\mu}=0} کو پرکھتے ہیں جو کہتا ہے کہ   کام کرنے کے دو مختلف حالات میں مزدور کی کارکردگی تقریباً ایک جیسی ہے۔\\
حل:\quad
ہم متبادل \عددی{\tilde{\mu}>0} اور معنی خیز سطح \عددی{\alpha=\SI{5}{\percent}} منتخب کرتے ہوئے۔اگر قیاس درست ہو تب مثبت فرق کا احتمال \عددی{p} اور منفی فرق کا احتمال ایک جیسے ہوں گے۔ یوں \عددی{p=0.5} ہو گا اور بلا منصوبہ متغیر
\begin{align*}
X=\text{\RL{$n$ قیمتوں میں مثبت قیمتوں کا مجموعہ}}
\end{align*}
کا تقسیم ثنائی ہو گا جس کا \عددی{p=0.5} ہو گا۔ہمارے نمونے میں \عددی{8} قیمتیں ہیں۔ہم \عددی{0} قیمتوں کو خارج کرتے ہیں چونکہ ان کا فیصلہ پر کوئی اثر نہیں پایا جاتا ہے۔تب \عددی{6} قیمتیں رہ جاتی ہیں۔یہ تمام قیمتیں مثبت ہیں۔۔چونکہ
\begin{align*}
P(X=6)=\binom{6}{6} (0.5)^6(0.5)^0=0.0156=\SI{1.56}{\percent}<\alpha
\end{align*}
ہے لہٰذا ہم قیاس نا منظور کرتے ہیں۔

اگر ان \عددی{6} قیمتوں میں صرف \عددی{1} قیمت منفی ہوتی تب
\begin{align*}
P(X\ge 5)=\binom{6}{5}(0.5)^5\cdot 0.5+\binom{6}{6}(0.5)^6=\SI{10.9}{\percent}
\end{align*}
ہوتا اور ہم قیاس کو نا منظور نہ کرتے۔
\انتہا{مثال}
%===========================
\ابتدا{مثال}\شناخت{مثال_شماریات_بلا_منصوبہ_رجحان}\quad \موٹا{بلا منصوبہ رجحان کے لئے پرکھ}\\
تار کو کاٹنے کے لئے ایک مشین استعمال کی جاتی ہے۔لگاتار کٹی لمبائیاں درج ذیل ہیں۔
\begin{align*}
\begin{array}{ccccc}
29&31&28&30&32
\end{array}
\end{align*} 
اس نمونہ کو استعمال کرتے ہوئے اس قیاس کو پرکھیں کہ مشین تار کو \ترچھا{بغیر کسی رجحان} کاٹتی ہے، یعنی مشین مسلسل بڑھتی یا مسلسل گھٹتی لمبائی کی تار نہیں کاٹتی ہے۔فرض کریں کہ مشین کی قسم سے ایسا ظاہر ہوتا ہے کہ یہ مسلسل بڑھتی لمبائی کی تار کاٹے گی (مثبت رجحان)۔\\
حل:\quad
جتنی بار کوئی بڑی قیمت کسی چھوٹی قیمت سے پہلے رونما ہو، ہم ان \ترچھا{تبدیلیوں} کی تعداد گنتے ہیں۔\\
\centerline{
\عددی{29} قیمت \عددی{28} قیمت سے پہلے آتی ہے:\quad  ($1$ تبدیلی)
}
\centerline{
\عددی{31} کی قیمت \عددی{28} اور \عددی{30} سے پہلے آتی ہے:\quad  ($2$ تبدیلیاں)
}
باقی تین قیمتیں بڑھتی رجحان رکھتی ہیں۔یوں نمونہ میں \عددی{1+2=3} تبدیلیاں پائی جاتی ہیں۔ہم اب بلا منصوبہ متغیر\\
\centerline{
\عددی{=T}تعداد تبدیلیاں
}
پر غور کرتے ہیں۔اگر قیاس درست ہو (غیر رجحانی)، تب پانچ اجزاء \عددی{1\,\,2\,\,3\,\,4\,\,5} کے \عددی{5!=120} ترتیبی اجتماعات میں ہر ایک کا احتمال
 \عددی{\tfrac{1}{120}} ہو گا۔ ہم ان ترتیبی اجتماعات کو ان کی تبدیلیوں کے لحاظ سے لکھتے ہیں: 
\begin{align*}
\text{وغیرہ}\quad
\begin{array}{ccccc}
\multicolumn{5}{c}{T=3}\\
1&2&5&4&3\\
1&3&4&5&2\\
1&3&5&2&4\\
1&4&2&5&3\\
1&4&3&2&5\\
1&5&2&3&4\\
2&1&4&5&3\\
2&1&5&3&4\\
2&3&1&5&4\\
2&3&4&1&5\\
2&4&1&3&5\\
3&1&2&5&4\\
3&1&4&2&5\\
3&2&1&4&5\\
4&1&2&3&5
\end{array}\quad
\begin{array}{ccccc}
\multicolumn{5}{c}{T=2}\\
1&2&4&5&3\\
1&2&5&3&4\\
1&3&2&5&4\\
1&3&4&2&5\\
1&4&2&3&5\\
2&1&3&5&4\\
2&1&4&3&5\\
2&3&1&4&5\\
3&1&2&4&5
\end{array}\quad
\begin{array}{ccccc}
\multicolumn{5}{c}{T=1}\\
1&2&3&5&4\\
1&2&4&3&5\\
1&3&2&4&5\\
2&1&3&4&5
\end{array}\quad
\begin{array}{ccccc}
\multicolumn{5}{c}{T=0}\\
1&2&3&4&5
\end{array}
\end{align*}
ان سے ہم درج ذیل حاصل کرتے ہیں
\begin{align*}
P(T\le 3)=\frac{1}{120}+\frac{4}{120}+\frac{9}{120}+\frac{15}{120}=\frac{29}{120}=\SI{24}{\percent}
\end{align*}
لہٰذا ہم قیاس کو نا منظور نہیں کرتے ہیں۔

ضمیمہ \حوالہ{ضمیمہ_جدول} کی جدول \حوالہ{ضمیمہ_بلا_رجحان} میں بلا رجحان صورت میں بلا منصوبہ متغیر \عددی{T} کی تقسیم دی گئی ہے۔  ہمارے تراکیب اور اس  جدول کی قیمتیں استمراری تقسیمات کے کئے ہیں۔یوں ہم توقع کرتے ہیں کہ نمونہ کی تمام قیمتیں ایک دوسرے سے مختلف ہوں گی۔پور و پور کی بنا عملاً چند نمونی قیمتیں ایک جیسی ہو سکتی ہیں۔اگر \عددی{m} قیمتیں ایک جیسی ہوں تب \عددی{\tfrac{m(m-1)}{4}} (\عددی{m} اجزاء کی ترتیبی اجتماعات میں تبدیلیوں کے تعداد کی اوسط) جمع کریں، یعنی، ایک جیسی قیمتوں کے ہر جوڑی کے لئے \عددی{\tfrac{1}{2}}، ایک جیسی تین قیمتوں کے لئے \عددی{\tfrac{3}{2}}، وغیرہ۔
\انتہا{مثال}
%==================
\حصہء{سوالات}
%=============
\ابتدا{سوال}\quad
\عددی{10} کوششوں میں سے \عددی{7} کوششوں میں قسم الف ہوئی چھلنی نے قسم ب ہوائی چھلنی سے زیادہ صاف ہوا پیدا کی، \عددی{1} کوشش میں چھلنی ب نے زیادہ صاف ہوا پیدا کی جبکہ \عددی{2} کوششوں میں دونوں کے نتائج ایک جیسے تھے۔کیا چھلنی الف زیادہ بہتر ہے؟\\
جواب:\quad
قیاس: الف اور ب ایک جیسی معیار رکھتی ہیں۔تب \عددی{8} کوششوں میں \عددی{7} یا \عددی{8} بار الف کے حق میں وقوعہ کا احتمال \عددی{\SI{3.5}{\percent}} ہے۔قیاس کو نا منظور کریں۔ 
\انتہا{سوال}
%=======================
\ابتدا{سوال}\quad
کن صورتوں میں ہم پرکھ علامت کو استمراری تقسیم کی اوسط پرکھنے کے لئے استعمال کر سکتے ہیں۔
\انتہا{سوال}
%======================
\ابتدا{سوال}\quad
پرکھ علامت کو سوال \حوالہ{سوال_شماریات_ٹلسٹار}  کے نمونہ پر لاگو کریں۔\\
جواب:\quad
$P(X\le 2)=0.5^6(1+6+15)=\SI{34}{\percent}$
قیاس \عددی{\tilde{\mu}=0} کو نا منظور نہ کریں۔
\انتہا{سوال}
%======================
\ابتدا{سوال}\quad
اگر \عددی{\tilde{\mu}=0} کی بجائے قیاس \عددی{\tilde{\mu}=\tilde{\mu}_0} ہو تب آپ پرکھ علامت  کو کس طرح استعمال کریں گے۔ (\عددی{\mu_0} کوئی بھی عدد ہو سکتا ہے۔)
\انتہا{سوال}
%==========================
\ابتدا{سوال}\quad
\عددی{16} جسامت کے نمونہ میں \عددی{10} مثبت، \عددی{4} منفی اور \عددی{2} قیمتیں صفر ہیں۔(ضمیمہ \حوالہ{ضمیمہ_جدول} کی جدول \حوالہ{ضمیمہ_ثنائی_تقسیم} میں درکار قیمتیں نہیں دی گئی ہیں۔آپ کو یہ قیمتیں حاصل کرنی ہوں گی۔)\\
جواب:\quad
اگر \عددی{\tilde{\mu}=0} ہو، \عددی{14} میں سے \عددی{4} یا \عددی{4} سے کم عدد قیمتیں منفی ہونے کا احتمال \عددی{\SI{9}{\percent}} ہے۔قیاس \عددی{\tilde{\mu}=0} کو نا منظور نہ کریں۔
\انتہا{سوال}
%===========================
\ابتدا{سوال}\شناخت{سوال_شماریات_سلاخ_لمبائی}\quad
\عددی{\tilde{\mu}=5} میٹر لمبائی سلاخ پیدا کرنے  کے عمل کے ایک نمونہ میں \عددی{4} سلاخوں کی لمبائی ٹھیک ہے، \عددی{15} کی لمبائی کم اور \عددی{3} کی لمبائی زیادہ ہے۔ کیا اس عمل کو درست کرنے کی ضرورت ہے؟ (عمومی تقسیم کو ثنائی تقسیم کا تخمین لیں۔ حصہ \حوالہ{حصہ_شماریات_عمومی_تقسیم})
\انتہا{سوال}
%=======================
\ابتدا{سوال}\quad
مسئلہ \حوالہ{مسئلہ_شماریات_ڈی_موے_لاپلاس_تحدیدی} استعمال کیے بغیر سوال  \حوالہ{سوال_شماریات_سلاخ_لمبائی} کو حل کریں۔\\
جواب:\quad
\عددی{3} یا اس سے کم  سلاخوں کی لمبائی \عددی{5} میٹر سے زیادہ ہونے کا ٹھیک  احتمال \عددی{\SI{0.38}{\percent}} ہے۔یہ سوال \حوالہ{سوال_شماریات_سلاخ_لمبائی} میں حاصل تخمینی احتمال سے کچھ کم ہے۔
\انتہا{سوال}
%=====================
\ابتدا{سوال}\شناخت{سوال_شماریات_عمومی_پرکھ_استعمال}\quad
\عددی{10} مریضوں میں سے ہر ایک کو دو مختلف نیند کی دوائیاں دی گئی۔درج ذیل جدول ان کے اثرات (سونے کے دورانیے میں گھنٹوں میں اضافہ) پیش کرتا ہے۔پرکھ علامت کی مدد سے دیکھیں کہ آیا ان میں فرق معنی خیز ہے۔\\
\begin{align*}
\begin{array}{c|rrrrrrrrrr}
A&1.9&0.8&1.1&0.1&-0.1&4.4&5.5&1.6&4.6&3.4\\
B&0.7&-1.6&-0.2&-1.2&-0.1&3.4&3.7&0.8&0.0&2.0\\
\hline
\text{فرق} &1.2&2.4&1.3&1.3&0.0&1.0&1.8&0.8&4.6&1.4
\end{array}
\end{align*}
\انتہا{سوال}
%=========================
\ابتدا{سوال}\quad
مثال \حوالہ{مثال_شماریات_نا_معلوم_تغیریت_عمومی_تقسیم_اوسط_کا_پرکھ} میں سمجھائے گیے پرکھ کو  سوال \حوالہ{سوال_شماریات_عمومی_پرکھ_استعمال} پر لاگو کریں ۔(سوال میں دیے گیے  نمونہ کی آبادی کو عمومی تصور کریں۔)  \\
جواب:\quad
قیاس \عددی{\mu=0}؛ متبادل \عددی{\mu>0}، \عددی{\overline{x}=1.58}،\\
 $t=\sqrt{10}\cdot \tfrac{1.58}{1.23}=4.06>c=1.83\,(\alpha=\SI{5}{\percent})$
؛قیاس نا منظور۔
\انتہا{سوال}
%=========================
\ابتدا{سوال}\quad
نچلی چوتھائی \عددی{q_{25}} (جس کی تعریف \عددی{F(q_{25})=0.25} ہے) کے لئے پرکھ علامت بنائیں۔ 
\انتہا{سوال}
%=====================
\ابتدا{سوال}\quad
\عددی{8} قیمتوں کا نمونہ جس میں \عددی{7} کی قیمت \عددی{\SI{20}{\celsius}} سے کم اور \عددی{1} کی قیمت \عددی{\SI{20}{\celsius}} سے زیادہ ہو استعمال کرتے ہوئے  خود کار حراری سوئچ ٹھیک \عددی{\SI{20}{\celsius}} پر مقرر ہونے کے قیاس کو  بالمقابل کہ سوئچ کم درجہ حرارت پر مقرر ہے، پرکھیں۔\\
جواب:\quad
$P(X\ge1)=0.5^8(1+8)=\SI{3.5}{\percent}<\alpha=\SI{5}{\percent}$
؛ اس قیاس کو نا منظور کریں کہ سوئچ ٹھیک درجہ حرارت پر مقرر ہے۔
\انتہا{سوال}
%======================
\ابتدا{سوال}\quad
وولٹ پیما کی پیمائش درجہ حرارت \عددی{T[\si{\celsius}]} سے آزاد ہے کے قیاس کو بالمقابل کہ اس کی پیمائش بڑھتے \عددی{T} کے ساتھ بڑھتی ہے پرکھیں۔مستقل برقی دباو مہیا کرتے ہوئے حاصل درج ذیل پیمائشوں کا نمونہ استعمال کریں۔
\begin{align*}
\begin{array}{c|ccccc}
\text{\RL{درجہ حرارت}}\, T[\si{\celsius}]&10&20&30&40&50\\
\hline
\text{پیمائش}\,V[\si{\volt}]&99.8&101.0&100.4&100.8&101.5
\end{array}
\end{align*}
\انتہا{سوال}
%======================
\ابتدا{سوال}\quad
\عددی{n=4} لیتے ہوئے مثال \حوالہ{مثال_شماریات_بلا_منصوبہ_رجحان} میں دی گئی جدول کی طرح جدول بنائیں۔
\انتہا{سوال}
%============================
\ابتدا{سوال}\quad
کیا کھاد سے گندم کی استعمال سے  پیداوار \عددی{X\,[\si{\kilo\gram/\text{رقبہ}}]} بڑھتی ہے؟کھاد کی بڑھتی مقدار کے لحاظ سے مرتب درج ذیل نمونہ استعمال کریں۔
\begin{align*}
\begin{array}{cccccccc}
15.2&16.8&13.2&16.6&17.2&17.5&17.3&18.1
\end{array}
\end{align*}
\انتہا{سوال}
%========================
\ابتدا{سوال}\quad
 مثال \حوالہ{مثال_شماریات_بلا_منصوبہ_رجحان} کے پرکھ کو درج ذیل نمونہ پر لاگو کریں۔(اون میں ڈائی سلفائڈ کی مقدار  \عددی{x} جس کو کیمیائی عمل سے نا گزاری گئی اوون میں مقدار کے فی صد میں ناپا گیا ہے۔اون میں پانی کی فی صد مقدار \عددی{y} ہے۔)
\begin{align*}
\begin{array}{r|rrrrrrrr}
x&10&15&30&40&50&55&80&100\\
\hline
y&50&46&43&42&36&39&37&33
\end{array}
\end{align*}
\انتہا{سوال}
%===========================

\حصہ{پیمائشوں کی جوڑیاں۔ سیدھے خطوط کو موافق بنانا}
ہم اب ایسی تجربات پر غور کرتے ہیں جن میں ہم جوڑی مقدار ناپتے یا ان کا مشاہدہ کرتے ہیں۔ہم تجربات کو درج ذیل دو اقسام میں تقسیم کر سکتے ہیں۔  
\begin{itemize}
\item
\اصطلاح{تجزیہ شماریاتی باہمی رشتہ} میں دونوں متغیرات بلا منصوبہ ہوں گے اور ہم ان کے درمیان رشتہ میں دلچسپی رکھتے ہیں۔(اس کتاب میں شماریات کی اس شاخ پر غور نہیں کی جائے گی۔)
\item
\اصطلاح{مصاعف رجعی تجزیہ} میں دو میں سے ایک متغیر، مثلاً \عددی{x}، کو عام متغیر تصور کیا جاتا ہے، یعنی، اس کی ناپ میں خاطر خواہ خلل نہیں پایا جاتا ہے۔دوسرا متغیر، \عددی{Y}، بلا منصوبہ متغیر ہے۔\عددی{x} کو غیر تابع متغیر کہتے ہیں اور ہم جاننا چاہتے ہیں کہ \عددی{Y}،  متغیر \عددی{x} کا کتنا تابع ہے؟اس کی ایک اچھی مثال فشار خون \عددی{Y} ہے جو انسان کے  عمر \عددی{x} کی تابع ہے۔
\end{itemize}

تجربہ کرنے والا پہلے \عددی{x} کی \عددی{n} قیمتیں \عددی{x_1,\cdots,x_n} منتخب کرتا ہے اور اس کے بعد ان \عددی{x} پر \عددی{Y} کی قیمتیں مشاہدے سے حاصل کرتا ہے۔یوں اس کو درج ذیل صورت کا نمونہ ملتا ہے۔
\begin{align*}
(x_1,y_1), (x_2,y_2),\cdots,(x_n,y_n)
\end{align*} 
مصاعف رجعی تجزیہ میں فرض کیا جاتا ہے کہ \عددی{Y} کی اوسط \عددی{\mu}، متغیر \عددی{x} کے تابع ہے، یعنی، ان کے مابین عام تعلق \عددی{\mu=\mu(x)} پایا جاتا ہے۔\عددی{\mu(x)} کی منحنی کو \ترچھا{\عددی{Y} کی \عددی{x} پر مصاعف رجعی منحنی} کہتے ہیں۔اس حصہ میں ہم سادہ ترین صورت پر غور کرتے ہیں جہاں \عددی{\mu(x)} خطی تفاعل \عددی{\mu(x)=\alpha+\beta x} ہے۔ہم نمونی قیمتوں کو \عددی{xY} مستوی پر ترسیم کر کے، ان پر سیدھی خط بٹھا کر، اس خط کو استعمال کرتے ہوئے کسی بھی \عددی{x} کے لحاظ سے \عددی{\mu(x)} کی اندازاً قیمت حاصل کرنا چاہیں گے تا کہ کسی بھی \عددی{x} سے حاصل \عددی{Y} کی متوقع قیمت ہم جان سکیں۔اگر نقطے بکھرے ہوں تب، خط کو آنکھ کی مدد سے ٹھیک بٹھانا غیر یقینی ہو گا لہٰذا ہمیں حسابی طریقہ درکار ہو گا جو صرف نقطوں پر منحصر یکتا نتیجہ دے۔ایک بہت زیادہ استعمال ہونے والی ترکیب، جس کو گاوس نے بنایا،  \اصطلاح{کمتر مربعوں کی ترکیب}\فرہنگ{کمتر مربع!ترکیب}\حاشیہب{method of least squares}\فرہنگ{least squares!method} کہلاتی ہے۔ہمارے موجودہ ضرورت کو مد نظر رکھتے ہوئے اس کو درج ذیل بیان کیا جا سکتا ہے۔
