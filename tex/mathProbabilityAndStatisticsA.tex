\جزوحصہء{عمومی تقسیم کی صورت میں پرکھ}
درج ذیل مثال عملاً اہم قیاس کے پرکھ کی وضاحت کرتا ہے۔ 

%==============
\ابتدا{مثال}\شناخت{مثال_شماریات_معلوم_تغیریت_عمومی_تقسیم_کی_اوسط}\quad \موٹا{(معلوم تغیریت کی عمومی تقسیم کی اوسط کا پرکھ)}\\
فرض کریں کہ \عددی{X} بلا منصوبہ متغیر ہے جس کی تغیریت \عددی{\sigma^2=9} ہے۔نمونی جسامت \عددی{n=10} لیتے ہوئے قیاس \عددی{\mu=\mu_0=24} کو درج ذیل تین متبادل کے بالمقابل پرکھیں۔
\begin{align*}
\text{(پ)}\, \mu \ne \mu_0 \quad \text{(ب)}\, \mu<\mu_0 \quad \text{(الف)}\,  \mu>\mu_0
\end{align*}
حل:\quad
ہم معنی خیز سطح \عددی{\alpha=0.05} منتخب کرتے ہیں۔اوسط کی اندازاً قیمت درج ذیل سے حاصل ہو گی۔
\begin{align*}
\overline{X}=\frac{1}{n}(X_1+\cdots,X_n)
\end{align*}
اگر قیاس درست ہو تب \عددی{X} عمومی ہو گا جس کی اوسط \عددی{\mu=24} اور تغیریت \عددی{\tfrac{\sigma^2}{n}=0.9} ہو گی (مسئلہ \حوالہ{مسئلہ_شماریات_عمومی_کی_شرط_اوسط_تغیریت})۔لہٰذا ہم فاصل قیمت \عددی{c} کو ضمیمہ \حوالہ{ضمیمہ_جدول} کی جدول \حوالہ{ضمیمہ_عمومی_تقسیم_ب} سے حاصل کر سکتے ہیں۔\\
\موٹا{صورت الف:}\quad
ہم \عددی{P(\overline{X}\le c)_{\mu=24}=\alpha=0.05} سے \عددی{c} تعین کرتے ہیں۔
\begin{align*}
P(\overline{X}\le c)_{\mu=24}=\Phi\big(\frac{c-24}{\sqrt{0.9}}\big)=1-\alpha=0.95
\end{align*}
ضمیمہ \حوالہ{ضمیمہ_جدول} کی جدول \حوالہ{ضمیمہ_عمومی_تقسیم_ب}  سے \عددی{\tfrac{c-24}{\sqrt{0.9}}=1.645} یعنی \عددی{c=25.56} حاصل ہوتا ہے جو \عددی{\mu_0} سے بڑی قیمت ہے (اور جو شکل \حوالہ{شکل_شماریات_خطہ_منظور_اور_نا_منظوری} میں سب سے اوپر دکھائی گئی صورت ہے)۔ اگر \عددی{\overline{x}\le 25.56} ہو تب قیاس کو منظور کیا جائے گا۔ اگر \عددی{\overline{x}>25.56} ہو تب قیاس کو نا منظور کیا جائے گا۔پرکھ کی طاقت درج ذیل ہو گی۔
\begin{gather}
\begin{aligned}
\eta(\mu)=P(\overline{X}>25.56)_{\mu}&=1-P(\overline{X}\le 25.56)_{\mu}\\
&=1-\Phi\big(\frac{25.56-\mu}{\sqrt{0.9}}\big)=1-\Phi(26.94-1.05\mu)
\end{aligned}
\end{gather}
\موٹا{صورت ب:}\quad
فاصل قیمت \عددی{c} کو درج ذیل مساوات سے حاصل کیا جا سکتا ہے۔
\begin{align*}
P(\overline{X}\le c)_{\mu=24}=\Phi\big(\frac{c-24}{\sqrt{0.9}}\big)=\alpha=0.05
\end{align*}
ضمیمہ \حوالہ{ضمیمہ_جدول} کی جدول \حوالہ{ضمیمہ_عمومی_تقسیم_ب}  سے \عددی{c=24-1.56=22.24} ملتا ہے۔اگر \عددی{\overline{x}\ge 22.44} ہو تب ہم قیاس کو منظور کرتے ہیں۔اگر \عددی{\overline{x}<22.44} ہو تب ہم قیاس کو نا منظور کرتے ہیں۔پرکھ کی طاقت درج ذیل ہے۔
\begin{align}
\eta(\mu)=P(\overline{X}\le 22.44)_{\mu}=\Phi\big(\frac{22.44-\mu}{\sqrt{0.9}}\big)=\Phi(23.65-1.05\mu)
\end{align}
\موٹا{صورت پ:}\quad
چونکہ عمومی تقسیم تشاکلی ہے، ہم \عددی{\mu=24} سے \عددی{c_1} اور \عددی{c_2} کو ایک جیسے فاصلے پر چن کر، مثلاً \عددی{c_1=24-k} اور \عددی{c_2=24+k}، مستقل \عددی{k} کو درج ذیل سے تعین کرتے ہیں۔
\begin{align*}
P(24-k\le \overline{X}\le 24+k)_{\mu=24}=\Phi\big(\frac{k}{\sqrt{0.9}}\big)-\Phi\big(-\frac{k}{\sqrt{0.9}}\big)=1-\alpha=0.95
\end{align*}
ضمیمہ \حوالہ{ضمیمہ_جدول} کی جدول \حوالہ{ضمیمہ_عمومی_تقسیم_ب} سے \عددی{\tfrac{k}{\sqrt{0.9}}=1.960} یعنی \عددی{k=1.86} حاصل ہو گا۔یوں \عددی{c_1=24-1.86=22.14} اور \عددی{c_2=24+1.86=25.86} ہوں گے۔اگر \عددی{\overline{x}} کی قیمت \عددی{c_1} سے چھوٹی نہ ہو اور \عددی{c_2} سے بڑی نہ ہو تب ہم قیاس کو منظور کرتے ہیں۔اس کے علاوہ ہم قیاس کو نا منظور کرتے ہیں۔پرکھ کی طاقت درج ذیل ہے۔
\begin{gather}
\begin{aligned}
\eta(\mu)&=P(\overline{X}<22.14)_{\mu}+P(\overline{X}>25.86)_{\mu}\\
&=P(\overline{X}<22.14)_{\mu}+1-P(\overline{X}\le 25.86)_{\mu}\\
&=1+\Phi\big(\frac{22.14-\mu}{\sqrt{0.9}}\big)-\Phi\big(\frac{25.86-\mu}{\sqrt{0.9}}\big)\\
&=1+\Phi(23.34-1.05\mu)-\Phi(27.26-1.05\mu)
\end{aligned}
\end{gather}
نتیجتاً خاصیت کارکردگی \عددی{\beta(\mu)=1-\eta(\mu)} درج ذیل ہو گی۔
\begin{align*}
\beta(\mu)&=\Phi\big(\frac{24.59-\mu}{\sqrt{0.9}}\big)-\Phi\big(\frac{23.41-\mu}{\sqrt{0.9}}\big)\\
&=\Phi(81.97-3.33\mu)-\Phi(78.03-3.33\mu)
\end{align*}
شکل سے ظاہر ہے کہ \عددی{n=10} کی خاصیت کارکردگی کی مطابقتی منحنی کی ڈھلوان زیادہ ہے۔اس کا مطلب ہے کہ \عددی{n} بڑھانے سے بہتر پرکھ حاصل ہوتا ہے۔کسی بھی عملی استعمال میں \عددی{n} کو کم سے کم لیکن اتنا زیادہ رکھا جاتا ہے کہ پرکھ \عددی{\mu} اور \عددی{\mu_0} میں انحراف، جس میں ہم دلچسپی رکھتے ہیں، کو  واضح  کرے۔ مثال کے طور پر اگر انحراف ہماری دلچسپی  \عددی{\mp 2} اکائی ہو، ہم شکل سے دیکھتے ہیں کہ \عددی{n=10} بہت کم ہو گا چونکہ جب \عددی{\mu=24-2=22} یا \عددی{\mu=24+2=26} ہو تب \عددی{\beta} تقریباً \عددی{\SI{50}{\percent}} ہو گا۔اس کے برعکس ، \عددی{n=100} اس صورت کافی ہو گا۔
\انتہا{مثال}
%==================

\ابتدا{مثال}\شناخت{مثال_شماریات_نا_معلوم_تغیریت_عمومی_تقسیم_اوسط_کا_پرکھ}\quad \موٹا{نا معلوم تغیریت کی عمومی تقسیم کی اوسط کا پرکھ}\\
رسی کی تنشی مضبوطی \عددی{n=16} کا نمونہ لے کر ناپی گئی۔نمونی اوسط \عددی{\overline{x}=\SI{4482}{\kilo\gram}} اور نمونی معیاری انحراف \عددی{s=\SI{115}{\kilo\gram}} حاصل ہوئے۔ ہم فرض کرتے ہیں کہ تنشی مضبوطی عمومی بلا منصوبہ  متغیر ہے۔ قیاس \عددی{\mu_0=\SI{4500}{\kilo\gram}} کو متبادل \عددی{\mu_1=\SI{4400}{\kilo\gram}} کے مقابلے میں پرکھیں۔یہاں \عددی{\mu_0} وہ قیمت ہو سکتی ہے جو بنانے والے نے فراہم کی ہو جبکہ \عددی{\mu_1} سابقہ تجربات کا نتیجہ ہو سکتا ہے۔\\
حل:\quad
ہم معنی خیز سطح \عددی{\alpha=\SI{5}{\percent}} منتخب کرتے ہیں۔اگر قیاس درست ہو تب مسئلہ \حوالہ{مسئلہ_شماریات_ٹی_تقسیم_اوسط_تغیریت} کے تحت یاس بلا منصوبہ متغیر
\begin{align*}
T=\sqrt{n}\,\frac{\overline{X}-\mu_0}{S}=4\, \frac{\overline{X}-4500}{S}
\end{align*}
کا \عددی{t} تقسیم \عددی{n-1=15} درجہ آزادی کا ہو گا۔ فاصل قیمت \عددی{c} کو درج ذیل مساوات سے حاصل کیا جائے گا۔
\begin{align*}
P(T<c)_{\mu_0}=\alpha=0.05
\end{align*}
ضمیمہ \حوالہ{ضمیمہ_جدول} کی جدول \حوالہ{ضمیمہ_ٹی_تقسیم} سے \عددی{c=-1.75} حاصل ہو گا۔نمونہ سے \عددی{T} کی مشاہدہ سے حاصل قیمت \عددی{t=\tfrac{4(4482-4500)}{115}=-0.626} ہے۔ہم دیکھتے ہیں کہ \عددی{t>c} ہے لہٰذا ہم قیاس کو نا منظور نہیں کرتے ہیں۔پرکھ کی طاقت کی اعدادی قیمتیں حاصل کرنے کی خاطر ہمیں مزید جدول بند قیمتیں درکار ہوں گی جن پر اس کتاب میں غور نہیں کیا جائے گا۔
\انتہا{مثال}
%=========================
\ابتدا{مثال}\quad \موٹا{(عمومی تقسیم کی تغیریت کی پرکھ)}\\
عمومی آبادی کے \عددی{n=15} جسامت اور نمونی تغیریت \عددی{s^2=13}  کے نمونہ سے قیاس \عددی{\sigma^2=\sigma^2_0=10} کو متبادل
 \عددی{\sigma^2=\sigma^2_1=20} میں مقابلے میں پرکھیں۔\\
حل:\quad
ہم معنی خیز سطح \عددی{\alpha=\SI{5}{\percent}} منتخب کرتے ہیں۔اگر قیاس درست ہو تب 
\begin{align*}
Y=(n-1)\frac{S^2}{\sigma^2_0}=14\frac{S^2}{10}=1.4S^2
\end{align*}
کا مربع خا تقسیم \عددی{n-1=14} درجہ آزادی کا ہو گا (مسئلہ \حوالہ{مسئلہ_شماریات_مربع_خا_تقسیم_کا_درجہ_آزادی})۔ضمیمہ \حوالہ{ضمیمہ_جدول} کی جدول \حوالہ{ضمیمہ_مربع_خا_تقسیم} اور درج ذیل سے  \عددی{14} درجہ آزادی کے لئے   \عددی{c=23.68} حاصل ہو گا 
\begin{align*}
P(Y>c)=\alpha=0.05\quad \implies \quad P(Y\le c)=0.95
\end{align*}
جو \عددی{Y} کی فاصل قیمت ہے۔یوں \عددی{S^2=\tfrac{\sigma^2_0 Y}{n-1}=0.714Y} کا مطابقتی فاصل قیمت \عددی{c^*=0.714\cdot 23.68=16.91} ہو گا۔چونکہ \عددی{s^2<c^*} ہے ہم قیاس کو نا منظور نہیں کرتے ہیں،

اگر متبادل درست ہو تب متغیر
\begin{align*}
Y_1=14\frac{S^2}{\sigma_1^2}=0.7S^2
\end{align*}
کے مربع خا تقسیم کا درجہ آزادی \عددی{14} ہو گا۔یوں ہمارے پرکھ کی طاقت
\begin{align*}
\eta=P(S^2>c^*)_{\sigma^2=20}=P(Y_1>0.7c^*)_{\sigma^2=20}=1-P(Y_1\le 11.84)_{\sigma^20}\approx \SI{62}{\percent}
\end{align*}
ہو گی اور ہم دیکھتے ہیں قسم دوم غلطی کا امکان (جو \عددی{\SI{38}{\percent}} ہے) بہت زیادہ ہے جس کو کم کرنے کے لئے نمونی جسامت بڑھانی ضروری ہے۔
\انتہا{مثال}
%===================
\ابتدا{مثال}\quad \موٹا{دو عمومی تقسیمات کی تغیریت کا آپس میں موازنہ}\\
نا معلوم اوسط \عددی{\mu_1} کی عمومی تقسیم کا نمونہ \عددی{x_1,\cdots,x_{n1}} اور دوسری عمومی تقسیم جس کی اوسط \عددی{\mu_2} نا معلوم ہو کا نمونہ \عددی{y_1,\cdots,y_{n2}} استعمال کرتے ہوئے ہم قیاس \عددی{\mu_1=\mu_2} کو متبادل مثلاً \عددی{\mu_1>\mu_2}  کے مقابلے میں پرکھنا چاہتے ہیں۔تغیرات جاننا ضروری نہیں ہے لیکن انہیں ایک جیسا\حاشیہد{اگر اگلے مثال کا پرکھ واضح کرے کہ تغیرات میں واضح فرق پایا جاتا ہے تب ایک جیسے \عددی{n_1=n_2=n}، مثلاً \عددی{n>30} منتخب کرتے ہوئے اس حقیقت کو استعمال کرتے ہوئے کہ مساوات تخمیناً عمومی بلا منصوبہ متغیر، جس کی اوسط \عددی{0} اور تغیریت \عددی{1} ہے، کی مشاہدے سے حاصل قیمت ہے، اور مثال \حوالہ{مثال_شماریات_معلوم_تغیریت_عمومی_تقسیم_کی_اوسط} کی طرز پر حل کریں۔} تصور کیا جاتا ہے۔دو صورتیں عملاً اہم ہیں۔\\
\موٹا{پہلی صورت:}\quad
\ترچھا{نمونوں کی جسامت ایک جیسی ہے۔مزید پہلے نمونہ کی ہر قیمت کا دوسرے نمونہ میں مطابقتی ٹھیک ایک قیمت پایا جاتا ہے}، چونکہ مطابقتی قیمتیں ایک ہی انسان یا چیز کی بدولت پائی جاتی ہیں (\اصطلاح{جوڑی دار موازنہ}\فرہنگ{موازنہ!جوڑی دار}\حاشیہب{paired comparison}\فرہنگ{comparison!paired})؛مثال کے طور پر ایک ہی چیز کی دو مختلف طریقوں سے ناپ، یا ایک ہی جانور کی دو آنکھوں کی ناپ، یا زیادہ عمومی طور پر جہاں ہم کہہ سکتے ہیں کہ  نمونوں کی جوڑی قیمتیں  ایک جیسے انسانوں یا چیزوں (مثلاً جڑواں بھائی،  گاڑھی کے اگلے ٹائر، وغیرہ) سے حاصل کی گئی ہوں۔ تب ہم مطابقتی قیمتوں کا فرق لے کر، مثال \حوالہ{مثال_شماریات_نا_معلوم_تغیریت_عمومی_تقسیم_اوسط_کا_پرکھ} میں دی ترکیب استعمال کرتے ہوئے، اس قیاس کو پرکھیں گے کہ ان فرق کی مطابقتی آبادی  کی اوسط \عددی{0}  ہے۔  اگر ممکن ہو تب ہم اسی ترکیب کو استعمال کریں گے ورنہ ہمیں درج ذیل ترکیب استعمال کرنی ہو گی۔\\
\موٹا{دوسری صورت:}\quad
\ترچھا{دونوں نمونے غیر تابع ہیں اور ان کی جسامت مختلف ہو سکتی ہے۔} تب ہم درج ذیل طریقے سے بڑھتے ہیں۔فرض کریں کہ متبادل \عددی{\mu_1>\mu_2} ہے۔ہم معنی خیز سطح \عددی{\alpha} منتخب کرتے ہیں۔ہم نمونی اوسط \عددی{\overline{x}}، \عددی{\overline{y}} اور \عددی{(n_1-1)s_1^2}، \عددی{(n-1)s_2^2} کا حساب کرتے ہیں جہاں \عددی{s_1^2} اور \عددی{s_2^2} نمونی تغیریت ہیں۔ضمیمہ \حوالہ{ضمیمہ_جدول} کی جدول \حوالہ{ضمیمہ_ٹی_تقسیم} میں \عددی{n_1+n_2-2} درجہ آزادی لیتے ہوئے ہم \عددی{c} کو
\begin{align}\label{مساوات_مثال_معنی_خیز_سطح_الف}
P(T\le c)=1-\alpha
\end{align}
سے تعین کرتے ہیں۔آخر میں ہم درج ذیل کا حساب کرتے ہیں۔
\begin{align}\label{مساوات_مثال_معنی_خیز_سطح_ب}
t_0=\sqrt{\frac{n_1n_2(n_1+n_2-2)}{n_1+n_2}}\, \frac{\overline{x}-\overline{y}}{\sqrt{(n_1-1)s_1^2+(n_2-1)s_2^2}}
\end{align}
یہ دکھایا جا سکتا ہے کہ اگر قیاس درست ہو تب یہ  \عددی{t} تقسیم کے  \عددی{n_1+n_2-2} درجہ آزادی کے  بلا منصوبہ متغیر کی مشاہدے سے حاصل قیمت ہے۔اگر \عددی{t_0\le c} ہو تب قیاس کو نا منظور نہیں کیا جاتا ہے۔اگر \عددی{t_0>c}ہو تب قیاس کو نا منظور کیا جاتا ہے۔

اگر متبادل \عددی{\mu_1 \ne \mu_2} ہو تب مساوات \حوالہ{مساوات_مثال_معنی_خیز_سطح_الف} کی جگہ درج ذیل استعمال کیا جائے گا۔
\begin{align*}\tag*{(*\ref{مساوات_مثال_معنی_خیز_سطح_الف})}
P(T\le c_1)=0.5\alpha,\quad P(T\le c_2)=1-0.5\alpha
\end{align*}
دھیان رہے کہ ایک جیسی نمونی جسامت \عددی{n_1=n_2=n} کے لئے  مساوات \حوالہ{مساوات_مثال_معنی_خیز_سطح_ب} درج ذیل صورت اختیار کرتی ہے۔
\begin{align}\label{مساوات_مثال_معنی_خیز_سطح_پ}
t_0=\sqrt{n}\,\frac{\overline{x}-\overline{y}}{\sqrt{s_1^2-s_2^2}}
\end{align}

اس کی وضاحت کے لئے آئیں درج ذیل دو نمونوں پر غور کرتے ہیں جو ایک ہی کام میں دو مختلف حالات میں مزدور کی کارکردگی ہے۔
\begin{align*}
\begin{array}{rrrrrrrr}
105&108&86&103&103&107&124&105\\
89&92&84&97&103&107&111&97
\end{array}
\end{align*}
فرض کریں کہ مطابقتی آبادی عمومی ہے اور ان کی تغیریت ایک جیسی ہے۔آئیں قیاس \عددی{\mu_1=\mu_2} کو متبادل \عددی{\mu_1\ne \mu_2} کے مقابلے میں پرکھیں۔ (تغیریت کی ایک جیسا ہونے کو اگلی مثال میں استعمال کیا جائے گا۔)\\
\موٹا{حل:}\quad
ہم درج ذیل حاصل کرتے ہیں۔
\begin{align*}
\overline{x}=105.125,\quad \overline{y}=97.500,\quad s_1^2=106.125,\quad s_2^2=84.000
\end{align*}
ہم معنی خیز سطح \عددی{\alpha=\SI{5}{\percent}} منتخب کرتے ہیں۔مساوات \حوالہ{مساوات_مثال_معنی_خیز_سطح_الف}* میں \عددی{0.5\alpha=\SI{2.5}{\percent}}، \عددی{1-0.5\alpha=\SI{97.5}{\percent}} اور  ضمیمہ \حوالہ{ضمیمہ_جدول} کی جدول \حوالہ{ضمیمہ_ٹی_تقسیم} میں \عددی{14} درجہ آزادی سے \عددی{c_1=-2.15} اور \عددی{c_2=2.15} حاصل ہوتے ہیں۔مساوات \حوالہ{مساوات_مثال_معنی_خیز_سطح_پ} میں \عددی{n=8} استعمال کرتے ہوئے درج ذیل قیمت حاصل ہوتی ہے۔
\begin{align*}
t_0=\frac{\sqrt{8}\cdot 7.625}{\sqrt{190.125}}=1.56
\end{align*}
چونکہ \عددی{c_1\le t_0\le c_2} ہے ہم دونوں صورتوں میں ایک جیسی اوسط کے  قیاس \عددی{\mu_1=\mu_2} کو نا منظور نہیں کرتے ہیں۔

پہلی صورت اس مثال پر لاگو ہوتی ہے چونکہ پہلی دونوں نمونوں کی پہلی نمونی قیمت ایک قسم کے کام کے لئے حاصل کی گئی۔اسی طرح دونوں نمونوں کی دوسری نمونی قیمت کسی دوسرے کام کے لئے حاصل کی گئی، وغیرہ۔یوں ہم ان نمونی قیمتوں کا مطابقتی فرق
\begin{align*}
\begin{array}{rrrrrrrr}
16&16&2&6&0&0&13&8
\end{array}
\end{align*}
اور مثال \حوالہ{مثال_شماریات_نا_معلوم_تغیریت_عمومی_تقسیم_اوسط_کا_پرکھ} کی ترکیب استعمال کرتے ہوئے قیاس \عددی{\mu=0} پرکھ سکتے ہیں جہاں \عددی{\mu} اس فرق کی اوسط ہے۔ہم اس کا منطقی متبادل \عددی{\mu \ne 0} لیتے ہیں۔نمونی اوسط \عددی{\overline{d}=7.625} اور نمونی تغیریت \عددی{s^2=45.696} ہے لہٰذا درج ذیل ہو گا۔
\begin{align*}
t=\frac{\sqrt{8}(7.625-0)}{\sqrt{45.696}}=3.19
\end{align*}
\عددی{P(T\le c_1)=\SI{2.5}{\percent}}، \عددی{P(T\le c_2)=\SI{97.5}{\percent}} اور ضمیمہ \حوالہ{ضمیمہ_جدول} کی جدول \حوالہ{ضمیمہ_ٹی_تقسیم} میں \عددی{n-1=7} درجہ آزادی سے \عددی{c_1=-2.37} اور \عددی{c_2=2.37} حاصل ہوتے ہیں لہٰذا ہم قیاس کو نا منظور کرتے ہیں چونکہ \عددی{t=3.19} معلوم شدہ \عددی{c_1} اور \عددی{c_2} کے بیچ نہیں پایا جاتا ہے۔اس طرح ہمارا موجودہ پرکھ، جو اسی نمونوں پر مبنی ہے لیکن زیادہ معلومات کو استعمال کرتا ہے، دکھاتا ہے کہ نتائج میں فرق کافی ہے۔ 
\انتہا{مثال} 
%=======================
\ابتدا{مثال}\quad \موٹا{(دو عمومی تقسیمات کی تغیریت کا موازنہ)}\\
گزشتہ مثال کے دو نمونے استعمال کرتے ہوئے قیاس \عددی{\sigma^2_1=\sigma^2_2}کو پرکھیں۔فرض کریں کہ مطابقتی آبادیاں عمومی ہیں اور تجربہ کی نوعیت سے متبادل \عددی{\sigma^2_1 >\sigma^2_2} ہو گا۔\\
\موٹا{حل:}\quad
ہم \عددی{s_1^2=106.125} اور \عددی{s_2^2=84.000} حاصل کرتے ہیں۔ہم معنی خیز سطح \عددی{\alpha=\SI{5}{\percent}} منتخب کرتے ہیں۔\عددی{P(V\le c)=1-\alpha=\SI{95}{\percent}} اور ضمیمہ \حوالہ{ضمیمہ_جدول} کی جدول \حوالہ{ضمیمہ_ایف_تقسیم} میں \عددی{(n_1-1,n_2-1)=(7,7)} درجہ آزادی  سے \عددی{c=3.79} تعین ہوتا ہے۔ہم آخر میں \عددی{v_0=\tfrac{s_1^2}{s_2^2}=1.26} حاصل کرتے ہیں۔چونکہ \عددی{v_0\le c} ہے ہم قیاس کو نا منظور نہیں کرتے ہیں۔اگر \عددی{v_0>c} ہوتا ہم اس کو نا منظور کرتے۔

قیاس درست ہونے کی صورت میں \عددی{v_0} ایسے بلا منصوبہ متغیر کی مشاہدے سے حاصل قیمت ہے جس کی تقسیم درجہ آزادی \عددی{(n_1-1,n_2-1)} کی \عددی{F} \اصطلاح{تقسیم}\فرہنگ{تقسیم!ایف}\حاشیہب{F-distribution}\فرہنگ{distribution!F} ہے۔  \عددی{(m,n)} درجہ آزادی کے \عددی{F} تقسیم کا تفاعل تقسیم درج ذیل ہے
\begin{align}
F(z)=
\begin{cases}
K_{mn}\int_{0}^{z} t^{\tfrac{m-2}{2}} (mt+n)^{-\tfrac{m+n}{2}}\dif t& z\ge 0\\
0&z<0
\end{cases}
\end{align}
جہاں 
$K_{mn}=m^{\tfrac{m}{2}}n^{\tfrac{n}{2}}\tfrac{\Gamma(\tfrac{m}{2}+\tfrac{n}{2})}{\Gamma(\tfrac{m}{2})\Gamma(\tfrac{n}{2})}$
ہے۔
\انتہا{مثال}
%===========================

\حصہء{سوالات}

