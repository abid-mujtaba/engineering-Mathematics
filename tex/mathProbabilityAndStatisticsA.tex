\جزوحصہء{عمومی تقسیم کی صورت میں پرکھ}
درج ذیل مثال عملاً اہم قیاس کے پرکھ کی وضاحت کرتا ہے۔ 

%==============
\ابتدا{مثال}\شناخت{مثال_شماریات_معلوم_تغیریت_عمومی_تقسیم_کی_اوسط}\quad \موٹا{(معلوم تغیریت کی عمومی تقسیم کی اوسط کا پرکھ)}\\
فرض کریں کہ \عددی{X} بلا منصوبہ متغیر ہے جس کی تغیریت \عددی{\sigma^2=9} ہے۔نمونی جسامت \عددی{n=10} لیتے ہوئے قیاس \عددی{\mu=\mu_0=24} کو درج ذیل تین متبادل کے بالمقابل پرکھیں۔
\begin{align*}
\text{(پ)}\,\,\, \mu \ne \mu_0 \quad \text{(ب)}\,\,\, \mu<\mu_0 \quad \text{(الف)}\,\,\,  \mu>\mu_0
\end{align*}
حل:\quad
ہم معنی خیز سطح \عددی{\alpha=0.05} منتخب کرتے ہیں۔اوسط کی اندازاً قیمت درج ذیل سے حاصل ہو گی۔
\begin{align*}
\overline{X}=\frac{1}{n}(X_1+\cdots,X_n)
\end{align*}
اگر قیاس درست ہو تب \عددی{X} عمومی ہو گا جس کی اوسط \عددی{\mu=24} اور تغیریت \عددی{\tfrac{\sigma^2}{n}=0.9} ہو گی (مسئلہ \حوالہ{مسئلہ_شماریات_عمومی_کی_شرط_اوسط_تغیریت})۔لہٰذا ہم فاصل قیمت \عددی{c} کو ضمیمہ \حوالہ{ضمیمہ_جدول} کی جدول \حوالہ{ضمیمہ_عمومی_تقسیم_ب} سے حاصل کر سکتے ہیں۔\\
\موٹا{صورت الف:}\quad
ہم \عددی{P(\overline{X}\le c)_{\mu=24}=\alpha=0.05} سے \عددی{c} تعین کرتے ہیں۔
\begin{align*}
P(\overline{X}\le c)_{\mu=24}=\Phi\big(\frac{c-24}{\sqrt{0.9}}\big)=1-\alpha=0.95
\end{align*}
ضمیمہ \حوالہ{ضمیمہ_جدول} کی جدول \حوالہ{ضمیمہ_عمومی_تقسیم_ب}  سے \عددی{\tfrac{c-24}{\sqrt{0.9}}=1.645} یعنی \عددی{c=25.56} حاصل ہوتا ہے جو \عددی{\mu_0} سے بڑی قیمت ہے (اور جو شکل \حوالہ{شکل_شماریات_خطہ_منظور_اور_نا_منظوری} میں سب سے اوپر دکھائی گئی صورت ہے)۔ اگر \عددی{\overline{x}\le 25.56} ہو تب قیاس کو منظور کیا جائے گا۔ اگر \عددی{\overline{x}>25.56} ہو تب قیاس کو نا منظور کیا جائے گا۔پرکھ کی طاقت درج ذیل ہو گی۔
\begin{gather}
\begin{aligned}
\eta(\mu)=P(\overline{X}>25.56)_{\mu}&=1-P(\overline{X}\le 25.56)_{\mu}\\
&=1-\Phi\big(\frac{25.56-\mu}{\sqrt{0.9}}\big)=1-\Phi(26.94-1.05\mu)
\end{aligned}
\end{gather}
\موٹا{صورت ب:}\quad
فاصل قیمت \عددی{c} کو درج ذیل مساوات سے حاصل کیا جا سکتا ہے۔
\begin{align*}
P(\overline{X}\le c)_{\mu=24}=\Phi\big(\frac{c-24}{\sqrt{0.9}}\big)=\alpha=0.05
\end{align*}
ضمیمہ \حوالہ{ضمیمہ_جدول} کی جدول \حوالہ{ضمیمہ_عمومی_تقسیم_ب}  سے \عددی{c=24-1.56=22.24} ملتا ہے۔اگر \عددی{\overline{x}\ge 22.44} ہو تب ہم قیاس کو منظور کرتے ہیں۔اگر \عددی{\overline{x}<22.44} ہو تب ہم قیاس کو نا منظور کرتے ہیں۔پرکھ کی طاقت درج ذیل ہے۔
\begin{align}
\eta(\mu)=P(\overline{X}\le 22.44)_{\mu}=\Phi\big(\frac{22.44-\mu}{\sqrt{0.9}}\big)=\Phi(23.65-1.05\mu)
\end{align}
\موٹا{صورت پ:}\quad
چونکہ عمومی تقسیم تشاکلی ہے، ہم \عددی{\mu=24} سے \عددی{c_1} اور \عددی{c_2} کو ایک جیسے فاصلے پر چن کر، مثلاً \عددی{c_1=24-k} اور \عددی{c_2=24+k}، مستقل \عددی{k} کو درج ذیل سے تعین کرتے ہیں۔
\begin{align*}
P(24-k\le \overline{X}\le 24+k)_{\mu=24}=\Phi\big(\frac{k}{\sqrt{0.9}}\big)-\Phi\big(-\frac{k}{\sqrt{0.9}}\big)=1-\alpha=0.95
\end{align*}
ضمیمہ \حوالہ{ضمیمہ_جدول} کی جدول \حوالہ{ضمیمہ_عمومی_تقسیم_ب} سے \عددی{\tfrac{k}{\sqrt{0.9}}=1.960} یعنی \عددی{k=1.86} حاصل ہو گا۔یوں \عددی{c_1=24-1.86=22.14} اور \عددی{c_2=24+1.86=25.86} ہوں گے۔اگر \عددی{\overline{x}} کی قیمت \عددی{c_1} سے چھوٹی نہ ہو اور \عددی{c_2} سے بڑی نہ ہو تب ہم قیاس کو منظور کرتے ہیں۔اس کے علاوہ ہم قیاس کو نا منظور کرتے ہیں۔پرکھ کی طاقت درج ذیل ہے۔
\begin{gather}
\begin{aligned}
\eta(\mu)&=P(\overline{X}<22.14)_{\mu}+P(\overline{X}>25.86)_{\mu}\\
&=P(\overline{X}<22.14)_{\mu}+1-P(\overline{X}\le 25.86)_{\mu}\\
&=1+\Phi\big(\frac{22.14-\mu}{\sqrt{0.9}}\big)-\Phi\big(\frac{25.86-\mu}{\sqrt{0.9}}\big)\\
&=1+\Phi(23.34-1.05\mu)-\Phi(27.26-1.05\mu)
\end{aligned}
\end{gather}
نتیجتاً خاصیت کارکردگی \عددی{\beta(\mu)=1-\eta(\mu)} درج ذیل ہو گی۔
\begin{align*}
\beta(\mu)&=\Phi\big(\frac{24.59-\mu}{\sqrt{0.9}}\big)-\Phi\big(\frac{23.41-\mu}{\sqrt{0.9}}\big)\\
&=\Phi(81.97-3.33\mu)-\Phi(78.03-3.33\mu)
\end{align*}
شکل سے ظاہر ہے کہ \عددی{n=10} کی خاصیت کارکردگی کی مطابقتی منحنی کی ڈھلوان زیادہ ہے۔اس کا مطلب ہے کہ \عددی{n} بڑھانے سے بہتر پرکھ حاصل ہوتا ہے۔کسی بھی عملی استعمال میں \عددی{n} کو کم سے کم لیکن اتنا زیادہ رکھا جاتا ہے کہ پرکھ \عددی{\mu} اور \عددی{\mu_0} میں انحراف، جس میں ہم دلچسپی رکھتے ہیں، کو  واضح  کرے۔ مثال کے طور پر اگر انحراف ہماری دلچسپی  \عددی{\mp 2} اکائی ہو، ہم شکل سے دیکھتے ہیں کہ \عددی{n=10} بہت کم ہو گا چونکہ جب \عددی{\mu=24-2=22} یا \عددی{\mu=24+2=26} ہو تب \عددی{\beta} تقریباً \عددی{\SI{50}{\percent}} ہو گا۔اس کے برعکس ، \عددی{n=100} اس صورت کافی ہو گا۔
\انتہا{مثال}
%==================

\ابتدا{مثال}\شناخت{مثال_شماریات_نا_معلوم_تغیریت_عمومی_تقسیم_اوسط_کا_پرکھ}\quad \موٹا{نا معلوم تغیریت کی عمومی تقسیم کی اوسط کا پرکھ}\\
رسی کی تنشی مضبوطی \عددی{n=16} کا نمونہ لے کر ناپی گئی۔نمونی اوسط \عددی{\overline{x}=\SI{4482}{\kilo\gram}} اور نمونی معیاری انحراف \عددی{s=\SI{115}{\kilo\gram}} حاصل ہوئے۔ ہم فرض کرتے ہیں کہ تنشی مضبوطی عمومی بلا منصوبہ  متغیر ہے۔ قیاس \عددی{\mu_0=\SI{4500}{\kilo\gram}} کو متبادل \عددی{\mu_1=\SI{4400}{\kilo\gram}} کے مقابلے میں پرکھیں۔یہاں \عددی{\mu_0} وہ قیمت ہو سکتی ہے جو پیداکار نے فراہم کی ہو جبکہ \عددی{\mu_1} سابقہ تجربات کا نتیجہ ہو سکتا ہے۔\\
حل:\quad
ہم معنی خیز سطح \عددی{\alpha=\SI{5}{\percent}} منتخب کرتے ہیں۔اگر قیاس درست ہو تب مسئلہ \حوالہ{مسئلہ_شماریات_ٹی_تقسیم_اوسط_تغیریت} کے تحت  بلا منصوبہ متغیر
\begin{align*}
T=\sqrt{n}\,\,\,\frac{\overline{X}-\mu_0}{S}=4\,\,\, \frac{\overline{X}-4500}{S}
\end{align*}
کا \عددی{t} تقسیم \عددی{n-1=15} درجہ آزادی کا ہو گا۔ فاصل قیمت \عددی{c} کو درج ذیل مساوات سے حاصل کیا جائے گا۔
\begin{align*}
P(T<c)_{\mu_0}=\alpha=0.05
\end{align*}
ضمیمہ \حوالہ{ضمیمہ_جدول} کی جدول \حوالہ{ضمیمہ_ٹی_تقسیم} سے \عددی{c=-1.75} حاصل ہو گا۔نمونہ سے \عددی{T} کی مشاہدہ سے حاصل قیمت \عددی{t=\tfrac{4(4482-4500)}{115}=-0.626} ہے۔ہم دیکھتے ہیں کہ \عددی{t>c} ہے لہٰذا ہم قیاس کو نا منظور نہیں کرتے ہیں۔پرکھ کی طاقت کی اعدادی قیمتیں حاصل کرنے کی خاطر ہمیں مزید جدول بند قیمتیں درکار ہوں گی جن پر اس کتاب میں غور نہیں کیا جائے گا۔
\انتہا{مثال}
%=========================
\ابتدا{مثال}\شناخت{مثال_شماریات_عمومی_تقسیم_پرکھ_تغیریت}\quad \موٹا{(عمومی تقسیم کی تغیریت کی پرکھ)}\\
عمومی آبادی کے \عددی{n=15} جسامت اور نمونی تغیریت \عددی{s^2=13}  کے نمونہ سے قیاس \عددی{\sigma^2=\sigma^2_0=10} کو متبادل
 \عددی{\sigma^2=\sigma^2_1=20} میں مقابلے میں پرکھیں۔\\
حل:\quad
ہم معنی خیز سطح \عددی{\alpha=\SI{5}{\percent}} منتخب کرتے ہیں۔اگر قیاس درست ہو تب 
\begin{align*}
Y=(n-1)\frac{S^2}{\sigma^2_0}=14\frac{S^2}{10}=1.4S^2
\end{align*}
کا مربع خا تقسیم \عددی{n-1=14} درجہ آزادی کا ہو گا (مسئلہ \حوالہ{مسئلہ_شماریات_مربع_خا_تقسیم_کا_درجہ_آزادی})۔ضمیمہ \حوالہ{ضمیمہ_جدول} کی جدول \حوالہ{ضمیمہ_مربع_خا_تقسیم} اور درج ذیل سے  \عددی{14} درجہ آزادی کے لئے   \عددی{c=23.68} حاصل ہو گا 
\begin{align*}
P(Y>c)=\alpha=0.05\quad \implies \quad P(Y\le c)=0.95
\end{align*}
جو \عددی{Y} کی فاصل قیمت ہے۔یوں \عددی{S^2=\tfrac{\sigma^2_0 Y}{n-1}=0.714Y} کا مطابقتی فاصل قیمت \عددی{c^*=0.714\cdot 23.68=16.91} ہو گا۔چونکہ \عددی{s^2<c^*} ہے ہم قیاس کو نا منظور نہیں کرتے ہیں،

اگر متبادل درست ہو تب متغیر
\begin{align*}
Y_1=14\frac{S^2}{\sigma_1^2}=0.7S^2
\end{align*}
کے مربع خا تقسیم کا درجہ آزادی \عددی{14} ہو گا۔یوں ہمارے پرکھ کی طاقت
\begin{align*}
\eta=P(S^2>c^*)_{\sigma^2=20}=P(Y_1>0.7c^*)_{\sigma^2=20}=1-P(Y_1\le 11.84)_{\sigma^20}\approx \SI{62}{\percent}
\end{align*}
ہو گی اور ہم دیکھتے ہیں قسم دوم غلطی کا امکان (جو \عددی{\SI{38}{\percent}} ہے) بہت زیادہ ہے جس کو کم کرنے کے لئے نمونی جسامت بڑھانی ضروری ہے۔
\انتہا{مثال}
%===================
\ابتدا{مثال}\quad \موٹا{دو عمومی تقسیمات کی تغیریت کا آپس میں موازنہ}\\
نا معلوم اوسط \عددی{\mu_1} کی عمومی تقسیم کا نمونہ \عددی{x_1,\cdots,x_{n1}} اور دوسری عمومی تقسیم جس کی اوسط \عددی{\mu_2} نا معلوم ہو کا نمونہ \عددی{y_1,\cdots,y_{n2}} استعمال کرتے ہوئے ہم قیاس \عددی{\mu_1=\mu_2} کو متبادل مثلاً \عددی{\mu_1>\mu_2}  کے مقابلے میں پرکھنا چاہتے ہیں۔تغیرات جاننا ضروری نہیں ہے لیکن انہیں ایک جیسا\حاشیہد{اگر اگلے مثال کا پرکھ واضح کرے کہ تغیرات میں واضح فرق پایا جاتا ہے تب ایک جیسے \عددی{n_1=n_2=n}، مثلاً \عددی{n>30} منتخب کرتے ہوئے اس حقیقت کو استعمال کرتے ہوئے کہ مساوات تخمیناً عمومی بلا منصوبہ متغیر، جس کی اوسط \عددی{0} اور تغیریت \عددی{1} ہے، کی مشاہدے سے حاصل قیمت ہے، اور مثال \حوالہ{مثال_شماریات_معلوم_تغیریت_عمومی_تقسیم_کی_اوسط} کی طرز پر حل کریں۔} تصور کیا جاتا ہے۔دو صورتیں عملاً اہم ہیں۔\\
\موٹا{پہلی صورت:}\quad
\ترچھا{نمونوں کی جسامت ایک جیسی ہے۔مزید پہلے نمونہ کی ہر قیمت کا دوسرے نمونہ میں مطابقتی ٹھیک ایک قیمت پایا جاتا ہے}، چونکہ مطابقتی قیمتیں ایک ہی انسان یا چیز کی بدولت پائی جاتی ہیں (\اصطلاح{جوڑی دار موازنہ}\فرہنگ{موازنہ!جوڑی دار}\حاشیہب{paired comparison}\فرہنگ{comparison!paired})؛مثال کے طور پر ایک ہی چیز کی دو مختلف طریقوں سے ناپ، یا ایک ہی جانور کی دو آنکھوں کی ناپ، یا زیادہ عمومی طور پر جہاں ہم کہہ سکتے ہیں کہ  نمونوں کی جوڑی قیمتیں  ایک جیسے انسانوں یا چیزوں (مثلاً جڑواں بھائی،  گاڑھی کے اگلے ٹائر، وغیرہ) سے حاصل کی گئی ہوں۔ تب ہم مطابقتی قیمتوں کا فرق لے کر، مثال \حوالہ{مثال_شماریات_نا_معلوم_تغیریت_عمومی_تقسیم_اوسط_کا_پرکھ} میں دی ترکیب استعمال کرتے ہوئے، اس قیاس کو پرکھیں گے کہ ان فرق کی مطابقتی آبادی  کی اوسط \عددی{0}  ہے۔  اگر ممکن ہو تب ہم اسی ترکیب کو استعمال کریں گے ورنہ ہمیں درج ذیل ترکیب استعمال کرنی ہو گی۔\\
\موٹا{دوسری صورت:}\quad
\ترچھا{دونوں نمونے غیر تابع ہیں اور ان کی جسامت مختلف ہو سکتی ہے۔} تب ہم درج ذیل طریقے سے بڑھتے ہیں۔فرض کریں کہ متبادل \عددی{\mu_1>\mu_2} ہے۔ہم معنی خیز سطح \عددی{\alpha} منتخب کرتے ہیں۔ہم نمونی اوسط \عددی{\overline{x}}، \عددی{\overline{y}} اور \عددی{(n_1-1)s_1^2}، \عددی{(n-1)s_2^2} کا حساب کرتے ہیں جہاں \عددی{s_1^2} اور \عددی{s_2^2} نمونی تغیریت ہیں۔ضمیمہ \حوالہ{ضمیمہ_جدول} کی جدول \حوالہ{ضمیمہ_ٹی_تقسیم} میں \عددی{n_1+n_2-2} درجہ آزادی لیتے ہوئے ہم \عددی{c} کو
\begin{align}\label{مساوات_مثال_معنی_خیز_سطح_الف}
P(T\le c)=1-\alpha
\end{align}
سے تعین کرتے ہیں۔آخر میں ہم درج ذیل کا حساب کرتے ہیں۔
\begin{align}\label{مساوات_مثال_معنی_خیز_سطح_ب}
t_0=\sqrt{\frac{n_1n_2(n_1+n_2-2)}{n_1+n_2}}\,\,\, \frac{\overline{x}-\overline{y}}{\sqrt{(n_1-1)s_1^2+(n_2-1)s_2^2}}
\end{align}
یہ دکھایا جا سکتا ہے کہ اگر قیاس درست ہو تب یہ  \عددی{t} تقسیم کے  \عددی{n_1+n_2-2} درجہ آزادی کے  بلا منصوبہ متغیر کی مشاہدے سے حاصل قیمت ہے۔اگر \عددی{t_0\le c} ہو تب قیاس کو نا منظور نہیں کیا جاتا ہے۔اگر \عددی{t_0>c}ہو تب قیاس کو نا منظور کیا جاتا ہے۔

اگر متبادل \عددی{\mu_1 \ne \mu_2} ہو تب مساوات \حوالہ{مساوات_مثال_معنی_خیز_سطح_الف} کی جگہ درج ذیل استعمال کیا جائے گا۔
\begin{align*}\tag*{(*\ref{مساوات_مثال_معنی_خیز_سطح_الف})}
P(T\le c_1)=0.5\alpha,\quad P(T\le c_2)=1-0.5\alpha
\end{align*}
دھیان رہے کہ ایک جیسی نمونی جسامت \عددی{n_1=n_2=n} کے لئے  مساوات \حوالہ{مساوات_مثال_معنی_خیز_سطح_ب} درج ذیل صورت اختیار کرتی ہے۔
\begin{align}\label{مساوات_مثال_معنی_خیز_سطح_پ}
t_0=\sqrt{n}\,\,\,\frac{\overline{x}-\overline{y}}{\sqrt{s_1^2-s_2^2}}
\end{align}

اس کی وضاحت کے لئے آئیں درج ذیل دو نمونوں پر غور کرتے ہیں جو ایک ہی کام میں دو مختلف حالات میں مزدور کی کارکردگی ہے۔
\begin{align*}
\begin{array}{rrrrrrrr}
105&108&86&103&103&107&124&105\\
89&92&84&97&103&107&111&97
\end{array}
\end{align*}
فرض کریں کہ مطابقتی آبادی عمومی ہے اور ان کی تغیریت ایک جیسی ہے۔آئیں قیاس \عددی{\mu_1=\mu_2} کو متبادل \عددی{\mu_1\ne \mu_2} کے مقابلے میں پرکھیں۔ (تغیریت کی ایک جیسا ہونے کو اگلی مثال میں استعمال کیا جائے گا۔)\\
\موٹا{حل:}\quad
ہم درج ذیل حاصل کرتے ہیں۔
\begin{align*}
\overline{x}=105.125,\quad \overline{y}=97.500,\quad s_1^2=106.125,\quad s_2^2=84.000
\end{align*}
ہم معنی خیز سطح \عددی{\alpha=\SI{5}{\percent}} منتخب کرتے ہیں۔مساوات \حوالہ{مساوات_مثال_معنی_خیز_سطح_الف}* میں \عددی{0.5\alpha=\SI{2.5}{\percent}}، \عددی{1-0.5\alpha=\SI{97.5}{\percent}} اور  ضمیمہ \حوالہ{ضمیمہ_جدول} کی جدول \حوالہ{ضمیمہ_ٹی_تقسیم} میں \عددی{14} درجہ آزادی سے \عددی{c_1=-2.15} اور \عددی{c_2=2.15} حاصل ہوتے ہیں۔مساوات \حوالہ{مساوات_مثال_معنی_خیز_سطح_پ} میں \عددی{n=8} استعمال کرتے ہوئے درج ذیل قیمت حاصل ہوتی ہے۔
\begin{align*}
t_0=\frac{\sqrt{8}\cdot 7.625}{\sqrt{190.125}}=1.56
\end{align*}
چونکہ \عددی{c_1\le t_0\le c_2} ہے ہم دونوں صورتوں میں ایک جیسی اوسط کے  قیاس \عددی{\mu_1=\mu_2} کو نا منظور نہیں کرتے ہیں۔

پہلی صورت اس مثال پر لاگو ہوتی ہے چونکہ پہلی دونوں نمونوں کی پہلی نمونی قیمت ایک قسم کے کام کے لئے حاصل کی گئی۔اسی طرح دونوں نمونوں کی دوسری نمونی قیمت کسی دوسرے کام کے لئے حاصل کی گئی، وغیرہ۔یوں ہم ان نمونی قیمتوں کا مطابقتی فرق
\begin{align*}
\begin{array}{rrrrrrrr}
16&16&2&6&0&0&13&8
\end{array}
\end{align*}
اور مثال \حوالہ{مثال_شماریات_نا_معلوم_تغیریت_عمومی_تقسیم_اوسط_کا_پرکھ} کی ترکیب استعمال کرتے ہوئے قیاس \عددی{\mu=0} پرکھ سکتے ہیں جہاں \عددی{\mu} اس فرق کی اوسط ہے۔ہم اس کا منطقی متبادل \عددی{\mu \ne 0} لیتے ہیں۔نمونی اوسط \عددی{\overline{d}=7.625} اور نمونی تغیریت \عددی{s^2=45.696} ہے لہٰذا درج ذیل ہو گا۔
\begin{align*}
t=\frac{\sqrt{8}(7.625-0)}{\sqrt{45.696}}=3.19
\end{align*}
\عددی{P(T\le c_1)=\SI{2.5}{\percent}}، \عددی{P(T\le c_2)=\SI{97.5}{\percent}} اور ضمیمہ \حوالہ{ضمیمہ_جدول} کی جدول \حوالہ{ضمیمہ_ٹی_تقسیم} میں \عددی{n-1=7} درجہ آزادی سے \عددی{c_1=-2.37} اور \عددی{c_2=2.37} حاصل ہوتے ہیں لہٰذا ہم قیاس کو نا منظور کرتے ہیں چونکہ \عددی{t=3.19} معلوم شدہ \عددی{c_1} اور \عددی{c_2} کے بیچ نہیں پایا جاتا ہے۔اس طرح ہمارا موجودہ پرکھ، جو اسی نمونوں پر مبنی ہے لیکن زیادہ معلومات کو استعمال کرتا ہے، دکھاتا ہے کہ نتائج میں فرق کافی ہے۔ 
\انتہا{مثال} 
%=======================
\ابتدا{مثال}\quad \موٹا{(دو عمومی تقسیمات کی تغیریت کا موازنہ)}\\
گزشتہ مثال کے دو نمونے استعمال کرتے ہوئے قیاس \عددی{\sigma^2_1=\sigma^2_2}کو پرکھیں۔فرض کریں کہ مطابقتی آبادیاں عمومی ہیں اور تجربہ کی نوعیت سے متبادل \عددی{\sigma^2_1 >\sigma^2_2} ہو گا۔\\
\موٹا{حل:}\quad
ہم \عددی{s_1^2=106.125} اور \عددی{s_2^2=84.000} حاصل کرتے ہیں۔ہم معنی خیز سطح \عددی{\alpha=\SI{5}{\percent}} منتخب کرتے ہیں۔\عددی{P(V\le c)=1-\alpha=\SI{95}{\percent}} اور ضمیمہ \حوالہ{ضمیمہ_جدول} کی جدول \حوالہ{ضمیمہ_ایف_تقسیم} میں \عددی{(n_1-1,n_2-1)=(7,7)} درجہ آزادی  سے \عددی{c=3.79} تعین ہوتا ہے۔ہم آخر میں \عددی{v_0=\tfrac{s_1^2}{s_2^2}=1.26} حاصل کرتے ہیں۔چونکہ \عددی{v_0\le c} ہے ہم قیاس کو نا منظور نہیں کرتے ہیں۔اگر \عددی{v_0>c} ہوتا ہم اس کو نا منظور کرتے۔

قیاس درست ہونے کی صورت میں \عددی{v_0} ایسے بلا منصوبہ متغیر کی مشاہدے سے حاصل قیمت ہے جس کی تقسیم درجہ آزادی \عددی{(n_1-1,n_2-1)} کی \عددی{F} \اصطلاح{تقسیم}\فرہنگ{تقسیم!ایف}\حاشیہب{F-distribution}\فرہنگ{distribution!F} ہے۔  \عددی{(m,n)} درجہ آزادی کے \عددی{F} تقسیم\حاشیہد{انگلستانی ماہر جینیات رونلڈ ایلمر فشر [1890-1962]} کا تفاعل تقسیم درج ذیل ہے
\begin{align}
F(z)=
\begin{cases}
K_{mn}\int_{0}^{z} t^{\tfrac{m-2}{2}} (mt+n)^{-\tfrac{m+n}{2}}\dif t& z\ge 0\\
0&z<0
\end{cases}
\end{align}
جہاں 
$K_{mn}=m^{\tfrac{m}{2}}n^{\tfrac{n}{2}}\tfrac{\Gamma(\tfrac{m}{2}+\tfrac{n}{2})}{\Gamma(\tfrac{m}{2})\Gamma(\tfrac{n}{2})}$
ہے۔
\انتہا{مثال}
%===========================

\حصہء{سوالات}
%==================
\ابتدا{سوال}\شناخت{سوال_شماریات_سکہ_منصفانہ_الف}\quad 
صفحہ \حوالہصفحہ{جدول_شماریات_سکہ_پھینکنے_کے_نتائج} پر جدول \حوالہ{جدول_شماریات_سکہ_پھینکنے_کے_نتائج} میں امجد کے مواد کو استعمال کرتے ہوئے اس قیاس کو پرکھیں کہ سکہ منصفانہ ہے، یعنی خط اور شیر کا احتمال ایک جیسا ہے۔\عددی{\alpha=\SI{5}{\percent}} منتخب کریں۔\\
جواب:\quad
اگر قیاس \عددی{p=0.5} درست ہو تب \عددی{4040} کوششوں میں خط کی تعداد\عددی{X=} تقریباً عمومی ہو گا جس کی اوسط \عددی{\mu=2020} اور تغیریت \عددی{\sigma^2=1010} ہو گی (حصہ \حوالہ{حصہ_شماریات_عمومی_تقسیم})۔\\
$P(X\le c)=\Phi(\tfrac{c-2020}{\sqrt{1010}})=0.95,\,\,\,c=2072>2048$
ہے لہٰذا قیاس نا منظور نہ کریں۔
\انتہا{سوال}
%=====================
\ابتدا{سوال}\quad
مشرف کا مواد استعمال کرتے ہوئے سوال \حوالہ{سوال_شماریات_سکہ_منصفانہ_الف} کو دوبارہ حل کریں۔
\انتہا{سوال}
%======================
\ابتدا{سوال}\شناخت{سوال_شماریات_قیاس_منظور_یا_نا_منظور_الف}\quad
عمومیت تصور کرتے ہوئے اور \عددی{\sigma^2=4} لیتے ہوئے قیاس \عددی{\mu=15.0} کو متبادل (الف) \عددی{\mu=12.0} اور (ب) \عددی{\mu=15.8} کے بالمقابل پرکھیں۔نمونی جسامت \عددی{10} اور نمونی اوسط \عددی{\overline{x}=14} لیں جبکہ \عددی{\alpha=\SI{5}{\percent}} منتخب کریں۔\\
جواب:\quad (الف) \عددی{c=13.96>12.00} ہے۔ قیاس کو نا منظور کریں۔\\
(ب) \عددی{c=16.04>15.80} ہے۔قیاس کو نا منظور نہ کریں۔
\انتہا{سوال}
%=========================
\ابتدا{سوال}\quad
اگر بڑی نمونی جسامت، مثلاً \عددی{100}، استعمال کی جائے تب سوال \حوالہ{سوال_شماریات_قیاس_منظور_یا_نا_منظور_الف} میں باقی مواد (\عددی{\overline{x}=14}، \عددی{\alpha=\SI{5}{\percent}}، وغیرہ) تبدیل کیے بغیر نتیجے میں کیا تبدیلی پیدا ہو گی؟ 
\انتہا{سوال}
%=============================
\ابتدا{سوال}\quad
دو طرفہ پرکھ، \عددی{\SI{5}{\percent}} سطح پر استعمال کرتے ہوئے سوال \حوالہ{سوال_شماریات_قیاس_منظور_یا_نا_منظور_الف} میں خطہ نا منظوری تلاش کریں؟\\
جواب:\quad \عددی{\mu<13.76} یا \عددی{\mu>16.24}
\انتہا{سوال}
%============================
\ابتدا{سوال}\quad
سوال \حوالہ{سوال_شماریات_قیاس_منظور_یا_نا_منظور_الف}-الف میں پرکھ کی طاقت تلاش کریں۔
\انتہا{سوال}
%=====================
\ابتدا{سوال}\quad
مثال \حوالہ{مثال_شماریات_معلوم_تغیریت_عمومی_تقسیم_کی_اوسط}-الف اور ب کی خاصیت کارکردگی کو ترسیم کریں۔  
\انتہا{سوال}
%======================
\ابتدا{سوال}\quad
دکھائیں کہ عمومی تقسیم میں قیاس \عددی{H_0:\mu=\mu_0} اور متبادل \عددی{H_1:\mu=\mu_1} کی پرکھ میں  دو اقسام کی غلطیوں کو نمونی جسامت کافی بڑھا کر  جتنا چاہیں کم (ما سوائے صفر کرنے کے) کیا جا سکتا ہے۔
\انتہا{سوال}
%=========================
\ابتدا{سوال}\quad
\عددی{\mu=0} کو \عددی{\mu>0} کے بالمقابل سطح \عددی{\alpha=\SI{5}{\percent}} پرکھیں۔ عمومیت فرض کرتے ہوئے نمونہ \عددی{1,-1,1,3,-8,6,0} لیں جو مصنوعی سیارہ \موٹا{ٹلسٹار} کی \عددی{143} ویں گردش میں مدار سے مضرب \عددی{0.01} ریڈیئن انحراف ہے۔\\
جواب:\quad
$t=\sqrt{7}\tfrac{0.286-0}{4.31}=0.18<c=1.94$
ہے لہٰذا قیاس کو نا منظور نہ کریں۔
\انتہا{سوال}
%==========================
\ابتدا{سوال}\quad
مثال \حوالہ{مثال_شماریات_اوسط} میں دیا گیا نمونہ استعمال کرتے ہوئے قیاس \عددی{\mu=\SI{0.80}{\centi\meter}} (ڈبے پر درج لمبائی) کو متبادل \عددی{\mu\ne \SI{0.80}{\centi\meter}} کے مقابل پرکھیں۔ (عمومیت تصور کرتے ہوئے \عددی{\alpha=\SI{5}{\percent}} لیں۔)
\انتہا{سوال}
%====================
\ابتدا{سوال}\quad
ایک مشین ڈبوں میں فی ڈبہ \عددی{\SI{1000}{\gram}} تیل بھرتی ہے۔آپ جاننا چاہتے ہیں کہ آیا \عددی{\alpha=\SI{5}{\percent}} سطح پر اوسط کی درکار کمیت \عددی{\SI{1000}{\gram}} سے  تجاوز زیادہ ہے۔اگر ایسا ہو تب مشین میں مطابقت پیدا کرنی ہو گی۔ایک قیاس اور متبادل  بنائیں اور انہیں پرکھیں۔ عمومیت فرض کرتے ہوئے نمونی جسامت \عددی{20} جس کی  اوسط \عددی{\SI{996}{\gram}} اور معیاری انحراف \عددی{\SI{5}{\gram}}  ہو استعمال کریں۔\\
جواب:\quad
متبادل \عددی{\mu\ne 1000}،
$t=\sqrt{20}\tfrac{996-1000}{5}=-3.58<c=-2.09$
(ضمیمہ \حوالہ{ضمیمہ_جدول} جدول  \حوالہ{ضمیمہ_ٹی_تقسیم} درجہ آزادی \عددی{19})۔ قیاس \عددی{\mu=\SI{1000}{\gram}} کو نا منظور کریں۔
\انتہا{سوال}
%======================
\ابتدا{سوال}\quad
ایک مخصوص ٹائر کی اوسط زندگی \عددی{\SI{32000}{\kilo\meter}} اور معیاری انحراف \عددی{\SI{4000}{\kilo\meter}} ہے۔کیا ٹائر کا پیداکار یہ دعویٰ کر سکتا ہے کہ اس کے بنائے ہوئے ٹائروں کی اوسط زندگی \عددی{\SI{30000}{\kilo\meter}} سے زیادہ ہے۔متبادل قیاس بناتے ہوئے اس کو \عددی{\SI{5}{\percent}} سطح پر پرکھیں۔
\انتہا{سوال}
%=====================
\ابتدا{سوال}\quad
برقی دباو کو بیک وقت دو عدد وولٹ پیما سے ناپا جاتا ہے۔ ان کے نتائج میں فرق 
\begin{align*}
0.8,0.2,-0.3,0.1,0.0,0.5,0.2
\end{align*}
 وولٹ  ہے۔عمومیت فرض کرتے ہوئے کیا ہم \عددی{\SI{5}{\percent}} سطح کے لحاظ سے کہہ سکتے ہیں کہ دونوں وولٹ پیما کی \ترچھا{پیمانہ بندی}\حاشیہب{calibration} میں کوئی معنی خیز فرق نہیں پایا جاتا ہے۔\\
جواب:\quad
\عددی{\mu=0} کو متبادل \عددی{\mu \ne 0} کے مقابلے میں پرکھیں۔
$t=2.11<c=2.37$
(درجہ آزادی \عددی{7})۔ قیاس کو نا منظور نہ کریں۔
\انتہا{سوال}
%=========================
\ابتدا{سوال}\quad
ایک معیاری دوائی ایک مخصوص مرض  میں مبتلا \عددی{\SI{70}{\percent}} مریضوں کو صحتیاب کرتی ہے اور ایک نئی دوائی پہلے \عددی{200} مریضوں میں سے \عددی{148} کو صحتیاب کرتی ہے۔  کیا \عددی{\alpha=\SI{5}{\percent}} لیتے ہوئے ہم کہہ سکتے ہیں کہ نئی دوائی زیادہ بہتر ہے؟
\انتہا{سوال}
%========================
\ابتدا{سوال}\quad
ماضی میں ایک مشین جو فی ڈبہ \عددی{\SI{25}{\kilo\gram}} چینی بھرتی تھی کا معیاری انحراف \عددی{\SI{0.4}{\kilo\gram}} تھا۔قیاس \عددی{H_0:\sigma=0.4} کو متبادل \عددی{H_1:\sigma>0.4}  کے بالمقابل پرکھیں۔عمومیت تصور کرتے ہوئے   نمونی جسامت \عددی{10}  جس کی معیاری انحراف \عددی{\sigma=0.4} ہو لیں اور 
\عددی{\alpha=\SI{5}{\percent}} منتخب کریں۔\\
جواب:\quad
$\alpha=\SI{5}{\percent}, c=16.92>\tfrac{9\cdot 0.5^2}{0.4^2}=14.06$
 ہے۔ قیاس کو نا منظور نہ کریں۔
\انتہا{سوال}
%=========================
\ابتدا{سوال}\quad
فرض کریں کہ معیاری انحراف کسی مخصوص حد سے کم، مثلاً، \عددی{5} گھنٹوں سے کم، ہونے کی صورت میں بیٹری سے چلنے والی مشینوں میں تمام بیٹریوں کو مخصوص مدت کے بعد بیک وقت تبدیل کرنا کم مہنگا پڑتا ہے  بہ نسبت ہر بیٹری کو اس وقت تبدیل کرنے کے جب وہ خراب ہو جائے۔ ایک موزوں پرکھ بنا کر اس قیاس کو پرکھیں۔عرصہ زندگی کے \عددی{28} قیمتیں جن کا معیاری انحراف \عددی{s=3.5} گھنٹے ہو استعمال کرتے ہوئے \عددی{\alpha=\SI{5}{\percent}} لیں۔عمومیت تصور کریں۔
\انتہا{سوال}
%===========================
\ابتدا{سوال}\quad
تیل کی قسم \عددی{A} کو \عددی{9} ایک جیسی گاڑیوں میں ایک جیسے حالات میں استعمال کیا گیا جنہوں نے اوسط \عددی{20.2} کلومیٹر فی لٹر اور معیاری انحراف \عددی{0.5} دیا۔انہیں حالات میں تیل کی بہتر قسم \عددی{B} کو اس جیسی \عددی{10} گاڑیوں میں استعمال کیا گیا جن سے اوسط \عددی{21.8} اور معیاری انحراف \عددی{0.6} حاصل ہوا۔کیا \عددی{B} بہت بہتر نتائج دیتا ہے؟اس قیاس کو \عددی{\alpha=\SI{5}{\percent}} پر پرکھیں۔عمومیت فرض کریں۔\\
جواب:\quad
$t_0=\sqrt{\tfrac{10\cdot9\cdot17}{19}}\tfrac{21.8-20.2}{\sqrt{9\cdot 0.6^2+8\cdot0.5^2}}=6.3>c=1.74$
(درجہ آزادی \عددی{17} ہے۔)
\انتہا{سوال}
%=======================
\ابتدا{سوال}\quad
ماسوائے عرصہ زندگی، بلب \عددی{A} اور \عددی{B}  ایک جیسے ہیں۔ایک خریدار دونوں قسم کے \عددی{100} بلب کو پرکھتا ہے۔قسم \عددی{A} کی  اوسط عرصہ زندگی \عددی{\SI{1120}{\hour}} اور معیاری انحراف \عددی{\SI{75}{\hour}} جبکہ \عددی{B} کی اوسط \عددی{\SI{1064}{\hour}} اور معیاری انحراف \عددی{\SI{82}{\hour}} حاصل ہوتے ہیں۔ کیا عرصہ زندگی میں معنی خیز فرق پایا جاتا ہے؟ (عمومیت فرض کرتے ہوئے \عددی{\alpha=\SI{5}{\percent}} سطح پر پرکھیں۔)
\انتہا{سوال}
%===========================
\ابتدا{سوال}\quad
نمونی جسامت \عددی{10} اور \عددی{16} اور تغیریت \عددی{s_1^2=50} اور \عددی{s_2^2=30} لیں۔ عمومیت تصور کرتے ہوئے \عددی{\alpha=\SI{5}{\percent}} سطح پر  قیاس \عددی{H_0:\sigma^2_1=\sigma_2^2} کو متبادل \عددی{H_1:\sigma^2_1>\sigma^2_2} کے بالمقابل پرکھیں۔\\
جواب:\quad
$v_0=\tfrac{50}{30}=1.67<c=2.59$
[درجہ آزادی \عددی{(9,15)} ہے۔] قیاس کو نا منظور نہ کریں۔
\انتہا{سوال}
%=============================
\ابتدا{سوال}\quad
دو نمونے \عددی{50,90,100,90,110,80} اور \عددی{110,110,120,110,130,110,120} لوہے کی ڈھلائی کے دوران دو مختلف بالٹیوں میں دو مختلف وقتوں پر درجہ حرارت \عددی{(\si{\celsius})} میں فرق دیتی ہیں۔ کیا پہلے نمونہ کی تغیریت دوسرے  سے زیادہ ہے؟ عمومیت فرض کریں اور \عددی{\alpha=\SI{5}{\percent}} لیں۔
\انتہا{سوال}
%=========================

\حصہ{ضبط معیار}
پیداوار کا کوئی بھی عمل اتنا ٹھیک نہیں ہوتا ہے کہ تمام پیداوار مکمل طور پر ایک جیسی ہو۔  بہت ساری معمولی، غیر قابو وجوہات کی بنا ان میں ہر صورت معمولی فرق پایا جاتا ہے جس کو امکانی فرق تصور کیا جا سکتا ہے۔یہ ضروری ہے کہ پیداوار کی درکار خاصیت کی قیمت درست ہو (مثلاً لمبائی، مضبوطی، یا جو بھی خاصیت کسی مخصوص صورت میں درکار ہو)۔ اس مقصد کے لئے اس قیاس کو پرکھا جاتا ہے کہ پیداوار درکار خاصیت، مثلاً \عددی{\mu=\mu_0}، رکھتے ہیں جہاں \عددی{\mu_0} درکار قیمت ہے۔اگر ایسا پوری کھیپ کی پیداوار (مثلاً، \عددی{100000} پیچوں کی کھیپ) کے بعد کیا جائے تب پرکھ ہمیں بتائے گا کہ  پیداوار کتنی اچھی یا کتنی خراب ہے لیکن ظاہر ہے کہ اس نتیجہ کو استعمال کرتے ہوئے ہم کوئی بہتری نہیں لا سکتے ہیں۔بہتری لانے کے لئے ضروری ہے کہ پرکھ دوران پیداوار کیا جائے۔ایسا عموماً مقررہ دورانیہ (مثلاً ہر \عددی{30} منٹ یا ہر گھنٹہ) بعد جاتا ہے اور اس کو \اصطلاح{ضبط معیار}\فرہنگ{ضبط معیار}\فرہنگ{معیار!ضبط}\حاشیہب{quality control}\فرہنگ{control!quality} کہتے ہیں۔ہر مرتبہ ایک جیسی جسامت (عملاً \عددی{3} یا \عددی{10} اجزاء) کا نمونہ لیا جاتا ہے۔قیاس نا منظور ہونے کی صورت میں عمل پیداوار روک کر اس وجہ کو تلاش کیا جاتا ہے جس کی بنا انحراف پیدا ہوا ہے۔

اگر ہم عمل پیدا وار کو روک دیں اگرچہ سب ٹھیک چل رہا ہو تب ہم غلطی قسم اول  کر رہے ہوں گے۔اگر خرابی کے باوجود ہم عمل پیداوار کو نا روکیں تب ہم غلطی قسم دوم  کر رہے ہوں گے (حصہ \حوالہ{حصہ_شماریات_قیاس_کی_پرکھ_فیصلے})۔

ہر پرکھ کا نتیجہ کو ترسیمی صورت میں \اصطلاح{نقشہ ضبط}\فرہنگ{ضبط!نقشہ}\حاشیہب{control chart}\فرہنگ{control!chart} پر ظاہر\حاشیہد{امریکی ماہر شماریات والٹر انڈرو شوہارٹ [1891-1967] نے یہ نقشہ  \سن{1924} میں تجویز کیا جو معیار کو قابو کرنے میں انتہائی موثر ثابت ہوا ہے۔} کیا جاتا ہے۔

\جزوحصہء{اوسط کا نقشہ ضبط}
شکل \حوالہ{شکل_شماریات_نقش_ضبط} میں نقشہ ضبط کی مثال دکھائی گئی ہے۔اوسط کے نقشہ ضبط پر \اصطلاح{نچلی حد ضبط}\فرہنگ{ضبط!نچلی حد}\حاشیہب{lower control limit ($\LCL$)}\فرہنگ{control!lower limit} \عددی{\LCL}، \اصطلاح{وسطی خط ضبط}\فرہنگ{ضبط!وسطی خط}\حاشیہب{central control line ($\textrm{CL}$)}\فرہنگ{control!central line} \عددی{\textrm{CL}} اور \اصطلاح{بالائی حد ضبط}\فرہنگ{ضبط!بالائی حد}\حاشیہب{upper control limit ($\UCL$)}\فرہنگ{control!upper limit} \عددی{\UCL} دکھائے گئے ہیں۔ یہ  حدود مثال \حوالہ{مثال_شماریات_معلوم_تغیریت_عمومی_تقسیم_کی_اوسط}-پ میں فاصل قیمتوں \عددی{c_1} اور \عددی{c_2} کے مطابقتی ہیں۔جیسے ہم نمونی اوسط نچلی حد ضبط یا بالائی حد ضبط سے تجاوز کر جائے ہم قیاس کو نا منظور کرتے ہوئے کہتے ہیں کہ عمل پیداوار "بے قابو" ہے، یعنی، ہم کہتے ہیں کہ عمل پیداوار میں تبدیلی رو نما ہوئی ہے۔جب بھی کوئی نقطہ حدود ضبط سے تجاوز کرے عمل پیداوار میں مداخلت کی ضرورت ہو گی۔

اگر ہم حدود ضبط ڈھیلے  رکھیں تب ہم عمل پیداوار میں نا پسندیدہ تبدیلی کو پکڑ نہیں پائیں گے۔اس کے برعکس حدود ضبط بہت سخت رکھنے سے  ہم بار بار عمل پیداوار کو روک کر نا پسندیدہ تبدیلی کی  غیر موجود وجہ تلاش کرتے رہیں گے جس سے پیداوار بری طرح متاثر ہو گی۔عموماً معنی خیز سطح \عددی{\alpha=\SI{1}{\percent}} منتخب کی جاتی ہے۔صفحہ \حوالہصفحہ{مسئلہ_شماریات_عمومی_کی_شرط_اوسط_تغیریت} پر مسئلہ \حوالہ{مسئلہ_شماریات_عمومی_کی_شرط_اوسط_تغیریت} اور ضمیمہ \حوالہ{ضمیمہ_جدول} کی جدول \حوالہ{ضمیمہ_عمومی_تقسیم_ب} سے ہم دیکھتے ہیں کہ عمومی تقسیم کی صورت میں اوسط کے مطابقتی حد ضبط
\begin{align}\label{مساوات_شماریات_حدود_ضبط_عمومی_تقسیم}
\LCL=\mu_0-2.58\,\,\,\frac{\sigma}{\sqrt{n}} \quad \text{اور}\quad \UCL=\mu_0+2.58\,\,\,\frac{\sigma}{\sqrt{n}}
\end{align} 
ہوں گے۔یہاں فرض کیا گیا ہے کہ ہمیں \عددی{\sigma} معلوم ہے۔اگر \عددی{\sigma} نا معلوم ہو تب پہلی \عددی{20} یا \عددی{30} نمونوں کی معیاری انحراف  حاصل کر کے ان کی اوسط کو \عددی{\sigma} کی تخمینی قیمت تصور کیا جا سکتا ہے۔شکل \حوالہ{شکل_شماریات_نقش_ضبط} میں اوسط  کو  لکیر سے جوڑا جاتا ہے جو محض نتائج کو واضح کرنے میں مدد دیتی ہے۔


\جزوحصہء{تغیریت کا نقشہ ضبط}
اوسط کے ساتھ ساتھ عموماً تغیریت، معیاری انحراف یا سعت کو بھی قابو رکھا جاتا ہے۔عمومی تقسیم کی صورت میں معیاری انحراف کا نقشہ ضبط بناتے ہوئے مثال \حوالہ{مثال_شماریات_عمومی_تقسیم_پرکھ_تغیریت} میں استعمال ترکیب بروئے کار لاتے ہوئے حدود ضبط تعین کیے جا سکتے ہیں۔روایتی طور پر صرف بالائی حد ضبط استعمال کیا جاتا ہے۔مثال \حوالہ{مثال_شماریات_عمومی_تقسیم_پرکھ_تغیریت} سے یہ حد
\begin{align}\label{مساوات_شماریات_تغیریت_بالائی_حد_ضبط}
\UCL=\frac{\sigma^{\,\,\,2} c}{n-1}
\end{align}
ہو گا جہاں \عددی{c} کو مساوات 
\begin{align*}
P(Y>c)=\alpha \quad  \implies  \quad P(Y\le c)=1-\alpha
\end{align*}
اور ضمیمہ \حوالہ{ضمیمہ_جدول} کی جدول \حوالہ{ضمیمہ_مربع_خا_تقسیم} (مربع خا تقسیم) سے \عددی{n-1} درجہ آزادی کے لئے  حاصل کیا جاتا ہے؛ یہاں نمونہ سے مشاہدے کے ذریعہ  \عددی{S^2} کی حاصل قیمت \عددی{s^2} کا بالائی حد ضبط سے تجاوز کا احتمال \عددی{\alpha} (\عددی{\SI{5}{\percent}} یا \عددی{\SI{1}{\percent}}) ہے۔

اگر ہم تغیریت کے نقشہ ضبط میں نچلی حد ضبط  اور بالائی حد ضبط استعمال کرنا چاہیں تب یہ حدود
\begin{align}\label{مساوات_شماریات_تغیریت_دونوں_حدود}
\LCL=\frac{\sigma^2 c_1}{n-1},\quad \UCL=\frac{\sigma^2 c_2}{n-1}
\end{align} 
ہوں گے جہاں \عددی{c_1} اور \عددی{c_2} کو  \عددی{n-1} درجہ آزادی کے لئے  ضمیمہ \حوالہ{ضمیمہ_جدول} کی جدول \حوالہ{ضمیمہ_مربع_خا_تقسیم}  اور درج ذیل مساوات سے حاصل کیا جائے گا۔
\begin{align}
P(Y\le c_1)=\frac{\alpha}{2},\quad P(Y\le c_2)=1-\frac{\alpha}{2}
\end{align}
%================================
\begin{figure}
\centering
\begin{subfigure}{1\textwidth}
\centering
\begin{tikzpicture}
\draw(0,-2)--(0,2);
\foreach \x in {1,2,3,4,5,6,7,8,9,10,11,12,13,14}{\draw[gray](\x/2.5,-2)--++(0,4);}
\foreach \y in {1,2,3,4,6,7,8,9}{\draw[gray](0,\y/5)--++(5.6,0) (0,-\y/5)--++(5.6,0);}
\draw(0,0/5)node[left]{$4.10$}--++(5.6,0);
\draw(0,5/5)node[left]{$4.15$}--++(5.6,0);
\draw[gray](0,10/5)node[left,black]{$4.20$}--++(5.6,0);
\draw(0,-5/5)node[left]{$4.05$}--++(5.6,0);
\draw[gray](0,-10/5)node[left,black]{$4.00$}--++(5.6,0);
\draw(5/2.5,-10/5)node[below]{$5$};
\draw(10/2.5,-10/5)node[below]{$10$};
\draw(5.6,5/5)node[right]{\RL{بالائی حد ضبط}};
\draw(5.6,0/5)node[right]{\RL{وسطی خط ضبط}};
\draw(5.6,-5/5)node[right]{\RL{نچلی حد ضبط}};
\draw(6.8,5/5)--++(0.85,0);
\draw(6.8,-5/5)--++(0.85,0);
\draw(1/2.5,20*4.080-20*4.10)node[ocirc]{}
\foreach \x/\y in {2/4.112,3/4.084,4/4.088,5/4.108,6/4.100,7/4.088,8/4.096,9/4.100,10/4.104,11/4.140,12/4.152}{--(\x/2.5,20*\y-20*4.10)node[ocirc]{}};
\draw[stealth-stealth] (7.54,5/5)--++(0,-2)node[pos=0.5,fill=white]{$\SI{99}{\percent}$};
\draw[stealth-] (7.54,5/5)--++(0,1)node[pos=0.5,fill=white]{$\SI{0.5}{\percent}$};
\draw[stealth-] (7.54,-5/5)--++(0,-1)node[pos=0.5,fill=white]{$\SI{0.5}{\percent}$};
\draw(-1,0.5)node[rotate=90]{اوسط};
\draw(3,-2)node[below]{\RL{شمار نمونہ}};
\end{tikzpicture}
\end{subfigure}
\begin{subfigure}{1\textwidth}
\centering
\begin{tikzpicture}
\draw(0,0)--(0,4);
\foreach \x in {1,2,3,4,5,6,7,8,9,10,11,12,13,14}{\draw[gray](\x/2.5,0)--++(0,4);}
\foreach \y in {0,1,2,3,4,5,6,7,8}{\draw[gray](0,\y/2)--++(5.6,0);}
\draw(0,2)node[left]{$0.02$}--++(5.6,0);
\draw(0,100*0.036)node[left]{$0.036$}--++(5.6,0);
\draw(0,0)node[left]{$0$};
\draw(0,100*0.01)node[left]{$0.01$};
\draw(0,100*0.03)node[left]{$0.03$};
\draw(0,100*0.04)node[left]{$0.04$};
\draw(5/2.5,0)node[below]{$5$};
\draw(10/2.5,0)node[below]{$10$};
%curve
\draw(1/2.5,100*0.014)node[ocirc]{}
\foreach \x/\y in{2/0.011,3/0.026,4/0.023,5/0.018,6/0.014,7/0.023,8/0.017,9/0.032,10/0.038}{--(\x/2.5,100*\y)node[ocirc]{}};
\draw(-1,2)node[rotate=90]{\RL{معیاری انحراف}};
\draw(3,0)node[below]{\RL{شمار نمونہ}};
\draw(5.6,3.65)node[right]{\RL{بالائی حد ضبط}};
\draw(6.8,3.65)--++(0.85,0);
\draw(6.5,0)--(7.65,0);
\draw[stealth-stealth] (7.54,0)--++(0,3.65)node[pos=0.5,fill=white]{$\SI{99}{\percent}$};
\draw[stealth-] (7.54,3.65)--++(0,1)node[pos=0.5,fill=white]{$\SI{1}{\percent}$};
\end{tikzpicture}
\end{subfigure}
\caption{اوسط اور معیاری انحراف کے نقشہ ضبط برائے جدول \حوالہ{جدول_شماریات_نلکی_کے_قطر_کا_نمونہ}}
\label{شکل_شماریات_نقش_ضبط}
\end{figure}

%====================
\جزوحصہء{معیاری انحراف کا نقشہ ضبط}
تغیریت کے نقشہ ضبط کی طرح ہمیں بالائی حد ضبط
\begin{align}\label{مساوات_شماریات_معیاری_انحراف_بالائی_حد}
\UCL=\frac{\sigma \sqrt{c}}{\sqrt{n-1}}
\end{align}
درکار ہو گا جس کو مساوات \حوالہ{مساوات_شماریات_تغیریت_بالائی_حد_ضبط} سے حاصل کیا گیا ہے۔مثال کے طور پر جدول \حوالہ{جدول_شماریات_نلکی_کے_قطر_کا_نمونہ} میں \عددی{n=5} ہے۔مطابقتی آبادی کو عمومی تصور کرتے ہوئے جس کی معیاری انحراف \عددی{\sigma=0.02} ہو، \عددی{\alpha=\SI{1}{\percent}}  منتخب کرتے ہوئے  \عددی{4} درجہ آزادی کے لئے  ضمیمہ \حوالہ{ضمیمہ_جدول} کی جدول \حوالہ{ضمیمہ_مربع_خا_تقسیم} اور مساوات
\begin{align*}
P(Y\le c)=1-\alpha=\SI{99}{\percent}
\end{align*}
سے فاصل قیمت \عددی{c=13.28} حاصل ہوتی ہے۔یوں مساوات \حوالہ{مساوات_شماریات_معیاری_انحراف_بالائی_حد} سے
\begin{align*}
\UCL=\frac{0.02\sqrt{13.28}}{\sqrt{4}}=0.0365
\end{align*}
حاصل ہو گا جس کو شکل \حوالہ{شکل_شماریات_نقش_ضبط} کے نچلے حصے میں دکھایا گیا ہے۔

معیاری انحراف کا نقشہ ضبط جس میں بالائی حد ضبط اور نچلا حد ضبط پائے جاتے ہوں کو مساوات \حوالہ{مساوات_شماریات_تغیریت_دونوں_حدود} سے حاصل کیا جا سکتا ہے۔

\begin{table}
\caption{بارہ نمونے جہاں ہر نمونہ $5$ قیمتوں (چھوٹی نلکیوں کے ملی میٹروں میں قطر) پر مشتمل ہے}
\label{جدول_شماریات_نلکی_کے_قطر_کا_نمونہ}
\centering
\begin{otherlanguage}{english}
\begin{tabular}{C|CCCCC|C|C|C}
\hline
\text{\RL{\urdufont{نمونی شمار}}} & \multicolumn{5}{C|}{\text{\RL{\urdufont{نمونی قیمتیں}}}}& \overline{x}&s&R\\
\hline
1&4.06&4.08&4.08&4.08&4.10&4.080&0.014&0.04\\
2&4.10&4.10&4.12&4.12&4.12&4.112&0.011&0.02\\
3&4.06&4.06&4.08&4.10&4.12&4.084&0.026&0.06\\
4&4.06&4.08&4.08&4.10&4.12&4.088&0.023&0.06\\
5&4.08&4.10&4.12&4.12&4.12&4.108&0.018&0.04\\
&&&&&&&&\\
6&4.08&4.10&4.10&4.10&4.12&4.100&0.014&0.04\\
7&4.06&4.08&4.08&4.10&4.12&4.088&0.023&0.06\\
8&4.08&4.08&4.10&4.10&4.12&4.096&0.017&0.04\\
9&4.06&4.08&4.10&4.12&4.14&4.100&0.032&0.08\\
10&4.06&4.08&4.10&4.12&4.16&4.104&0.038&0.10\\
&&&&&&&&\\
11&4.12&4.14&4.14&4.14&4.16&4.140&0.014&0.04\\
12&4.14&4.14&4.16&4.16&4.16&4.152&0.011&0.02\\
\hline
\end{tabular}
\end{otherlanguage}
\end{table}
%======================================
\جزوحصہء{سعت کا نقشہ ضبط}
اگر ہم \عددی{\sigma^2} یا \عددی{\sigma} کو قابو رکھتے ہوں تب ہمیں بالترتیب \عددی{s^2} یا \عددی{s} کا حساب کرنا ہو گا۔ایسا کرنا  غیر تربیت یافتہ شخص کے لئے مشکل ہوتا ہے لہٰذا ہم تغیریت یا معیاری انحراف کی حد ضبط کی جگہ سعت \عددی{R=}(نمونہ کی زیادہ سے زیادہ قیمت منفی نمونہ کی کم سے کم قیمت) استعمال کرنا چاہیں گے۔عمومی تقسیم کی صورت میں یہ دکھایا جا سکتا ہے کہ معیاری انحراف \عددی{\sigma} کی قیمت بلا منصوبہ متغیر \عددی{R^*} کی  توقع  کے راست متناسب ہے جس کی مشاہدے سے حاصل قیمت \عددی{R} ہو، یعنی \عددی{\sigma=\lambda_nE(R^*)}، جہاں جزو \عددی{\lambda_n} کی قیمت نمونی جسامت پر منحصر ہے اور اس کی قیمتیں درج ذیل ہیں۔
\begin{align*}
\begin{array}{cccccccccc}
n&2&3&4&5&6&7&8&9&10\\
\hline
\lambda_n=\sigma/E(R^*)&0.89&0.59&0.49&0.43&0.40&0.37&0.35&0.34&0.32\\[1ex]
n&12&14&16&18&20&30&40&50&\\
\hline
\lambda_n=\sigma/E(R^*)&0.31&0.29&0.28&0.28&0.27&0.25&0.23&0.22&
\end{array}
\end{align*}

چونکہ \عددی{R} صرف دو نمونی قیمتوں پر منحصر ہے لہٰذا یہ نمونے کے بارے میں \عددی{s} کے لحاظ سے کم معلومات فراہم کرتا ہے۔ظاہر ہے کہ نمونی جسامت \عددی{n} جتنی بڑی ہو گی، \عددی{s} کی جگہ \عددی{R} استعمال کرنے سے، اتنی زیادہ معلومات ہم ضائع کریں گے۔عملاً اگر \عددی{n} کی قیمت \عددی{10} سے زائد ہو تب \عددی{s} استعمال کیا جاتا ہے۔

دھیان رہے کہ سعت سے  معیاری انحراف کا جلدی سے  اندازہ لگانا عملی استعمال میں کار آمد ثابت ہوتا ہے۔

%==================
\حصہء{سوالات}
  
%=====================
\ابتدا{سوال}\شناخت{سوال_شماریات_حد_ضبط_الف}\quad 
ایک مشین چکنا تیل کو ٹین کی بوتل میں یوں بھرتی ہے کہ عمومی آبادی حاصل ہو جس کی اوسط \عددی{1} لٹر اور معیاری انحراف \عددی{0.03} لٹر ہو۔ اوسط کے لئے شکل \حوالہ{شکل_شماریات_نقش_ضبط} کی طرح نقشہ درکار ہے۔ نمونی جسامت \عددی{6} فرض کرتے ہوئے نچلی حد ضبط اور بالائی حد ضبط تلاش کریں۔\\
جواب:\quad
نچلی حد ضبط 
$\LCL=1-\tfrac{2.58\cdot 0.03}{\sqrt{6}}=0.968$
جبکہ بالائی حد ضبط \عددی{\UCL=1.032}
\انتہا{سوال}
%=========================
\ابتدا{سوال}\quad
سوال \حوالہ{سوال_شماریات_حد_ضبط_الف} میں دکھائیں کہ \عددی{\alpha=\SI{0.3}{\percent}} سطح سے درج ذیل  حاصل ہوتے ہیں۔ان کی اعدادی قیمتیں تلاش کریں۔
\begin{align*}
\LCL=\mu-\tfrac{3\sigma}{\sqrt{n}},\quad \UCL=\mu+\tfrac{3\sigma}{\sqrt{n}}
\end{align*}
\انتہا{سوال}
%========================
\ابتدا{سوال}\quad
معنی خیز سطح تبدیل کیے بغیر ہمیں سوال \حوالہ{سوال_شماریات_حد_ضبط_الف} میں نمونی جسامت کتنی رکھنی ہو گی تا کہ بالائی اور نچلی حد ضبط قریب قریب ہوں، مثلاً \عددی{\UCL-\LCL=0.05}\\
جواب:\quad
$n=10$
\انتہا{سوال}
%======================
\ابتدا{سوال}\quad
اگر ہم غیر عمومی آبادی کے  لئے مساوات \حوالہ{مساوات_شماریات_حدود_ضبط_عمومی_تقسیم} کے حدود ضبط والا نقشہ ضبط استعمال کریں تب ان حدود کا کیا مطلب ہو گا؟
\انتہا{سوال}
%=============================
\ابتدا{سوال}\quad
عمومی آبادی کی اوسط قابو کرتے ہوئے  \عددی{\UCL-\LCL} کو نصف کرنے کی خاطر نمونی جسامت کو کس طرح تبدیل کرنا ہو گا؟\\
جواب:\quad
نمونی جسامت کو \عددی{4} گنا بڑھانا ہو گا۔
\انتہا{سوال}
%===========================
\ابتدا{سوال}\quad
قابلوں کی پیداوار میں سے \عددی{2} جسامت کے \عددی{10} نمونے لئے گئے۔ان کی لمبائی ملی میٹروں میں درج ذیل ہے۔
\begin{align*}
\begin{array}{ccccccccccc}
\text{\RL{نمونی شمار}}&1&2&3&4&5&6&7&8&9&10\\
\hline
\multirow{2}{*}{\text{\RL{لمبائی}}}&27.4&27.4&27.5&27.3&27.9&27.6&27.6&27.8&27.5&27.3\\
&27.6&27.4&27.7&27.4&27.5&27.5&27.4&27.3&27.4&27.7
\end{array}
\end{align*}
فرض کریں کہ آبادی عمومی ہے جس کی اوسط \عددی{27.5} اور تغیریت \عددی{0.024} ہے۔مساوات \حوالہ{مساوات_شماریات_حدود_ضبط_عمومی_تقسیم} استعمال کرتے ہوئے اوسط کے لئے نقش ضبط بنائیں اور نمونی اوسط اس پر ترسیم کریں۔\\
جواب:\quad
$\tfrac{2.58\sqrt{0.024}}{\sqrt{2}}=0.283,\UCL=27.783,\LCL=27.217$
\انتہا{سوال}
%==========================
\ابتدا{سوال}\شناخت{سوال_شماریات_موٹائی_الف}\quad
لوہے کی چادر موٹائی کے درج ذیل نمومے \عددی{30} منٹ کے وقفوں پر حاصل کیے گئے۔ان کی اوسط کو نقش ضبط پر ترسیم کریں۔فرض کریں کہ آبادی عمومی ہے جس کی اوسط \عددی{5} اور معیاری انحراف \عددی{1.55} ہے۔
\begin{align*}
\begin{array}{ccccccccccc}
\text{\RL{نمونی شمار}} &1&2&3&4&5&6&7&8&9&10\\
\hline
\multirow{4}{*}{\RL{نمونی قیمتیں}} &3&3&5&7&7&4&5&6&5&5\\
&4&6&2&5&3&4&6&4&5&2\\
&8&6&5&4&6&3&4&6&6&5\\
&4&8&6&4&5&6&6&4&4&3
\end{array}
\end{align*} 
\انتہا{سوال}
%=======================
\ابتدا{سوال}\quad
سعت کے نقشہ ضبط پر سوال \حوالہ{سوال_شماریات_موٹائی_الف} کے نمونی سعت کو ترسیم کریں۔
\انتہا{سوال}
%===========================
\ابتدا{سوال}\quad
\عددی{\lambda_n=\tfrac{\sigma}{E(R^*)}} بالمقابل \عددی{n} ترسیم کریں۔\عددی{\lambda_n} متغیر \عددی{n} کا یک سر گھٹتا تفاعل ہے۔اس کی وجہ بیان کریں۔
\انتہا{سوال}
%=============================
\ابتدا{سوال}\quad
حدود ضبط کے باہر اوسط کا نقطہ نظام میں خرابی کو ظاہر کرتی ہے۔اگر ہم (الف) \عددی{1\sigma} حد، (ب) \عددی{2\sigma} حد، منتخب کریں تب ہم کتنی بار نظام میں غیر موجود خرابی کو تلاش کرنے کی کوشش کریں گے۔(عمومیت فرض کریں۔)\\
جواب:\quad
تقریباً \عددی{\SI{30}{\percent}\,\,\,(\SI{5}{\percent})} صورتوں میں
\انتہا{سوال}
%===========================
\ابتدا{سوال}\quad
ایک خود کار خراد کی مشین پر قابلے بنائے جاتے ہیں۔ مسلسل رگڑ سے پیدا تبدیلی، اوسط کی نقش ضبط پر کس طرح رونما ہو گی؟ خراد کی مشین میں یک دم تبدیلی کس طرح نقش ضبط پر نظر آئے گی؟   
\انتہا{سوال}
%==========================
\ابتدا{سوال}\quad \موٹا{(عیب داروں کی تعداد)}\quad
\عددی{3\sigma} حدود ضبط کے لحاظ سے  \عددی{\UCL}، \عددی{\textrm{CL}} اور \عددی{\LCL} کے کلیات عیب دار کے نقشہ ضبط کے لئے  تلاش کریں۔(فرض کریں کہ شماریاتی ضبط میں \عددی{p} عیب دار کو ظاہر کرتا ہے۔)\\
$\UCL=np+3\sqrt{np(1-p)},\textrm{CL}=np,\LCL=np-3\sqrt{np(1-p)}$
\انتہا{سوال}
%=======================
\ابتدا{سوال}\quad \موٹا{خاصیت کی نقش ضبط} \quad
برتنوں کی پیداوار سے جسامت \عددی{100} کے نمونے حاصل کیے گئے۔عیب دار (رستا برتنوں) کی تعداد (اسی ترتیب سے) درج ذیل تھی۔
\begin{align*}
3\,\,\,7\,\,\,6\,\,\,1\,\,\,4\,\,\,5\,\,\,4\,\,\,9\,\,\,7\,\,\,0\,\,\,5\,\,\,6\,\,\,13\,\,\,4\,\,\,9\,\,\,0\,\,\,2\,\,\,1\,\,\,12\,\,\,8
\end{align*} 
گزشتہ تجربہ سے ہم جانتے ہیں کہ اگر عمل پیداوار میں خرابی نہ ہو تب  عیب دار کی اوسط تعداد \عددی{p\si{\percent}} ہوتی ہے۔ثنائی تقسیم استعمال کرتے ہوئے عیب دار نقشہ ضبط (جس کو \عددی{p} نقشہ بھی کہتے ہیں) بنائیں، یعنی، \عددی{\LCL=0} لیں اور  \عددی{3\sigma} حدود کے لئے حاصل عیب دار (فی صد)  کو \عددی{\UCL} لیں، جہاں بلا منصوبہ متغیر \عددی{=\overline{X}}\ترچھا{نمونہ میں فی صد عیب دار} کی تغیریت \عددی{\sigma^2} ہے۔کیا عمل پیداوار قابو میں ہے؟
\انتہا{سوال}
%=========================
\ابتدا{سوال}\quad \موٹا{فی اکائی عیب دار کی تعداد} \quad
فی اکائی عیب دار کے نقشہ (جس کو \عددی{c} نقشہ بھی کہتے ہیں) کو فی اکائی عیب دار \عددی{X} (مثلاً \عددی{10} میٹر کاغذ میں عیبوں کی تعداد، جہاز کے ایک پر میں غیر موجود کیلوں کی تعداد، وغیرہ) کو قابو کرنے کے لئے استعمال کیا جاتا ہے۔ (الف)   \عددی{X} کی تقسیم کو  پوئسن تقسیم تصور کرتے ہوئے  \عددی{\mu\mp 3\sigma} کے لحاظ سے \عددی{\textrm{CL}}، \عددی{\LCL}  اور \عددی{\UCL} کے کلیات بنائیں۔ (ب) شیشے کی چادر میں عیب کے لئے \موٹا{عمل قابو}\فرہنگ{عمل قابو}\فرہنگ{قابو!عمل}\حاشیہب{control process}\فرہنگ{control process} کے لئے \عددی{\textrm{CL}}، \عددی{\LCL} اور \عددی{\UCL} تلاش کریں؛ فرض کریں کہ جب عمل پیداوار شماریاتی قابو میں ہو تب اوسطاً یہ عدد \عددی{2.5} فی چادر ہے۔ 
\انتہا{سوال}
%=========================

\حصہ{قبولیت نمونہ}
بڑے پیمانہ پر پیداوار میں، جہاں پیداکار خریدار کو \عددی{N} اشیاء کی کھیپ مہیا کرتا ہے، قبولیت نمونہ لاگو کیا جاتا ہے۔ایسی صورت میں ہر کھیپ کو قبول یا مسترد کرنے کا فیصلہ کرنا ہو گا۔  کھیپ سے \عددی{n} جسامت کے نمونے کا معائنہ کرتے ہوئے \موٹا{عیب دار اشیاء} ، جنہیں مختصراً \اصطلاح{عیب دار}\فرہنگ{عیب دار}\حاشیہب{defectives}\فرہنگ{defectives} کہتے ہیں،  کی تعداد کو مد نظر رکھ کر عموماً  فیصلہ کیا جاتا ہے۔ جو اشیاء درکار تخصیص  (بیان کردہ خواص، مثلاً جسامت، رنگ، مضبوطی، یا جو بھی اہمیت رکھتا ہو) پر پورا نہیں اترتے ہیں، انہیں عیب دار تصور کیا جاتا ہے۔ اگر نمونہ میں عیب دار اشیاء  کی تعداد \عددی{x} طے شدہ عدد \عددی{c\, (<n)} سے زیادہ نہ ہو تب کھیپ کو قبول کیا جاتا ہے۔اگر \عددی{x>c} ہو تب کھیپ کو مسترد کیا جاتا ہے۔\عددی{c} کو \ترچھا{عیب دار کی قابل قبول تعداد} یا \اصطلاح{تعداد قبولیت}\فرہنگ{تعداد قبولیت}\حاشیہب{acceptance number}\فرہنگ{acceptance number} کہتے ہیں۔ ظاہر ہے کہ پیدا کار اور خریدار کو \اصطلاح{نمونی منصوبہ}\فرہنگ{نمونی منصوبہ}\حاشیہب{sampling plan}\فرہنگ{sampling plan}  پر اتفاق کرنا ہو گا، یعنی، نمونی جسامت \عددی{n} کی قیمت  اور تعداد قبولیت \عددی{c} کی قیمت۔چونکہ یہ ایک نمونہ پر منحصر ہے لہٰذا اس  کو \اصطلاح{واحد نمونی منصوبہ}\فرہنگ{نمونی!واحد منصوبہ}\حاشیہب{single sampling plan}\فرہنگ{sampling!single plan} کہتے ہیں۔ \اصطلاح{دوہرا نمونی منصوبہ} پر بعد میں غور کیا جائے گا۔

فرض کریں کہ کھیپ قبول ہونے کا وقوعہ \عددی{A} ہے۔ظاہر ہے کہ مطابقتی احتمال \عددی{P(A)} نا صرف \عددی{n} اور \عددی{c} بلکہ کھیپ میں عیب داروں کی تعداد \عددی{M} پر بھی منحصر ہے۔فرض کریں کہ نمونہ میں عیب داروں کی تعداد بلا منصوبہ متغیر \عددی{X} ہے اور ہم بغیر واپس رکھے  نمونہ حاصل کرتے ہیں۔ تب (حصہ \حوالہ{حصہ_شماریات_ثنائی_پوئسن_بیش_ہندسی_تقسیم})
\begin{align}\label{مساوات_شماریات_نسبت_عیب_دار_الف}
P(A)=P(X\le c)=\sum_{x=0}^{c}\frac{\binom{M}{x}\binom{N-M}{n-x}}{\binom{N}{n}}
\end{align}
ہو گا۔اگر \عددی{M=0} (کھیپ میں کوئی عیب دار نہیں ہے) ہو تب \عددی{X} کی قیمت لازماً \عددی{0} ہو گی اور
\begin{align*}
P(A)=\frac{\binom{0}{0}\binom{N}{n}}{\binom{N}{n}}=1
\end{align*}
ہو گا۔مقررہ \عددی{n} اور \عددی{c} اور بڑھتے \عددی{M} کی صورت میں احتمال \عددی{P(M)} گھٹتا ہے۔اگر \عددی{M=N} (کھیپ میں تمام اشیاء عیب دار) ہو، تب \عددی{X} کی قیمت لازماً \عددی{n} ہو گی اور \عددی{P(A)=P(X\le c)=0} ہو گا چونکہ \عددی{c<n} ہے۔

نسبت \عددی{\theta=\tfrac{M}{N}} کو کھیپ میں \اصطلاح{نسبت عیب دار}\فرہنگ{عیب دار!نسبت}\حاشیہب{fraction defective}\فرہنگ{defective!fraction} کہتے ہیں۔دھیان رہے کہ \عددی{M=N\theta} ہے اور مساوات \حوالہ{مساوات_شماریات_نسبت_عیب_دار_الف} کو درج ذیل لکھا جا سکتا ہے۔
\begin{align}\label{مساوات_شماریات_نسبت_عیب_دار_ب}
P(A;\theta)=\sum_{x=0}^{c}\frac{\binom{N\theta}{x}\binom{N-N\theta}{n-x}}{\binom{N}{n}}
\end{align}
چونکہ \عددی{\theta} کی قیمت \عددی{N+1} قیمتوں \عددی{0,\tfrac{1}{N},\tfrac{2}{N},\cdots,\tfrac{N}{N}} میں سے ایک ہو سکتی ہے، احتمال \عددی{P(A)}  صرف ان قیمتوں  کے لئے  معین ہو گا۔مقررہ \عددی{n} اور \عددی{c} کے لئے ہم \عددی{P(A)} بالمقابل \عددی{\theta} ترسیم کر سکتے ہیں۔یہ \عددی{N+1} نقطے ہوں گے۔ان نقطوں سے ہموار منحنی گزاری جا سکتی ہے جس کو مد نظر نمونی منصوبہ کی \اصطلاح{منحنی خاصیت کارکردگی}\فرہنگ{منحنی!خاصیت کارکردگی}\حاشیہب{operating characteristic curve}\فرہنگ{operating characteristic curve} (\عددی{\OC})  کہتے ہیں۔

%========================
\ابتدا{مثال}\شناخت{مثال_شماریات_منحنی_خاصیت_الف}\quad
ایک مخصوص قسم کی ورموں کو \عددی{20} فی ڈبیا بند کیا جاتا ہے اور مذکورہ زیر نمونی منصوبہ استعمال کیا جاتا ہے۔\عددی{2} ورموں کا نمونہ حاصل کیا جاتا ہے اور دونوں ورمے غیر عیب دار ہونے کی صورت میں ڈبے کو قبول کیا جاتا ہے۔یہاں \عددی{N=20}، \عددی{n=2}، \عددی{c=0} ہیں لہٰذا مساوات \حوالہ{مساوات_شماریات_نسبت_عیب_دار_ب} درج ذیل صورت اختیار کرے گا۔
\begin{align*}
P(A;\theta)=\frac{\binom{20\theta}{0}\binom{20-20\theta}{2}}{\binom{20}{2}}=\frac{(20-20\theta)(19-20\theta)}{380}
\end{align*}
اعدادی قیمتیں درج ذیل ہیں۔
\begin{align*}
\begin{array}{ccccccc}
\theta&0.00&0.05&0.10&0.15&0.20&\cdots\\
\hline
P(A;\theta)&1.00&0.90&0.81&0.72&0.63&\cdots
\end{array}
\end{align*}
منحنی خاصیت کارکردگی کو شکل \حوالہ{شکل_شماریات_منحنی_خاصیت_کارکردگی} میں دکھایا گیا ہے۔
\begin{figure}
\centering
\begin{tikzpicture}
\begin{axis}[small,axis lines*=middle,xlabel={$\theta$},ylabel={$P(A;\theta)$},ylabel style={rotate=-90},ylabel style={at={(current axis.above origin)},anchor=south east},xtick={0.1,0.2,0.3,0.4,0.5,0.6,0.7,0.8,0.9,1},xticklabels={,,,,$0.5$,,,,,$1$},ytick={0.1,0.2,0.3,0.4,0.5,0.6,0.7,0.8,0.9,1},yticklabels={,,,,$0.5$,,,,,$1$}]
\addplot[domain=0:1,samples=21,mark=*,mark size=1pt] {(20-20*x)*(19-20*x)/380}%
node[pos=0.5,pin={45:\RL{مثال \حوالہ{مثال_شماریات_منحنی_خاصیت_الف}}}]{};
\addplot[domain=0:0.5] {e^(-20*x)*(1+20*x)}node[pos=0.6,shift={(45:0.3cm)},rotate=-70]{\RL{مثال \حوالہ{مثال_شماریات_منحنی_خاصیت_ب}}};
\end{axis}
\end{tikzpicture}
\caption{منحنیات خاصیت کارکردگی برائے مثال \حوالہ{مثال_شماریات_منحنی_خاصیت_الف} اور مثال \حوالہ{مثال_شماریات_منحنی_خاصیت_ب}}
\label{شکل_شماریات_منحنی_خاصیت_کارکردگی}
\end{figure}
\انتہا{مثال}
%============================
عملی صورتوں میں عموماً \عددی{\theta}  چھوٹا ہو گا (\عددی{\SI{10}{\percent}} سے کم)۔ عموماً صورتوں میں جسامت کھیپ \عددی{N} بہت بڑا (\عددی{1000}، \عددی{10000}، وغیرہ) ہو گا لہٰذا ہم مساوات \حوالہ{مساوات_شماریات_نسبت_عیب_دار_الف} اور  مساوات \حوالہ{مساوات_شماریات_نسبت_عیب_دار_ب} میں بیش ہندسی تقسیم  کو تخمیناً  ثنائی تقسیم سے ظاہر کر سکتے ہیں جس میں \عددی{p=\theta} لیا جائے گا۔اب اگر \عددی{n} ایسا ہو کہ \عددی{n\theta} معتدل (مثلاً \عددی{20} سے کم) ہو،  تب ہم اس تقسیم کو \عددی{\mu=np}  اوسط کی  پوئسن تقسیم سے ظاہر کر سکتے ہیں۔یوں مساوات \حوالہ{مساوات_شماریات_نسبت_عیب_دار_ب} سے  درج ذیل حاصل ہو گا۔
\begin{align}\label{مساوات_شماریات_نسبت_عیب_دار_پ}
P(A;\theta)\sim e^{-\mu} \sum_{x=0}^{c} \frac{\mu^x}{x!}\quad\quad\quad (\mu=n\theta)
\end{align}

%===========================
\ابتدا{مثال}\شناخت{مثال_شماریات_منحنی_خاصیت_ب}\quad
فرض کریں کہ بری کھیپ کے لئے مذکورہ ذیل واحد نمونی منصوبہ استعمال کیا جاتا ہے۔\عددی{n=20} نمونہ لیا جاتا ہے۔اگر اس میں \عددی{1} سے زیادہ عیب دار نہ ہوں تب کھیپ کو قبول کیا جاتا ہے۔اگر نمونہ میں \عددی{2} یا اس سے زیادہ عیب دار ہوں تب کھیپ کو مسترد کیا جاتا ہے۔اس منصوبہ میں مساوات \حوالہ{مساوات_شماریات_نسبت_عیب_دار_پ} درج ذیل دیتا ہے
\begin{align*}
P(A;\theta)\sim e^{-20\theta} (1+20\theta)
\end{align*}
جس کی مطابقتی منحنی شکل \حوالہ{شکل_شماریات_منحنی_خاصیت_کارکردگی} میں دکھائی گئی ہے۔
\انتہا{مثال}
%==============================

ہم اب قبولیت نمونہ میں دو اقسام کے غلطیوں پر غور کرتے ہیں اور \عددی{n} اور \عددی{c} منتخب کرنے کی تفصیل پیش کرتے ہیں۔قبولیت نمونہ میں پیداکار اور خریدار کے غرض مختلف ہوں گے۔پیداکار چاہے گا کہ "اچھی" یا "قابل قبول" کھیپ کی مسترد ہونے کا احتمال، جس کو ہم \عددی{\alpha} سے ظاہر کرتے ہیں، کم سے کم  عدد  ہو۔ خریدار چاہے گا کہ "خراب" یا "نا قابل قبول" کھیپ کے قبول ہونے کا احتمال، جس کو ہم \عددی{\beta} سے ظاہر کرتے ہیں، کم سے کم عدد ہو۔ یہ کہنا زیادہ درست ہو گا کہ دونوں اس پر اتفاق کرتے ہیں کہ جس کھیپ کے لئے \عددی{\theta} کی قیمت ایک مخصوص عدد \عددی{\theta_0} سے تجاوز نہ کرے تب کھیپ "قابل قبول"  ہو  گا جبکہ وہ کھیپ جس کے لئے \عددی{\theta}  کی قیمت ایک مخصوص عدد  \عددی{\theta_1} کے برابر یا اس  سے زیادہ ہو تب کھیپ "نا قابل قبول" ہو گا۔تب وہ کھیپ جس کے لئے \عددی{\theta\le \theta_0} ہو کے مسترد ہونے  کا احتمال \عددی{\alpha} ہو گا جس کو \اصطلاح{خطرہ پیداکار}\فرہنگ{خطرہ پیداکار}\حاشیہب{producer's risk}\فرہنگ{producer's risk} کہتے ہیں۔یہ قیاس کی پرکھ کی قسم اول غلطی  کے مترادف ہے (حصہ \حوالہ{حصہ_شماریات_قیاس_کی_پرکھ_فیصلے})۔  وہ کھیپ جس کے لئے \عددی{\theta\ge \theta_1} ہو کے قبول ہونے  کا احتمال \عددی{\beta} ہو گا جس کو \اصطلاح{خطرہ خریدار}\فرہنگ{خطرہ خریدار}\حاشیہب{consumer's risk}\فرہنگ{consumer's risk} کہتے ہیں۔یہ حصہ \حوالہ{حصہ_شماریات_قیاس_کی_پرکھ_فیصلے} میں قسم دوم  غلطی  کے مترادف ہے۔شکل میں ان کی وضاحت کی گئی ہے۔ \عددی{\theta_0} کو \اصطلاح{قابل قبول سطح معیار}\فرہنگ{معیار!قابل قبول سطح}\حاشیہب{acceptable quality level}\فرہنگ{quality!acceptable level} اور \عددی{\theta_1} کو \اصطلاح{قابل مسترد سطح معیار}\فرہنگ{معیار!قابل مسترد سطح}\حاشیہب{rejectable quality level}\فرہنگ{quality!rejectable level} کہتے ہیں جبکہ کھیپ \عددی{\theta_0<\theta<\theta_1} کو \اصطلاح{لا تعلق کھیپ}\فرہنگ{لا تعلق کھیپ}\حاشیہب{indifferent lot}\فرہنگ{indifferent lot} کہتے ہیں۔

شکل سے ہم دیکھتے ہیں کہ نقطہ \عددی{(\theta_0,1-\alpha)} اور  نقطہ \عددی{(\theta_1,\beta)} \ترچھا{منحنی خاصیت کارکردگی} پر پائے جاتے ہیں۔یہ دکھایا جا سکتا ہے کہ بڑی کھیپ کے لئے ہم \عددی{\theta_0}، \عددی{\theta_1\,(>\theta_0)}، \عددی{\alpha}، \عددی{\beta} منتخب کرتے ہوئے \عددی{n} اور \عددی{c} یوں تعین کر سکتے ہیں کہ منحنی خاصیت کارکردگی ان نقطوں کے قریب سے گزرتی ہو۔متعین \عددی{\alpha}، \عددی{\beta}، \عددی{\theta_0} اور \عددی{\theta_1} کے لئے نمونی منصوبے شائع کیے گئے ہیں۔ 

پرکھ قیاس اور معائنہ نمونہ میں  قریبی تعلق پایا جاتا ہے جس کو جدول \حوالہ{جدول_پرکھ_نمونہ_تعلق} میں دکھایا گیا ہے۔  
\begin{table}
\caption{پرکھ قیاس اور معائنہ نمونہ کا تعلق}
\label{جدول_پرکھ_نمونہ_تعلق}
\centering
\begin{tabular}{r|r}
معائنہ نمونہ& پرکھ قیاس\\
\hline
قابل قبول سطح معیار \عددی{\theta=\theta_0} & قیاس \عددی{\theta=\theta_0}\\
قابل مسترد سطح معیار \عددی{\theta=\theta_1} & متبادل \عددی{\theta=\theta_1}\\
عیب دار کی قابل قبول تعداد \عددی{c} & فاصل قیمت \عددی{c}\\
\عددی{\theta\le \theta_0} کھیپ مسترد ہونے کا  احتمال \عددی{\alpha} (خطرہ پیداکار)& قسم اول غلطی کا احتمال \عددی{\alpha} (معنی خیز سطح)\\
\عددی{\theta\ge \theta_1} کھیپ قبول ہونے کا احتمال \عددی{\beta} (خطرہ خریدار) &   قسم دوم غلطی کا احتمال \عددی{\beta}
\end{tabular}
\end{table}

نمونی عمل ازخود خریدار کو مکمل تحفظ فراہم نہیں کرتا ہے۔درحقیقت اگر پیداکار کو اجازت ہو کہ وہ خراب کھیپ کو دوبارہ قبول ہونے کے لئے پیش کرے تب آخر کار خراب کھیپ بھی قبول ہو جائیں گے۔خریدار کو اس صورت حال سے بچانے کی خاطر پیداکار اس بات سے اتفاق کر سکتا ہے  کہ مسترد کھیپ کو \اصطلاح{سدھارا}\فرہنگ{سدھارا}\حاشیہب{rectified}\فرہنگ{rectified} جائے گا یعنی اس کا \عددی{\SI{100}{\percent}} معائنہ کرتے ہوئے ہر جزو کو پرکھا جائے گا اور کھیپ میں تمام عیب دار اشیاء کی جگہ بے عیب اشیاء رکھے جائیں گے\حاشیہد{ظاہر ہے کہ اگر معائنہ سے اشیاء تباہ ہوتے  ہوں یا ہر جزو کا معائنہ کرنا اشیاء کی قیمت سے زیادہ مہنگا پڑتا ہو تب ہر جزو کے معائنے کی بجائے مسترد کھیپ کو کم دام فروخت کیا جائے گا۔}۔ فرض کریں ایک کارخانہ \عددی{100\theta\si{\percent}} عیب دار اشیاء بناتا ہے اور مسترد کھیپ کو سدھارا جاتا ہے۔ تب \عددی{N} جسامت کے \عددی{K} کھیپ میں \عددی{KN} اشیاء ہوں گے جن میں سے \عددی{KN\theta} عیب دار ہوں گے۔کھیپوں میں سے  \عددی{KP(A;\theta)} قبول کیے جائیں گے؛ ان میں کل \عددی{KPN\theta} عیب دار اجزاء ہوں گے۔مسترد اور سدھارے گئے کھیپ میں کوئی عیب دار جزو نہیں پایا جاتا ہے۔یوں سدھارنے کے بعد \عددی{K} کھیپ میں عیب دار کی تناسب
 \عددی{\tfrac{KPN\theta}{KN}=\theta P(A;\theta)} ہو گا۔ \عددی{\theta} کی اس تفاعل کو \اصطلاح{اوسط خارجی معیار}\فرہنگ{معیار!اوسط خارجی}\حاشیہب{average outgoing quality}\فرہنگ{quality!average outgoing} کہتے ہیں  جس کو \عددی{\AOQ(\theta)} سے ظاہر کیا جاتا ہے، یعنی:
\begin{align}
\AOQ(\theta)=\theta P(A;\theta)
\end{align}
 
اگر نمونی منصوبہ دیا گیا ہو تب یہ تفاعل اور \ترچھا{منحنی اوسط خارجی معیار} کو \عددی{P(A;\theta)}  اور \ترچھا{منحنی خاصیت کارکردگی} سے حاصل کیا جا سکتا ہے۔اس کی مثال شکل میں دکھائی گئی ہے۔

ظاہر ہے کہ \عددی{\AOQ(0)=0} ہو گا۔چونکہ \عددی{P(A;1)=0} ہے لہٰذا  \عددی{\AOQ(1)=0} ہو گا۔اس سے اور \عددی{\AOQ(\theta)\ge 0} سے ہم یہ نتیجہ حاصل کرتے ہیں کہ کسی \عددی{\theta=\theta^*} پر اس تفاعل کی زیادہ سے زیادہ قیمت پائی جائے گی جس کی مطابقتی قیمت \عددی{\AOQ(\theta^*)} کو \اصطلاح{اوسط خارجی حد معیار}\فرہنگ{معیار!اوسط خارجی حد}\حاشیہب{average outgoing quality limit}\فرہنگ{quality!average outgoing limit} کہتے ہیں۔ یہ خراب ترین معیار ہے جو سدھارنے کے عمل کے ساتھ قابل قبول ہو گا۔ 
