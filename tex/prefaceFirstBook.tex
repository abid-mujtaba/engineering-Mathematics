\باب{میری پہلی کتاب کا دیباچہ}
گزشتہ چند برسوں سے حکومتِ پاکستان اعلیٰ تعلیم کی طرف توجہ دے رہی ہے جس سے ملک کی تاریخ میں پہلی مرتبہ اعلیٰ تعلیمی اداروں میں تحقیق کا رجحان پیدا ہوا ہے۔امید کی جاتی ہے کہ یہ سلسلہ جاری رہے گا۔

پاکستان میں اعلیٰ تعلیم کا نظام انگریزی زبان میں رائج ہے۔دنیا میں تحقیقی کام کا بیشتر حصہ انگریزی زبان میں ہی چھپتا ہے۔انگریزی زبان میں ہر موضوع پر لاتعداد کتابیں پائی جاتی ہیں جن سے طلبہ و طالبات استفادہ کر سکتے ہیں۔

ہمارے ملک میں طلبہ و طالبات کی ایک بہت بڑی تعداد بنیادی تعلیم اردو زبان میں حاصل کرتی ہے۔ان کے لئے انگریزی زبان میں موجود مواد سے استفادہ حاصل کرنا تو ایک طرف، انگریزی زبان ازخود ایک رکاوٹ کے طور پر ان کے سامنے آتی ہے۔یہ طلبہ و طالبات ذہین ہونے کے باوجود آگے بڑھنے اور قوم و ملک کی بھر پور خدمت کرنے کے قابل نہیں رہتے۔ایسے طلبہ و طالبات کو اردو زبان میں نصاب کی اچھی کتابیں درکار ہیں۔ہم نے قومی سطح پر ایسا کرنے کی کوئی خاطر خواہ کوشش نہیں کی۔ 

میں برسوں تک اس صورت حال کی وجہ سے پریشانی کا شکار رہا۔کچھ کرنے کی نیت رکھنے کے باوجود کچھ نہ کر سکتا تھا۔میرے لئے اردو میں ایک صفحہ بھی لکھنا ناممکن تھا۔آخر کار ایک دن میں نے اپنی اس کمزوری کو کتاب نہ لکھنے کا جواز بنانے سے انکار کر دیا اور یوں یہ کتاب وجود میں آئی۔

یہ کتاب اردو زبان میں تعلیم حاصل کرنے والے طلبہ و طالبات کے لئے نہایت آسان اردو میں لکھی گئی ہے۔کوشش کی گئی ہے کہ اسکول کی سطح پر نصاب میں استعمال تکنیکی الفاظ ہی استعمال کئے جائیں۔جہاں ایسے الفاظ موجود نہ تھے وہاں روز مرہ میں استعمال ہونے والے الفاظ چنے گئے۔تکنیکی الفاظ کی چنائی کے وقت اس بات کا دہان رکھا گیا کہ ان کا استعمال دیگر مضامین میں بھی ممکن ہو۔

کتاب میں بین الاقوامی نظامِ اکائی استعمال کی گئ ہے۔اہم متغیرات کی علامتیں وہی رکھی گئی ہیں جو موجودہ نظامِ تعلیم کی نصابی کتابوں میں رائج ہیں۔یوں اردو میں لکھی اس کتاب اور انگریزی میں اسی مضمون پر لکھی کتاب پڑھنے والے طلبہ و طالبات کو ساتھ کام کرنے میں دشواری نہیں ہو گی۔ 

امید کی جاتی ہے کہ یہ کتاب ایک دن خالصتاً اردو زبان میں انجنیئرنگ کی نصابی کتاب کے طور پر استعمال کی جائے گی۔اردو زبان میں الیکٹریکل انجنیئرنگ کی مکمل نصاب کی طرف یہ پہلا قدم ہے۔ 

اس کتاب کے پڑھنے والوں سے گزارش کی جاتی ہے کہ اسے زیادہ سے زیادہ طلبہ و طالبات تک پہنچانے میں مدد دیں اور انہیں جہاں اس کتاب میں غلطی نظر آئے وہ اس کی نشاندہی میری ای-میل پر کریں۔میں ان کا نہایت شکر گزار ہوں گا۔

اس کتاب میں تمام غلطیاں مجھ سے ہی ڈلی ہیں البتہ اسے درست بنانے میں بہت لوگوں کا ہاتھ ہے۔میں ان سب کا شکریہ ادا کرتا ہوں۔ یہ سلسلہ ابھی جاری ہے اور مکمل ہونے پر ان حضرات کے تاثرات یہاں شامل کئے جائیں گے۔  

میں یہاں کامسیٹ یونیورسٹی اور ہائر ایجوکیشن کمیشن کا شکریہ ادا کرنا چاہتا ہوں جن کی وجہ سے ایسی سرگرمیاں ممکن ہوئیں۔	
\vspace{5mm}

{\raggedleft{
خالد خان یوسفزئی

28 اکتوبر 2011}}
