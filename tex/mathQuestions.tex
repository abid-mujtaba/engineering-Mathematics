\باب{سوالات}

%==============
\ابتدا{سوال}\شناخت{سوال_سادہ_اول_نادر_حل_الف}
نادر حل: بعغ اوقات سادہ تفرقی مساوات کا ایسا حل بھی پایا جاتا ہے جس کو عمومی حل سے حاصل نہیں کیا جا سکتا۔ایسیے حل کو \اصطلاح{نادر حل}\فرہنگ{نادر!حل}\فرہنگ{حل!نادر}\حاشیہب{singular solution}\فرہنگ{solution!singular} کہا جاتا ہے۔مساوات \عددی{y'^2-xy'+y=0} کا عمومی حل \عددی{y=cx-c^{2}} ہے جبکہ اس کا  نادر حل \عددی{y=\tfrac{x^2}{4}} ہے۔ ان حل کا تفرق لیتے ہوئے تفرقی مساوات میں پر کرتے ہوئے ثابت کریں کہ یہ تفرقی مساوات کے حل ہیں۔
\انتہا{سوال}
%=================
