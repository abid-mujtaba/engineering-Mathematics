\باب{سمتیات عارضی باب}
%+++++++++++++++++++++++++++++++++++++++++++++++++++++++++++++++++
%++++++++++++++++++++++++++++++++++++++++++++++++++++++++++++++++
this is sec 9.1 to 9.4 of the latest addition. i shall palce it at the very beginning of my 7th chapter.this resolves all the issues.
%++++++++++++++++++++++++++++++++++++++++++++++++++++++++
%++++++++++++++++++++++++++++++++++++++++++++++++++++++++++
%++++++++++++++++++++++++++++++++++++++++++++++++++++++
%++++++++++++++++++++++++++++++++++++++++++++++++++++++++++++++++
\حصہ{غیر سمتیات اور سمتیات}
طبیعیات اور جیومیٹری میں ایسی قیمتیں پائی جاتی ہیں جنہیں ان کی مقدار سے مکمل طور پر بیان کیا جا سکتا ہے۔مثلاً  کمیت، درجہ حرارت، برقی بار، وقت، رقبہ، حجم، فاصلہ، برقی دباو وغیرہ۔ان میں سے ہر ایک کو (مقدار کی موزوں اکائی چن کر) ایک عدد  سے ظاہر کیا جا سکتا ہے۔ ایسی تمام مقداروں کو \اصطلاح{غیر سمتیات}\فرہنگ{غیر سمتیات}\فرہنگ{سمتیہ!غیر}\حاشیہب{scalars}\فرہنگ{scalars} کہتے ہیں۔غیر سمتی مقدار کی قیمت پر چننی گئی محدد کا کوئی اثر نہیں ہو گا۔

اس کے برعکس طبیعیات اور جیومیٹری میں ایسی قیمتیں بھی پائی جاتی ہیں جن کی مکمل اظہار کے لئے ان کی قیمت کے علاوہ ان کی سمت بھی درکار ہوتی ہے۔ان کی ایک مثال میکانی  قوت ہے۔ آپ جانتے ہیں کہ قوت کو تیر کی نشان سے ظاہر کیا جا سکتا ہے جہاں تیر کی سمت، قوت کی سمت اور تیر کی لمبائی (کسی پیمائش کے تحت) قوت کی مقدار کو ظاہر کرتی ہے۔ شکل \حوالہ{شکل_الجبرا_قوت_سمتی_رفتار}-الف میں ہلکے دھاگے سے بندھی ہوئی کمیت \عددی{m} کی دائری حرکت دکھائی گئی ہے۔کمیت کی لمحاتی سمتی رفتار \عددی{\bM{v}} کو تیر سے دکھایا گیا ہے۔اس تیر کی سمت، کمیت کی لمحاتی سمتی رفتار دیتی ہے جبکہ تیر کی لمبائی (کسی موزوں تناسب سے) لمحاتی سمتی رفتار کی قیمت دیتی ہے۔شکل میں کمیت کی اسراع \عددی{\bM{a}} بھی دکھائی گئی ہے جہاں \عددی{\bM{a}} کی لمبائی (کسی موزوں تناسب سے) لمحاتی اسراع کی قیمت دیتی ہے۔

شکل \حوالہ{شکل_الجبرا_قوت_سمتی_رفتار}-ب میں سیدھی سطح پر رہتے ہوئے تکون کی (بلا گھومے) منتقلی دکھائی گئی ہے۔اس حرکت کو (تکون کے ہر نقطے کی) طے فاصلے  کی مقدار اور سمت سے ظاہر کیا جا سکتا ہے۔تکون پر کسی نقطے کی ابتدائی مقام \عددی{A} سے اختتامی مقام \عددی{B} تک تیر کی نشان سے اس حرکت کو ظاہر کیا جا سکتا ہے۔


اس سے سمتیہ کی درج ذیل تعریف بیان کی جا سکتی ہے۔
\begin{figure}
\centering
\begin{subfigure}{0.5\textwidth}
\centering
\begin{tikzpicture}
\draw[dashed] (0,0) circle (1cm);
\draw[gray](0,0)--++(30:1cm)node[circ]{}node[right,color=black]{\RL{کمیت $m$}};
\draw[-latex](30:1cm)--++(120:0.75cm)node[above]{$\bM{v}$};
\draw[-latex](30:1cm)--++(30:-0.5cm)node[below]{$\bM{a}$};
\end{tikzpicture}
\caption*{(الف) سمتی رفتار اور اسراع۔}
\end{subfigure}%
\begin{subfigure}{0.5\textwidth}
\centering
\begin{tikzpicture}
\draw(0,0)--++(1.5,0)--++(0,1)--++(-1.5,-1);
\draw(15:3)--++(1.5,0)--++(0,1)--++(-1.5,-1);
\draw[-latex](1.5,0.25)node[ocirc]{}node[below right]{$A$}--++(15:3)node[ocirc]{}node[below right]{$B$}node[pos=0.4,below]{$\bM{b}$};
\end{tikzpicture}
\caption*{(ب) سمتیہ کی دم اور  سر۔}
\end{subfigure}%
\caption{سمتیہ کی تفصیل۔}
\label{شکل_الجبرا_قوت_سمتی_رفتار}
\end{figure}
%=================
\ابتدا{تعریف}\quad سمتیہ\\
سمتی خط کو \اصطلاح{سمتیہ}\فرہنگ{سمتیہ}\حاشیہب{vector}\فرہنگ{vector} کہتے ہیں۔اس کی لمبائی کو سمتیہ کی \اصطلاح{لمبائی} اور سمت کو سمتیہ کی \اصطلاح{سمت} کہتے ہیں۔دو سمتیات صرف اور صرف اس صورت ایک دوسرے کے برابر ہوں گے جب ان کی لمبائی ایک جیسی ہو اور ان کی سمت ایک جیسی ہو۔

سمتیہ کی لمبائی کو سمتیہ کی \اصطلاح{اقلیدسی معیار}\فرہنگ{اقلیدسی معیار}\حاشیہب{Euclidean norm}\فرہنگ{Euclidean norm} (یا معیار) اور سمتیہ کی \اصطلاح{مقدار}\فرہنگ{مقدار!سمتیہ}\فرہنگ{سمتیہ!مقدار}\حاشیہب{magnitude}\فرہنگ{magnitude} بھی کہتے ہیں۔
\انتہا{تعریف}  
%=========================

سمتیہ کی ابتدائی نقطے کو سمتیہ کی \اصطلاح{دم}\فرہنگ{دم}\فرہنگ{سمتیہ!دم}\حاشیہب{tail}\فرہنگ{tail} اور اختتامی نقطے کو سمتیہ کا \اصطلاح{سر}\فرہنگ{سر}\فرہنگ{سمتیہ!سر}\حاشیہب{head}\فرہنگ{head} کہتے ہیں۔ یوں شکل \حوالہ{شکل_الجبرا_قوت_سمتی_رفتار}-ب میں نقطہ \عددی{B} سمتیہ \عددی{\bM{b}} کی دم  ہے جبکہ نقطہ \عددی{A} اس کا سر  ہے۔

ہم سمتیات کو موٹی لکھائی میں چھوٹی حروف تہجی مثلاً \عددی{\bM{a}}، \عددی{\bM{b}}، \عددی{\bM{v}}، وغیرہ،  سے ظاہر کرتے ہیں۔ قلم و کاغذ استعمال کرتے ہوئے سمتیہ پر تیر  یا آدھے تیر کا نشان بنایا جاتا ہے یوں اسراع کو \عددیء{\vec{a}} یا  
$\overset{\rightharpoonup}{\rule{0pt}{.9ex}\smash{a}}$
لکھا جاتا ہے۔سمتیہ \عددی{\bM{a}} کی مقدار کو \عددی{\abs{\bM{a}}} لکھا جاتا ہے۔

سمتیہ کی تعریف سے ظاہر ہے کہ ہم سمتیہ کو بغیر گھمائے  ایک جگہ سے دوسری جگہ منتقل کر سکتے ہیں\حاشیہد{یہاں یہ بتلانا ضروری ہے کہ طبیعیات اور جیومیٹری میں ایسی صورتیں پائی جاتی ہیں جہاں سمتیہ  کو ایک جگہ سے دوسری  جگہ منتقل کرنا ممکن نہیں ہوتا ہے۔آپ میکانیات سے جانتے ہیں کہ کسی بھی غیر لچکدار مادے پر قوت کا اطلاق، قوت کی سمت میں لکیر پر رہتے ہوئے،  کسی بھی نقطے پر کیا جا سکتا ہے۔اس سے \اصطلاح{قابل منتقلی سمتیہ}\فرہنگ{سمتیہ!قابل منتقلی}\فرہنگ{vector!sliding} کا تصور پیدا ہوتا ہے۔اس کے برعکس، لچکدار مادے پر قوت کے اطلاق کا نقطہ تبدیل کرنے سے نتائج تبدیل ہوں گے  جو نا قابل قبول بات ہے۔یہ حقیقت \اصطلاح{مقید سمتیہ}\فرہنگ{مقید!سمتیہ}\فرہنگ{سمتیہ!مقید}\فرہنگ{bound vector}\فرہنگ{vector!bound} کی تصور کو جنم دیتی ہے۔اس کتاب میں صرف قابل منتقلی سمتیات پر بات کی جائے گی۔} یعنی ہم سمتیہ کی دم کہیں پر بھی منتقل کر سکتے ہیں۔ظاہر ہے کہ سمتیہ کی دم کا مقام مقرر کرنے سے اس کی سر کا مقام بھی مقرر ہو گا۔

اگر دو سمتیات \عددی{\bM{a}} اور \عددی{\bM{b}} ایک دوسرے کے برابر ہوں تب ہم درج ذیل لکھتے ہیں
\begin{align}
\bM{a}=\bM{b}
\end{align}
اور اگر یہ آپس میں برابر نہ ہوں تب ہم درج ذیل لکھتے ہیں۔
\begin{align}
\bM{a}\ne\bM{b}
\end{align}
کسی بھی سمتیہ کو ترسیمی طور پر موزوں  لمبائی اور سمت کی سمتی خط سے ظاہر کیا جا سکتا ہے۔

ایسا سمتیہ جس کی لمبائی اکائی \عددی{(1)} ہو \اصطلاح{اکائی سمتیہ}\فرہنگ{اکائی!سمتیہ}\فرہنگ{سمتیہ!اکائی}\حاشیہب{unit vector}\فرہنگ{vector!unit}\فرہنگ{unit!vector} کہلاتا ہے۔

%======================
\حصہ{سمتیہ کے اجزاء}

