\باب{حافظ زاویہ نقشہ کشی}
اگر \عددی{z} سطح میں دائرہ کار \عددی{D} میں مخلوط تفاعل \عددی{w=f(z)} معین ہو، تب \عددی{D} میں ہر نقطہ کا مطابقتی نقطہ \عددی{w} سطح میں پایا جاتا ہے۔یوں \عددی{D}
 کا مطابقتی، \عددی{f(z)} کے حلقہ کا نقشہ، \عددی{w} سطح پر حاصل ہو گا۔جیومیٹریائی نقشہ ذہن میں تفاعل کی تصویر قائم کرتا ہے۔مختلف منحنیات اور خطوں کے نقوش دیکھ کر مخلوط تفاعل سمجھنے میں مدد ملتی ہے۔

جیسا ہم دیکھیں گے، اگر \عددی{f(z)} \ترچھا{تحلیلی} ہو تب \عددی{f(z)} سے حاصل نقشے میں زاویے تبدیل نہیں ہوں گے ماسوائے ان نقطوں پر جہاں \عددی{f'(z)=0} ہو۔ایسا نقشہ \اصطلاح{حافظ زاویہ نقشہ}\فرہنگ{حافظ زاویہ نقشہ}\فرہنگ{نقشہ!حافظ زاویہ}\حاشیہب{conformal map}\فرہنگ{conformal!map}\فرہنگ{mapping!conformal} کہلاتا ہے۔ 

\اصطلاح{حافظ زاویہ نقشہ گشی}\فرہنگ{حافظ زاویہ نقشہ کشی}\حاشیہب{conformal mapping}\فرہنگ{conformal!mapping} کے ذریعہ دیے گیے پیچیدہ خطے کا تبادل سادہ خطے میں کرتے ہوئے نظریہ مخفی قوہ کی دو بعدی سرحدی مسائل حل کیے جاتے ہیں۔اسی وجہ سے حافظ زاویہ نقشہ گشی انجینئری میں اہمیت رکھتی ہے۔

ہم نقشہ گشی کی تعریف پیش کرنے کے بعد نقشہ گشی کا عمل سکھائیں گے۔اس کے بعد کئی بنیادی تحلیلی تفاعل  کے نقوش پیش کریں گے۔

\حصہ{نقشہ گشی}
حقیقی متغیرہ \عددی{x} کے حقیقی تفاعل \عددی{y=f(x)} کی منحنی  کو کارتیسی \عددی{xy} سطح پر  کھینچا جا سکتا ہے۔اس خط کو تفاعل کی \ترچھا{ترسیم} کہتے ہیں۔چونکہ مخلوط متغیرہ \عددی{z} کو جیومیٹریائی طور پر مخلوط سطح میں نقاط سے ظاہر کیا جاتا ہے اور یہی کچھ  \عددی{w} کے لئے بھی درست ہے لہٰذا مخلوط تفاعل
\begin{align}
w=f(z)=u(x,y)+iv(x,y)\quad \quad \quad (z=x+iy)
\end{align}
کی صورت حال زیادہ پیچیدہ ہے۔اس سے ہمیں خیال آتا ہے کہ ہم ان دو متغیرات کے لئے دو علیحدہ علیحدہ مخلوط سطحیں استعمال کریں۔ایک \عددی{z} سطح جس میں \عددی{z=x+iy} دکھایا جائے اور دوسری \عددی{w} سطح جس میں مطابقتی \عددی{w=u+iv} دکھایا جائے۔یوں  \عددی{f(z)} کی دائرہ کار \عددی{D} میں ہر \عددی{z}  کے لئے تفاعل \عددی{f(z)}  سطح \عددی{w}   میں قیمت \عددی{w=f(z)}  مختص کرے گا۔اس معین تعلق کو \عددی{f} کی دائرہ کار کی سطح \عددی{w} \اصطلاح{میں}\فرہنگ{میں}\حاشیہب{into}\فرہنگ{into}  \اصطلاح{نقشہ گشی}\فرہنگ{نقشہ گشی}\حاشیہب{mapping}\فرہنگ{mapping} (یا \ترچھا{تبادل}) کہتے ہیں، یا \عددی{f} کی دائرہ کار کا \عددی{f} کے حلقہ \اصطلاح{پر}\فرہنگ{پر}\حاشیہب{onto}\فرہنگ{onto} نقشہ کشی کہتے ہیں۔

\عددی{w_0=f(z_0)} جو نقطہ \عددی{z_0} کا مطابقتی نقطہ ہے،  \عددی{f(z)} کے لحاظ سے نقشے میں، \عددی{z_0} کا  \اصطلاح{عکس نقطہ}\فرہنگ{عکس نقطہ}\حاشیہب{image point}\فرہنگ{image!point} یا \اصطلاح{عکس}\فرہنگ{عکس}\حاشیہب{image}\فرہنگ{image} کہلاتا ہے۔ اگر \عددی{z} کسی منحنی پر حرکت کرے اور \عددی{f(z)} استمراری (نا کہ مستقل)  ہو تب مطابقتی نقطہ \عددی{w=f(z)} عمومی طور پر سطح \عددی{w} میں منحنی \عددی{C^*} پر حرکت کرے گا۔اس منحنی کو منحنی \عددی{C} کا عکس کہیں گے۔لفظ "عکس" کسی بھی نقطوں کے سلسلے اور خطہ کے لئے بھی استعمال کیا جاتا ہے۔ 

ہم دیکھیں گے کہ ایسی نقشہ کشی کی خواص کی تفتیش،  \عددی{z} سطح میں منحنیات اور خطے اور  \عددی{w} سطح میں ان کے عکس پر غور اور \عددی{w} سطح میں منحنیات اور خطے اور  \عددی{z} سطح میں ان کے عکس پر غور  کرنے سے  کی جا سکتی ہے۔ اس طرح  انفرادی نقطوں پر غور کرنے سے حاصل معلومات سے  زیادہ معلومات حاصل ہو گی۔

اگرچہ  \عددی{w} اور \عددی{z} کو دو علیحدہ علیحدہ سطحوں سے ظاہر کیا جاتا ہے، بعض اوقات یوں  سوچنا زیادہ بہتر ثابت ہوتا ہے  کہ اصل اور نقش ایک ہی سطح پر پائے جاتے ہوں اور  عمومی اصطلاحات مثلاً " گھومنا" اور "مستقیم حرکت" استعمال کرنا۔یوں \عددی{w=z+3}   مستقیم حرکت کہلائے گی جو \عددی{z} سطح میں ہر نقطہ کو دائیں جانب تین اکایاں منتقل کرتی ہے۔

تحلیل تفاعل \عددی{w=u+iv=f(z)} جس نقشہ کو ظاہر کرتا ہو، کی کسی مخصوص  خاصیت  جاننے کے لئے ہم \عددی{z} سطح میں سیدھے لکیروں \عددی{x=\text{مستقل}} اور \عددی{y=\text{مستقل}} کا \عددی{w} سطح میں عکس پر غور کر سکتے ہیں۔اسی طرح ہم دائرہ \عددی{\abs{z}=\text{مستقل}} یا مبدا سے گزرتی سیدھی لکیروں کی عکس پر غور کر سکتے ہیں۔اس کے علاوہ ہم \عددی{u(x,y)=\text{مستقل}} اور \عددی{v(x,y)=\text{مستقل}} منحنیات پر \عددی{z} سطح میں غور کر سکتے ہیں۔ان منحنیات کو \عددی{u} اور \عددی{v} کی \اصطلاح{ہموار منحنیات}\فرہنگ{ہموار!منحنی}\حاشیہب{level curves}\فرہنگ{curve!level} کہتے ہیں۔ ہم سادہ اشکال مثلاً چکور، تکون، مستطیل وغیرہ اور ان کے عکس پر بھی غور کر سکتے ہیں۔

آئیں چند مثالوں کی مدد سے ان حقائق کو بہتر سمجھنے کی کوشش کرتے ہیں۔

%====================
\ابتدا{مثال}\شناخت{مثال_نقش_خطی_مستقیم}\quad \موٹا{خطی تبادل \عددی{w=ax+b}}\\
درج ذیل نقش \اصطلاح{مستقیم حرکت} کو ظاہر کرتا ہے۔
\begin{align}\label{مساوات_نقش_خطی_مستقیم_الف}
w=z+b
\end{align}
شکل \حوالہ{شکل_مثال_نقش_خطی_مستقیم} میں مساوات \حوالہ{مساوات_نقش_خطی_مستقیم_الف} کو \عددی{w=z+2+i} کے لئے دکھایا گیا ہے جہاں مستطیل اور اس کا عکس دکھائے گئے ہیں جو یکساں ہیں (کیوں؟)۔\عددی{A} کا عکس \عددی{A^*}، وغیرہ۔زیادہ پیچیدہ اشکال میں نقطوں کو اس طرح ظاہر کرنا مفید ثابت ہوتا ہے۔ مساوات \حوالہ{مساوات_نقش_خطی_مستقیم_الف} میں \عددی{b=0} پر کرنے سے \اصطلاح{مماثل تبادل}\فرہنگ{تبادل!مماثل}\حاشیہب{identity transformation}\فرہنگ{transformation!identity} 
\begin{align*}
w=z
\end{align*}
حاصل ہوتا ہے جو ہر نقطے کو اپنے آپ پر نقش کرتا ہے۔
%
\begin{figure}
\centering
\begin{tikzpicture}
\draw(0,0)--++(3,0)node[below]{$x$};
\draw(0,0)node[below]{$0$}--++(0,2)node[right]{$y$};
\draw(1,0.5)node[left]{$A$}--++(0.5,0)node[right]{$B$}--++(0,1)node[right]{$C$}--++(-0.5,0)node[left]{$D$}--++(0,-1);
\foreach \x in {0.5,1,1.5,2,2.5}{\draw (\x,0)--++(0,0.1);}
\foreach \y in {0.5,1,1.5}{\draw (0,\y)--++(0.1,0);}
\draw(2,0)node[below]{$4$};
\draw(0,1)node[left]{$2$};
\draw(1.5,-0.5)node[below]{\text{\RL{($\,z$ مستوی)}}};
\begin{scope}[shift={(4cm,0)}]
\draw(0,0)--++(3,0)node[below]{$u$};
\draw(0,0)node[below]{$0$}--++(0,2)node[right]{$v$};
\draw(2,1)node[left]{$A^*$}--++(0.5,0)node[right]{$B^*$}--++(0,1)node[right]{$C^*$}--++(-0.5,0)node[left]{$D^*$}--++(0,-1);
\foreach \x in {0.5,1,1.5,2,2.5}{\draw (\x,0)--++(0,0.1);}
\foreach \y in {0.5,1,1.5}{\draw (0,\y)--++(0.1,0);}
\draw(2,0)node[below]{$4$};
\draw(0,1)node[left]{$2$};
\draw(1.5,-0.5)node[below]{\text{\RL{($\,w$ مستوی)}}};
\end{scope}
\end{tikzpicture}
\caption{مستقیم حرکت \عددی{w=z+2+i}}
\label{شکل_مثال_نقش_خطی_مستقیم}
\end{figure}

درج ذیل تبادل
\begin{align*}
w=az\quad \quad \quad (\abs{a}=1)
\end{align*}
مقررہ زاویہ \عددی{\phase{a}} سے گھومنے کو ظاہر کرتا ہے۔شکل \حوالہ{شکل_نقش_گھومنا} میں \عددی{w=iz} یعنی  گھڑی کی سوئیوں کی گھومنے کی الٹ رخ \عددی{\tfrac{\pi}{2}} زاویہ سے گھومنا دکھایا گیا ہے۔
\begin{figure}
\centering
\begin{tikzpicture}
\draw(0,0)--++(3,0)node[below]{$x$};
\draw(0,0)node[below]{$0$}--++(0,2.5)node[right]{$y$};
\draw([shift={(0:1)}]0,0) arc (0:90:1);
\draw([shift={(0:2)}]0,0) arc (0:90:2);
\draw[dashed](0,0)--++(30:2.5);
\draw[dashed](0,0)--++(60:2.5);
\foreach \x in {0.5,1,1.5,2,2.5}{\draw(\x,0)--++(0,0.1);}
\foreach \y in {0.5,1,1.5,2}{\draw(0,\y)--++(0.1,0);}
\draw(2.5,0)node[below]{$5$};
\draw(0,1.5)node[left]{$3$};
\draw(1,0)node[below]{$A$};
\draw(2,0)node[below]{$B$};
\draw(0,1)node[left]{$C$};
\draw(0,2)node[left]{$D$};
\draw(1.5,-0.5)node[below]{\text{\RL{($\,z$ مستوی)}}};
%
\begin{scope}[xshift={8cm}]
\draw(-3,0)--(0.75,0)node[below]{$u$};
\draw(0,0)node[below]{$0$}--++(0,2.5)node[right]{$v$};
\draw([shift={(90:1)}]0,0) arc (90:180:1);
\draw([shift={(90:2)}]0,0) arc (90:180:2);
\draw[dashed](0,0)--++(120:2.5);
\draw[dashed](0,0)--++(150:2.5);
\foreach \x in {0.5,-0.5,-1,-1.5,-2,-2.5}{\draw(\x,0)--++(0,0.1);}
\foreach \y in {0.5,1,1.5,2}{\draw(0,\y)--++(0.1,0);}
\draw(-2.5,0.1)node[above]{$-5$};
\draw(0.1,1.5)node[right]{$3$};
\draw(0,1)node[right]{$A^*$};
\draw(0,2)node[right]{$B^*$};
\draw(-1,0)node[below]{$C^*$};
\draw(-2,0)node[below]{$D^*$};
\draw(-1.5,-0.5)node[below]{\text{\RL{($\,w$ مستوی)}}};
\end{scope}
\end{tikzpicture}
\caption{گھڑی کی الٹ رخ گھومنے کا زاویہ \عددی{\tfrac{\pi}{2}} ہے۔}
\label{شکل_نقش_گھومنا}
\end{figure}

درج ذیل تبادل
\begin{align*}
w=az \quad \quad \quad (\text{\RL{مثبت حقیقی $a$}})
\end{align*}
میں \عددی{a>1} اتساع جبکہ \عددی{0<a<1} سکڑاو کو ظاہر کرتا ہے۔اسی طرح 
\begin{align}
w=az\quad \quad \quad (\text{\RL{اختیاری $a$}})
\end{align}
زاویہ \عددی{\phase{a}} سے گھومنے کو اور ساتھ ہی یکساں اتساع یا سکڑاو کو ظاہر کرتا ہے۔ درج ذیل تبادل
\begin{align}
w=az+b
\end{align}
\اصطلاح{خطی تبادل}\فرہنگ{تبادل!خطی}\حاشیہب{linear transformation}\فرہنگ{transformation!linear} کہلاتا ہے جو گھومنے کے ساتھ اتساع یا سکڑاو \عددی{w_1=az} کے ساتھ ساتھ مستقیم حرکت \عددی{w=w_1+b} کو ظاہر کرتا ہے۔ شکل \حوالہ{شکل_نقش_خطی_تبادل} میں \عددی{w=(1+i)z+2i} تبادل دکھایا گیا ہے جو گھڑی کی الٹ رخ  \عددی{\tfrac{\pi}{4}} زاویے کے گھومنے اور \عددی{\abs{1+i}=\sqrt{2}} تناسب کی اتساع کے بعد اوپر کی رخ مستقیم حرکت کو ظاہر کرتا ہے۔  
\begin{figure}
\centering
\begin{tikzpicture}
\draw(0,0)--++(2,0)node[below]{$x$};
\draw(0,0)--++(0,2)node[left]{$y$};
\draw([shift={(0:0.5)}]0,0) arc (0:90:0.5);
\draw([shift={(0:1)}]0,0) arc (0:90:1);
\draw([shift={(0:1.5)}]0,0) arc (0:90:1.5);
\draw(0.5,0)node[below]{$1$};
\draw(1,0)node[below]{$2$};
\draw(1.5,0)node[below]{$3$};
\draw(0,0.5)node[left]{$1$};
\draw(0,1)node[left]{$2$};
\draw(0,1.5)node[left]{$3$};
\draw[thick](0,0)--++(1.5,0)node[shift={(0.2,0.2)}]{$B$};
\draw[thick](0,0)node[shift={(0.2,0.2)}]{$A$}--++(0,1.5)node[shift={(0.2,0.2)}]{$C$};
\draw(1,-1)node[]{$z=x+iy$};
\draw(1,-1.5)node[]{\text{\RL{($\,z$ مستوی)}}};
\begin{scope}[xshift={(4cm)}]
\draw(-1.5,0)--(1.5,0)node[below]{$u_1$};
\draw(0,0)--++(0,2.75)node[right]{$v_1$};
\draw([shift={(45:0.707)}]0,0) arc (45:135:0.707);
\draw([shift={(45:1.4142)}]0,0) arc (45:135:1.4142);
\draw([shift={(45:2.121)}]0,0) arc (45:135:2.121);
\draw(0,0)node[below]{$0$};
\draw[thick](0,0)node[shift={(0.5,0.2)}]{$A^*$}--++(45:2.212)node[right]{$B^*$};
\draw[thick](0,0)--++(135:2.212)node[left]{$C^*$};
\foreach \x in {-1,-0.5,1}{\draw(\x,0)--++(0,0.1);}
\foreach \y in {0.5,1,1.5,2,2.5}{\draw(0,\y)--++(0.1,0);}
\draw(0,2.5)node[left]{$5$};
\draw(1,-1)node[]{$w_1=u_1+iv_1$};
\draw(1,-1.5)node[]{\text{\RL{($\,w_1$ مستوی)}}};
\end{scope}
\begin{scope}[xshift={(8cm)}]
\draw(-1.5,0)--(1.5,0)node[below]{$u$};
\draw(0,0)--++(0,3.75)node[right]{$v$};
\draw([shift={(45:0.707)}]0,1) arc (45:135:0.707);
\draw([shift={(45:1.4142)}]0,1) arc (45:135:1.4142);
\draw([shift={(45:2.121)}]0,1) arc (45:135:2.121);
\draw(0,0)node[below]{$0$};
\draw[thick](0,1)node[shift={(0.5,0.2)}]{$A^{**}$}--++(45:2.212)node[right]{$B^{**}$};
\draw[thick](0,1)--++(135:2.212)node[left]{$C^{**}$};
\foreach \x in {-1,-0.5,0.5,1}{\draw(\x,0)--++(0,0.1);}
\foreach \y in {0.5,1,1.5,2,2.5,3.5}{\draw(0,\y)--++(0.1,0);}
\draw(0,2.6)node[left]{$5$};
\draw(1,-1)node[]{$w=u+iv$};
\draw(1,-1.5)node[]{\text{\RL{($\,w$ مستوی)}}};
\end{scope}
\end{tikzpicture}
\caption{خطی تبادل $w=(1+i)z+2i$ جس میں  گھومنا،  اتساع $w_1=(1+i)z$ اور مستقیم حرکت $w=w_1+2i$ شامل ہے۔}
\label{شکل_نقش_خطی_تبادل}
\end{figure}
\انتہا{مثال}
%========================
\ابتدا{مثال}\quad \موٹا{نقش \عددی{w=z^2}}\\

\انتہا{مثال}
%===========================
