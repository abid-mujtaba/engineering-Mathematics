\حصہ{تہرا تکمل۔ گاوس کا مسئلہ پھیلاو}
دہرا تکمل (حصہ \حوالہ{حصہ_خطی_تکمل_دوہرا_تکمل}) کی تصور کو وسعت دیتے ہوئے تہرا تکمل حاصل کیا جا سکتا ہے۔ فرض کریں کہ فضا کے کسی بند محدود\حاشیہد{"بند" سے مراد ہے کہ وقفے کی سرحد بھی وقفے کا حصہ ہے اور "محدود" سے مراد ہے کہ پورے وقفے کو معقول وسعت کی کرہ میں گھیرا جا سکتا ہے۔} خطہ \عددی{T} میں تفاعل \عددی{f(x,y,z)} معین ہے۔ہم تینوں محور کے متوازی سطحوں سے \عددی{T} کو ٹکڑوں میں تقسیم کرتے ہیں۔ہم \عددی{T} کے متوازی السطوح ٹکڑوں  کو ہم \عددی{1} تا \عددی{n} سے ظاہر  کرتے ہیں۔ایسے ہر ٹکڑے کے اندر ہم بے قاعدگی سے کوئی نقطہ منتخب کرتے ہیں، مثلاً ٹکڑا \عددی{k} میں نقطہ  \عددی{(x_k,y_k,z_k)} چنا جاتا ہے، اور درج ذیل مجموعہ حاصل کرتے ہیں
\begin{align*}
J_n=\sum_{k=1}^{n} f(x_k,y_k,z_k)\Delta H
\end{align*} 
جہاں ٹکڑا \عددی{k} کی حجم \عددی{\Delta H_k} ہے۔ہم مثبت عدد صحیح \عددی{n} کی قیمت بتدریج بڑھاتے ہوئے  بالکل آزادانہ طریقے سے اس طرح کے مجموعے حاصل کرتے ہیں پس اتنا خیال رکھا جاتا ہے کہ جیسے جیسے \عددی{n} کی قیمت لامتناہی کے قریب پہنچتی ہو، مستطیلی ٹکڑوں کی وتر کی زیادہ سے زیادہ لمبائی صفر تک پہنچتی ہو۔  یوں ہمیں حقیقی اعداد \عددی{J_{n1}}، \عددی{J_{n2}}، \نقطے کا سلسلہ حاصل ہو گا۔ہم فرض کرتے ہیں کہ کسی  ایسے خطہ میں، جس کا \عددی{T} حصہ ہو، \عددی{f(x,y,z)} استمراری ہے اور \عددی{T} کو لامتناہی تعداد  کی ہموار سطحیں گھیرتی ہیں۔ایسی صورت میں یہ ثابت کیا جا سکتا ہے کہ  حقیقی اعداد \عددی{J_{n1}}، \عددی{J_{n2}}، \نقطے کا سلسلہ مرتکز ہو گا جس کا حد ٹکڑوں کی چنائی یا ٹکڑوں میں نقطوں \عددی{(x_,y_k,z_k)} کی چنائی سے بالکل آزاد ہو گا (مثال \حوالہ{مثال_سمتی_تکمل_رقبہ_اور_تکمل} کی طرح)۔ اس حد کو خطہ \عددی{T} پر \عددی{f(x,y,z)} کا \اصطلاح{تہرا تکمل}\فرہنگ{تہرا!تکمل}\فرہنگ{تکمل!تہرا}\حاشیہب{triple integral}\فرہنگ{integral!triple} کہتے ہیں جس کو درج ذیل سے ظاہر کیا جاتا ہے۔
\begin{align*}
\iiint\limits_T f(x,y,z)\dif x\dif y\dif z \quad \text{یا}\quad \iiint\limits_T f(x,y,z)\dif H
\end{align*}

ہم اب  ثاب کرتے ہیں کہ  ایسا استمراری سمتی تفاعل \عددی{\bM{u}} جس کے استمراری ایک درجی جزوی تفرق پائے جاتے ہوں کی پھیلاو  کا فضا میں خطہ \عددی{T} پر تہرا تکمل کا تبادلہ \عددی{T} کی سطح پر \عددی{\bM{u}}  کے عمودی  جزو کی سطحی تکمل میں کیا جا سکتا ہے۔ایسا مسئلہ پھیلاو کی مدد سے کیا جاتا ہے جو دو بعدی مسئلہ گرین کا  تین بعدی مماثل ہے۔ مسئلہ پھیلاو کئی نظریاتی اور عملی مسائل میں بنیادی اہمیت رکھتی ہے۔

%============================
\ابتدا{مسئلہ}\شناخت{مسئلہ_خطی_تکمل_مسئلہ_پھیلاو}\quad گاوس کا مسئلہ پھیلاو (حجمی تکمل سے سطحی تکمل اور سطحی تکمل سے حجمی تکمل کا حصول)\\
فرض کریں کہ فضا میں بند محدود خطہ \عددی{T} کی سرحد \عددی{S} ٹکڑوں میں ہموار (حصہ \حوالہ{حصہ_خطی_تکمل_سطحیں}) اور قابل سمت بند ہے۔مزید فرض کریں کہ خطہ \عددی{T} میں \عددی{\bM{u}(x,y,z)} ایک استمراری سمتی تفاعل ہے جس کے \عددی{T} میں استمراری ایک درجی جزوی تفرق پائے جاتے ہیں۔ تب درج ذیل ہو گا
\begin{align}\label{مساوات_خطی_تکمل_مسئلہ_گاوس_الف}
\iiint\limits_T \nabla \cdot \bM{u}\dif H=\iint\limits_S u_n\dif A
\end{align}
جہاں \عددی{T} کی لحاظ سے سطح \عددی{S} پر \عددی{\bM{u}} کا باہر رخ عمودی جزو
\begin{align}\label{مساوات_خطی_تکمل_مسئلہ_گاوس_ب}
u_n=\bM{u}\cdot \bM{n}
\end{align}
ہے اور \عددی{\bM{n}} سطح \عددی{S} کا باہر رخ  اکائی عمودی سمتیہ ہے۔
\انتہا{مسئلہ}
%============================
\ابتدا{ثبوت}
ہم \عددی{\bM{u}} اور \عددی{\bM{n}} کو ارکان کی صورت میں لکھتے ہیں
\begin{align*}
\bM{u}=u_1\bM{i}+u_2\bM{j}+u_3\bM{k} \quad \bM{n}=\cos \alpha \,\bM{i}+\cos \beta \,\bM{j}+\cos \gamma\,\bM{k} 
\end{align*}
جہاں \عددی{\bM{n}} اور مثبت \عددی{x}، \عددی{y}، \عددی{z} محور کے مابین زاویے بالترتیب \عددی{\alpha}، \عددی{\beta}، \عددی{\gamma} ہیں۔یوں مساوات \حوالہ{مساوات_خطی_تکمل_مسئلہ_گاوس_الف} درج ذیل لکھی جا سکتی ہے
\begin{align}\label{مساوات_خطی_تکمل_مسئلہ_گاوس_پ}
\iiint\limits_T \big(\frac{\partial u_1}{\partial x}+\frac{\partial u_2}{\partial y}+\frac{\partial u_3}{\partial z}\big)\dif x\dif y\dif z=\iint\limits_S (u_1\cos \alpha+u_2\cos \beta+u_3\cos \gamma)\dif A
\end{align}
جسے مساوات \حوالہ{مساوات_خطی_تکمل_عمودی_سمتیہ_اجزاء_استعمال} کی مدد سے درج ذیل لکھی جا سکتی ہے۔
\begin{align}\label{مساوات_خطی_تکمل_مسئلہ_گاوس_ت}
\iiint\limits_T \big(\frac{\partial u_1}{\partial x}+\frac{\partial u_2}{\partial y}+\frac{\partial u_3}{\partial z}\big)\dif x\dif y\dif z=\iint\limits_S (u_1\dif y\dif z+u_2\dif x\dif z +u_3\dif x\dif y)
\end{align}
اب ظاہر ہے کہ اگر درج ذیل تین تعلقات یک وقت درست ہوں تب مساوات \حوالہ{مساوات_خطی_تکمل_مسئلہ_گاوس_پ} درست ہو گا۔
\begin{align}
\iiint\limits_T \frac{\partial u_1}{\partial x}\dif x\dif y\dif z&=\iint\limits_S u_1\cos \alpha \dif A \label{مساوات_خطی_تکمل_مسئلہ_گاوس_ٹ}\\
\iiint\limits_T \frac{\partial u_2}{\partial y}\dif x\dif y\dif z&=\iint\limits_S u_2\cos \beta \dif \label{مساوات_خطی_تکمل_مسئلہ_گاوس_ث}\\
\iiint\limits_T \frac{\partial u_3}{\partial z}\dif x\dif y\dif z&=\iint\limits_S u_3\cos \gamma \dif A\label{مساوات_خطی_تکمل_مسئلہ_گاوس_ج}
\end{align}

ہم مساوات \حوالہ{مساوات_خطی_تکمل_مسئلہ_گاوس_ج} کو ایک خصوصی خطہ \عددی{T} کے لئے ثابت کرتے ہیں جس کی سرحد ٹکڑوں میں ہموار قابل سمت بند سطح \عددی{S} ہے۔اس مخصوص \عددی{T} کی خاصیت ہے کہ \عددی{x}، \عددی{y} یا \عددی{z} محور کے متوازی کوئی بھی خط جو  \عددی{T} کو قطع کرتی ہو، کا زیادہ سے زیادہ صرف ایک  حصہ (یا صرف ایک نقطہ) \عددی{T} کے ساتھ مشترک ہو گا۔ اس خاصیت کا مطلب ہے کہ \عددی{T} کو درج ذیل روپ میں لکھا جا سکتا ہے
\begin{align}\label{مساوات_خطی_تکمل_مسئلہ_گاوس_چ}
g(x,y)\le z\le h(x,y)
\end{align}
جہاں \عددی{xy} مستوی پر \عددی{T} کے قائمہ الزاویہ سائے \عددی{\overline{R}} میں نقطہ \عددی{(x,y)} ہو گا۔ظاہر ہے کہ \عددی{g(x,y)} سطح \عددی{S} کی نچلی سطح \عددی{S_2} کو ظاہر کرتی ہے جبکہ \عددی{h(x,y)} سطح \عددی{S} کی بالائی سطح \عددی{S_1} کو ظاہر کرتی ہے (شکل \حوالہ{شکل_خطی_تکمل_مخصوص_خطہ})۔عین ممکن ہے کہ \عددی{S} کا کوئی کھڑا حصہ \عددی{S_3} بھی پایا جاتا ہو۔(حصہ \عددی{S_3} کی انحطاطی شکل ایک منحنی ہو سکتی ہے مثلاً  کروی \عددی{T} کی صورت میں \عددی{S_3} ایک گول دائرہ ہو گا۔)   
\begin{figure}
\centering
\begin{tikzpicture}
\draw plot[smooth] coordinates {(0,0)(2.5,-0.5) (4,1) (2,1)(0,0)};
\draw plot[smooth] coordinates {(0,0) (1,-1.5) (4,-1.5) (2.5,-0.5)};
\draw(4,-1.5)--(4,1);
%
\draw[-latex](2,0.25)coordinate(kT)--++(130:1.5)node[left]{$\bM{n}$};
\draw[dashed] (kT)node[ocirc,solid]{}--++(0,1.5);
\draw([shift={(90:0.5)}]kT) arc (90:130:0.5);
\draw(kT)++(90+20:0.9)node{$\gamma$};
\draw(3,0.5)node{$S_1$};
%
\draw[-latex](1.75,-1)coordinate(kB)--++(-140:1.5)node[above left]{$\bM{n}$};
\draw[dashed] (kB)node[ocirc,solid]{}--++(0,1);
\draw([shift={(90:0.5)}]kB) arc (90:220:0.5);
\draw(kB)++(160:0.8)node{$\gamma$};
\draw(2.75,-1.25)node[]{$S_2$};
%
\draw[-latex](3.75,-0.75)node[ocirc]{}--++(0:1.5)node[right]{$\bM{n}$};
\draw(3.75,0)node{$S_3$};
%
\draw[dashed](0,0)--++(0,-3.5);
\draw[dashed](4,-1.5)--++(0,-1);
%axis
\draw(-0.5,-2.25)coordinate(kO)--++(-0.5,-0.5)node[left]{$x$};
\draw(kO)--(-0.1,-2.25)(0.1,-2.25)--(3.9,-2.25)(4.1,-2.25)--(5.5,-2.25)node[right]{$y$};
\draw(kO)node[ocirc]{}--++(0,1)node[left]{$z$};
\begin{scope}[shift={(0,-3.5cm)}]
%shadow
\draw plot[smooth] coordinates {(0,0)(2.5,-0.5) (4,1) (2,1)(0,0)};
\draw(2,0.25)node{$\overline{R}$};
\end{scope}
\end{tikzpicture}
\caption{مخصوص خطہ}
\label{شکل_خطی_تکمل_مخصوص_خطہ}
\end{figure}

مساوات \حوالہ{مساوات_خطی_تکمل_مسئلہ_گاوس_ج} کو مساوات \حوالہ{مساوات_خطی_تکمل_مسئلہ_گاوس_چ} کی مدد سے ثابت کرتے ہیں۔چونکہ کسی خطہ جس کا \عددی{T} حصہ ہے میں \عددی{\bM{u}} استمراری قابل تفرق ہے لہٰذا درج ذیل ہو گا۔
\begin{align}\label{مساوات_خطی_تکمل_مسئلہ_گاوس_ح}
\iiint\limits_T \frac{\partial u_3}{\partial z} \dif x\dif y\dif z=\iint\limits_{\overline{R}}\big[\int_{g(x,y)}^{h(x,y)} \frac{\partial u_3}{\partial z}\dif z\big]\dif x\dif y
\end{align}
اس میں اندرونی تکمل لیتے ہیں۔
\begin{align*}
\int_g^h \frac{\partial u_3}{\partial z}\dif z=u_3(x,y,h)-u_3(x,y,g)
\end{align*}
یوں مساوات \حوالہ{مساوات_خطی_تکمل_مسئلہ_گاوس_ح} کا بایاں ہاتھ درج ذیل کے برابر ہو گا۔
\begin{align}\label{مساوات_خطی_تکمل_مسئلہ_گاوس_خ}
\iint\limits_{\overline{R}}u_3[x,y,h(x,y)]\dif x\dif y-\iint\limits_{\overline{R}}u_3[x,y,g(x,y)]\dif x\dif y
\end{align}
آئیں اب ثابت کرتے ہیں کہ  مساوات \حوالہ{مساوات_خطی_تکمل_مسئلہ_گاوس_ج} کا دایاں ہاتھ بھی اسی کے برابر ہے۔ چونکہ \عددی{S_3} پر \عددی{\gamma=\tfrac{\pi}{2}}  ہے لہٰذا \عددی{\cos \gamma=0} ہو گا اور یوں مساوات \حوالہ{مساوات_خطی_تکمل_مسئلہ_گاوس_ح} کے دائیں ہاتھ  \عددی{S_3} پر سطحی تکمل صفر کے برابر ہو گا۔یوں درج ذیل رہ جاتا ہے۔
\begin{align*}
\iint\limits_S u_3\cos \gamma \dif A=\iint\limits_{S_1}u_3\cos \gamma \dif A+\iint\limits_{S_2}u_3\cos \gamma \dif A
\end{align*}
\عددی{S_1} پر \عددی{\gamma} زاویہ حادہ ہے لہٰذا \عددی{\sigma=\gamma} لیتے ہوئے مساوات \حوالہ{مساوات_خطی_تکمل_صریح_تفاعل_رقبہ_پ} سے \عددی{\dif A=\sec \gamma \dif x\dif y} ملتا ہے۔چونکہ \عددی{\cos \gamma\sec\gamma=1} کے برابر ہے لہٰذا یوں
\begin{align*}
\iint\limits_{S_1}u_3\cos \gamma \dif A=\iint\limits_{\overline{R}}u_3[x,y,h(x,y)]\dif x\dif y
\end{align*}
حاصل ہو گا جو مساوات \حوالہ{مساوات_خطی_تکمل_مسئلہ_گاوس_خ} میں پہلی دوہرا  تکمل کے برابر ہے۔اسی طرح \عددی{S_2} پر \عددی{\gamma} زاویہ منفرجہ ہے لہٰذا \عددی{\pi-\gamma} مساوات  \حوالہ{مساوات_خطی_تکمل_صریح_تفاعل_رقبہ_پ} میں زاویہ حادہ \عددی{\sigma} کے مترادف ہو گا۔یوں
\begin{align*}
\dif A=\sec(\pi-\gamma)\dif x\dif y=-\sec \gamma \dif x\dif y
\end{align*}
لکھتے ہوئے
\begin{align}
\iint\limits_{S_2}u_3\cos \gamma \dif A=-\iint\limits_{\overline{R}}u_3[x,y,g(x,y)]\dif x\dif y
\end{align}
ہو گا جو عین \حوالہ{مساوات_خطی_تکمل_صریح_تفاعل_رقبہ_پ} میں دوسرے دوہرا تکمل کے برابر ہے۔یوں مساوات \حوالہ{مساوات_خطی_تکمل_مسئلہ_گاوس_ج} ثابت ہوا۔

مساوات \حوالہ{مساوات_خطی_تکمل_مسئلہ_گاوس_ٹ} اور مساوات \حوالہ{مساوات_خطی_تکمل_مسئلہ_گاوس_ث} کو بالکل اسی طرح ثابت کیا جا سکتا ہے جہاں مساوات \حوالہ{مساوات_خطی_تکمل_مسئلہ_گاوس_چ} کی طرح \عددی{T} کو درج ذیل سے ظاہر کیا جائے گا۔
\begin{align*}
\tilde{g}(y,z) \le x\le \tilde{h}(y,z) \quad \text{اور}\quad g^*(x,z) \le y\le h^*(x,z)
\end{align*}
اس طرح مسئلہ پھیلاو کا مخصوص خطے میں ثبوت مکمل ہوتا ہے۔

ایسا خطہ \عددی{T} جس کو اضافی سطحوں کی مدد سے محدود تعداد کی مخصوص ٹکڑوں میں تقسیم کرنا ممکن ہو کے ہر ٹکڑے پر مسئلہ پھیلاو لاگو کرتے ہوئے تمام جوابات کو مجموعہ لینے سے  پوری خطے پر مسئلہ ثابت ہو گا۔اس ترکیب بالکل مسئلہ گرین میں استعمال کی گئی ترکیب کی طرح ہے۔ہر اضافی سطح پر دو مرتبہ حاصل سطحی تکمل کے جوابات  کا مجموعہ صفر کے برابر ہو گا جبکہ باقی سطحوں پر سطحی تکمل \عددی{T} کی پوری سطح \عددی{S} پر سطحی تکمل ہی ہو گا۔\عددی{T} کے تمام ٹکڑوں کے حجمی تکملات کا مجموعہ \عددی{T} کے حجمی تکمل کے برابر ہو گا۔

یوں کسی بھی عملی استعمال کے محدود خطہ \عددی{T} کے لئے مسئلہ پھیلاو کا ثبوت مکمل ہوتا ہے۔ 


مسئلہ کو ایسی عمومی خطہ \عددی{T} جو مسئلہ کی شرائط پر پورا اترتا ہو کے  لئے ثابت کرنے کی خاطر ہم  \عددی{T} کو تخمیناً ایسی خطوں میں تقسیم کرتے ہیں جو ان شرائط پر پورا اترتے ہوں اور تحدیدی طریقہ اختیار کرتے ہیں۔ یہ ترکیب مسئلہ گرین کی ثبوت میں اختیار کی گئی ترکیب کی طرح ہے۔
\انتہا{ثبوت}
%============================

مسئلہ گرین خطی تکمل کے حل میں کار آمد ثابت ہوتا ہے۔اسی طرح مسئلہ پھیلاو سطحی تکمل کے حل میں کار آمد ثابت ہوتا ہے۔

%=================
\ابتدا{مثال}\quad سطحی تکمل کا حصول بذریعہ مسئلہ پھیلاو\\
درج ذیل کو تہرا  تکمل میں تبدیل کرتے ہوئے حل کریں جہاں \عددی{S} بیلن \عددی{x^2+y^2=a^2\, (0\le z\le b)} اور اس کے دونوں اطراف کی ڈھکنوں کی سطح ہے۔
\begin{align*}
I=\iint\limits_S (x^3\dif y\dif z+x^2y\dif x\dif z+x^2z\dif x\dif y)
\end{align*}
حل:یہاں مساوات \حوالہ{مساوات_خطی_تکمل_مسئلہ_گاوس_پ} اور مساوات \حوالہ{مساوات_خطی_تکمل_مسئلہ_گاوس_ت} میں \عددی{u_1=x^3}، \عددی{u_2=x^2y}، \عددی{u_3=x^2z} ہیں۔یوں خطہ \عددی{T} کی تشاکل کو دیکھ کر ہم درج ذیل لکھ سکتے ہیں۔
\begin{align*}
\iiint\limits_T (3x^2+x^2+x^2)\dif x\dif y\dif z=4\cdot 5\int_0^b \int_0^a\int_0^{\sqrt{a^2-y^2}}x^2\dif x\dif y\dif z
\end{align*}
اندرونی تکمل \عددی{\tfrac{1}{3}(a^2-y^2)^{\tfrac{3}{2}}} کے برابر ہے۔یوں \عددی{y=a\cos t} چنتے ہوئے
\begin{align*}
\dif y=-a\sin t\dif t,\quad (a^2-y^2)^{\frac{3}{2}}=a^3\sin^3 t
\end{align*}
لکھا جا سکتا ہے۔اب \عددی{y} پر تکمل
\begin{align*}
\frac{1}{3}\int_0^a (a^2-y^2)^{\frac{3}{2}}\dif y=-\frac{1}{3} a^4\int_{\frac{\pi}{2}}^{0}\sin^4 t \dif t=\frac{\pi a^4}{16}
\end{align*}
ہو گا اور آخر میں \عددی{z} پر تکمل جزو \عددی{b} دیتا  ہے لہٰذا جواب درج ذیل ہو گا۔
\begin{align*}
I=4\cdot 5\frac{\pi a^4}{16}b=\frac{5}{4}\pi a^4b
\end{align*}
\انتہا{مثال}
%===================

\حصہ{مسئلہ پھیلاو کے نتائج اور استعمال}
مسئلہ پھیلاو کی عملی استعمال اور اس کے چند اہم نتائج کی مثالیں اس حصے میں پیش کی جائیں گی۔ان مثالوں میں فرض کیا جاتا ہے کی تفاعل اور خطہ  مسئلہ پھیلاو کے شرائط  پر پورا اترتے ہیں۔ مزید کہ سطح \عددی{S} پر خطہ \عددی{T} کا باہر رخ اکائی عمودی سمتیہ  \عددی{\bM{n}} ہے۔

%=================
\ابتدا{مثال}\quad محدد سے آزاد پھیلاو\\
مسئلہ پھیلاو کی (مساوات \حوالہ{مساوات_خطی_تکمل_مسئلہ_گاوس_الف}) کے دونوں اطراف کو خطہ \عددی{T} کی حجم \عددی{H(T)} سے تقسیم کرتے ہوئے
\begin{align}\label{مساوات_خطی_تکمل_پھیلاو_مثال_الف}
\frac{1}{H(T)}\iiint\limits_T \nabla \cdot \bM{u}\dif H=\frac{1}{H(T)}\iint\limits_{S(T)}u_n \dif A
\end{align}
ملتا ہے جہاں \عددی{T} کی سرحدی سطح \عددی{S(T)} ہے۔دوہرا تکمل کی خصوصیات کو حصہ \حوالہ{حصہ_خطی_تکمل_دوہرا_تکمل} میں بیان کیا گیا۔تہرا تکمل بھی یہی خصوصیات رکھتا ہے۔بالخصوص تہرا تکمل کا \اصطلاح{مسئلہ اوسط قیمت}\فرہنگ{مسئلہ!اوسط قیمت} کہتا ہے کہ خطہ \عددی{T} میں کسی بھی استمراری تفاعل \عددی{f(x,y,z)}  کے لئے \عددی{T} میں ایسا نقطہ \عددی{Q: (x_0,y_0,z_0)} پایا جائے گا کہ درج ذیل درست ہو گا۔
\begin{align*}
\iiint\limits_T f(x,y,z)\dif H=f(x_0,y_0,z_0) H(T)
\end{align*} 
یوں \عددی{f=\nabla \cdot \bM{u}} پر کرتے ہوئے مساوات \حوالہ{مساوات_خطی_تکمل_پھیلاو_مثال_الف} سے درج ذیل ملتا ہے۔
\begin{align}\label{مساوات_خطی_تکمل_پھیلاو_مثال_ب}
\frac{1}{H(T)}\iiint\limits_T \nabla \cdot \bM{u}\, \dif H=\nabla \cdot \bM{u}(x_0,y_0,z_0)
\end{align}
فرض کریں کہ \عددی{T} میں \عددی{N:(x_1,y_1,z_1)} کوئی مقررہ نقطہ ہے اور \عددی{T} نقطہ \عددی{N} کے گرد یوں سکڑتا ہے کہ \عددی{N} سے \عددی{T} کے  دور ترین نقطے کا فاصل \عددی{d(T)} صفر کے قریب پہنچے۔اس طرح نقطہ \عددی{Q} نقطہ \عددی{N} کے قریب پہنچے گا اور مساوات \حوالہ{مساوات_خطی_تکمل_پھیلاو_مثال_الف} اور مساوات \حوالہ{مساوات_خطی_تکمل_پھیلاو_مثال_ب} سے ظاہر کہ کہ نقطہ \عددی{N} پر \عددی{\bM{u}} کی پھیلاو درج ذیل ہو گی۔
\begin{align}\label{مساوات_خطی_تکمل_پھیلاو_مثال_پ}
\nabla \cdot \bM{u}(x_1,y_1,z_1)=\lim_{d(T)\to 0}\frac{1}{V(T)}\iint\limits_{S(T)}u_n\dif A
\end{align}
اس کلیہ کو بعض اوقات پھیلاو کی تعریف تصور کیا جاتا ہے۔جہاں حصہ \حوالہ{حصہ_الاحصاء_پھیلاو} میں پھیلاو کی تعریف میں \عددی{x}، \عددی{y}، \عددی{z} محدد پائے جاتے ہیں مساوات \حوالہ{مساوات_خطی_تکمل_پھیلاو_مثال_پ} میں دی گئی پھیلاو کی تعریف محدد سے پاک ہے۔اس سے یک دم اخذ کیا جا سکتا ہے کہ پھیلاو کی قیمت پر محددی نظام کی انتخاب کا کوئی اثر نہیں پایا جاتا ہے۔
\انتہا{مثال}
%===================
\ابتدا{مثال}\quad پھیلاو کا طبعی مفہوم\\
مسئلہ پھیلاو سے سمتیہ کی پھیلاو کا مفہوم سمجھا جا سکتا ہے۔ایسا ہی کرنے کی خاطر ہم اکائی کمیتی کثافت \عددی{\rho=1} کی  داب نا پذیر  سیال کی برقرار حال (وقت کے ساتھ نہ تبدیل ہوتا) بہاو پر غور کرتے ہیں (مثال \حوالہ{مثال_الاحصاء_حرکت_سیال} بھی دیکھیں)۔ کسی بھی نقطہ \عددی{N} پر ایسی بہاو کا تعین اس نقطہ پر سمتی رفتار سمتیہ \عددی{\bM{v}(N)} سے کیا جاتا ہے۔

فرض کریں کہ فضا میں خطہ \عددی{T} کی سرحدی سطح \عددی{S} ہے اور \عددی{\bM{n}} باہر رخ \عددی{S} کا اکائی عمودی سمتیہ ہے۔اس سطح کے چھوٹے حصہ \عددی{\Delta S} جس کا رقبہ \عددی{\Delta A} ہے سے،  اندرون \عددی{S} سے بیرون \عددی{S}  رخ، اکائی وقت میں کمیت کی اخراج \عددی{v_n \Delta A}\حاشیہد{کسی نقطہ پر \عددی{v_n}  منفی ہو سکتا ہے لہٰذا ایسے نقطے پر سیال \عددی{S} میں داخل ہو گا۔} ہو گی جہاں \عددی{v_n=\bM{v}\cdot \bM{n}} سمتیہ \عددی{\bM{v}} کا \عددی{\bM{n}}  رخ جزو ہے (یعنی \عددی{S} کا عمودی جزو ہے) اور \عددی{\bM{n}} کو \عددی{\Delta S} کے کسی موزوں نقطے پر لیا گیا ہے۔یوں \عددی{T} سے کل اخراج جو \عددی{S} سے گزرتا ہے سطحی تکمل
\begin{align*}
\iint\limits_S v_n \dif A
\end{align*}
سے حاصل ہو گا۔یہ تکمل \عددی{T} کا کل اخراج دیتا ہے۔یوں \عددی{T} کی اوسط اخراج
\begin{align}\label{مساوات_خطی_تکمل_پھیلاو_مثال_ت}
\frac{1}{H}\iint\limits_S v_n\dif A
\end{align}
ہو گی جہاں \عددی{T} کا حجم \عددی{H} ہے۔چونکہ بہاو برقرار حال ہے اور سیال داب نا پذیر ہے لہٰذا \عددی{T} سے  اخراج برابر کمیت \عددی{T} کو  مہیا کی جاتی ہو گی۔یوں اگر مساوات \حوالہ{مساوات_خطی_تکمل_پھیلاو_مثال_ت} کے تکمل کی قیمت غیر صفر ہو تب \عددی{T} میں \اصطلاح{منبع}\فرہنگ{منبع}\حاشیہب{source}\فرہنگ{source} (مثبت منبع یا منفی منبع) پایا جاتا ہو گا جہاں سیال پیدا یا غائب ہوتا ہے۔

اگر ہم \عددی{T} کو ایک نقطہ \عددی{N} مانند کر دیں تب  مساوات \حوالہ{مساوات_خطی_تکمل_پھیلاو_مثال_ت} ہمیں \عددی{N} پر \اصطلاح{شدت منبع}\فرہنگ{شدت منبع}\حاشیہب{source intensity}\فرہنگ{source intensity} دیگا (مساوات \حوالہ{مساوات_خطی_تکمل_پھیلاو_مثال_پ} کا دائیں ہاتھ جہاں \عددی{v_n} کی جگہ \عددی{u_n} لکھا گیا ہے)۔ اس سے ظاہر ہے کہ داب نا پذیر سیال کی برقرار حال سمتی رفتار سمتیہ \عددی{\bM{v}} کا نقطہ \عددی{N} پر پھیلاو سے مراد \عددی{N} پر شدت منبع ہے۔صرف اور صرف اس صورت \عددی{T} میں کوئی منبع نہ ہو گا جب \عددی{\nabla \cdot \bM{v} \equiv 0} ہو اور ایسی صورت میں \عددی{} میں کسی بھی بند سطح \عددی{S^*} کے لئے درج ذیل درست ہو گا۔
\begin{align*}
\iint\limits_{S^*}v_n\dif A=0
\end{align*}
آپ نے دیکھا کہ  کسی نقطہ سے سیال کی اخراج کو اس نقطہ پر \عددی{\bM{v}} کی پھیلاو ظاہر کرتی ہے۔ہم کہتے ہیں سیال اس نقطہ سے نکل کر پھیلتا ہے۔اسی سے اس عمل کو \اصطلاح{پھیلاو}\فرہنگ{پھیلاو} کہتے ہیں۔ 
\انتہا{مثال}
%=====================
\ابتدا{مثال}\quad مساوات حرارت۔ حراری بہاو\\
ہم جانتے ہیں کہ کسی بھی جسم میں حراری توانائی کا بہاو  گرم سے سرد مقام کے رخ  ہو گا۔اس کا مطلب ہے کہ حراری بہاو کی سمتی رفتار \عددی{\bM{v}} درج طرز کی ہو گی
\begin{align}\label{مساوات_خطی_تکمل_حراری_اخراج_الف}
\bM{v}=-K\,\nabla U
\end{align}
جہاں \عددی{U(x,y,z,t)} لمحہ \عددی{t} پر نقطہ \عددی{(x,y,z)} کا درجہ حرارت ہے اور \عددی{K} جسم کی \اصطلاح{حراری موصلیت}\فرہنگ{موصلیت!حراری}\فرہنگ{حراری موصلیت}\حاشیہب{thermal conductivity}\فرہنگ{thermal conductivity} ہے۔عمومی طبعی حالات میں \عددی{K} ایک مستقل ہو گا۔
 
فرض کریں کہ جسم میں \عددی{R} کوئی خطہ ہے جس کی سرحدی سطح \عددی{S} ہے۔یوں اکائی وقت میں \عددی{R} سے کل حراری توانائی کا اخراج
\begin{align*}
\iint\limits_S v_n \dif A
\end{align*}
ہو گا جہاں \عددی{v_n=\bM{v}\cdot \bM{n}} سرحد \عددی{S} پر باہر رخ اکائی عمودی سمتیہ \عددی{\bM{n}} کی رخ  \عددی{\bM{v}} کا جزو ہے۔یہ تعلق گزشتہ مثال کی حاصل کیا گیا ہے۔ مساوات \حوالہ{مساوات_خطی_تکمل_حراری_اخراج_الف} اور مسئلہ پھیلاو سے درج ذیل لکھا جا سکتا ہے (مساوات \حوالہ{مساوات_الاحصاء_لاپلاسی_ڈھلوان_اور_پھیلاو})۔
\begin{align}
\iint\limits_S v_n \dif A=-K\iiint\limits_R \nabla \cdot (\nabla U)\dif x\dif y\dif z=-K\iiint\limits_R \nabla^{\,2}U \dif x\dif y\dif z
\end{align}
 \عددی{R} میں کل حراری توانائی \عددی{W} درج ذیل ہے
\begin{align*}
W=\iiint\limits_R \sigma \rho U \dif x\dif y\dif z
\end{align*}
جہاں \عددی{\sigma} جسم کے مواد کی \اصطلاح{خصوصی حراری استعداد}\فرہنگ{حراری!خصوصی استعداد}\حاشیہب{specific heat capacity}\فرہنگ{heat!specific capacity} ہے جبکہ \عددی{\rho} جسم کی کمیتی کثافت (کمیت فی اکائی حجم) ہے۔یوں جسم میں حراری توانائی کی وقت کے ساتھ گھٹاو 
\begin{align*}
-\frac{\partial W}{\partial t}=-\iiint\limits_R \sigma \rho \frac{\partial U}{\partial t} \dif x\dif y\dif z
\end{align*}
ہو گی جو عین \عددی{R} سے توانائی کی اخراج کے برابر ہو گا یعنی
\begin{align*}
-\iiint\limits_R \sigma \rho \frac{\partial U}{\partial t} \dif x\dif y\dif z=-K\iiint\limits_R \nabla^{\,2}U \dif x\dif y\dif z
\end{align*} 
یا:
\begin{align*}
\iiint\limits_R \big(\sigma \rho\frac{\partial U}{\partial t}-K\nabla^{\,2}U\big)\dif x\dif y\dif z=0
\end{align*}
 چونکہ یہ مساوات کسی بھی خطہ \عددی{R} کے لئے درست ہے لہٰذا متکمل (اگر استمراری ہو تب) تمام \عددی{R} میں صفر کے برابر ہو گا یعنی:
\begin{align}
\frac{\partial U}{\partial t}=c^2\nabla^{\,2}U\quad \quad \quad (c^2=\frac{K}{\sigma \rho})
\end{align}
یہ \اصطلاح{حراری مساوات}\فرہنگ{حراری!مساوات}\حاشیہب{heat equation}\فرہنگ{heat!equation} کہلاتی ہے جو حراری بہاو کی بنیادی مساوات ہے۔
\انتہا{مثال}
%=====================
\ابتدا{مثال}\quad لاپلاسی مساوات کے حل کی بنیادی خصوصیت\\
مسئلہ پھیلاو کی مساوات
\begin{align}\label{مساوات_خطی_تکمل_مسئلہ_پھیلاو_دوبارہ_الف}
\iiint\limits_T \nabla \bM{u}\dif H=\iint\limits_S u_n \dif A
\end{align}
پر غور کریں۔فرض کریں کہ \عددی{\bM{u}} کسی غیر سمتی تفاعل کی ڈھلوان \عددی{\bM{u}=\nabla f} ہے۔یوں
\begin{align*}
\nabla \cdot \bM{u}=\nabla \cdot (\nabla f)=\nabla^{\,2}f
\end{align*}
ہو گا (مساوات \حوالہ{مساوات_الاحصاء_لاپلاسی_ڈھلوان_اور_پھیلاو})۔مزید 
\begin{align*}
u_n=\bM{u}\cdot \bM{n}=\bM{n}\cdot \nabla f
\end{align*}
لکھا جائے گا جو مساوات \حوالہ{مساوات_الاحصاء_سمتی_تفرق_ج} کے تحت \عددی{S} کے باہر رخ \عددی{f} کا سمتی تفرق ہے جس کو \عددی{\tfrac{\partial f}{\partial n}} سے ظاہر کرتے ہوئے  مساوات \حوالہ{مساوات_خطی_تکمل_مسئلہ_پھیلاو_دوبارہ_الف} کو درج ذیل لکھا جا سکتا ہے۔
\begin{align}\label{مساوات_خطی_تکمل_مسئلہ_پھیلاو_دوبارہ_ب}
\iiint\limits_T \nabla^{\,2}f\dif H=\iint\limits_S \frac{\partial f}{\partial n}\dif A
\end{align}
ظاہر ہے کہ یہ مساوات \حوالہ{مساوات_خطی_تکمل_لاپلاسی_قائمہ_تبادل} کی تین بعدی مماثل ہے۔
\انتہا{مثال}
%=========================

مسئلہ پھیلاو کے لئے درکار شرائط کو مد نظر رکھتے ہوئے  مساوات \حوالہ{مساوات_خطی_تکمل_مسئلہ_پھیلاو_دوبارہ_ب} سے درج ذیل اخذ کیا جا سکتا ہے۔ 

%==========================
\ابتدا{مسئلہ}\شناخت{مسئلہ_خطی_تکمل_لاپلاس_حل_خصوصیت}\quad (لاپلاسی مساوات کے حل کی خصوصیت)\\
فرض کریں کہ کسی خطہ \عددی{D} میں تفاعل \عددی{f(x,y,z)} لاپلاسی مساوات
\begin{align*}
\nabla^{\,2}f=\frac{\partial^{\,2} f}{\partial x^2}+\frac{\partial^{\,2} f}{\partial y^2}+\frac{\partial^{\,2} f}{\partial z^2}=0
\end{align*}
کا حل ہے اور \عددی{D} میں \عددی{f} کے دو درجی جزوی تفرق استمراری ہیں۔تب \عددی{D} میں کسی بھی ٹکڑوں میں ہموار  بند اور قابل سمت بند سطح \عددی{S} پر \عددی{f} کے عمودی  (سمتی) تفرق کا تکمل صفر ہو گا۔
\انتہا{مسئلہ}
%========================
\ابتدا{مثال}\quad مسئلہ گرین\\
فرض کریں کہ \عددی{f} اور \عددی{g} ایسے غیر سمتی تفاعل ہیں کہ کسی خطہ \عددی{T} میں \عددی{\bM{u}=f\nabla g} مسئلہ پھیلاو کی شرائط پر پورا اترتا ہو۔تب درج ذیل ہو گا (سوال \حوالہ{سوال_الاحصاء_پھیلاو_تعلق_پ})۔
\begin{align*}
\nabla \cdot \bM{u}=\nabla \cdot(f\nabla g)=f\nabla^{\,2}g+\nabla f \cdot \nabla g
\end{align*}
مزید
\begin{align*}
\bM{u}\cdot \bM{n}=\bM{n}\cdot(f\nabla g)=f(\bM{n}\cdot \nabla g)
\end{align*}
ہو گا جہاں \عددی{\bM{n}\cdot \nabla g} سے مراد مسئلہ پھیلاو کی سطح \عددی{S} پر باہر رخ اکائی عمودی سمتیہ \عددی{\bM{n}} کی سمت میں \عددی{g} کا سمتی تفرق ہے۔ اس سمتی تفرق کو \عددی{\tfrac{\partial g}{\partial n}} لکھنے سے مسئلہ پھیلاو کی مساوات درج ذیل صورت اختیار کرتی ہے
\begin{align}\label{مساوات_خطی_تکمل_گرین_کلیہ_اول}
\iiint\limits_T (f\nabla^{\,2}g+\nabla f\cdot \nabla g)\dif H=\iint\limits_S f\frac{\partial g}{\partial n}\dif A
\end{align}
جس کو \اصطلاح{گرین کلیہ اول}\فرہنگ{گرین!کلیہ اول}\حاشیہب{Green's first formula}\فرہنگ{Green!first formula}  یا (لاگو شرائط کو شامل کرتے ہوئے) \ترچھا{مسئلہ گرین کی پہلی صورت} کہتے ہیں۔

\عددی{f} اور \عددی{g} کو آپس میں بدلنے سے اسی طرح کی دوسری مساوات حاصل ہوتی ہے  جس کو  مساوات \حوالہ{مساوات_خطی_تکمل_گرین_کلیہ_اول} سے منفی کرتے ہوئے درج ذیل ملتا ہے
\begin{align}\label{مساوات_خطی_تکمل_گرین_کلیہ_دوم}
\iiint\limits_T (f\nabla^{\,2}g-g\nabla^{\,2}f)\dif H=\iint\limits_S \big(f\frac{\partial g}{\partial n}-g\frac{\partial f}{\partial n}\big)\dif A
\end{align} 
جس کو \اصطلاح{گرین کلیہ دوم}\فرہنگ{گرین!کلیہ دوم}\حاشیہب{Green's second formula}\فرہنگ{Green!second formula} یا (لاگو شرائط کو شامل کرتے ہوئے) \ترچھا{مسئلہ گرین کی دوسری صورت} کہتے ہیں۔
\انتہا{مثال}
%=========================
\ابتدا{مثال}\quad لاپلاس مساوات کی حل کی یکتائی\\
فرض کریں کہ \عددی{f} مسئلہ \حوالہ{مسئلہ_خطی_تکمل_لاپلاس_حل_خصوصیت} کے شرائط پر پورا اترتا ہے اور \عددی{D} میں ٹکڑوں میں ہموار بند اور قابل سمت بند سطح \عددی{S} پر ہر جگہ صفر کے برابر ہے۔تب مساوات \حوالہ{مساوات_خطی_تکمل_گرین_کلیہ_اول} میں \عددی{f=g} پر کرتے ہوئے اور \عددی{S} کے اندرونی حصہ کو \عددی{T} سے ظاہر کرتے ہوئے
\begin{align*}
\iiint\limits_T \nabla f\cdot \nabla f\, \dif H=\iiint\limits_T \abs{\nabla f}^2\dif H=0
\end{align*}
ملتا ہے جہاں مسئلہ \حوالہ{مسئلہ_خطی_تکمل_لاپلاس_حل_خصوصیت} میں دیے شرط کے مطابق \عددی{\nabla^{\,2}f=0} لیا گیا ہے اور مساوات \حوالہ{مساوات_خطی_تکمل_گرین_کلیہ_اول} کے دائیں ہاتھ چونکہ سطح \عددی{S} پر \عددی{f=0} ہے لہٰذا اس سطحی تکمل کو صفر لیا گیا ہے۔اب چونکہ ہمارے مفروضہ کے تحت  \عددی{T} کے اندر  اور \عددی{S} پر \عددی{\abs{\nabla f}} استمراری اور غیر منفی  ہے لہٰذا یہ ضرور پورے \عددی{T} میں  ہر جگہ صفر کے برابر ہو گا۔یوں \عددی{f_x=f_y=f_z=0} ہو گا لہٰذا \عددی{T} میں \عددی{f} ایک مستقل ہو گا اور چونکہ \عددی{f} استمراری ہے لہٰذا \عددی{T} کے اندر اس کی قیمت وہی ہو گی جو \عددی{S} پر ہے یعنی \عددی{f=0} ہو گا۔
\انتہا{مثال}
%=========================
اس سے درج ذیل ثابت ہوتا ہے۔

%======================
\ابتدا{مسئلہ}\شناخت{مسئلہ_خطی_تکمل_لاپلاسی_خصوصیات_ب}\\
اگر تفاعل \عددی{f(x,y,z)} مسئلہ \حوالہ{مسئلہ_خطی_تکمل_لاپلاس_حل_خصوصیت} کے شرائط پر پورا اترتا ہے اور \عددی{D} میں ٹکڑوں میں ہموار بند اور قابل سمت بند سطح \عددی{S} پر ہر جگہ صفر کے برابر ہے۔ تب \عددی{S} کے احاطہ  خطہ \عددی{T} میں \عددی{f=0} ہو گا۔
\انتہا{مسئلہ}
%=========================
اس مسئلہ کے اہم نتائج پائے جاتے ہیں۔فرض کریں کہ تفاعل \عددی{f_1} اور \عددی{f_2} مسئلہ \حوالہ{مسئلہ_خطی_تکمل_لاپلاس_حل_خصوصیت} کے شرائط پر پورا اترتے ہیں اور \عددی{S} پر دونوں یکساں ہوں۔تب ان کا فرق \عددی{f_1-f_2} بھی ان شرائط پر پورا اترتا ہے  اور پوری \عددی{S} پر اس کی قیمت صفر کے برابر ہے۔یوں مسئلہ \حوالہ{مسئلہ_خطی_تکمل_لاپلاسی_خصوصیات_ب}  کے تحت پوری \عددی{T} میں \عددی{f_1-f_2=0} ہو گا جس سے درج ذیل نتیجہ حاصل ہوتا ہے۔

%=========================
\ابتدا{مسئلہ}\شناخت{مسئلہ_خطی_تکمل_لاپلاسی_خصوصیات_پ}\quad (لاپلاس مساوات کی حل کی یکتائی)\\
فرض کریں کہ \عددی{f} لاپلاس مساوات کا حل ہے اور خطہ \عددی{D} میں اس کے ایک درجی اور دو درجی جزوی تفرق پائے جاتے ہیں۔مزید فرض کریں کہ \عددی{D} میں خطہ \عددی{T}  مسئلہ پھیلاو کی شرائط پر پورا اترتا ہے۔تب \عددی{T} میں \عددی{f} کی قیمت یکتا ہو گی اور یہ \عددی{T} کی سرحدی سطح \عددی{S} پر قیمت کے برابر ہو گی۔
\انتہا{مسئلہ}
%==========================

\حصہء{سوالات}
%=============================
سوال \حوالہ{سوال_خطی_تکمل_حجم_تہرا_الف} تا سوال \حوالہ{سوال_خطی_تکمل_حجم_تہرا_ب} میں حجم بذریعہ تہرا تکمل دریافت کریں۔

%================================
\ابتدا{سوال}\شناخت{سوال_خطی_تکمل_حجم_تہرا_الف}\quad چو سطحہ جس کے کونے \عددی{A:(0,0,0)}، \عددی{B:(3,0,0)}، \عددی{C:(0,2,0)}، \عددی{D:(0,0,1)} ہیں۔\\
جواب:\quad یہ چو سطح ربع اول میں جس سطح  کے نیچے پایا جاتا ہے پہلے اس (بالائی) سطح کی مساوات حاصل کرتے ہیں۔ \عددی{B} تا \عددی{C} سمتیہ \عددی{\bM{r}_1=-3\bM{i}+2\bM{j}} اور \عددی{B} تا \عددی{D} سمتیہ \عددی{\bM{r}_2=-3\bM{i}+\bM{k}} دونوں جو سطح کی اس بالائی سطح پر پائے جاتے ہیں لہٰذا دونوں سطح کے مماسی سمتیات ہیں۔ان سے بالائی سطح کی اکائی عمودی سمتیہ \عددی{\bM{n}} حاصل کرتے ہیں۔
\begin{align*}
\bM{n}=\frac{\bM{r}_1\times \bM{r}_2}{\abs{\bM{r}_1\times \bM{r}_2}}=\frac{2\bM{i}+3\bM{j}+6\bM{k}}{7}
\end{align*}
یوں بالائی سطح کی مساوات \عددی{[(x-3)\bM{i}+y\bM{j}+z\bM{k}]\cdot \bM{n}=0} سے
\begin{align*}
2x+3y+6z=6
\end{align*}
حاصل ہوتی ہے۔اس طرح چو سطح کا حجم درج ذیل ہو گا (سوال \حوالہ{سوال_خطی_تکمل_حجم_الف} دیکھیں)۔
\begin{align*}
H=\int_0^3\int_0^{2-\tfrac{2}{3}x}\int_0^{1-\tfrac{x}{3}-\tfrac{y}{2}} \dif z\dif y\dif x&=\int_0^3\int_0^{2-\tfrac{2}{3}x}\big(1-\frac{x}{3}-\frac{y}{2}\big)\dif y\dif x\\
&=\int_0^3 \big(\frac{1}{9}x^2-\frac{2}{3}x+1\big)\dif x=1
\end{align*}
\انتہا{سوال} 
%=============================
\ابتدا{سوال}\quad ربع اول میں وہ خطہ جس کی سرحدیں \عددی{y=x}، \عددی{y=x^2} اور \عددی{z=3-2x} ہیں۔\\
جواب:\quad 
$\tfrac{1}{3}$
\انتہا{سوال}
%====================
\ابتدا{سوال}\quad سطح \عددی{z=1-x^2-y^2} اور \عددی{xy} مستوی  کے مابین خطہ۔\\
جواب:\quad 
$\tfrac{\pi}{2}$
\انتہا{سوال}
%====================
\ابتدا{سوال}\شناخت{سوال_خطی_تکمل_حجم_تہرا_ب}\quad بیلن \عددی{x^2+y^2=1} اور \عددی{x^2+z^2=1} کا مشترکہ حصہ۔\\
جواب:\quad 
$\tfrac{16}{3}$
\انتہا{سوال}
%====================
سوال \حوالہ{سوال_خطی_تکمل_کمیتی_کثافت_سے_کمیت_الف} تا سوال \حوالہ{سوال_خطی_تکمل_کمیتی_کثافت_سے_کمیت_ب} میں کمیتی کثافت \عددی{\sigma} دیا گیا ہے۔خطہ \عددی{T} میں کل کمیت دریافت کریں۔

%==========================
\ابتدا{سوال}\شناخت{سوال_خطی_تکمل_کمیتی_کثافت_سے_کمیت_الف} \quad 
$\sigma=xy,\quad T:\, 0\le x\le 1,\,\, 0 \le y \le 1,\,\, 0\le z \le 1 \, \text{مکعب}$\\
جواب:\quad 
$\tfrac{1}{4}$
\انتہا{سوال}
%=========================
\ابتدا{سوال} \quad چو سطح جس کے کونے  \عددی{(0,0,0)}، \عددی{(3,0,0)}، \عددی{(0,2,0)}، \عددی{(0,0,1)} ہیں اور \عددی{\sigma=x+y+z} ہے۔\\
جواب:\quad 
$\tfrac{3}{2}$
\انتہا{سوال}
%=========================
\ابتدا{سوال} \quad چو سطح جس کے کونے  \عددی{(0,0,0)}، \عددی{(3,0,0)}، \عددی{(0,2,0)}، \عددی{(0,0,1)} ہیں اور \عددی{\sigma=xy} ہے۔\\
جواب:\quad 
$\tfrac{3}{10}$
\انتہا{سوال}
%=========================
\ابتدا{سوال}\شناخت{سوال_خطی_تکمل_کمیتی_کثافت_سے_کمیت_ب} \quad ربع اول میں \عددی{y=1-x^2} اور \عددی{z=x} کے درمیان \عددی{T} جہاں \عددی{\sigma=xy} ہے۔\\
جواب:\quad 
$\tfrac{4}{105}$
\انتہا{سوال}
%=========================
سوال \حوالہ{سوال_خطی_تکمل_جمودی_معیار_اثر_اکائی_کثافت_الف} تا سوال \حوالہ{سوال_خطی_تکمل_جمودی_معیار_اثر_اکائی_کثافت_ب} میں خطہ \عددی{T} میں کمیتی کثافت \عددی{\sigma=1} لیتے ہوئے  \عددی{z} محور کے لحاظ سے جمودی معیار اثر \عددی{I_z=\iiint\limits_T (x^2+y^2)\sigma \dif x\dif y\dif z} دریافت کریں۔

%======================
\ابتدا{سوال}\شناخت{سوال_خطی_تکمل_جمودی_معیار_اثر_اکائی_کثافت_الف}\quad مکعب 
$0\le x\le c,\,\, 0 \le y \le c,\,\, 0\le z \le c\,\,$\\
جواب:\quad
$\tfrac{2}{3}c^5$
\انتہا{سوال}
%=======================
\ابتدا{سوال}\شناخت{سوال_خطی_تکمل_بیلن_سطح_الف}\quad بیلن 
$x^2+y^2 \le c^2,\, 0\le z \le h\,\,$\\
جواب:\quad
$\tfrac{1}{2}\pi c^4 h$
\انتہا{سوال}
%=======================
\ابتدا{سوال}\quad بیلن 
$x^2+z^2 \le c^2,\, 0\le y \le h\,\,$\\
جواب:\quad
$\tfrac{\pi c^2h}{12}(4h^2+3c^2)$
\انتہا{سوال}
%=======================
\ابتدا{سوال}\quad مخروط
$x^2+y^2\le z^2,\,\, 0\le z \le h\,\,$\\
جواب:\quad
$\tfrac{\pi h^5}{10}$
\انتہا{سوال}
%=======================
\ابتدا{سوال}\شناخت{سوال_خطی_تکمل_جمودی_معیار_اثر_اکائی_کثافت_ب}\quad اندرون کرہ
$x^2+y^2+z^2=c^2\,\,$\\
جواب:\quad
$\tfrac{4}{15}\pi c^5$
\انتہا{سوال}
%=======================
\ابتدا{سوال}\شناخت{سوال_خطی_تکمل_حجم_کے_کلیات_الف}\quad مسئلہ پھیلاو استعمال کرتے ہوئے ثابت کریں کہ خطہ \عددی{T} جس کی سرحد سطح \عددی{S} ہو کا حجم \عددی{H} درج ذیل ہے۔
\begin{align*}
H=\iint\limits_S x\dif y\dif z=\iint\limits_S y\dif x \dif z=\iint\limits_S z\dif x\dif y=\frac{1}{3}\iint\limits_S (x\dif y\dif z+y\dif x\dif z+z\dif x\dif y)
\end{align*}
\انتہا{سوال}
%=======================
\ابتدا{سوال}\quad مکعب کا حجم سوال \حوالہ{سوال_خطی_تکمل_حجم_کے_کلیات_الف} کے کلیات کی مدد سے حاصل کریں۔
\انتہا{سوال}
%=======================
\ابتدا{سوال}\quad بیلن \عددی{x^2+y^2\le 1,\,\, 0\le z \le h} کا حجم سوال \حوالہ{سوال_خطی_تکمل_حجم_کے_کلیات_الف} کے کلیات کی مدد سے حاصل کریں۔
\انتہا{سوال}
%=======================
\ابتدا{سوال}\شناخت{سوال_خطی_تکمل_حجم_بذریعہ_سطحی_تکمل_زاویہ_رداس_الف}\quad 
\عددی{\bM{u}=x\bM{i}+y\bM{j}+z\bM{k}} لیتے ہوئے مساوات \حوالہ{مساوات_خطی_تکمل_مسئلہ_گاوس_الف} کی مدد سے ثابت کریں کہ خطہ \عددی{T} جس کی سرحدی سطح \عددی{S} ہو کا حجم درج ذیل ہے
\begin{align*}
H=\frac{1}{3}\iint\limits_S r\cos \theta \dif A
\end{align*}
جہاں \عددی{S} پر نقطہ \عددی{N:(x,y,z)} کا مبدا \عددی{O} سے فاصلہ \عددی{r} ہے اور \عددی{O} سے \عددی{N} تک سمتی خط  اور \عددی{N} پر باہر رخ  عمودی سمتیہ کے مابین زاویہ \عددی{\theta} ہے۔  
\انتہا{سوال}
%=====================
\ابتدا{سوال}\quad رداس \عددی{a} کی کرہ کا حجم سوال \حوالہ{سوال_خطی_تکمل_حجم_بذریعہ_سطحی_تکمل_زاویہ_رداس_الف} کے کلیے کی مدد سے دریافت کریں۔
\انتہا{سوال}
%====================
سوال \حوالہ{سوال_خطی_تکمل_مسئلہ_پھیلاو_سے_سطحی_تکمل_الف} تا سوال \حوالہ{سوال_خطی_تکمل_مسئلہ_پھیلاو_سے_سطحی_تکمل_ب} میں \عددی{S} مسئلہ پھیلاو کی شرط کے مطابق سمت بند ہے۔ سطحی تکمل کو مسئلہ پھیلاو کی مدد سے حل کریں۔ 

%=================
\ابتدا{سوال}\شناخت{سوال_خطی_تکمل_مسئلہ_پھیلاو_سے_سطحی_تکمل_الف}
\begin{align*}\iint\limits_S [(x+z)\dif y \dif z+(y+z)\dif x\dif z+(x+y)\dif x\dif y],\quad S:\, x^2+y^2+z^2=a^2\end{align*}
جواب:\quad
$\tfrac{8\pi a^3}{3}$
\انتہا{سوال}
%=====================
\ابتدا{سوال}\quad سطح مکعب سوال \حوالہ{سوال_خطی_تکمل_کمیتی_کثافت_سے_کمیت_الف}\quad 
$\iint\limits_S (x\dif y \dif z+y\dif x\dif z+z\dif x\dif y)$\\
جواب:\quad
$3$
\انتہا{سوال}
%=====================
\ابتدا{سوال}\quad سطح بیلن سوال \حوالہ{سوال_خطی_تکمل_بیلن_سطح_الف}\quad 
$\iint\limits_S (x^2\dif y \dif z+y^2\dif x\dif z+z^2\dif x\dif y)$\\
جواب:\quad
$\pi c^2h^2$
\انتہا{سوال}
%=====================
\ابتدا{سوال}\quad سطح سوال \حوالہ{سوال_خطی_تکمل_مسئلہ_پھیلاو_سے_سطحی_تکمل_الف}\quad
$\iint\limits_S (yz^2\dif y \dif z+xz\dif x\dif z+x^2y^2\dif x\dif y)$\\
جواب:\quad
$0$
\انتہا{سوال}
%======================
\ابتدا{سوال}\شناخت{سوال_خطی_تکمل_ایک_جزو_الف}\quad متوازی السطوح
$\iint\limits_S x(y+z)\dif y \dif z,\, S:0\le x \le 3, 0\le y \le 2,0\le z \le 4\,\,$\\
جواب:\quad
$72$
\انتہا{سوال}
%======================
\ابتدا{سوال}\quad سطح سوال \حوالہ{سوال_خطی_تکمل_ایک_جزو_الف}
$\iint\limits_S [x\cos y\,\dif y \dif z+(y-\sin y)\dif x\dif z]\,\,$\\
جواب:\quad
$24$
\انتہا{سوال}
%======================
\ابتدا{سوال}\شناخت{سوال_خطی_تکمل_مسئلہ_پھیلاو_سے_سطحی_تکمل_ب}\quad سطح سوال \حوالہ{سوال_خطی_تکمل_مسئلہ_پھیلاو_سے_سطحی_تکمل_الف}
$\iint\limits_S [(y\cos^2 x+y^3)\dif x\dif z+z(\sin^2 x-3y^2)\dif x\dif y]$\\
جواب:\quad
$\tfrac{4}{3}\pi a^3$
\انتہا{سوال}
%========================
سوال \حوالہ{سوال_خطی_تکمل_فقرے_ثبوت_الف} تا سوال \حوالہ{سوال_خطی_تکمل_فقرے_ثبوت_پ} میں \عددی{T} بند محدود خطہ ہے جس کی سرحدی سطح \عددی{S} ہے۔مسئلہ پھیلاو استعمال کرتے ہوئے دیے  گئے فقرے ثابت کریں جہاں \اصطلاح{ہارمونی}\فرہنگ{ہارمونی}\حاشیہب{harmonic}\فرہنگ{harmonic} سے مراد لاپلاس مساوات کا حل ہے جس کے \عددی{T} میں استمراری دو درجی جزوی تفرق پائے جاتے ہوں۔

%===================
\ابتدا{سوال}\شناخت{سوال_خطی_تکمل_فقرے_ثبوت_الف}\quad اگر کسی خطہ جس کا \عددی{T} حصہ ہو  میں \عددی{g} ہارمونی ہو تب درج ذیل ہو گا۔
\begin{align*}
\iint\limits_S \frac{\partial g}{\partial n}\dif A=0
\end{align*}
جواب:\quad مساوات \حوالہ{مساوات_خطی_تکمل_گرین_کلیہ_اول} میں \عددی{f=1} پر کریں۔
\انتہا{سوال}
%=====================
\ابتدا{سوال}\شناخت{سوال_خطی_تکمل_فقرے_ثبوت_ب}\quad اگر کسی خطہ جس کا \عددی{T} حصہ ہو  میں \عددی{g} ہارمونی ہو تب درج ذیل ہو گا۔
\begin{align*}
\iint\limits_S g\frac{\partial g}{\partial n}\dif A=\iiint \limits_T \abs{\nabla g}^2 \dif H
\end{align*}
جواب:\quad مساوات \حوالہ{مساوات_خطی_تکمل_گرین_کلیہ_اول} میں \عددی{f=g} پر کریں۔
\انتہا{سوال}
%=====================
\ابتدا{سوال}\quad اگر کسی خطہ جس کا \عددی{T} حصہ ہو  میں \عددی{g} ہارمونی ہو  اور \عددی{S} پر  \عددی{\tfrac{\partial g}{\partial n}=0} ہو  تب \عددی{T} میں \عددی{g} ایک مستقل ہو گا۔ \\
جواب:\quad سوال \حوالہ{سوال_خطی_تکمل_فقرے_ثبوت_ب} کو استعمال کریں۔
\انتہا{سوال}
%=====================
\ابتدا{سوال}\quad  اگر کسی خطہ جس کا \عددی{T} حصہ ہو  میں \عددی{g} اور \عددی{f} ہارمونی ہوں اور \عددی{S} پر \عددی{\tfrac{\partial f}{\partial n}=\tfrac{\partial g}{\partial n}} ہو تب \عددی{T} میں \عددی{f=g+c} ہو گا جہاں \عددی{c} مستقل قیمت ہے۔

\انتہا{سوال}
%===================
\ابتدا{سوال}\شناخت{سوال_خطی_تکمل_فقرے_ثبوت_پ}\quad  اگر کسی خطہ جس کا \عددی{T} حصہ ہو  میں \عددی{g} اور \عددی{f} ہارمونی ہوں تب درج ذیل ہو گا۔
\begin{align*}
\iint\limits_S \big(f\frac{\partial g}{\partial n}-g\frac{\partial f}{\partial n}\big)\dif A=0
\end{align*}

\انتہا{سوال}
%===================
\ابتدا{سوال}\quad ثابت کریں کہ لاپلاسی کو محدد سے پاک صورت میں درج ذیل لکھا جا سکتا ہے
\begin{align*}
\nabla^{\,2}f=\lim_{d(T)\to 0}\frac{1}{H(T)}\iint\limits_{S(T)} \frac{\partial f}{\partial n}\dif A
\end{align*}
جہاں جس نقطے پر لاپلاسی درکار ہو، اس نقطے سے \عددی{T} میں دور ترین نقطے کا فاصلہ \عددی{d(T)} ہے اور \عددی{H(T)} خطہ \عددی{T} کا حجم ہے جس کی سرحدی سطح \عددی{S(T)} ہے۔ (اشارہ: مساوات \حوالہ{مساوات_خطی_تکمل_پھیلاو_مثال_پ} میں \عددی{\bM{u}=\nabla f} پر کرتے ہوئے \عددی{\bM{b}=\bM{n}} لیتے ہوئے مساوات \حوالہ{مساوات_الاحصاء_سمتی_تفرق_ج} استعمال کریں جہاں \عددی{\bM{n}} سطح \عددی{S} کا باہر رخ اکائی عمودی سمتیہ ہے۔)
\انتہا{سوال}
%========================

\حصہ{مسئلہ سٹوکس}
ہم نے حصہ \حوالہ{حصہ_خطی_تکمل_دوہرا_خطی_تبادل} میں دیکھا کہ مستوی پر دوہرا تکمل کو سطح کی سرحد پر خطی تکمل میں تبدیل کیا جا سکتا ہے۔آئیں اس نتیجے کو عمومی بناتے ہوئے سطحی تکمل کے تبادل پر غور کریں۔

%================================
\ابتدا{مسئلہ}\quad مسئلہ سٹوکس\حاشیہد{آئرستانی ریاضی دان اور ماہر طبیعیات جارج جبرائیل سٹوکس [1819-1903]} (سطحی تکمل سے خطی تکمل اور خطی تکمل سے سطحی تکمل کا حصول)\\
فرض کریں کہ فضا میں ٹکڑوں میں ہموار سمت بند سطح \عددی{S} کی سرحد \عددی{C} ٹکڑوں میں ہموار سادہ بند منحنی  ہے۔ مزید فرض کریں کہ کسی ایسے خطہ میں جس کا \عددی{S} حصہ ہو، \عددی{\bM{v}(x,y,z)} استمراری سمتی تفاعل ہے اور اس خطے میں اس  کے استمراری ایک درجی جزوی تفرق پائے جاتے ہیں۔تب \اصطلاح{مسئلہ سٹوکس}\فرہنگ{مسئلہ!سٹوکس}\فرہنگ{سٹوکس!مسئلہ}\حاشیہب{Stokes' theorem}\فرہنگ{Stokes' theorem} کہتا ہے کہ
\begin{align}\label{مساوات_خطی_تکمل_سٹوکس_الف}
\iint\limits_S (\nabla \times \bM{v})_n \dif A=\int\limits_C v_t \dif s
\end{align}
ہو گا جہاں \عددی{S} کے اکائی عمودی سمتیہ \عددی{\bM{n}} کی سمت میں \عددی{\nabla \times \bM{v}} کا جزو \عددی{(\nabla \times \bM{v})_n=(\nabla \times \bM{v})\cdot \bM{n}} ہے؛ \عددی{C} پر تکمل کا رخ شکل \حوالہ{شکل_خطی_تکمل_مسئلہ_سٹوکس}-الف میں دکھایا گیا ہے جہاں \عددی{C} کی مماس \عددی{\bM{r}'} (شکل \حوالہ{شکل_خطی_تکمل_مسئلہ_سٹوکس}-الف) کی سمت میں  \عددی{\bM{v}} کا جزو \عددی{v_t} ہے ۔
\begin{figure}
\centering
\begin{subfigure}{0.5\textwidth}
\centering
\begin{tikzpicture}
\draw[->-=0.6](0,0) to [out=-90,in=140] (0.2,-0.35) to [out=-40,in=180](0.6,-0.5) to [out=0,in=180](1.5,-0.3) to [out=0,in=180](2.7,-0.4) to [out=0,in=-90](3,-0.2) to [out=90,in=0](1.5,0.75) to [out=180,in=90](0,0);
\draw[-latex](0.2,-0.35)node[ocirc]{}--++(-40:0.75)node[right]{$\bM{r}'$};
\draw[-latex](1.5,0.25)node[ocirc]{}--++(70:1)node[left]{$\bM{n}$};
\draw(2,0)node{$S$};
\draw(2.5,0.5)node[above]{$C$};
%
\begin{scope}[shift={(0,-2cm)}]
\draw[-<-=0.6,name path=kA](0,0) to [out=-90,in=140] (0.2,-0.35) to [out=-40,in=180](0.6,-0.5) to [out=0,in=180](1.5,-0.3) to [out=0,in=180](2.7,-0.4) to [out=0,in=-90](3,-0.2) to [out=90,in=0](1.5,0.75) to [out=180,in=90](0,0);
\draw[-latex](0.2,-0.35)node[ocirc]{}--++(140:0.75)node[above]{$\bM{r}'$};
\path[name path=kB](1.5,0.25)--++(70:-1);
\draw[dashed,name intersections={of=kA and kB}](1.5,0.25)node[ocirc,solid]{}--(intersection-1);
\draw[-latex](intersection-1)--++(70:-0.5)node[right]{$\bM{n}$};
\draw(2,0)node{$S$};
\draw(2.5,0.5)node[above]{$C$};
\end{scope}
\end{tikzpicture}
\caption*{(الف) سمت بندی (مسئلہ سٹوکس)}
\end{subfigure}%
\begin{subfigure}{0.5\textwidth}
\centering
\begin{tikzpicture}
\draw[->-=0.25](0,0) to [out=-90,in=140] (0.2,-0.35) to [out=-40,in=180](0.6,-0.5) to [out=0,in=180](1.5,-0.3) to [out=0,in=180](2.7,-0.4) to [out=0,in=-90](3,-0.2) to [out=90,in=0](1.5,0.75) to [out=180,in=90](0,0);
\draw[dashed](1.5,0.25)--++(0,1.3);
\draw[-latex](1.5,0.25)node[ocirc]{}--++(60:1.3)node[right]{$\bM{n}$};
\draw([shift={(60:0.65)}]1.5,0.25) arc (60:90:0.65);
\draw(1.5,0.25)++(75:0.9)node{$\gamma$};
\draw(2.2,0.2)node{$S$};
\draw(1.5,-0.3,0)node[below]{$C$};
%axis
\draw(-0.5,-1.25)coordinate(kO)--(-0.1,-1.25)(0.1,-1.25)--(2.9,-1.25)(3.1,-1.25)--(4,-1.25)node[right]{$y$};
\draw(kO)--++(0,1.5)node[left]{$z$};
\draw(kO)node[ocirc]{}--++(-0.75,-0.75)node[left]{$x$};
%
\draw[dashed](0,0)--++(0,-2);
\draw[dashed](3,-0.2)--++(0,-2);
%
\begin{scope}[shift={(0,-2cm)}]
\draw[->-=0.25,name path=kA](0,0) to [out=-90,in=140] (0.2,-0.3) to [out=-40,in=180](0.6,-0.4) to [out=0,in=180](1.5,-0.2) to [out=0,in=180](2.7,-0.3) to [out=0,in=-90](3,-0.15) to [out=90,in=0](1.5,0.6) to [out=180,in=90](0,0);
\draw(2.2,0.2)node{$S^*$};
\draw(1.5,-0.2)node[below]{$C^*$};
\end{scope}
\end{tikzpicture}
\caption*{(ب) ثبوت مسئلہ سٹوکس}
\end{subfigure}%
\caption{مسئلہ سٹوکس}
\label{شکل_خطی_تکمل_مسئلہ_سٹوکس}
\end{figure}
\انتہا{مسئلہ}
%===============================
\ابتدا{ثبوت}
ہم مسئلہ سٹوکس کو ایسی سطح \عددی{S} کے لئے ثابت کرتے ہیں جس کو درج ذیل تینوں طریقوں سے ظاہر کرنا ممکن ہو
\begin{align}\label{مساوات_خطی_تکمل_سٹوکس_ب}
\text{(الف)}\,\,z=f(x,y),\quad \text{(ب)}\,\,y=g(x,z),\quad \text{(پ)}\,\,x=h(y,z)
\end{align}
جہاں \عددی{f}، \عددی{g} اور \عددی{h} اپنے آزاد متغیرات کے استمراری تفاعل ہیں اور ان کے استمراری ایک درجی جزوی تفرقات پائے جاتے ہیں۔فرض کریں کہ \عددی{S} کا بالائی رخ اکائی عمودی سمتیہ \عددی{\bM{n}} درج ذیل
\begin{align}\label{مساوات_خطی_تکمل_سٹوکس_پ}
\bM{n}=\cos \alpha \,\bM{i}+\cos \beta \,\bM{j}+\cos \gamma \,\bM{k}
\end{align}
ہے (شکل \حوالہ{شکل_خطی_تکمل_مسئلہ_سٹوکس}-ب) اور \عددی{\bM{v}=v_1\bM{i}+v_2\bM{j}+v_3\bM{k}} ایک سمتی تفاعل ہے۔اگر \عددی{C} کو \عددی{\bM{r}=\bM{r}(s)} لکھا جائے جہاں قوس لمبائی \عددی{s} تکمل کے رخ بڑھتی ہو تب اکائی مماسی سمتیہ
\begin{align*}
\frac{\dif \bM{r}}{\dif s}=\frac{\dif x}{\dif s}\bM{i}+\frac{\dif y}{\dif s}\bM{j}+\frac{\dif z}{\dif s}\bM{k}
\end{align*}
ہو گا لہٰذا
\begin{align*}
v_t=\bM{v}\cdot \frac{\dif \bM{r}}{\dif s}=v_1\frac{\dif x}{\dif s}+v_2\frac{\dif y}{\dif s}+v_3\frac{\dif z}{\dif s}
\end{align*}
لکھا جا سکتا ہے جس سے درج ذیل ملتا ہے۔
\begin{align*}
v_t \dif s=\bM{v}\cdot \frac{\dif \bM{r}}{\dif s}\dif s=v_1\dif x+v_2\dif y+v_3\dif z
\end{align*}
یوں  گردش کو دائیں ہاتھ کارتیسی نظام (حصہ \حوالہ{حصہ_الاحصاء_غیر_سمتی-_اور_سمتی_میدان}) میں لکھتے ہوئے مسئلہ سٹوکس کی مساوات کو درج ذیل لکھا جا سکتا ہے
\begin{multline}\label{مساوات_خطی_تکمل_سٹوکس_ت}
\iint\limits_S\big[\big(\frac{\partial v_3}{\partial y}-\frac{\partial v_2}{\partial z}\big)\cos \alpha+\big(\frac{\partial v_1}{\partial z}-\frac{\partial v_3}{\partial x}\big)\cos \beta+\big(\frac{\partial v_2}{\partial x}-\frac{\partial v_1}{\partial y}\big)\cos \gamma\big]\dif A\\
=\int\limits_C (v_1\dif x+v_2\dif y+v_3\dif z)
\end{multline} 
جہاں \عددی{\alpha}، \عددی{\beta}، \عددی{\gamma} مساوات \حوالہ{مساوات_خطی_تکمل_سٹوکس_پ} میں بیان کیے گئے ہیں۔

ہم ثابت کرتے ہیں کہ مساوات \حوالہ{مساوات_خطی_تکمل_سٹوکس_ت} میں دونوں اطراف وہ تکمل جن میں \عددی{v_1} پایا جاتا ہے عین برابر ہیں یعنی:
\begin{align}\label{مساوات_خطی_تکمل_سٹوکس_ٹ}
\iint\limits_S \big(\frac{\partial v_1}{\partial z}\cos \beta-\frac{\partial v_1}{\partial y}\cos \gamma\big)\dif A=\int\limits_C v_1\dif x
\end{align}
فرض کریں کہ \عددی{xy} مستوی پر \عددی{S} کا قائمہ سایہ \عددی{S^*} ہے جس کی سرحد \عددی{C^*}  کی سمت بندی  شکل \حوالہ{شکل_خطی_تکمل_مسئلہ_سٹوکس}-ب میں دکھائی گئی ہے۔\عددی{S} کو مساوات \حوالہ{مساوات_خطی_تکمل_سٹوکس_ب}-الف سے ظاہر کرتے ہوئے \عددی{C} پر خطی تکمل کو \عددی{C^*} پر خطی تکمل لکھتے ہیں۔
\begin{align*}
\int\limits_C v_1(x,y,a)\dif x=\int\limits_{C^*}v_1[x,y,f(x,y)]\dif x
\end{align*}
اب مسئلہ گرین (حصہ \حوالہ{حصہ_خطی_تکمل_دوہرا_خطی_تبادل}) کو [\عددی{f} اور \عددی{g} کی بجائے] \عددی{v_1[x,y,f(x,y)]} اور \عددی{0} پر لاگو کرتے ہوئے درج ذیل لکھا جا سکتا ہے۔
\begin{align*}
\int\limits_{C^*}v_1[x,y,f(x,y)]\dif x=-\iint\limits_{S^*}\frac{\partial v_1}{\partial y}\dif x\dif y
\end{align*}
دائیں ہاتھ تکمل میں
\begin{align*}
\frac{\partial v_1[x,y,f(x,y)]}{\partial y}=\frac{\partial v_1(x,y,z)}{\partial y}+\frac{\partial v_1(x,y,z)}{\partial z}\frac{\partial f}{\partial y}\quad\quad [z=f(x,y)]
\end{align*}
لکھتے ہوئے درج ذیل ملتا ہے۔
\begin{align}\label{مساوات_خطی_تکمل_سٹوکس_ث}
\int\limits_C v_1(x,y,z)\dif x=-\iint\limits_{S^*}\big(\frac{\partial v _1}{\partial y}+\frac{\partial v_1}{\partial z}\frac{\partial f}{\partial y}\big)\dif x\dif y
\end{align}

ہم ثابت کرتے ہیں کہ مساوات \حوالہ{مساوات_خطی_تکمل_سٹوکس_ٹ} کے بائیں ہاتھ کا تکمل مساوات \حوالہ{مساوات_خطی_تکمل_سٹوکس_ث} کے دائیں ہاتھ کے تکمل کے برابر ہے۔ہم پہلے تکمل میں \عددی{x} اور \عددی{y} کو بطور متغیرات تکمل متعارف کرتے ہیں۔مساوات \حوالہ{مساوات_خطی_تکمل_سٹوکس_ب}-الف کو
\begin{align*}
F(x,y,z)=z-f(x,y)=0
\end{align*}
لکھتے ہوئے
\begin{align*}
\nabla F=-\frac{\partial f}{\partial x}\,\bM{i}-\frac{\partial F}{\partial y}\,\bM{j}+\bM{k}
\end{align*}
ملتا ہے جس سے ڈھلوان \عددی{F} کی لمبائی \عددی{a} لکھتے ہیں۔
\begin{align*}
a=\abs{\nabla F}=\sqrt{1+\big(\frac{\partial f}{\partial x}\big)^2+\big(\frac{\partial f}{\partial y}\big)^2}
\end{align*}
چونکہ \عددی{\nabla F} سطح \عددی{S} کو عمودی ہے لہٰذا  \عددی{S} کی اکائی عمودی سمتیات \عددی{\bM{n}} درج ذیل حاصل ہوتی ہیں۔
\begin{align*}
\bM{n}=\mp \frac{\nabla F}{a}
\end{align*}
اب مثبت \عددی{z} رخ میں \عددی{\bM{n}} اور \عددی{\nabla F} دونوں کے اجزاء مثبت ہیں لہٰذا
\begin{align*}
\bM{n}=+ \frac{\nabla F}{a}
\end{align*}
ہو گا۔ \عددی{xyz} کارتیسی نظام میں \عددی{\bM{n}} اور \عددی{\nabla F} کی روپ سے یوں درج ذیل لکھا جا سکتا ہے۔
\begin{align*}
\cos \alpha=-\frac{1}{a}\frac{\partial f}{\partial x},\quad  \cos \beta=-\frac{1}{a}\frac{\partial f}{\partial y},\quad \cos \gamma=\frac{1}{a}
\end{align*}
مزید مساوات \حوالہ{مساوات_خطی_تکمل_صریح_تفاعل_رقبہ_ب} کے تحت مساوات \حوالہ{مساوات_خطی_تکمل_سٹوکس_ٹ} میں \عددی{\dif A=a\dif x\dif y} ہو گا لہٰذا
\begin{align*}
\iint\limits_S \big(\frac{\partial v_1}{\partial z}\cos \beta-\frac{\partial v_1}{\partial y}\cos \gamma\big)\dif A=\iint\limits_{S^*}\big[\frac{\partial v_1}{\partial z}\big(-\frac{1}{a}\frac{\partial f}{\partial y}\big)-\frac{\partial v_1}{\partial y}\frac{1}{a}\big]a\dif x\dif y
\end{align*}
لکھا جا سکتا ہے جو مساوات \حوالہ{مساوات_خطی_تکمل_سٹوکس_ث} کے دائیں ہاتھ تکمل برابر ہے۔یوں مساوات \حوالہ{مساوات_خطی_تکمل_سٹوکس_ٹ} ثابت ہوتی ہے۔

اگر \عددی{-\bM{n}} کو مثبت اکائی عمودی سمتیہ چنا جاتا تب \عددی{C} کی مثبت سمت الٹ رخ ہوتی لہٰذا حاصل جواب پر کوئی اثر نہ ہوتا۔یوں مساوات \حوالہ{مساوات_خطی_تکمل_سٹوکس_ٹ} \عددی{S} کے دونوں مثبت اکائی عمودی سمتیات کے لئے درست ہے۔

 مساوات \حوالہ{مساوات_خطی_تکمل_سٹوکس_ب}-ب اور مساوات \حوالہ{مساوات_خطی_تکمل_سٹوکس_ب}-پ میں دیے روپ استعمال کر کر بالکل اسی طرح درج ذیل ثابت ہوں گے۔
\begin{align}
\iint\limits_S \big(\frac{\partial v_2}{\partial x}\cos \gamma-\frac{\partial v_2}{\partial z}\cos \alpha\big)\dif A
&=\int\limits_C v_2\dif y \label{مساوات_خطی_تکمل_سٹوکس_ج}\\
\iint\limits_S \big(\frac{\partial v_3}{\partial y}\cos \alpha-\frac{\partial v_3}{\partial x}\cos \beta\big)\dif A
&=\int\limits_C v_3\dif z\label{مساوات_خطی_تکمل_سٹوکس_چ}
\end{align}
مساوات \حوالہ{مساوات_خطی_تکمل_سٹوکس_ٹ}، مساوات \حوالہ{مساوات_خطی_تکمل_سٹوکس_ج} اور مساوات \حوالہ{مساوات_خطی_تکمل_سٹوکس_چ} جمع کرتے ہوئے مساوات \حوالہ{مساوات_خطی_تکمل_سٹوکس_الف} ملتا ہے۔اس طرح مسئلہ سٹوکس ایسی سطح  \عددی{S} کے لئے ثابت ہوتا ہے جس کو بیک وقت   مساوات \حوالہ{مساوات_خطی_تکمل_سٹوکس_ب}-الف،  مساوات \حوالہ{مساوات_خطی_تکمل_سٹوکس_ب}-ب اور  مساوات \حوالہ{مساوات_خطی_تکمل_سٹوکس_ب}-پ کی روپ  میں لکھنا ممکن ہو۔

مسئلہ پھیلاو کی طرح موجودہ ثبوت کو وسعت دیتے ہوئے اسے ایسی سطح پر لاگو کیا جا سکتا ہے جس کو محدود تعداد کے ایسی ٹکڑوں میں تقسیم کرنا ممکن ہو کہ ہر ٹکڑے کو مساوات \حوالہ{مساوات_خطی_تکمل_سٹوکس_ب} کی روپ میں لکھا جا سکے۔عموماً عملاً استعمال کی سطحیں ایسی ہی ہوتی ہیں۔

مسئلہ کو ایسی عمومی سطح \عددی{S} جو مسئلہ کی شرائط پر پورا اترتا ہو کے  لئے ثابت کرنے کی خاطر ہم  \عددی{S} کو تخمیناً ایسی سطحوں میں تقسیم کرتے ہیں جو ان شرائط پر پورا اترتے ہوں اور تحدیدی طریقہ اختیار کرتے ہیں۔ یہ ترکیب مسئلہ گرین کی ثبوت میں اختیار کی گئی ترکیب کی طرح ہے۔
\انتہا{ثبوت}
%============================

\حصہ{مسئلہ سٹوکس کے نتائج اور عملی استعمال}
%======================
\ابتدا{مثال}\quad سطح میں مسئلہ گرین درحقیقت مسئلہ سٹوکس کی خصوصی شکل ہے\\
فرض کریں کہ سمتی تفاعل \عددی{\bM{v}=v_1\bM{i}+v_2\bM{j}+v_3\bM{k}} مستوی \عددی{xy}  میں کسی ایسا خطہ میں استمراری قابل تفرق ہے  جس میں سادہ تعلق بند محدود خطہ \عددی{S} پایا جاتا ہے جس کی سرحد \عددی{C} ٹکڑوں میں ہموار بند سادہ منحنی ہے۔تب مساوات \حوالہ{مساوات_سمتی_تفرق_گردش_تعریف} کے تحت
\begin{align*}
(\nabla \times \bM{v})_n=\frac{\partial v_2}{\partial x}-\frac{\partial v_1}{\partial y}
\end{align*}
ہو گا۔مزید \عددی{v_t\dif s=v_1\dif x+v_2\dif y} لیتے ہوئے  مساوات \حوالہ{مساوات_خطی_تکمل_سٹوکس_الف} درج ذیل لکھا جائے گا
\begin{align*}
\iint\limits_S \big(\frac{\partial v_2}{\partial x}_\frac{\partial v_1}{\partial y}\big)\dif A=\int\limits_C (v_1\dif x+v_2\dif y)
\end{align*}
جو سطح میں مسئلہ گرین (حصہ \حوالہ{حصہ_خطی_تکمل_دوہرا_خطی_تبادل}) ہے۔
\انتہا{مثال}
%=============================
\ابتدا{مثال}\شناخت{مثال_خطی_تکمل_گردش_مفہوم}\quad گردش کا طبعی مفہوم\\
فرض کریں کہ رداس \عددی{r} کے دائری قرص \عددی{S_r} کا مرکز \عددی{N} اور سرحد دائرہ \عددی{C_r} ہے (شکل \حوالہ{شکل_مثال_خطی_تکمل_گردش_مفہوم})۔مزید فرض کریں کہ کسی خطہ جس کا \عددی{S_r}  حصہ ہو میں \عددی{\bM{v}(Q)=\bM{v}(x,y,z)} استمراری قابل تفرق سمتی تفاعل  ہے۔تب مسئلہ سٹوکس اور سطحی تکمل کے اوسط قیمت مسئلہ کے تحت درج ذیل ہو گا
\begin{align*}
\int\limits_{C_r} v_t\dif s=\iint\limits_{S_r} (\nabla \times \bM{v})_n\dif A=[\nabla \times \bM{v}(N^*)]_n A_r
\end{align*}
جہاں \عددی{S_r} کا رقبہ \عددی{A_r} اور \عددی{S_r} میں \عددی{N^*} کوئی موزوں نقطہ ہے۔اس کو یوں
\begin{align*}
[\nabla \times \bM{v}(N^*)]_n=\frac{1}{A_r}\int\limits_{C_r} v_t\dif s
\end{align*}
بھی لکھا جا سکتا ہے۔سیال کی حرکت کی صورت میں تکمل
\begin{align*}
\int\limits_{C_r}v_t\dif s
\end{align*}
\عددی{C_r} پر سیال کی \اصطلاح{دائری بہاو}\فرہنگ{دائری بہاو}\حاشیہب{circulation}\فرہنگ{circulation} کی ناپ ہے۔اب  \عددی{r} کو صفر مانند کرنے سے
\begin{align}
[\nabla \times \bM{v}(N)]_n=\lim_{r\to 0}\frac{1}{A_r}\int\limits_{C_r} v_t\dif s
\end{align}
ملتا ہے جس کو \عددی{N} پر فی اکائی رقبہ  دائری بہاو کہا جا سکتا ہے۔ 
% 
\begin{figure}
\centering
\begin{tikzpicture}
\begin{scope}[rotate=45]
\draw[->-=0.725](0,0)node[right]{$N$} ellipse (1cm and 0.5cm);
\draw[-latex](0,0)node[ocirc]{}--++(90:1)node[left]{$\bM{n}$};
\draw(-0.25,-0.75)node{$C_r$};
\end{scope}
\end{tikzpicture}
\caption{قرص (مثال \حوالہ{مثال_خطی_تکمل_گردش_مفہوم})}
\label{شکل_مثال_خطی_تکمل_گردش_مفہوم}
\end{figure}
\انتہا{مثال}
%========================
\ابتدا{مثال}\quad خطی تکمل کا حصول بذریعہ مسئلہ سٹوکس\\
تکمل \عددی{\int_C v_t \dif s} حل کریں جہاں مبدا سے دیکھتے ہوئے  دائیں ہاتھ کارتیسی نظام میں دائرہ  \عددی{C:\,x^2+y^2=4,\, z=-3} گھڑی کی الٹ رخ سمت بند ہے جبکہ \عددی{\bM{v}} درج ذیل ہے۔
\begin{align*}
\bM{v}=y\bM{i}+xz^3\bM{j}-zy^3\bM{k}
\end{align*}
ہم \عددی{C} کے احاطہ \عددی{S} کو مستوی دائری قرص \عددی{x^2+y^2\le 4,\, z=-3} لیتے ہیں۔یوں مسئلہ سٹوکس میں \عددی{\bM{n}} کی سمت مثبت \عددی{z} رخ ہو گی لہٰذا \عددی{\bM{n}=\bM{k}} ہو گا۔یوں \عددی{(\nabla \times \bM{v})_n} سے مراد مثبت \عددی{z} رخ میں \عددی{\nabla \times \bM{v}} کا جزو۔چونکہ \عددی{z=-3} پر کرتے ہوئے \عددی{\bM{v}} کے اجزاء \عددی{v_1=y}، \عددی{v_2=-27x} اور \عددی{v_3=3y^3} ملتے ہیں لہٰذا
\begin{align*}
(\nabla \times \bM{v})_n=\frac{\partial v_2}{\partial x}-\frac{\partial v_1}{\partial y}=-27-1=-28
\end{align*}
ہو گا۔یوں مسئلہ سٹوکس میں تکمل  کی قیمت قرص کا رقبہ   ضرب \عددی{-28}  یعنی \عددی{-112\pi} ہو گی۔

یہاں مسئلہ سٹوکس کی افادیت جاننے کی خاطر آپ سے گزارش کی جاتی ہے کہ مسئلہ سٹوکس استعمال کیے بغیر اس تکمل کو حل کریں۔
\انتہا{مثال}
%============================

\حصہء{سوالات}

