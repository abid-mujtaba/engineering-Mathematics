%draws left and right arrows where needed e.g.  
% \draw[->-=0.5] (0,0)--(3,0); draws arrow at the middle
\tikzset{->-/.style={decoration={markings, mark=at position #1 with {\arrow{latex}}},postaction={decorate}}}
\tikzset{-<-/.style={decoration={markings, mark=at position #1 with {\arrow{latex reversed}}},postaction={decorate}}}


%draws right angles \RightAngle{A}{B}{C}
\providecommand{\RightAngle}[4][5pt]{\draw[gray] ($#3!#1!#2$)--($ #3!2!($($#3!#1!#2$)!.5!($#3!#1!#4$)$) $) --($#3!#1!#4$) ;     }
%colours
\definecolor{lgray}{cmyk}{0,0,0,0.2}
\definecolor{dgray}{cmyk}{0,0,0,0.7}
%draws a cross just like ocirc, circ; usage \fill(0,2)circle(1.5py);\draw(0,2)node[kcross]{};
\tikzset{kcross/.style={cross out, draw, 
         minimum size=2*(2pt-\pgflinewidth), 
         inner sep=0pt, outer sep=0pt}}


%tikz, pgfplot TABLE
\pgfplotsset{select coords between index/.style 2 args={
    x filter/.code={
        \ifnum\coordindex<#1\def\pgfmathresult{}\fi
        \ifnum\coordindex>#2\def\pgfmathresult{}\fi
    }
}}
%
%boxed circuits
%=========================================
%\leftBox[K]{3,2}   draws a box with lower end at (3,2) and the terminals called Ka and Kb
\newcommand{\boxLeft}[2][p]{
\coordinate (a) at (#2);
\draw (a)++(-0.025,0.5) coordinate (b);
\draw (a)++(-0.04,1) coordinate (c);
\draw (a)++(-0.12,1.5) coordinate (d);
\draw (a)++(-0.2,2) coordinate (e);
\draw (a)++(-0.15,2.5) coordinate (f);
\draw (a)++(0.5,3) coordinate (g);

\draw (a)++(0.7,2.5)coordinate(h);
\draw (a)++(0.6,2)coordinate(i);
\draw (a)++(0.75,1.5)coordinate(j);
\draw (a)++(0.7,1)coordinate(k);
\draw (a)++(0.7,0.5)coordinate(l);
\draw (a)++(0.6,0)coordinate(m);
\draw plot [smooth cycle] coordinates {(a) (b) (c) (d) (e) (f) (g) (h) (i) (j) (k) (l) (m)};
\draw (h)coordinate(#1a);
\draw(l)coordinate(#1b);
}
%===================
%\rightBox[J]{3,2}   draws a box with lower end at (3,2) and the terminals called Ja and Jb
\newcommand{\boxRight}[2][p]{
\coordinate (aa) at (#2);
\draw (aa)++(0.025,0.5) coordinate(ba);
\draw (aa)++(0.04,1)coordinate(ca);
\draw (aa)++ (0.12,1.5)coordinate(da);
\draw (aa)++(0.13,2)coordinate(ea);
\draw (aa)++(0.1,2.5)coordinate(fa);
\draw (aa)++(-0.5,3)coordinate(ga);

\draw (aa)++(-0.8,2.5) coordinate(ha);
\draw (aa)++(-0.8,2) coordinate(ia);
\draw (aa)++ (-0.75,1.5) coordinate(ja);
\draw (aa)++(-0.7,1) coordinate(ka);
\draw (aa)++(-0.7,0.5) coordinate(la);
\draw (aa)++(-0.5,0) coordinate(ma);
\draw plot [smooth cycle] coordinates {(aa) (ba) (ca) (da) (ea) (fa) (ga) (ha) (ia) (ja) (ka) (la) (ma)};
\draw (ha)coordinate(#1a);
\draw(la)coordinate(#1b);
}
%===================
%writes text above matrix entries (outside the matrix bars)
\newcommand\bovermat[2]{%
  \makebox[0pt][r]{$\raisebox{16pt}[0pt][0pt]{\text{\RL{#1}}}$}#2}
\newcommand\covermat[2]{%
  \makebox[0pt][c]{$\raisebox{16pt}[0pt][0pt]{\text{\RL{#1}}}$}#2}
\newcommand\partialphantom{\vphantom{\frac{\partial e_{P,M}}{\partial w_{1,1}}}}

%=============================
%when a table is all math, instead of using $$ in each cell use the following.Text can be entered in a cell with \text{} 
%usage \begin{matrix}{C|L} ;not needed in array as array is $$ by default
\newcolumntype{L}{>{$}l<{$}}
\newcolumntype{C}{>{$}c<{$}}
\newcolumntype{R}{>{$}r<{$}}
