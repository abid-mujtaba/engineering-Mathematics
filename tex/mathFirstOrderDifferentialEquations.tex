\باب{پہلے درجے کے تفرقی مساوات}
عموماً طبعی تعلقات کو تفرقی مساوات کی صورت میں لکھا جا سکتا ہے۔اسی طرح عموماً انجنیئرنگ مسائل تفرقی مساوات کی صورت میں پیش آتے ہیں۔اسی لئے  اس کتاب کی ابتدا تفرقی مساوات اور ان کے حل سے کی جاتی ہے۔

\اصطلاح{سادہ تفرقی مساوات}\فرہنگ{تفرقی!سادہ مساوات}\حاشیہب{ordinary differential equation}\فرہنگ{differential!ordinary equation} سے مراد ایسی تفرقی مساوات ہے جس میں ایک عدد آزاد متغیرہ پایا جاتا ہو۔اس کے برعکس \اصطلاح{جزوی تفرقی مساوات}\فرہنگ{تفرقی!جزوی مساوات}\حاشیہب{partial differential equation}\فرہنگ{differential!partial equation} ایک سے زائد آزاد متغیرات پر منحصر ہوتی ہے۔جزوی تفرقی مساوات کا حل نسبتاً مشکل ثابت ہوتا ہے۔

کسی بھی حقیقی صورت حال یا مشاہدے کی نقشہ کشی کرتے ہوئے  اس کا \اصطلاح{ریاضی نمونہ}\فرہنگ{نمونہ!ریاضی}\حاشیہب{mathematical model}\فرہنگ{model!mathematical} حاصل کیا جا سکتا ہے۔سائنس کے مختلف میدان مثلاً انجنیئرنگ، طبیعیات، علم کیمیا، حیاتیات، کمپیوٹر وغیرہ میں درپیش مسائل کی صحیح تفرقی مساوات کا حصول اور ان کے حل پر تفصیلاً غور کیا جائے گا۔

باب-20 میں سادہ تفرقی مساوات کا حل بذریعہ کمپیوٹر  پیش کیا جائے گا۔یہ باب بقایا کتاب سے مکمل طور پر علیحدہ رکھا گیا ہے۔ یوں کتاب کے پہلے  دو باب کے بعد باب-20 پڑھا جا سکتا ہے۔

پہلے باب کا آغاز  درجہ اول کے سادہ تفرقی مساوات کے حصول، مساوات کے حل اور حل کی تشریح  سے کیا جاتا ہے۔پہلے درجے کی سادہ تفرقی مساوات میں صرف ایک عدد نا معلوم تفاعل کا ایک درجی تفرق پایا جاتا ہے۔ایسی مساوات میں ایک سے زیادہ درجے کا تفرق نہیں پایا جاتا۔نا معلوم تفاعل کو \عددی{y(t)} یا \عددی{y(x)} سے ظاہر کیا جائے گا جہاں غیر تابع متغیرہ  \عددی{t}  وقت کو ظاہر کرتی ہے۔باب کے اختتام میں تفرقی مساوات کے حل کی \اصطلاح{وجودیت}\فرہنگ{وجودیت}\حاشیہب{existence}\فرہنگ{existence}  اور \اصطلاح{یکتائی}\فرہنگ{یکتائی}\حاشیہب{uniqueness}\فرہنگ{uniqueness} پر غور کیا جائے گا۔

تفرقی مساوات سمجھنے کی خاطر ضروری ہے کہ انہیں کاغذ اور قلم سے حل کیا جائے البتہ کمپیوٹر کی مدد سے آپ حاصل جواب کی درستگی دیکھنا چاہیں تو اس میں کوئی حرج نہیں ہے۔

\حصہ{نمونہ کشی}
شکل \حوالہ{شکل_سادہ_اول_نمونہ_کشی_عمل_الف} کو دیکھیے۔ انجنیئرنگ مسئلے کا حل تلاش کرنے میں پہلا قدم مسئلے کو مساوات کی صورت میں بیان کرنا ہے۔مسئلے کو مختلف متغیرات اور تفاعل کے تعلقات کی صورت میں لکھا جاتا ہے۔اس مساوات کو \اصطلاح{ریاضی نمونہ}\فرہنگ{نمونہ!ریاضی}\حاشیہب{mathematical model}\فرہنگ{model!mathematical} کہا جاتا ہے۔ریاضی نمونے کا حصول، نمونے کا ریاضیاتی حل اور حل کی تشریح کے عمل کو \اصطلاح{نمونہ کشی}\فرہنگ{نمونہ کشی}\حاشیہب{modeling}\فرہنگ{modeling} کہا جاتا ہے۔
\begin{figure}
\centering
\begin{tikzpicture}[node distance = 2.5cm, auto]
\tikzstyle{block} = [rectangle, draw, 
    text width=5em, text centered, rounded corners, minimum height=4em]
 \node [block] (ka) {\RL{طبعی نظام}};
\node[block,right of=ka](kb){\RL{ریاضی نمونہ}};
\node[block,right of=kb](kc){\RL{ریاضی حل}};
\node[block,right of=kc](kd){\RL{تشریح}};
\draw[-latex] (ka)--(kb);
\draw[-latex] (kb)--(kc);
\draw[-latex] (kc)--(kd);
\end{tikzpicture}
\caption{نمونہ کشی، حل اور تشریح۔}
\label{شکل_سادہ_اول_نمونہ_کشی_عمل_الف}
\end{figure}
نمونہ کشی کی صلاحیت تجربے سے حاصل ہوتی ہے۔کسی بھی نمونہ کی حل میں کمپیوٹر مدد کر سکتا ہے البتہ نمونہ کشی میں کمپیوٹر عموماً کوئی مدد فراہم نہیں کر پاتا۔

عموماً طبعی مقدار مثلاً اسراع اور رفتار درحقیقت میں تفرق کو ظاہر کرتے ہیں لہٰذا بیشتر ریاضی نمونے مختلف متغیرات اور تفاعل کے تفرق پر مشتمل ہوتے ہیں جنہیں \اصطلاح{تفرقی مساوات}\فرہنگ{تفرقی مساوات}\حاشیہب{differential equation}\فرہنگ{differential equation} کہا جاتا ہے۔تفرقی مساوات کے حل سے مراد ایسا تفاعل ہے جو اس تفرقی مساوات پر پورا اترتا ہو۔تفرقی مساوات کا حل جانتے ہوئے مساوات میں موجود متغیرات اور تفاعل کے ترسیم کھینچے جا سکتا ہے اور ان پر غور کیا جا سکتا ہے۔تفرقی مساوات پر غور سے پہلے چند بنیادی تصورات تشکیل دیتے ہیں جو اس باب میں استعمال کی جائیں گی۔


\اصطلاح{سادہ تفرقی مساوات}\فرہنگ{تفرقی!سادہ مساوات} سے مراد ایسی مساوات ہے جس میں نا معلوم تفاعل کی ایک درجی یا بلند درجی تفرق پائے جاتے ہوں۔نا معلوم تفاعل کو \عددی{y(t)} یا \عددی{y(x)} سے ظاہر کیا جائے گا جہاں غیر تابع متغیر \عددی{t} وقت کو ظاہر کرتی ہے۔اس مساوات میں نا معلوم تفاعل \عددی{y} اور غیر تابع متغیرہ \عددی{x} (یا \عددی{t}) کے تفاعل بھی پائے جا سکتے ہیں۔درج ذیل چند سادہ تفرقی مساوات ہیں
\begin{align}
&y'=\sin x \label{مساوات_سادہ_اول_اول_درجہ}\\
&y'+\frac{6}{7}y=4e^{-\frac{3}{2}x} \label{مساوات_سادہ_اول_دوسرا_درجہ}\\
&y'\!'\!'+2y'-11y'^2=0 \label{مساوات_سادہ_اول_تیسرا_درجہ}
\end{align}
جہاں \عددی{y'=\tfrac{\dif y}{\dif x}}، \عددی{y'\!'=\tfrac{\dif{\,^2} y}{\dif x^2}}، وغیرہ ہیں۔

دو یا دو سے زیادہ  متغیرات کے تابع تفاعل کے تفرق پر مشتمل مساوات کو جزوی تفرقی مساوات کہتے ہیں۔ان کا حل سادہ تفرقی مساوات  سے زیادہ مشکل ثابت ہوتا ہے۔جزوی تفرقی مساوات پر بعد میں غور کیا جائے گا۔غیر تابع متغیرات \عددی{x} اور \عددی{y} پر منحصر تابع تفاعل \عددی{u(x,y)} کی جزوی تفرقی مساوات کی مثال درج ذیل ہے۔
\begin{align}
\frac{\partial^2 u}{\partial x^2}+\frac{\partial^2 u}{\partial y^2}=u
\end{align}
\عددی{n} درجی تفرقی مساوات سے مراد ایسی مساوات ہے جس میں نا معلوم تفاعل \عددی{y} کی بلند تر تفرق \عددی{n} درجے کی ہو۔یوں مساوات \حوالہ{مساوات_سادہ_اول_اول_درجہ} اول درجے کی مساوات ہے، مساوات \حوالہ{مساوات_سادہ_اول_دوسرا_درجہ} دوسرے درجے  جبکہ مساوات \حوالہ{مساوات_سادہ_اول_تیسرا_درجہ} تیسرے درجے کی مساوات ہے۔

اس باب میں پہلے درجے کی سادہ تفرقی مساوات پر غور کیا جائے گا۔ایسی مساوات میں اکائی درجہ تفرق \عددی{y'} کے علاوہ نا معلوم تفاعل \عددی{y} اور غیر تابع متغیرہ کا کوئی بھی تفاعل  پایا جا سکتا ہے۔ایک درجے کی سادہ تفرقی مساوات کو
\begin{align}\label{مساوات_سادہ_اول_درجہ_خفی}
F(y,y',x)=0
\end{align}
یا 
\begin{align}\label{مساوات_سادہ_اول_درجہ_صریح}
y'=f(x,y)
\end{align}
لکھا جا سکتا ہے۔مساوات \حوالہ{مساوات_سادہ_اول_درجہ_خفی} \اصطلاح{خفی}\فرہنگ{خفی}\حاشیہب{implicit}\فرہنگ{implicit}  صورت کہلاتی ہے جبکہ مساوات \حوالہ{مساوات_سادہ_اول_درجہ_صریح} \اصطلاح{صریح}\فرہنگ{صریح}\حاشیہب{explicit}\فرہنگ{explicit} صورت کہلاتی ہے۔یوں خفی مساوات\عددی{x^{2}y'-2y^3=0} کی صریح صورت \عددی{y'=2\tfrac{y^3}{x^2}} ہے۔

\حصہء{حل کا تصور}
ایک تفاعل
\begin{align}
y=h(x)
\end{align}
جو \اصطلاح{کھلے وقفہ}\فرہنگ{کھلا!وقفہ}\فرہنگ{وقفہ!کھلا}\حاشیہب{open interval}\فرہنگ{open!interval}\فرہنگ{interval!open} \عددی{a\le x \le b} پر \اصطلاح{معین}\فرہنگ{معین}\حاشیہب{defined}\فرہنگ{defined} ہو اور پورے وقفے پر اس کا تفرق پایا جاتا ہو، اس صورت مساوات \حوالہ{مساوات_سادہ_اول_درجہ_خفی} کا حل ہو گا جب \عددی{h(x)} اور \عددی{h'(x)} کو  مساوات \حوالہ{مساوات_سادہ_اول_درجہ_خفی} میں بالترتیب \عددی{y(x)} اور \عددی{y'(x)} کی جگہ پر کرنے سے مساوات کے دونوں اطراف برابر ہوں۔تفاعل \عددی{h(x)} کا خط \اصطلاح{منحنی حل}\فرہنگ{منحنی حل}\حاشیہب{solution curve}\فرہنگ{solution curve} کہلائے گا۔

یہاں کھلے وقفے سے مراد ایسا وقفہ ہے جس کے آخری سر \عددی{a} اور \عددی{b} وقفے کا حصہ نہ ہوں۔کھلا وقفہ لامتناہی ہو سکتا ہے مثلاً \عددی{-\infty \le x \le b} یا \عددی{a \le x \le \infty} اور  یا \عددی{-\infty \le x \le \infty} یعنی حقیقی محور۔
%====================
\ابتدا{مثال}
ثابت کریں کہ وقفہ \عددی{-\infty \le x \le \infty} پر تفاعل \عددی{y=cx} تفرقی مساوات \عددی{y=y' x} کا حل ہے جہاں \عددی{c} ایک مستقل ہے۔

حل:پورے وقفے پر \عددی{y=cx} معین ہے۔اسی طرح اس کا تفرق \عددی{y'=c} بھی پورے وقفے پر پایا جاتا ہے۔ان بنیادی شرائط پر پورا اترنے کے بعد تفاعل اور تفاعل کے تفرق کو دیے گئے تفرقی مساوات میں پر کرتے ہیں۔
\begin{align*}
y&=cx\\
(cx)&=(c)x
\end{align*}
مساوات کے دونوں اطراف برابر ہیں لہٰذا \عددی{y=cx} دیے گئے تفرقی مساوات کا حل ہے۔
\انتہا{مثال}
%=======================
