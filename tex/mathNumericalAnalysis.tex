\باب{اعدادی تجزیہ}
انجینئری حساب کا نتیجہ آخرکار اعدادی ہوتا ہے لہٰذا انجینئری طالب علم  کے لئے بنیادی \اصطلاح{اعدادی تراکیب}\فرہنگ{اعدادی!تراکیب}\حاشیہب{numerical methods}\فرہنگ{numerical!methods} جاننا ضروری ہیں جن کی مدد سے دیے گئے مواد سے اعدادی جوابات اخذ کرنا ممکن ہو۔

بعض اوقات نظریہ سے حاصل کردہ جوابات عملاً قابل استعمال نہیں ہوتے ہیں، مثلاً یک درجی خطی تفرقی مساوات کے حل کا تکملی کلیہ (حصہ \حوالہ{حصہ_سادہ_اول_خطی_اور_برنولی})، خطی الجبرائی مساوات کے نظام کا مقطع کی مدد سے حل بذریعہ قاعدہ کریمر (حصہ \حوالہ{حصہ_الجبرا_قاعدہ_کریمر})۔کئی بار نظریہ صرف حل کی وجودیت کی یقین دہانی کرتا ہے لیکن اصل حل حاصل کرنے کے بارے میں کوئی مدد فراہم نہیں کرتا ہے۔

اعدادی تراکیب کی اہمیت کمپیوٹر کی ایجاد کی نظر ہے۔ ہم ان تراکیب کے نظریہ اور عملی استعمال پر غور کریں گے۔\اصطلاح{تجزیہ خلل}\فرہنگ{تجزیہ!خلل}\فرہنگ{خلل!تجزیہ}\حاشیہب{error analysis}\فرہنگ{error!analysis} پر بھی غور کیا جائے گا جو اعدادی تراکیب میں زیادہ اہمیت کے حامل ہے۔     

%==============================
\حصہ{خلل اور غلطیاں۔ کمپیوٹر}\شناخت{حصہ_اعدادی_خلل_غلطیاں_کمپیوٹر}
چونکہ اعدادی تراکیب میں متناہی تعداد کے اعداد استعمال کرتے ہوئے متناہی تعداد کے چال کے بعد جواب حاصل کیا جاتا ہے لہٰذا  یہ تراکیب \اصطلاح{متناہی چال}\فرہنگ{متناہی چال}\حاشیہب{finite processes}\فرہنگ{finite process}  ہیں جو اصل (نا معلوم) بالکل درست حل کا \اصطلاح{تقریب}\فرہنگ{تقریب}\حاشیہب{approximation}\فرہنگ{approximation} پیش کرتے ہیں ماسوائے ان چند صورتوں میں جب اصل جواب کافی سادہ ناطق عدد ہو اور ہم کوئی ایسا اعدادی ترکیب استعمال کریں جو یہی بالکل درست جواب فراہم کرتا ہو۔

اگر کسی مقدار کی اندازاً قیمت \عددی{a^*} ہو اور اس کی اصل قیمت \عددی{a} ہو تب فرق \عددی{\epsilon=a^*-a} کو \عددی{a^*} کا \ترچھا{حتمی خلل} یا مختصراً \عددی{a^*} کا \اصطلاح{خلل}\فرہنگ{خلل}\حاشیہب{error}\فرہنگ{error}  کہتے ہیں۔یوں
\begin{align*}
a^*=a+\epsilon\quad \quad \text{\RL{خلل $\,\,+\,\,$ اصل قیمت $\,\,=\,\,$ تقریب}}
\end{align*}
ہو گا۔ \عددی{a^*} کی \اصطلاح{اضافی خلل}\فرہنگ{خلل!اضافی}\حاشیہب{relative error}\فرہنگ{error!relative} \عددی{\epsilon_r} کی تعریف درج ذیل ہے۔
\begin{align*}
\epsilon_r=\frac{\epsilon}{r}=\frac{a^*-a}{a}=\frac{\text{خلل}}{\text{\RL{اصل قیمت}}}\quad \quad (a\ne 0)
\end{align*}
ظاہر ہے اگر \عددی{\abs{\epsilon}} کی قیمت \عددی{\abs{a^*}} کی قیمت سے بہت کم ہو تب \عددی{\epsilon_r\approx \tfrac{\epsilon}{a^*}} ہو گا۔ہم ایک نئی مقدار \عددی{\gamma=a-a^*=-\epsilon} متعارف کرتے ہیں جس کو ہم \اصطلاح{درستگی}\فرہنگ{درستگی}\حاشیہب{correction}\فرہنگ{correction}\حاشیہد{بعض اوقات خلل کی تعریف \عددی{\gamma=-\epsilon} لی جاتی ہے۔آپ کسی ایک تعریف کو تسلیم کرتے ہوئے آگے بڑھ سکتے ہیں۔ہم خلل کی تعریف \عددی{\epsilon} لیں گے۔} کہیں گے۔یوں
\begin{align*}
a=a^*+\gamma\quad \quad  \text{\RL{درستگی $\,\,+\,\,$ تقریب $\,\,=\,\,$ اصل قیمت}}
\end{align*}
ہو گا۔آخر میں \عددی{a^*} کی \اصطلاح{حد خلل}\فرہنگ{حد!خلل}\فرہنگ{خلل!حد}\حاشیہب{error bound}\فرہنگ{error!bound} سے مراد عدد \عددی{\beta} ہے  جس کی تعریف درج ذیل ہے۔
\begin{align*}
\abs{a^*-a}\le \beta \quad \implies \quad  \abs{\epsilon}\le \beta
\end{align*}

خلل کی تین قسمیں تجربی خلل، قطع چال خلل اور تعداد اعداد خلل ہیں۔\اصطلاح{تجربی خلل}\فرہنگ{خلل!تجربی}\حاشیہب{Experimental errors}\فرہنگ{error!experimental} سے مراد مواد میں خلل ہے (جو تجربی ناپ کی وجہ سے ہو سکتے ہیں)۔بالکل درست جواب تک پہنچنے کی خاطر متناہی (یا لامتناہی) تعداد کے حسابی چال (قدم) درکار ہوں گے۔حقیقت میں کسی خاص تعداد کے چال بعد حساب روک دیا جاتا ہے اور یوں \اصطلاح{قطع چال خلل}\فرہنگ{خلل!قطع چال}\حاشیہب{Truncation error}\فرہنگ{error!truncation} پیدا ہو گا۔  ہر قدم پر حساب کے دوران کمپیوٹر متناہی تعداد کے اعداد استعمال کرتے ہوئے کمتر ہندسہ سے کم قیمتوں کو رد کرتا ہے جس سے \اصطلاح{تعداد ہندسہ خلل}\فرہنگ{خلل!تعداد ہندسہ}\حاشیہب{rounding error}\فرہنگ{error!round off} پیدا ہو گا جس پر ہم اب غور کرتے ہیں۔

اعشاری نظام میں ہر عدد کو متناہی یا لامتناہی تعداد کے اعشاری ہندسوں سے ظاہر کیا جاتا ہے۔کمپیوٹر لامتناہی تعداد کے ہندسوں کو ذخیرہ نہیں کر سکتا ہے لہٰذا کمپیوٹر استعمال کرتے ہوئے کسی بھی عدد کو متناہی تعداد کی ہندسوں سے ظاہر کیا جاتا ہے۔ان اعداد کو دو طریقوں سے کمپیوٹر میں  ذخیرہ کیا جاتا ہے۔ \اصطلاح{مقررہ نقطہ}\فرہنگ{مقررہ نقطہ نظام}\حاشیہب{fixed point}\فرہنگ{fixed point} نظام میں نقطہ اعشاریہ کے بعد مقررہ تعداد کے ہندسے  پائے جاتے ہیں مثلاً \عددی{35.143}، \عددی{5.000}، \عددی{0.076} جبکہ \اصطلاح{غیر مقررہ نقطہ}\فرہنگ{غیر مقررہ نقطہ نظام}\حاشیہب{floating point}\فرہنگ{floating point} نظام میں \اصطلاح{ملحوظ ہندسوں}\فرہنگ{ملحوظ ہندسے}\حاشیہب{significant digits}\فرہنگ{significant digits} کی تعداد متعین ہوتی ہے مثلاً \عددی{0.6723\times10^2}، \عددی{-0.2354\times10^{-4}} اور \عددی{-0.1000\times10^1} جہاں ملحوظ ہندسوں کی تعداد چار ہے۔عدد \عددی{c} کے ملحوظ ہندسہ سے مراد \عددی{c} کا ہر ہندسہ ہے ماسوائے پہلا غیر صفر عدد کی بائیں جانب صفر جو اعشاریہ کا مقام تعین کرتا ہو۔ (یوں اس کے علاوہ ہر صفر بھی \عددی{c} کا ملحوظ ہندسہ ہو گا۔) مثال کے طور پر \عددی{5420}، \عددی{1.340} اور \عددی{0.001460} میں سے ہر ایک میں چار ملحوظ ہندسے\حاشیہد{ایسا جدول جو \عددی{k} ملحوظ ہندسے دیتا ہو میں، جب تک کہا نا جائے کہ ایسا نہیں ہے،  ہم فرض کرتے ہیں کہ  دیا گیا عدد \عددی{a^*}،  بالکل درست قیمت \عددی{a} سے آخری ہندسے کی \عددی{\mp 0.5} اکایاں مختلف ہو سکتا ہے۔مثال کے طور پر اگر \عددی{a=1.1996} ہو تب چار ملحوظ ہندسوں کا جدول \عددی{a^*=1.200} دے گا۔} ہیں۔

تعداد ہندسہ خلل کا قاعدہ اب بیان کرتے ہیں۔(\عددی{k} ملحوظ ہندسوں تک قطع کرنے کی تعریف بھی یہی ہے پس اس میں ہندسہ کی جگہ ملحوظ ہندسہ پر کریں۔)

\عددی{k+1} واں ہندسہ اور اس کے بعد تمام ہندسوں کو رد کریں۔اگر رد شدہ عدد مقام \عددی{k} کی اکائی کی نصف سے کم ہو تب مقام \عددی{k} پر ہندسہ کو تبدیل نہ کریں ("گھٹانا")۔اگر رد شدہ عدد مقام \عددی{k} کی اکائی کی نصف سے زیادہ ہو تب تب مقام \عددی{k} کی ہندسے کے ساتھ \عددی{1} جمع کریں ("بڑھانا")۔اگر رد شدہ عدد مقام \عددی{k} کی اکائی کا نصف ہو تب اگر مقام \عددی{k} کا ہندسہ طاق ہو تب اس کو بڑھا کر جفت بنائیں۔(مثال کے طور پر \عددی{3.45} اور \عددی{3.55} کو اشاریہ کے بعد ایک ہندسہ تک قطع کرتے ہوئے بالترتیب \عددی{3.4} اور \عددی{3.6} حاصل ہو گا۔)

اس قاعدہ کا آخری حصہ یقینی بناتا ہے کہ  عدد کا کمتر حصہ رد کرتے ہوئے اوسطاً برابر مرتبہ عدد بڑھایا اور گھٹایا جاتا ہے۔ 

اگر ہم \عددی{1.2535} کو \عددی{3}، \عددی{2} اور \عددی{1} اشاریہ تک قطع کریں تب ہمیں بالترتیب \عددی{1.254}، \عددی{1.25} اور \عددی{1.3} حاصل ہو گا لیکن، بغیر مزید  معلومات کے، \عددی{1.25} کو ایک اشاریہ تک قطع کرنے سے ہمیں \عددی{1.2} ملتا ہے۔

تعداد ہندسہ خلل کی وجہ سے کوئی بھی حساب مکمل غلط ہو سکتا ہے۔عموماً چال کی تعداد بڑھانے سے یہ خلل بڑھتا ہے۔یوں حسابی پروگرام کو اس خلل کی نقطہ نظر سے دیکھنا ضروری ہو گا اور اس خلل کو کم سے کم کرنا لازم ہو گا۔

\حصہ{دہرانے سے مساوات کا حل}
ہمیں عموماً مساوات 
\begin{align}\label{مساوات_اعدادی_تفاعل_الف}
f(x)=0
\end{align}
کے حل درکار ہوتے ہیں یعنی ایسے عدد \عددی{X_0} کہ \عددی{f(X_0)} صفر کے برابر ہو جہاں \عددی{f} دیا گیا تفاعل ہے۔مثال کے طور پر \عددی{x^2-3x+2=0}، \عددی{x^3+x=1}، \عددی{\sin x=0.5 x}، \عددی{\tan x=x}، \عددی{\cosh x=\sec x} اور \عددی{\cosh x\cos x=-1} کو مساوات \حوالہ{مساوات_اعدادی_تفاعل_الف} کی صورت میں لکھا جا سکتا ہے۔ ان میں پہلے دو میں \عددی{f} کثیر رکنی ہے لہٰذا یہ دونوں \اصطلاح{الجبرائی مساوات}\فرہنگ{الجبرائی!مساوات}\حاشیہب{algebraic equations}\فرہنگ{algebraic equations} ہیں جن کے حل کو \اصطلاح{جذر}\فرہنگ{جذر}\حاشیہب{roots}\فرہنگ{roots} کہتے ہیں۔باقی \اصطلاح{ماورائی مساوات}\فرہنگ{مساوات!ماورائی}\فرہنگ{ماورائی!مساوات}\حاشیہب{transcendental equations}\فرہنگ{transcendental!equations} ہیں جن میں ماورائی تفاعل استعمال ہوئے ہیں۔حقیقتاً صرف انتہائی سادہ صورتوں میں مکمل درست حل نکالنے والے کلیات موجود ہوں گے۔عموماً ہم دہرانے کی ترکیب یا دیگر تراکیب سے اصل حل کے قریب قریب حل حاصل کریں گے۔

اعدادی دہرانے کے طریقہ میں ہم اختیاری \عددی{x_0} منتخب کرتے ہوئے  درج ذیل روپ کلیہ
\begin{align}\label{مساوات_اعدادی_تفاعل_ب}
x_{n+1}=g(x_n)\quad \quad \quad (n=0,1,2,\cdots)
\end{align}
سے، بار بار حل کرتے ہوئے، ترتیب \عددی{x_0, x_1, x_2,\cdots} حاصل کرتے ہیں جہاں \عددی{g} کسی ایسے وقفہ پر معین ہے جس پر \عددی{x_0} پایا جاتا ہو اور \عددی{g}  کا حلقہ اسی وقفہ پر ہے۔ یوں ہم یک بعد دیگرے \عددی{x_1=g(x_0)}،  \عددی{x_2=g(x_1)}، \عددی{x_3=g(x_2)}، \نقطے حاصل کرتے ہیں۔

اس حصہ میں دائرہ کار اور حلقہ \عددی{g(x)} دونوں حقیقی لکیر پر ہوں گے۔زیادہ عمومی معمہ میں \عددی{x} یا \عددی{g} اور یا دونوں سمتیات ہو سکتے ہیں۔

دہرانے کے تراکیب اعدادی تجزیہ کے لئے انتہائی اہم ہیں۔

مساوات \حوالہ{مساوات_اعدادی_تفاعل_الف} کو حل کرنے کے لئے  دہرانے کے تراکیب کئی طریقوں سے حاصل کیے جا سکتے ہیں۔ہم ان میں سے تین خصوصاً اہم طریقوں پر غور کرتے ہیں۔

\موٹا{الجبرائی تبادل۔} ہم مساوات \حوالہ{مساوات_اعدادی_تفاعل_الف} کو الجبرائی طور پر تبدیل کرتے ہوئے درج ذیل روپ حاصل کر سکتے ہیں
\begin{align}\label{مساوات_اعدادی_تفاعل_پ}
x=g(x)
\end{align}
جو مساوات \حوالہ{مساوات_اعدادی_تفاعل_ب}  کی روپ میں ہے۔مساوات \حوالہ{مساوات_اعدادی_تفاعل_پ} کے حل کو \عددی{g} کا \اصطلاح{مقررہ نقطہ}\فرہنگ{مقررہ نقطہ}\حاشیہب{fixed point}\فرہنگ{fixed point} کہتے ہیں۔دیے گئے مساوات \حوالہ{مساوات_اعدادی_تفاعل_الف} کے کئی مطابقتی مساوات \حوالہ{مساوات_اعدادی_تفاعل_پ} ہو سکتے ہیں جن کے ترتیب \عددی{x_0,x_1,\cdots} مختلف (اور \عددی{x_0} کے تابع) ہوں گے۔آئیں ایک سادہ مثال دیکھتے ہیں جس میں یہ حقائق ابھر کر سامنے آتے ہیں۔

%=================
\ابتدا{مثال}\شناخت{مثال_اعدادی_دہرانا_الف}\quad \موٹا{دہرانے کی ترکیب}\\
مساوات \عددی{f(x)=x^2-3x+1=0} کے لئے دہرانے کی ترکیب عمل میں لائیں۔چونکہ ہمیں اس مساوات کے حل
\begin{align*}
x=1.5\mp\sqrt{1.25},\quad x_1=\num{2.618034},\quad x_2=\num{0.381966}
\end{align*}
معلوم ہیں، ہم دہرانے کے عمل کے دوران خلل کا رویہ دیکھ سکتے ہیں۔ہم دیے گئے مساوات سے
\begin{align}\label{مساوات_مثال_اعدادی_دہرانا_الف}
x=g_1(x)=\frac{1}{3}(x^2+1)\quad \implies \quad x_{n+1}=\frac{1}{3}(x_n^2+1)
\end{align}
لکھ سکتے ہیں۔یوں \عددی{x_0=1} منتخب کرتے ہوئے ہمیں درج ذیل ترتیب ملتی ہے
\begin{align*}
x_0=\num{1.000},\quad x_1=\num{0.667},\quad x_2=\num{0.481},\quad x_3=\num{0.411},\quad x_4=\num{0.390}, \cdots
\end{align*}
جو چھوٹے جذر کی طرف گامزن ہے (شکل \حوالہ{شکل_مثال_اعدادی_دہرانا_الف}-الف)۔اگر ہم \عددی{x_0=3.000} منتخب کریں تب درج ذیل ملتا ہے
\begin{align*}
x_0=\num{3.000},\quad x_1=\num{3.333},\quad x_2=\num{4.037},\quad x_3=\num{5,766},\quad x_4=\num{11.414}, \cdots
\end{align*}
جو منفرج ترتیب ہے (شکل \حوالہ{مثال_اعدادی_دہرانا_الف}-الف)۔دی گئی مساوات سے درج ذیل بھی حاصل کیا جا سکتا ہے۔
\begin{align}\label{مساوات_مثال_اعدادی_دہرانا_ب}
x=g_2(x)=3-\frac{1}{x}\quad \implies \quad x_{n+1}=3-\frac{1}{x_n}
\end{align}
اب \عددی{x_0} منتخب کرتے ہوئے
\begin{align*}
x_0=\num{1.000},\quad x_1=\num{2.000},\quad x_2=\num{2.500},\quad x_3=\num{2.600},\quad x_4=\num{2.615}, \cdots
\end{align*}
حاصل ہوتا ہے جو بڑے جذر کی طرف گامزن ترتیب ہے (شکل \حوالہ{مثال_اعدادی_دہرانا_الف}-ب)۔اسی طرح \عددی{x_0=3} منتخب کرتے ہوئے
\begin{align*}
x_0=\num{3.000},\quad x_1=\num{2.667},\quad x_2=\num{2.625},\quad x_3=\num{2.619},\quad x_4=\num{2.618}, \cdots
\end{align*}
حاصل ہوتا ہے (شکل \حوالہ{مثال_اعدادی_دہرانا_الف}-ب)۔شکل کو دیکھ کر واضح ہوتا ہے کہ مرکوزیت اس صورت ہو گی جب حل کی پڑوس میں منحنی \عددی{g(x)} کی ڈھلوان سیدھے خط \عددی{y=x} کی ڈھلوان سے کم ہو۔ہم اب دیکھتے ہیں کہ مرکوزیت کے لئے \عددی{\abs{g'(x)<1}}  کی شرط کافی ہے (جہاں خط \عددی{y=x} کی ڈھلوان \عددی{y'=1} ہے)۔
\begin{figure}
\centering
\begin{subfigure}{0.5\textwidth}
\centering
\begin{tikzpicture}
\draw(0,0)--(5,0)node[below]{$x$};
\draw(0,0)--(0,5);
\draw(0,0)--(5,5);
\draw[thick,domain=0:3.74] plot ({\x},{1/3*(\x^2+1)});
\draw[->-=0.7](1,1)--(1,2/3);
\draw(1,2/3)--(2/3,2/3)--(2/3,0.481)--(0.481,0.481)--(0.481,0.411);
\draw[->-=0.7](3,3)--(3,3.333);
\draw(3,3.333)--(3.333,3.333)--(3.333,4.037)--(4.037,4.035)--(4.037,5);
\draw(3.74,5)node[below left, xshift={(-0.1cm)}]{$g_1(x)$};
\foreach \x in {1,2,3,4}{\draw(\x,0)node[below]{$\x$}--++(0,0.2);}
\foreach \y in {1,2,3,4}{\draw(0,\y)node[left]{$\y$}--++(0.2,0);}
\end{tikzpicture}
\caption*{(الف)}
\end{subfigure}%
\begin{subfigure}{0.5\textwidth}
\centering
\begin{tikzpicture}
\draw(0,0)--(5,0)node[below]{$x$};
\draw(0,0)--(0,5);
\draw(0,0)--(5,5)node[below left,xshift={(-0.2cm)}]{$y=x$};
\draw[smooth,thick,domain=1/3:5] plot ({\x},{(3-1/(\x))});
\draw[->-=0.7](1,1)--(1,2);
\draw(1,2)--(2,2)--(2,2.5)--(2.5,2.5)--(2.5,2.6)--(2.6,2.6)--(2.6,2.615);
\draw[->-=0.7](3,3)--(3,2.667);
\draw(3,2.667)--(2.667,2.667)--(2.667,2.625);
\draw(5,2.8)node[below]{$g_2(x)$};
\foreach \x in {1,2,3,4}{\draw(\x,0)node[below]{$\x$}--++(0,0.2);}
\foreach \y in {1,2,3,4}{\draw(0,\y)node[left]{$\y$}--++(0.2,0);}
\end{tikzpicture}
\caption*{(ب)}
\end{subfigure}%
\caption{اشکال برائے مثال \حوالہ{مثال_اعدادی_دہرانا_الف}}
\label{شکل_مثال_اعدادی_دہرانا_الف}
\end{figure}
\انتہا{مثال}
%=========================

اگر \عددی{x_0} کا  مطابقتی  مساوات \حوالہ{مساوات_اعدادی_تفاعل_ب} سے حاصل کردہ ترتیب \عددی{x_0,x_1,\cdots} مرتکز ہو تب ہم کہتے ہیں کہ مساوات \حوالہ{مساوات_اعدادی_تفاعل_ب} میں دی گئی دہرانے کی ترکیب \اصطلاح{مرتکز}\فرہنگ{مرتکز} ہے۔

ارتکاز کے لئے کافی شرط درج ذیل مسئلہ پیش کرتا ہے جس کے کئی اہم عملی استعمال پائے جاتے ہیں۔

%=====================
\ابتدا{مسئلہ}\شناخت{مسئلہ_اعدادی_ارتکاز_شرط}\quad \موٹا{(ارتکاز)}\\
فرض کریں کہ \عددی{x=g(x)} کا حل \عددی{x=s} ہے اور فرض کریں کہ کسی ایسے وقفہ \عددی{J}، جس میں \عددی{s} پایا جاتا ہو، پر \عددی{g(x)} کا استمراری تفرق پایا جاتا ہے۔ اب اگر \عددی{J} میں \عددی{\abs{g'(x)}\le \alpha<1} ہو تب مساوات \حوالہ{مساوات_اعدادی_تفاعل_ب} میں دی گئی دہرانے کی ترکیب \عددی{J} میں ہر \عددی{x_0} کے لئے مرتکز ہو گی۔
\انتہا{مسئلہ}
%==========================
\ابتدا{ثبوت}\quad
تفرقی علم الاحصاء کے مسئلہ اوسط قیمت کے تحت \عددی{x} اور \عددی{s} کے درمیان ایسا \عددی{\xi} پایا جائے گا جو درج ذیل کو مطمئن کرے گا،
\begin{align*}
g(x)-g(s)=g'(\xi)(x-s) 
\end{align*}
جہاں \عددی{x} وقفہ \عددی{J} میں پایا جاتا ہے۔  چونکہ \عددی{g(s)=s} اور \عددی{x_1=g(x_0)}، \عددی{x_2=g(x_1)}، \نقطے ہیں لہٰذا ہمیں درج ذیل ملتا ہے۔
\begin{align*}
\abs{x_n-s}&=\abs{g(x_{n-1})-g(s)}=\abs{g'(\xi)}\abs{x_{n-1}-s}\le \alpha \abs{x_{n-1}-s}\\
&\le \alpha^2\abs{x_{n-2}-s}\le \cdots \le \alpha^n\abs{x_{0}-s}
\end{align*}
چونکہ \عددی{\alpha<1} ہے لہٰذا  \عددی{n\to \infty} کرنے سے \عددی{\alpha^n\to 0} اور \عددی{\abs{x_n-s}\to 0} ہوں گے۔یوں ثبوت مکمل ہوتا ہے۔
\انتہا{ثبوت}
%=======================
\ابتدا{مثال}\شناخت{مثال_اعدادی_دہرانا_ب}\quad \موٹا{دہرانے کا طریقہ۔ مسئلہ \حوالہ{مسئلہ_اعدادی_ارتکاز_شرط}}\\
دہرانے کے طریقہ سے \عددی{f(x)=x^3+x-1=0} کا حل  تلاش کریں۔اس مساوات کا جلدی سے خاکہ  بنا کر آپ دیکھ سکتے ہیں کہ اس کا جذر \عددی{x=1} کے قریب پایا جاتا ہے۔  ہم اس  مساوات سے درج ذیل لکھ سکتے ہیں۔
\begin{align*}
x=g_1(x)=\frac{1}{1+x^2}\quad \implies \quad x_{n+1}=\frac{1}{1+x_n^2}
\end{align*}
یوں کسی بھی \عددی{x} کے لئے \عددی{\abs{g'_1(x)}=\tfrac{2\abs{x}}{(1+x^2)^2}<1} ہو گا لہٰذا تمام \عددی{x} پر مرکوزیت پائی جائے گی۔ہم \عددی{x_0=1} منتخب کرتے ہوئے درج ذیل حاصل کرتے ہیں (شکل \حوالہ{شکل_مثال_اعدادی_دہرانا_ب})
\begin{align*}
x_1=0.500,\quad x_2=0.800,\quad x_3=0.610,\quad x_4=0.729,\quad x_5=0.653,\quad x_6=0.701,\cdots
\end{align*}
جبکہ چھ ہندسوں تک درست اصل جذر \عددی{s=\num{0.682328}} ہے۔ ہم مساوات سے درج ذیل بھی لکھ سکتے ہیں۔
\begin{align*}
x=g_2(x)=1-x^3,\quad \abs{g_2'(x)}=3x^2 
\end{align*}
جذر کے قریب \عددی{ \abs{g'_2}} کی قیمت اکائی سے زیادہ ہے لہٰذا ہم ارتکاز کی توقع نہیں کر سکتے ہیں۔آپ \عددی{x_0=1}، \عددی{x_0=0.5}، \عددی{x_0=2} سے شروع کرتے ہوئے اپنی تسلی کر سکتے ہیں۔
\begin{figure}
\centering
\begin{tikzpicture}
\pgfmathsetmacro{\k}{5}
\draw(0,0)--(5.75,0)node[right]{$x$};
\draw(0,0)--(0,5.5);
\draw(0,0)--(5,5);
\draw[thick,domain=0:1.1] plot ({\k*\x},{\k/(\x^2+1)});
\draw[->-=0.5](\k*1,0)--(\k*1,\k*0.5);
\draw[->-=0.5](\k*1,\k*0.5)--(\k*0.5,\k*0.5)node[below right]{$x_1$};
\draw[->-=0.5](\k*0.5,\k*0.5)--(\k*0.5,\k*0.8);
\draw[->-=0.5](\k*0.5,\k*0.8)--(\k*0.8,\k*0.8)node[above left]{$x_2$};
\draw[->-=0.5](\k*0.8,\k*0.8)--(\k*0.8,\k*0.61);
\draw[->-=0.7](\k*0.8,\k*0.61)--(\k*0.61,\k*0.61);
\draw[->-=0.7](\k*0.61,\k*0.61)--(\k*0.61,\k*0.729);
\draw(1,5)node[right]{$g_1(x)$};
\foreach \x in {1,2,...,11}{\draw(\x/2,0)--++(0,0.2);}
\foreach \y in {1,2,...,10}{\draw(0,\y/2)--++(0.2,0);}
\draw(5/2,0)node[below]{$0.5$};
\draw(10/2,0)node[below]{$1.0$};
\draw(0,5/2)node[left]{$0.5$};
\draw(0,10/2)node[left]{$1.0$};
\draw(0,0)node[below]{$0$};
\draw(0,0)node[left]{$0$};
\end{tikzpicture}
\caption{شکل برائے مثال \حوالہ{مثال_اعدادی_دہرانا_ب}}
\label{شکل_مثال_اعدادی_دہرانا_ب}
\end{figure}%
\انتہا{مثال}
%=========================

مساوات \عددی{f(x)=0}، جہاں \عددی{f} قابل تفرق ہے، کو \موٹا{ترکیب نیوٹن} سے بھی حل کیا جا سکتا ہے۔اس ترکیب میں ہم \عددی{f(x)} کا تخمینہ اس کے موزوں مماس سے حاصل کرتے ہیں۔اس ترکیب میں ہم \عددی{f} کی ترسیم سے حاصل \عددی{x_0} پر \عددی{f} کا مماس بناتے ہیں۔یہ مماس \عددی{x} محور کو \عددی{x_1} پر قطع کرتا ہے (شکل \حوالہ{شکل_اعدادی_ترکیب_نیوٹن})۔یوں
\begin{align*}
\tan\beta=f'(x_0)=\frac{f(x_0)}{x_0-x_1}\quad \implies \quad x_1=x_0-\frac{f(x_0)}{f'(x_0)}
\end{align*}
ہو گا۔اگلے قدم پر ہم
\begin{align*}
x_2=x_1-\frac{f(x_1)}{f'(x_1)}
\end{align*}
حاصل کرتے ہیں۔اسی طرح چلتے ہوئے جذر تک پہنچا جاتا ہے۔یوں دہرانے کے طریقے کا عمومی کلیہ درج ذیل ہو گا۔
\begin{align}\label{مساوات_اعدادی_ترکیب_نیوٹن}
x_{n+1}=x_n-\frac{f(x_n)}{f'(x_n)}\quad \quad \quad (n=0,1,\cdots)
\end{align}
%
\begin{figure}
\centering
\begin{tikzpicture}
\draw(0,-0.5)--(0,3)node[left]{$y$};
\draw[name path=kA](0,0)--(5.5,0)node[right]{$x$};
\draw[thick,name path=kCurve] (0.5,-0.5) to [out=10,in=-120] coordinate[pos=0.9](pA)(5,3)node[right]{$y=f(x)$};
\draw[dashed] (pA)--($(0,0)!(pA)!(5,0)$)coordinate(pAA)node[below]{$x_0$};
\path[name path=kB](pA)--++(60:-3);
\draw[name intersections={of={kA and kB}}] (pA)--(intersection-1)coordinate(pB)node[below]{$x_1$};
\path[name path=kC] (intersection-1)--++(0,1);
\draw[dashed,name intersections={of=kCurve and kC}] (pB)--(intersection-1)coordinate(pC);
\path[name path=kD](pC)--++(40:-2);
\draw[name intersections={of={kD and kA}}] (pC)--(intersection-1)coordinate(pD)node[below]{$x_2$};
\path[name path=kE] (pD)--++(0,1);
\draw[dashed,name intersections={of=kE and kCurve}] (pD)--(intersection-1);
%
\draw[name intersections={of=kA and kCurve}] (intersection-1)node[ocirc]{};
\draw [decorate,decoration={brace,amplitude=10pt},xshift=-4pt,yshift=0pt]
(pA) -- (pAA) node [black,midway,xshift=0.8cm] {\footnotesize $f(x_0)$};
%
\draw(pA)node[ocirc]{};
\draw(pC)node[ocirc]{};
\draw(0,0)node[ocirc]{};
%
\draw[-stealth]([shift={(0:0.5)}]pB) arc (0:60:0.5);
\draw(pB)++(30:0.8)node[]{$\beta$};
\end{tikzpicture}
\caption{ترکیب نیوٹن}
\label{شکل_اعدادی_ترکیب_نیوٹن}
\end{figure}

%=====================
\ابتدا{مثال}\شناخت{مثال_اعدادی_دہرانا_پ}\quad \موٹا{جذر المربع}\\
کسی مثبت حقیقی عدد \عددی{c} کا جذر المربع حاصل کرنے کے لئے دہرانے کی ترکیب بنائیں۔اس ترکیب کو استعمال کرتے ہوئے \عددی{c=2} کا جذر المربع تلاش کریں۔ہمارے پاس \عددی{\sqrt{c}}  یعنی \عددی{f(x)=x^2-c=0} ہے لہٰذا \عددی{f'(x)=2x} ہو گا۔یوں مساوات \حوالہ{مساوات_اعدادی_ترکیب_نیوٹن} درج ذیل صورت اختیار کرتی ہے۔
\begin{align*}
x_{n+1}=x_n-\frac{x_n^2-c}{2x_n}=\frac{1}{2}\big(x_n+\frac{c}{x_n}\big)
\end{align*}
اب اس ترکیب سے \عددی{c=2} کا جذر المربع تلاش کرتے ہیں۔ہم \عددی{x_0=1} منتخب کرتے ہوئے درج ذیل حاصل کرتے ہیں۔
\begin{align*}
x_1=\num{1.500000},\quad x_2=\num{1.416667},\quad x_3=\num{1.414216},\quad x_4=\num{1.414214},\cdots
\end{align*}
\عددی{2} کا جذر المربع \عددی{\num{1.414213562}} ہے اور آپ دیکھ سکتے ہیں کہ \عددی{x_4} چھ ملحوظ ہندسوں تک درست جواب دیتا ہے۔
\انتہا{مثال}
%===========================
\ابتدا{مثال}\شناخت{مثال_اعدادی_ماورائی_الف}\quad \موٹا{ماورائی مساوات کا دہرانے کی ترکیب سے حل}\\
مساوات \عددی{2\sin x=x} کا مثبت حل تلاش کریں۔ہم \عددی{f(x)=x-2\sin x} لکھتے ہوئے \عددی{f'(x)=1-2\cos x} حاصل کرتے ہیں۔یوں مساوات \حوالہ{مساوات_اعدادی_ترکیب_نیوٹن} کی صورت درج ذیل ہو گی۔
\begin{align*}
x_{n+1}=x_n-\frac{x_n-2\sin x_n}{1-2\cos x_n}=\frac{2(\sin x_n-x_n\cos x_n)}{1-2\cos x_n}=\frac{N_n}{D_n}
\end{align*}
\عددی{f} کی ترسیم سے ہم دیکھتے ہیں کہ اس کا حل \عددی{x_0=2} کے قریب ہے۔یوں ہم جدول \حوالہ{جدول_مثال_اعدادی_ماورائی_الف} حاصل کرتے ہیں۔چار ملحوظ ہندسوں تک درست جواب \عددی{1.8955} ہے۔
\begin{table}
\caption{جدول برائے مثال \حوالہ{مثال_اعدادی_ماورائی_الف}}
\label{جدول_مثال_اعدادی_ماورائی_الف}
\centering
\begin{tabular}{c|cccc}
$n$& $x_n$& $N_n$ &$D_n$  & $x_{n+1}$\\
\hline
$0$& $2.000$& $3.483$ & $1.832$ & $1.901$\\
$1$& $1.901$ & $3.125$ & $1.648$ & $1.896$\\
$2$& $1.896$ & $3.107$ & $1.639$ & $1.896$
\end{tabular}
\end{table}
\انتہا{مثال}
%===========================
\ابتدا{مثال}\quad \موٹا{ترکیب نیوٹن کا الجبرائی مساوات پر اطلاق}
مساوات \عددی{f(x)=x^3+x-1=0} کو ترکیب نیوٹن سے حل کریں۔مساوات \حوالہ{مساوات_اعدادی_ترکیب_نیوٹن} سے درج ذیل ہو گا۔
\begin{align*}
x_{n+1}=x_n-\frac{x_n^3+x_n-1}{3x_n^2+1}=\frac{2x_n^3+1}{3x_n^2+1}
\end{align*}
\عددی{x_0=1} سے شروع کرتے ہوئے درج ذیل حاصل ہو گا۔
\begin{align*}
x_1=\num{0.750000},\quad x_2=\num{0.686047},\quad x_3=\num{0.682340},\quad x_4=\num{0.682328},\cdots
\end{align*}
\عددی{x_4} چھ ملحوظ ہندسوں تک درست ہے۔مثال \حوالہ{مثال_اعدادی_دہرانا_ب} کے ساتھ موازنہ کرنے سے آپ دیکھ سکتے ہیں کہ موجودہ مثال بہت تیزی کے ساتھ اصل حل پر مرکوز ہوتا ہے۔اس سے دہرانے کی ترکیب کے درجہ کا تصور پیدا ہوتا ہے جس پر اب بات کی جائے گی۔
\انتہا{مثال}
%==================================

فرض کریں کہ مساوات \عددی{x=g(x)} کا حل \عددی{s} ہے اور  \عددی{x_{n+1}=g(x_n)} ایک دہرانے کی ترکیب ہے جو اس حل کے قریب قریب قیمت \عددی{x_n} دیتی ہے۔تب \عددی{x_n=s+\epsilon_n} ہو گا جہاں \عددی{x_n} میں خلل \عددی{\epsilon_n} ہے۔فرض کریں کہ \عددی{g} متعدد بار قابل تفرق ہے لہٰذا ٹیلر کے کلیہ سے 
\begin{align*}
x_{n+1}=g(x_n)&=g(s)+g'(s)(x_n-s)+\frac{1}{2}g''(s)(x_n-s)^2+\cdots\\
&=g(s)+g'(s)\epsilon_n+\frac{1}{2}g''(s)\epsilon_n^2+\cdots
\end{align*}
لکھا  جا سکتا ہے۔جزو \عددی{g(s)} کے بعد پہلی غیر صفر جزو میں \عددی{\epsilon} کے قوت نما کو دہرانے کی ترکیب (جس کو \عددی{g} تعین کرتا ہے) کا \اصطلاح{درجہ}\فرہنگ{درجہ!دہرانے کی ترکیب}\حاشیہب{order}\فرہنگ{order!iteration method} کہتے ہیں۔چونکہ \عددی{x_{n+1}-g(s)=x_{n+1}-s=\epsilon_{n+1}} یعنی \عددی{x_{n+1}} کا خلل ہے، اور ارتکاز کی صورت میں بڑی \عددی{n} کے لئے \عددی{\epsilon_n} چھوٹا ہو گا لہٰذا ترکیب کا درجہ اس کی مرکوزیت کی ناپ ہے۔

\موٹا{ترکیب نیوٹن دو درجی ہے}\\
ترکیب نیوٹن کے لئے درج ذیل ہے
\begin{align*} 
g(x)=x-\frac{f(x)}{f'(x)},\quad g'(x)=1-\frac{f'f'-ff''}{(f')^2}=\frac{f(x)f''(x)}{f'(x)^2}
\end{align*} 
اور چونکہ \عددی{f(s)=0} ہے لہٰذا \عددی{g'(s)=0} ہو گا؛ یوں ترکیب نیوٹن کم از کم دو درجی ہے۔ایک اور تفرق کے بعد \عددی{g''(s)=\tfrac{f''(s)}{f'(s)}} ملتا ہے جو عموماً غیر صفر ہو گا۔ مثال \حوالہ{مثال_اعدادی_دہرانا_ب} میں \عددی{g_1(x)=\tfrac{1}{1+x^2}} اور \عددی{g'(x)=-\tfrac{2x}{(1+x^2)^2}} ہیں لہٰذا یہ یک درجی دہرانے کی ترکیب ہے۔

\عددی{f(x)=0} کے حل کے قریب \عددی{f'(x)=0} ہونے کی صورت میں ترکیب نیوٹن مشکلات پیدا کرتا ہے لیکن حل کے قریب \عددی{f(x)} کی ترسیم  کو دیکھتے ہوئے، ترکیب نیوٹن کی جیومیٹریائی تصور کو مد نظر رکھتے ہوئے عموماً  اس مشکل سے چھٹکارا حاصل کرنا ممکن ہو گا۔ اگر درکار حل کے قریب \عددی{f'(x)=0} ہو تب \عددی{x_{n+1}} کی بہتر قیمت حاصل کرنے کی خاطر  \عددی{f(x_n)} اور \عددی{f'(x_n)} کے زیادہ درست قیمتیں حاصل کرنا ضروری ہو گا۔ ایسی مساوات کو \اصطلاح{بد خو}\فرہنگ{بد خو}\حاشیہب{ill-conditioned}\فرہنگ{ill-conditioned} کہتے ہیں۔

\عددی{f(x)=0} کو  حل کرنے کی تیسری ترکیب جس کو \اصطلاح{مقام غلط کی ترکیب}\فرہنگ{دہرانا!مقام غلط}\حاشیہب{method of false position}\فرہنگ{iteration!method of false position} کہتے ہیں پر اب غور کرتے ہیں۔اس ترکیب میں منحنی \عددی{f(x)} کا مشابہ  وتر تصور کیا جاتا ہے (شکل \حوالہ{شکل_اعدادی_مشابہ})۔یہ وتر  محور \عددی{x} کو
\begin{align}\label{مساوات_اعدادی_غلط_مقام_الف}
x_1=\frac{x_0f(b)-bf(x_0)}{f(b)-f(x_0)}
\end{align}
پر قطع کرتا ہے جو \عددی{f(x)=0} کے حل \عددی{X_0} کے قریب ہو گا۔اگلے قدم پر اس سے بہتر حل
\begin{align}\label{مساوات_اعدادی_غلط_مقام_ب}
x_2=\frac{x_1f(b)-bf(x_1)}{f(b)-f(x_1)}
\end{align}
حاصل کیا جاتا ہے۔اسی طرح بتدریج بہتر حل حاصل کیے جا سکتے ہیں۔\عددی{b} کو \عددی{X_0} کے قریب کرنے سے ارتکاز کو بہتر  بنایا جا سکتا ہے۔عموماً قیاس کے ذریعہ ایسا کرنا ممکن ہو گا۔
\begin{figure}
\centering
\begin{tikzpicture}
\draw(0,-2)--(0,3)node[left]{$y$};
\draw[name path=cX](0,0)--(5.5,0)node[right]{$x$};
\draw[name path=cA](0.25,-2) to [out=10, in=-100] coordinate[pos=0.1](kA)coordinate[pos=0.9](kB)(5,3);
\draw[name path=cS](kA)--(kB);
\draw[dashed](kB)--($(0,0)!(kB)!(5,0)$)node[below]{$b$};
\draw[dashed](kA)node[ocirc,solid]{}--($(0,0)!(kA)!(5,0)$)node[above]{$x_0$};
\draw[name intersections={of={cS and cX}}] (intersection-1)coordinate(kM);
\path[name path=cL](kM)--++(0,-1.5);
\draw[dashed,name intersections={of=cL and cA}] (kM)node[ocirc,solid]{}node[above,xshift={(-0.05cm)}]{$x_1$}--(intersection-1);
\draw[name path=cR] (kB)--(intersection-1)node[ocirc,solid]{};
\draw[name intersections={of={cR and cX}}](intersection-1) node[ocirc]{}node[above,xshift={(-0.1cm)}]{$x_2$};
\draw[name intersections={of={cA and cX}}] (intersection-1)node[ocirc]{}node[below,xshift={(0.1cm)}]{$X_0$};
\end{tikzpicture}
\caption{منحنی کا مشابہ وتر سے کیا گیا ہے}
\label{شکل_اعدادی_مشابہ}
\end{figure}


%=====================
\ابتدا{مثال}\quad
مساوات \عددی{f(x)=x^3+x-1=0} کا وہ جذر تلاش کریں جو \عددی{x=1} کے قریب واقع ہے (مثال \حوالہ{مثال_اعدادی_دہرانا_ب})۔ چونکہ \عددی{f(0.5)=-0.375} اور \عددی{f(1)=1} ہیں لہٰذا ہم \عددی{x_0=0.5} اور \عددی{b=1} منتخب کر سکتے ہیں۔مساوات \حوالہ{مساوات_اعدادی_غلط_مقام_الف} سے
\begin{align*}
x_1=\frac{0.5\cdot 1-1\cdot(-0.375)}{1-(-0.375)}=0.64
\end{align*}
حاصل ہو گا جبکہ مساوات \حوالہ{مساوات_اعدادی_غلط_مقام_ب} سے\عددی{x_2=0.672} ملتا ہے۔ہم اسی طرح بتدریج بہتر حل تلاش کر سکتے ہیں۔
\انتہا{مثال}
%===================

\حصہء{سوالات}
%======================
\ابتدا{سوال}\شناخت{سوال_اعدادی_نیوٹن_الف}\quad
\عددی{x^3-3.9x^2+4.79x-1.881=0} کا جذر ترکیب نیوٹن میں \عددی{x_0=1} لے کر تین قدم چلتے ہوئے  تلاش کریں۔\\
جواب:\quad 
$x_1=\num{1.900000}$
\انتہا{سوال}
%=============================
\ابتدا{سوال}\quad
\عددی{x^3-1.2x^2+2x-2.4=0} کا جذر ترکیب نیوٹن میں \عددی{x_0=2} لے کر تین قدم چلتے ہوئے تلاش کریں۔\\
جواب:\quad
$x_1=\num{1.478261}$
\انتہا{سوال}
%=============================
\ابتدا{سوال}\quad
سوال \حوالہ{سوال_اعدادی_نیوٹن_الف} میں دیے گئے مساوات کے جذر \عددی{0.9}، \عددی{1.1} اور \عددی{1.9} ہیں۔اگرچہ \عددی{x_0=1} جذر \عددی{0.9} اور \عددی{1.1} کے قریب ہے لیکن ترکیب نیوٹن ان کی جگہ جذر \عددی{1.9} تلاش کرتا ہے۔ایسا کیوں ہے؟ \عددی{x_0} کی کوئی اور قیمت منتخب کرتے ہوئے ترکیب نیوٹن سے جذر \عددی{1.1} حاصل کریں۔\\
جواب:\quad
تفاعل \عددی{f(x)} کو \عددی{x_0=1} پر مماس \عددی{x} محور کو عین \عددی{x=1.9} پر قطع کرتا ہے۔آپ \عددی{x_0=1.2} یا کوئی اور عدد منتخب کر سکتے ہیں۔
\انتہا{سوال}
%========================
سوال \حوالہ{سوال_اعدادی_ترکیب_نیوٹن_تمام_جذر_الف} تا سوال \حوالہ{سوال_اعدادی_ترکیب_نیوٹن_تمام_جذر_ب} میں  دیے مساوات کی ترکیب نیوٹن کی مدد سے تمام جذر تلاش کریں۔

%=====================
\ابتدا{سوال}\شناخت{سوال_اعدادی_ترکیب_نیوٹن_تمام_جذر_الف}\quad
$\cos x=x$\\
جواب:\quad
$0.739$
\انتہا{سوال}
%==========================
\ابتدا{سوال}\quad
$x+\ln x-2$\\
جواب:\quad
$1.577$
\انتہا{سوال}
%==========================
\ابتدا{سوال}\quad
$2x+\ln x-1$\\
جواب:\quad
$0.687$
\انتہا{سوال}
%==========================
\ابتدا{سوال}\شناخت{سوال_اعدادی_ترکیب_نیوٹن_تمام_جذر_ب}\quad
$x^4-0.1x^3-0.82x^2-0.1x-1.82$\\
جواب:\quad
$-1.3,\quad 1.4$
\انتہا{سوال}
%==========================
\ابتدا{سوال}\quad
دکھائیں کہ مثال \حوالہ{مثال_اعدادی_دہرانا_ب} میں \عددی{\abs{g_1'(x)}} کی زیادہ سے زیادہ قیمت  \عددی{\tilde{x}=\mp\tfrac{1}{\sqrt{3}}} پر حاصل ہو گی اور کہ یہ قیمت \عددی{\abs{g'(\tilde{x})}=\tfrac{3\sqrt{3}}{8}=0.65} کے برابر ہے۔
\انتہا{سوال}
%=========================
\ابتدا{سوال}\quad
ایسا کیوں ہے کہ مثال \حوالہ{مثال_اعدادی_دہرانا_الف} میں یک سر ترتیب حاصل ہوتی ہے لیکن مثال \حوالہ{مثال_اعدادی_دہرانا_ب} میں ایسا نہیں ہوتا ہے؟
\انتہا{سوال}
%========================
\ابتدا{سوال}\quad
مثال \حوالہ{مثال_اعدادی_دہرانا_ب} کی آخر میں دہرانے کی ترکیب سے حاصل قیمتوں کو از خود حاصل کریں اور شکل \حوالہ{شکل_مثال_اعدادی_دہرانا_ب} کی طرز کا شکل بنائیں۔ 
\انتہا{سوال}
%========================
\ابتدا{سوال}\شناخت{سوال_اعدادی_دہرانا_درکار_الف}\quad
مساوات \عددی{x^5=x+0.2} کو  مساوات \حوالہ{مساوات_اعدادی_تفاعل_ب} کی صورت میں لکھ کر \عددی{x_0=0} سے شروع کرتے ہوئے اس کا جذر تلاش کریں۔\\
جواب:\quad
\عددی{x_1=\num{-0.2}}، \عددی{x_2=\num{-0.20032}}، \عددی{x_3=\num{-0.200323}}
\انتہا{سوال}
%=======================
\ابتدا{سوال}\شناخت{سوال_اعدادی_دہرانا_درکار_ب}\quad
سوال \حوالہ{سوال_اعدادی_دہرانا_درکار_الف} میں دیے گئے مساوات کا جذر  \عددی{x=1} کے قریب پایا جاتا ہے۔مساوات کو \عددی{x=\sqrt[\leftroot{-2}5]{x+0.2}} لکھ کر \عددی{x_0=1} سے شروع کرتے ہوئے اس جذر کو تلاش کریں۔\\
جواب:\quad
$\num{1.0447}$ 
\انتہا{سوال}
%=========================
\ابتدا{سوال}\quad
سوال \حوالہ{سوال_اعدادی_دہرانا_درکار_ب} میں اگر آپ \عددی{x=x^5-0.2} لکھ کر \عددی{x_0=1} سے شروع کریں تو کیا حاصل ہو گا؟\\
جواب:\quad
$\num{-0.200322}$
\انتہا{سوال}
%========================
\ابتدا{سوال}\quad
دہرانے کی ترکیب استعمال کرتے ہوئے دکھائیں کی مساوات \عددی{x=\tan x} کا کم تر جذر تقریباً \عددی{4.49} ہے۔اشارہ۔ مساوات کی ترسیم سے اخذ کریں کہ جذر 
\عددی{x_0=\tfrac{3\pi}{2}} کے قریب پایا جاتا ہے؛ مساوات کو \عددی{x=\pi+\tan^{-1} x} (کیوں؟) لکھ کر آگے بڑھیں۔
\انتہا{سوال}
%==========================
\ابتدا{سوال}\شناخت{سوال_اعدادی_دہرانا_درکار_پ}\quad
\عددی{x_0=2} سے شروع کرتے ہوئے \عددی{\sqrt{5}} کو مثال \حوالہ{مثال_اعدادی_دہرانا_پ} کی ترکیب سے حاصل کرتے ہوئے \عددی{x_1, x_2, x_3,x_4} تلاش کریں۔اب \عددی{\sqrt{5}=\num{2.236068}} استعمال کرتے ہوئے خلل حاصل کریں۔\\
جواب:\quad
\عددی{\epsilon_1=\num{0.236068}}، \عددی{\epsilon_2=\num{0.013932}}، \عددی{\epsilon_3=\num{0.000043}}،
 \عددی{\epsilon_4=\num{0.000000}}
\انتہا{سوال}
%============================
\ابتدا{سوال}\quad
دکھائیں کہ مثال \حوالہ{مثال_اعدادی_دہرانا_پ} میں ہمارے پاس
\begin{align*}
x^2_{n+1}-c=\frac{1}{4}\big(x_n-\frac{c}{x_n}\big)^2
\end{align*}
ہے جو درستگی کی ناپ ہے۔دکھائیں کہ تقریباً
\begin{align*}
\abs{x_n-\sqrt{c}}\approx \frac{1}{2}\abs{x_n-\frac{c}{x_n}}
\end{align*}
ہو گا۔ اس کا اطلاق سوال \حوالہ{سوال_اعدادی_دہرانا_درکار_پ} پر کریں۔
\انتہا{سوال}
%=============================
\ابتدا{سوال}\quad
مثبت \عددی{x} محور پر ایسا وقفہ تلاش کریں کہ \عددی{c=2} لیتے ہوئے مسئلہ \حوالہ{مسئلہ_اعدادی_ارتکاز_شرط} کی شرط کو  مثال \حوالہ{مثال_اعدادی_دہرانا_پ} کے دہرانے کی ترکیب مطمئن کرتی ہو۔ \\
جواب:\quad
$x\ge \sqrt{\tfrac{2}{1+2\alpha}}\,\,,\quad \alpha<1$
\انتہا{سوال}
%===============================
\ابتدا{سوال}\quad
جذر الکعب کے لئے ترکیب نیوٹن بنائیں۔اس ترکیب کو استعمال کرتے ہوئے \عددی{x_0=2} سے شروع کر کے تین قدم چل کر  \عددی{\sqrt[\leftroot{-2}3]{7}} تلاش کریں۔
\انتہا{سوال}
%========================
\ابتدا{سوال}\quad
مثبت عدد \عددی{c} کا \عددی{k} واں جذر حاصل کرنے کے لئے ترکیب نیوٹن بنائیں۔\\
جواب:\quad
$f(x)=x^k-c,\quad x_{n+1}=(1-\tfrac{1}{k})x_n+\tfrac{c}{kx_n^{k-1}}$
\انتہا{سوال}
%===========================
\ابتدا{سوال}\شناخت{سوال_اعدادی_غلط_مقام_درکار_الف}\quad
\عددی{x^4=2} کا حقیقی جذر بذریعہ غلط مقام دہرانے کی ترکیب حاصل کریں۔ \\
جواب:\quad
$0,\quad 1$
\انتہا{سوال}
%=========================
\ابتدا{سوال}\quad
\عددی{x^4=2x} کا حقیقی جذر بذریعہ غلط مقام دہرانے کی ترکیب حاصل کریں۔ \\
جواب:\quad
$0,\quad 1.260$
\انتہا{سوال}
%=========================
\ابتدا{سوال}\quad
\عددی{3\sin x=2x} کا حقیقی جذر بذریعہ غلط مقام دہرانے کی ترکیب حاصل کریں۔ \\
جواب:\quad
$0,\quad 1.49$
\انتہا{سوال}
%=========================
\ابتدا{سوال}\quad
سوال \حوالہ{سوال_اعدادی_غلط_مقام_درکار_الف} میں حاصل کردہ مثبت جذر ہر صورت اصل جذر سے معمولی کم ہو گا۔ایسا کیوں ہے؟
\انتہا{سوال}
%========================
\ابتدا{سوال}\quad
ترکیب نیوٹن میں \عددی{f'(x)} کا حساب کرنا ہوتا ہے۔عملی استعمال میں کبھی کبھار یہ قدم کافی پیچیدہ ثابت ہو سکتا ہے۔\عددی{f'(x)} سے چھٹکارا حاصل کرنا کا ایک طریقہ یہ ہے کہ اس کی جگہ \عددی{\tfrac{f(x_n)-f(x_{n-1})}{x_n-x_{n-1}}} استعمال کیا جائے۔یوں حاصل کردہ کلیہ کا غلط مقام کلیہ کے ساتھ کیا تعلق پایا جاتا ہے؟
\انتہا{سوال}
%======================
\ابتدا{سوال}\quad
فرض کریں بند وقفہ \عددی{I} میں \عددی{g} استمراری ہے اور اس کا حلقہ بھی \عددی{I} میں پایا جاتا ہے۔ دکھائیں کہ مساوات \عددی{x=g(x)} کا کم از کم ایک حل اس وقفہ میں پایا جائے گا۔دکھائیں کہ اس وقفہ میں مساوات کے زیادہ جذر بھی ممکن ہیں۔
\انتہا{سوال}
%==================

\حصہ{متناہی فرق}
متناہی فرق کا استعمال اعدادی تجزیہ کے کئی شاخوں میں پایا جاتا ہے مثلاً دو قیمتوں کے درمیان قیمت کا تخمینہ لگانے میں، جدول کی جانچ پڑتال میں، تخمینہ لگانے میں، تفرق میں، اور تفرقی مساوات کے حل میں۔ ہم فرض کرتے ہیں کہ ہمیں تفاعل \عددی{f} کی اعدادی قیمتوں  \عددی{f_j=f(x_j)} کا جدول دیا گیا ہے جہاں نقطے \عددی{x_j} ایک جیسے  فاصلے پر ہیں۔ 
\begin{align*}
x_0,\quad x_1=x_0+h,\quad x_2=x_0+2h,\quad x_3=x_0+3h,\cdots \quad (h>0, \text{مقررہ})
\end{align*}
\عددی{f(x_j)} کو عموماً کسی کلیہ یا تجربہ سے حاصل کیا جاتا ہے۔ ہم جدول میں ہر  \عددی{f(x)} کو اگلی (بڑی) \عددی{x} کی مطابقتی قیمت سے تفریق کرتے ہوئے \اصطلاح{پہلا فرق}\فرہنگ{فرق!پہلا}\حاشیہب{first difference}\فرہنگ{difference!first} حاصل کرتے ہیں۔جدول \حوالہ{جدول_اعدادی_فرق_الف} میں اس کی مثال پیش کی گئی ہے جہاں \عددی{f(x)=x^3,\,\,x=-3(1)3} ہیں\حاشیہد{\عددی{x=a(h)b} کا مطلب ہے کہ تفاعل کی قیمتیں \عددی{x=a}، \عددی{x=x+h}،\عددی{x=a+2h}، \نقطے، \عددی{x=b} پر دی گئی ہیں۔}۔یہی طریقہ پہلی فرق پر لاگو کرتے ہوئے \عددی{f} کا \اصطلاح{دوسرا فرق}\فرہنگ{فرق!دوسرا}\حاشیہب{second difference}\فرہنگ{difference!second} حاصل کیا جاتا ہے۔اسی طرح باقی فرق بھی حاصل کیے جاتے ہیں۔جدول فرق میں ہر فرق کو اپنی قطار میں گزشتہ قطار (جس سے فرق حاصل کیا گیا ہے) کی اندراج کی درمیان برابر  مقام پر درج کیا جاتا ہے۔نقطہ اعشاریہ اور فرق کی بائیں صفروں کو نظر انداز کیا جاتا ہے (جدول \حوالہ{جدول_اعدادی_فرق_ب})۔ 

\begin{table}
\caption{\RL{تفاعل \عددی{f(x)=x^3,\,x=-3(1)3} کا جدول فرق}}
\label{جدول_اعدادی_فرق_الف}
\centering
\begin{otherlanguage}{english}
\begin{tabular}{|RR | RRRR|}
\hline
\Tstrut
x&f(x)=x^3&\text{\urdufont{\RL{پہلا فرق}}}& \text{\urdufont{\RL{دوسرا فرق}}}&\text{\urdufont{\RL{تیسرا فرق}}}&\text{\urdufont{\RL{چوتھا فرق}}}\\
\hline
\Tstrut 
-3&-27&&&&\\
&&19&&&\\
-2&-8&&-12&&\\
&&7&&6&\\
-1&-1&&-6&&0\\
&&1&&6&\\
0&0&&0&&0\\
&&1&&6&\\
1&1&&6&&0\\
&&7&&6&\\
2&8&&12&&\\
&&19&&&\\
3&27&&&&\\
\hline
\end{tabular}
\end{otherlanguage}
\end{table}

%
\begin{table}
\caption{تفاعل \عددی{f(x)=\frac{1}{x},\,\, x=1(0.2)2} کا جدول فرق۔ملحوظ ہندسوں کی تعداد چار ہے۔}
\label{جدول_اعدادی_فرق_ب}
\centering
\begin{otherlanguage}{english}
\begin{tabular}{|RR | RRR|}
\hline
x&f(x)=x^3&\text{\urdufont{\RL{پہلا فرق}}}& \text{\urdufont{\RL{دوسرا فرق}}}&\text{\urdufont{\RL{تیسرا فرق}}}\\
\hline
\Tstrut 
1.0&1.0000&&&\\
&&-1667&&\\
1.2&0.8333&&477&\\
&&-1190&&-180\\
1.4&0.7143&&297&\\
&&-893&&-98\\
1.6&0.6250&&199&\\
&&-694&&-61\\
1.8&0.5556&&138&\\
&&-556&&\\
2.0&0.5000&&&\\
\hline
\end{tabular}
\end{otherlanguage}
\end{table}

جدول فرق میں فرق کو ظاہر کرنے  کے تین مختلف  طریقے رائج ہیں۔ان میں سے جو بھی طریقہ استعمال کیا جائے، جدول میں نہ کوئی فرق تبدیل ہو گا اور نا ہی اس کا مقام۔پہلی (اور غالباً اہم ترین) اظہار  جس کو \اصطلاح{وسطی فرق}\فرہنگ{فرق!وسطی}\حاشیہب{central difference}\فرہنگ{difference!central} کہتے ہیں درج ذیل ہے
\begin{equation*}
\begin{array}{lllll}
x_{-2}&f_{-2}&&&\\
&&\delta f_{\!-3/2}&&\\
x_{-1}&f_{-1}&&\delta^2f_{-1}&\\
&&\delta f_{-1/2}&&\delta^3f_{-1/2}\\
x_0&f_0&&\delta^2f_{0}&\\
&&\delta f_{-1/2}&&\delta^3f_{1/2}\\
x_1&f_0&&\delta^2f_{1}&\\
&&\delta f_{3/2}&&\\
x_2&f_2&&&\\
&&&&
\end{array}
\end{equation*}
جہاں \عددی{\delta f_{-3/2}=f_{-1}-f_{-2}} اور \عددی{\delta f_{-1/2}=f_{0}-f_{-1}} ہیں۔ وسطی فرق کا  عمومی جزو
\begin{align}
\delta f_{m+1/2}=f_{m+1}-f_m
\end{align}
ہے جہاں دائیں ہاتھ دو زیر نوشت کا مجموعہ  بائیں ہاتھ کا زیر نوشت دے گا۔اسی طرح
\begin{align*}
\delta^2 f_{m}=\delta f_{m+1/2}-\delta f_{m-1/2}
\end{align*}
ہو گا۔دیگر فرق بھی اسی طرح حاصل کیے جاتے ہیں۔ایک جیسی زیر نوشت والے اجزاء ایک ہی صف میں پائے جاتے ہیں۔(دھیان رہے کہ ضروری نہیں ہے کہ جدول میں \عددی{x} کی سب سے چھوتی قیمت \عددی{x_0} ہو۔مثال کے طور پر جدول \حوالہ{جدول_اعدادی_فرق_ب} میں ہم \عددی{x_0=1.6} لے سکتے ہیں؛ تب \عددی{f_0=0.6250}، \عددی{\delta f_{1/2}=-0.0694}، \عددی{\delta^2 f_0=0.0199}، \نقطے ہوں گے۔)

دوسری اظہار جس کو \اصطلاح{آگے فرق}\فرہنگ{فرق!آگے}\حاشیہب{forward difference}\فرہنگ{difference!forward} کہتے ہیں درج ذیل ہے
\begin{equation*}
\begin{array}{lllll}
x_{-2}&f_{-2}&&&\\
&&\Delta f_{-2}&&\\
x_{-1}&f_{-1}&&\Delta^{\,2} f_{-2}&\\
&&\Delta f_{-1}&&\Delta^{\,3} f_{-2}\\
x_0&f_0&&\Delta^{\,2} f_{-1}&\\
&&\Delta f_{0}&&\Delta^{\,3} f_{-1}\\
x_1&f_1&&\Delta^{\,2} f_{0}&\\
&&\Delta f_{1}&&\\
x_2&f_2&&&
\end{array}
\end{equation*}
جس میں \عددی{\Delta f_{-2}=f_{-1}-f_{-2}}، \عددی{\Delta f_{-1}=f_{0}-f_{-1}}  اور \عددی{\Delta f_{0}=f_{1}-f_{0}} ہیں۔اس کا عمومی جزو
\begin{align}
\Delta f_m=f_{m+1}-f_m
\end{align}
ہے۔اسی طرح
\begin{align*}
\Delta^{\,2}f_m=\Delta f_{m+1}-\Delta f_m
\end{align*}
ہو گا۔مثال کے طور پر اگر جدول \حوالہ{جدول_اعدادی_فرق_ب} میں \عددی{x_0=1.6} لیا جائے تب \عددی{f_0=0.6250}، \عددی{\Delta f_0=-0.0694}،
 \عددی{\Delta^{\,2} f_0=0.0138} ہوں گے۔ایک جیسے زیر نوشت والے اجزاء ترچھی لکیروں نیچے کی رخ یا جدول میں \ترچھا{آگے رخ} لکیروں پر پائے جائیں گے۔ 

تیسری اظہار جس کو \اصطلاح{پیچھے فرق}\فرہنگ{فرق!پیچھے}\حاشیہب{backward difference}\فرہنگ{difference!backward} کہتے ہیں درج ذیل ہے
\begin{equation*}
\begin{array}{lllll}
x_{-2}&f_{-2}&&&\\
&&\nabla f_{-1}&&\\
x_{-1}&f_{-1}&&\nabla^{\,2} f_{0}&\\
&&\nabla f_{0}&&\nabla^{\,3} f_{1}\\
x_0&f_0&&\nabla^{\,2} f_{1}&\\
&&\nabla f_{1}&&\nabla^{\,3} f_{2}\\
x_1&f_1&&\nabla^{\,2} f_{2}&\\
&&\nabla f_{2}&&\\
x_2&f_2&&&
\end{array}
\end{equation*}
جہاں \عددی{\nabla f_{-1}=f_{-1}-f_{-2}}، \عددی{\nabla f_{0}=f_{0}-f_{-1}} اور \عددی{\nabla f_{1}=f_{1}-f_{0}} ہیں۔عمومی جزو
\begin{align}
\nabla f_m=f_m-f_{m-1}
\end{align}
ہو گا۔اسی طرح 
\begin{align*}
\nabla^{\,2}f_m=\nabla f_m-\nabla f_{m-1}
\end{align*}
ہو گا۔ باقی اجزاء بھی اسی طرح حاصل کیے جاتے ہیں۔
ایک جیسے زیر نوشت والے اجزاء ترچھی لکیروں پر اوپر رخ یا جدول میں \ترچھا{پیچھے رخ} لکیروں پر پائے جاتے ہیں۔ جدول کی آخر میں حساب کے دوران پیچھے فرق عموماً زیادہ مدد گار ثابت ہوتا ہے۔ 

جدول میں کسی بھی فرق کو اب تین مختلف علامتوں سے ظاہر کیا جا سکتا ہے۔مثال کے طور پر جدول \حوالہ{جدول_اعدادی_فرق_ب} میں ہم \عددی{x_0=1.6} لیں تب \عددی{-0.0893=\delta f_{-1/2}=\Delta f_{-1}=\nabla f_0} ہو گا۔یوں عمومی طور پر 
\begin{align*}
\delta^n f_m=\Delta^{\,n}f_{m-n/2}=\nabla^{\,n}f_{m+n/2}
\end{align*}
ہو گا۔

جدول میں \موٹا{غلطیوں کی نشاندہی} کرنے کے لئے  فرق کا سہارا لیا جاتا ہے۔جیسا جدول \حوالہ{جدول_اعدادی_فرق_پ} میں دکھایا گیا ہے، تفاعل میں خلل \عددی{\epsilon} جلد تمام فرق میں پھیل جاتا ہے۔یوں فرق میں بہت زیادہ اتار چڑھاو  تفاعل کی قیمت میں غلطی کو ظاہر کرتی ہے۔ظاہر ہے کہ کم تعداد کی ملحوظ ہندسوں کی بنا معمولی اتار چڑھاو ہر صورت پائی جائے گی۔
\begin{table}
\caption{غلطی تمام فرق میں پھیل جاتی ہے۔یہاں تفاعل \عددی{f(x)=\sqrt{x},\,\,x=2.0(0.1)2.6} ہے اور ملحوظ ہندسے چار ہیں۔ غلطی \عددی{f(2.3)} میں ہے۔}
\label{جدول_اعدادی_فرق_پ}
\centering
\begin{otherlanguage}{english}
\begin{tabular}{|R|RRRR|RRRR|RRRR|}
\hline
\Tstrut 
x&\sqrt{x}&\multicolumn{3}{C|}{\text{\urdufont{فرق}}} &\sqrt{x}&\multicolumn{3}{C|}{\text{\urdufont{فرق}}}&\multicolumn{4}{C|}{\text{\urdufont{غلطی \عددی{\epsilon} کا پھیلنا}}}\\
\hline
\Tstrut 
2.0&1.4142&&&&1.41412&&&&&&&\\
&&349&&&&349&&&&&&\\
2.1&1.4491&&-8&&1.4491&&8&&&&&\\
&&341&&1&&341&&\underline{11}&&&&\epsilon\\
2.2&1.4832&&-7&&1.4832&&\underline{3}&&&&\epsilon&\\
&&334&&-1&&\underline{344}&&-\underline{31}&&\epsilon&&-3\epsilon\\
2.3&1.5166&&-8&&\underline{1.5176}&&-\underline{28}&&\epsilon&&-2\epsilon&\\
&&326&&1&&\underline{316}&&\underline{31}&&-\epsilon&&3\epsilon\\
2.4&1.5492&&-7&&1.5492&&\underline{3}&&&&\epsilon&\\
&&319&&2&&319&&-\underline{8}&&&&-\epsilon\\
2.5&1.5811&&-5&&1.5811&&-\underline{5}&&&&&\\
&&314&&&&314&&&&&&\\
2.6&1.6125&&&&1.6125&&&&&&&\\
\hline
\end{tabular}
\end{otherlanguage}
\end{table}

تفاعل کو \ترچھا{کثیر رکنی} سے ظاہر کرنے میں بھی فرق اہم کردار ادا کرتا ہے۔قدم \عددی{h} لیتے ہوئے \عددی{n} درجی کثیر رکنی \عددی{p_n(x)=a_0x^n+a_1x^{n-1}+\cdots+a_n} کے جدول فرق میں تمام \عددی{n} ویں فرق مستقل (\عددی{n!h^na_0} کے برابر) ہوں گے اور ان سے بلند فرق صفر ہوں گے۔ایسا اس لئے ہو گا کہ پہلے فرق 
\begin{align*}
p_n(x+h)-p_n(x)=a_0[(x+h)^n-x^n]+\cdots=a_0nhx^{n-1}+\cdots
\end{align*} 
کا درجہ \عددی{n-1} ہے، دوسرے فرق کے کثیر رکنی کا درجہ \عددی{n-2} ہو گا اور اس کے پہلے جزو کا عددی سر \عددی{a_0n(n-1)h^2} ہو گا، وغیرہ وغیرہ۔یوں  اگر تفاعل \عددی{f} کے جدول فرق میں \عددی{n} ویں فرق کسی حلقہ میں تقریباً مستقل ہوں تب جدول کی قیمتوں کو اس حلقہ میں \عددی{n} درجی کثیر رکنی \عددی{p_n} سے ظاہر کیا جا سکتا ہے۔آئیں دیے  گیے  \عددی{f} کی صورت میں کثیر رکنی \عددی{p_n} کے حصول کی ایک ترکیب دیکھیں۔     

%======================
\ابتدا{مثال}\شناخت{مثال_اعدادی_تفاعل_اظہار_کثیر_رکنی}\quad \موٹا{تفاعل کو کثیر رکنی سے ظاہر کرنا}\\
جدول \حوالہ{جدول_اعدادی_فرق_پ} میں دوسرا فرق تقریباً مستقل (\عددی{-7} کے برابر) ہیں۔یوں ہم دو درجی کثیر رکنی \عددی{p_2} تلاش کر سکتے ہیں جو دیے گیے تفاعل کے مشابہ ہو گا۔ہم  پہلے جدول فرق بناتے ہیں۔ یہ فرض کرتے ہوئے کہ تمام دوسرے فرق ٹھیک ٹھیک \عددی{-7} کے برابر ہیں ہم حلقہ کے وسط میں تفاعل کی کوئی قیمت اور پہلا فرق منتخب کرتے ہیں مثلاً \عددی{1.5166} اور \عددی{334} جس سے جدول \حوالہ{جدول_اعدادی_فرق_ت} حاصل ہوتا ہے۔\عددی{p_2} کے پہلے عددی سر کو
\begin{align*}
a_02!h^2=a_0\cdot2\cdot0.1^2=-0.0007=\text{\RL{دوسرا فرق}}
\end{align*}
 سے حاصل کیا جاتا ہے۔یوں \عددی{a_0=-\tfrac{0.0007}{0.02}=-0.035} ملتا ہے۔اس طرح
\begin{align*}
p_1(x)=p_2(x)+0.035x^2
\end{align*}
درجہ اول ہو گا اور جدول \حوالہ{جدول_اعدادی_فرق_ت} سے ہم حساب لگا کر دیکھتے ہیں کہ اس کے پہلے صفر تقریباً مستقل (\عددی{0.04915}) ہیں اور ہم جانتے ہیں کہ یہ \عددی{a_h} کے برابر ہے۔یوں \عددی{a_1=\tfrac{0.04915}{0.1}=0.4915} حاصل ہوتا ہے۔آخر میں \عددی{p_1(x)-0.4915x=a_2=0.5713} ہو گا لہٰذا
\begin{align*}
p_2(x)=-0.0350x^2+0.4915x+0.5713
\end{align*}
ہو گا۔اس مثال سے آپ دیکھ سکتے ہیں کہ فرق کو استعمال کرتے ہوئے کثیر رکنی حاصل کرنے سے پہلے،  مشابہ کثیر رکنی کی درستگی کا  معیار جانا جا سکتا ہے۔مشابہ کثیر رکنی کی حصول کے دیگر  تراکیب پر اگلے حصے میں غور کیا جائے گا۔
%
\begin{table}
\caption{تفاعل \عددی{f(x)=\sqrt{x}} کو دو درجی کثیر رکنی \عددی{p_2} سے ظاہر کرنا}
\label{جدول_اعدادی_فرق_ت}
\centering
\begin{otherlanguage}{english}
\begin{tabular}{|R|RRR|}
\hline
\Tstrut 
x&p_2(x)&\multicolumn{2}{C|}{\text{\urdufont{فرق}}}\\
\hline
\Tstrut 
2.0&1.4143&&\\
&&348&\\
2.1&1.4491&&-7\\
&&341&\\
2.2&1.4832&&-7\\
&&\underline{334}&\\
2.3&\underline{1.5166}&&-7\\
&&327&\\
2.4&1.5493&&-7\\
&&320&\\
2.5&1.5813&&-7\\
&&313&\\
2.6&1.6126&&\\
\hline
\end{tabular}
\end{otherlanguage}
\end{table}


\انتہا{مثال}
%===========================

\حصہء{سوالات}
%==================
\ابتدا{سوال}\quad
قلم و کاغذ استعمال کرتے ہوئے جدول \حوالہ{جدول_اعدادی_فرق_الف} حاصل کریں۔
\انتہا{سوال}
%==============================
\ابتدا{سوال}\quad
قلم و کاغذ استعمال کرتے ہوئے جدول \حوالہ{جدول_اعدادی_فرق_ب} حاصل کریں۔
\انتہا{سوال}
%==============================
\ابتدا{سوال}\quad
جدول \حوالہ{جدول_اعدادی_فرق_ب} میں \عددی{x_0=1.2} منتخب کرتے ہوئے (الف) وسطی فرق، (ب) آگے فرق اور (پ) پیچھے فرق کے جدول مکمل کریں۔
\انتہا{سوال}
%=======================
\ابتدا{سوال}\quad
\عددی{x_0=2} منتخب کرتے ہوئے تفاعل \عددی{f(x)=x^3} کا \عددی{x=0(1)5} کے لئے   (الف) وسطی فرق، (ب) آگے فرق اور (پ) پیچھے فرق کے جدول مکمل کریں۔
\انتہا{سوال}
%=========================
\ابتدا{سوال}\quad
درج ذیل دکھائیں۔
\begin{align*}
\delta^2 f_m&=f_{m+1}-2f_m+f_{m-1}\\
\delta^3 f_{m+1/2}&=f_{m+2}-3f_{m+1}+3f_m-f_{m-1}
\end{align*}
\انتہا{سوال}
%========================
\ابتدا{سوال}\quad
\عددی{f(x)=\tfrac{1}{x+1}} کی قیمتیں \عددی{x=0(0.2)1} کے لئے (الف) دو ملحوظ ہندسوں، (ب) تین ملحوظ ہندسوں اور (پ) چار ملحوظ ہندسوں تک حاصل کریں۔ان کے مطابقتی جدول فرق میں تعداد ہندسہ خلل کا آپس میں موازنہ کریں۔ 
\انتہا{سوال}
%==========================
\ابتدا{سوال}\quad
\عددی{x=0(1)10} کے لئے \عددی{f(x)=x^2} کا جدول فرق مکمل کریں۔ایک اور جدول میں \عددی{f(5)=25} کی جگہ \عددی{26} لکھتے ہوئے پہلا فرق، دوسرا فرق، تیسرا فرق اور چوتھا فرق تلاش کریں۔جدول میں غلطی کا پھیلنا دیکھیں۔  
\انتہا{سوال}
%===============
\ابتدا{سوال}\quad
فرق استعمال کرتے ہوئے درج ذیل جدول کی جانچ پڑتال کریں۔
\begin{align*}
\begin{array}{c|cccccc}
x&4.0&4.1&4.2&4.3&4.4&4.5\\
\hline
f(x)&0.250&0.244&0.242&0.233&0.227&0.222
\end{array}
\end{align*}
\انتہا{سوال}
%=====================
\ابتدا{سوال}\quad
مثال \حوالہ{مثال_اعدادی_تفاعل_اظہار_کثیر_رکنی} میں کی گئی  تمام حساب خود کریں۔  
\انتہا{سوال}
%========================

\حصہ{باہمی تحریف}
عموماً تفاعل \عددی{f(x)} کی قیمتوں کا جدول دیا گیا ہو گا اور ہمیں ان \عددی{x} پر تفاعل کی قیمت درکار ہو گی جو جدول میں دیے گئے \عددی{x} کی قیمتوں کے درمیان پائے جاتے ہوں۔ایسی قیمتوں کے حصول کی عمل کو ہم \اصطلاح{باہمی تحریف}\فرہنگ{تحریف!باہمی}\حاشیہب{interpolation}\فرہنگ{interpolation} کہیں گے۔اس عمل میں \عددی{f(x)} کی استعمال ہونے والی قیمتوں کو \اصطلاح{چول قیمتیں}\فرہنگ{چول!قیمت}\حاشیہب{pivotal values}\فرہنگ{pivotal value} کہتے ہیں۔باہمی تحریف کی ترکیب اس مفروضہ پر مبنی ہے کہ نقطہ \عددی{x} کے قریب تفاعل \عددی{f(x)} کو کثیر رکنی \عددی{p} سے ظاہر کرنا ممکن ہے لہٰذا \عددی{x} کے قریب کسی بھی نقطے پر \عددی{p} کی قیمت کو اس نقطے پر تفاعل کی قیمت تصور کیا جا سکتا ہے۔

\begin{figure}
\centering
\begin{tikzpicture}
\draw(-0.5,0)--(4,0);
\draw[thick](0,1) to [out=30,in=-170] (3,2);
\draw(0,1)--(3,2)coordinate[pos=0.35](kA);
\draw(0,1)node[ocirc]{}--(0,0)node[below]{$x_0$}node[pos=0.5,left]{$f_0$};
\draw(3,2)node[ocirc]{}--(3,0)node[below]{$x_1$}node[pos=0.3,right]{$f_1$};
\draw(kA)--($(-0.5,0)!(kA)!(4,0)$)coordinate[pos=0.9](kB);
\draw[stealth-stealth](0,0.8)--++(3,0)node[pos=0.75,above]{$h$};
\draw[stealth-stealth] (kB)--($(0,0)!(kB)!(0,1)$)node[pos=0.5,above]{$rh$};
\end{tikzpicture}
\caption{خطی باہمی تحریف}
\label{شکل_اعدادی_خطی_باہمی_تحریف}
\end{figure}

سادہ ترین طریقہ \اصطلاح{خطی باہمی تحریف}\فرہنگ{تحریف!باہمی خطی}\حاشیہب{linear interpolation}\فرہنگ{interpolation!linear} ہے۔اس ترکیب میں جدول میں درکار \عددی{x} کی دونوں جانب درج نقطوں \عددی{x_0} اور \عددی{x_1} کے مابین سیدھی قطع سے اس خطہ میں  \عددی{f(x)} کو ظاہر کیا جاتا ہے (شکل \حوالہ{شکل_اعدادی_خطی_باہمی_تحریف})۔یوں جیسا ہم چھوٹی جماعتوں کی حساب سے جانتے ہیں، نقطہ \عددی{x=x_0+rh} پر \عددی{f} کی قیمت تقریباً
\begin{align}\label{مساوات_اعدادی_خطی_باہمی_تحریف}
f(x)\approx p_1(x)=f_0+r(f_1-f_0)=f_0+r\Delta f_0\quad \quad (r=\frac{x-x_0}{h},\,\,0\le r\le 1)
\end{align}
ہو گی۔یوں اگر \عددی{\ln 9.0=2.197} اور \عددی{\ln 9.5=2.251} ہوں تب \عددی{\ln 9.2} حاصل کرنے کی خاطر ہم  \عددی{r=\tfrac{0.2}{0.5}=0.4} حاصل کر کے
\begin{align*}
\ln 9.2=\ln 9.0+0.4(\ln 9.5-\ln 9.0)=2.219
\end{align*}
حاصل کرتے ہیں۔  

خطی باہمی تحریف اس صورت تسلی بخش ہو گی جب جدول میں \عددی{x} کی قیمتیں اتنی قریب قریب ہوں کہ ان کے مابین منحنی سے سیدھی قطعات کی انحراف کم ہو، مثلاً ہر  \عددی{x_0} اور \عددی{x_1} کے درمیان ہر \عددی{x} کے لئے انحراف جدول میں آخری ہندسہ کی اکائی کی نصف (\عددی{\tfrac{1}{2}}) سے کم ہو۔

\اصطلاح{دو درجی باہمی تحریف}\فرہنگ{تحریف!باہمی دو درجی}\حاشیہب{quadratic interpolation}\فرہنگ{interpolation!quadratic} میں ہم \عددی{x_0} اور \عددی{x_2=x_0+2h} کے درمیان منحنی \عددی{f} کو ایسی دو درجی  قطع مکافی سے ظاہر کرتے ہیں جو نقطہ \عددی{(x_0,f_0)}، \عددی{(x_1,f_1)} اور \عددی{(x_2,f_2)} سے گزرتی ہو۔یوں بہتر کلیہ
\begin{align}\label{مساوات_اعدادی_دو_درجی_باہمی_تحریف}
f(x)\approx p_2(x)=f_0+r\Delta f_0+\frac{r(r-1)}{2}\Delta^2 f_0\quad \quad (r=\frac{x-x_0}{h},\,\, 0\le r\le 2)
\end{align} 
اخذ ہوتا ہے جہاں \عددی{x=x_0+rh} ہے۔یوں \عددی{x=x_0\,\,(r=0)} کے لئے دایاں ہاتھ \عددی{f_0} کے برابر ہو گا؛ \عددی{x=x_1\,\,(r=1)} کے لئے بایاں ہاتھ \عددی{f_0+\Delta f_0=f_1} کے برابر ہو گا اور \عددی{x=x_2\,\,(r=2)} کے لئے اس کی قیمت
\begin{align*}
f_0+2(f_1-f_0)+[(f_2-f_1)-(f_1-f_0)]=f_2
\end{align*}
ہو گی۔

%=====================
\ابتدا{مثال}\quad \موٹا{خطی اور دو درجی باہمی تحریف}\\
اگر \عددی{\ln 9.0=2.1972} اور \عددی{\ln 9.5=2.2513} ہوں تب  مساوات \حوالہ{مساوات_اعدادی_خطی_باہمی_تحریف} سے \عددی{\ln 9.2=2.2188} حاصل ہوتا ہے جو تین ملحوظ ہندسوں تک درست ہے جبکہ \عددی{\ln 10.0=2.3026} لیتے ہوئے  مساوات \حوالہ{مساوات_اعدادی_دو_درجی_باہمی_تحریف} 
\begin{align*}
\ln 9.2=2.1972+0.4\cdot 0.0541+\frac{0.4\cdot(-0.6)}{2}(-0.0028)=2.2192
\end{align*}
دیتی ہے جو چار ملحوظ ہندسوں تک درست جواب ہے۔
 \انتہا{مثال}
%======================

مزید بہتر جوابات حاصل کرنے کی خاطر زیادہ بلند درجی کثیر رکنی استعمال کرنی ہو گی۔\عددی{n+1} مختلف نقطوں پر قیمتوں سے  یکتا \عددی{n} درجی کثیر رکنی حاصل ہو گی۔ہمیں یہاں ایسی  کثیر رکنی \عددی{p_n} درکار ہے  کہ
\begin{align*}
p_n(x_0)=f_0, \cdots, p_n(x_n)=f_n
\end{align*}
 ہوں جہاں \عددی{f_0=f(x_0)}، \نقطے،\عددی{f_n=f(x_n)} جدول میں \عددی{f} کی قیمتیں ہیں۔یہ کثیر رکنی \اصطلاح{آگے فرق، باہمی تحریف کلیہ نیوٹن}\فرہنگ{نیوٹن!آگے فرق، باہمی تحریف کلیہ}\حاشیہب{Newton's forward-difference interpolation formula}\فرہنگ{Newton!forward-difference interpolation formula}
\begin{gather}
\begin{aligned}\label{مساوات_اعدادی_نیوٹن_باہمی_تحریف_الف}
f(x)\approx p_n(x)=f_0+r\Delta f_0&+\frac{r(r-1)}{2!}\Delta^2 f_0+\frac{r(r-1)(r-2)}{3!}\Delta ^3 f_0+\cdots\\
&\quad +\frac{r(r-1)\cdots (r-n+1)}{n!}\Delta ^n f_0\\
(x=x_0+rh,\,\, r&=\frac{x-x_0}{h},\,\, 0\le r\le n)
\end{aligned}
\end{gather}
دیتی ہے۔اس کلیہ میں \عددی{n=1} پر کرنے سے  مساوات \حوالہ{مساوات_اعدادی_خطی_باہمی_تحریف} اور \عددی{n=2} پر کرنے سے  مساوات \حوالہ{مساوات_اعدادی_دو_درجی_باہمی_تحریف} حاصل ہوتا ہے۔ہمیں اب \عددی{p_n(x_k)=f_k\,\, (k=0,1,\cdots,n)} ثابت کرنا ہو گا۔ مساوات \حوالہ{مساوات_اعدادی_نیوٹن_باہمی_تحریف_الف} کے دائیں ہاتھ سے
\begin{align}\label{مساوات_اعدادی_نیوٹن_باہمی_تحریف_ب}
f_k=\binom{k}{0}f_0+\binom{k}{1}\Delta f_0+\binom{k}{2}\Delta^2 f_0+\cdots +\binom{k}{k}\Delta^k f_0
\end{align}
لکھا جا سکتا ہے جہاں \اصطلاح{ثنائی عددی}\فرہنگ{ثنائی سر}\حاشیہب{binomial coefficients}\فرہنگ{binomial!coefficients} سر  درج ذیل ہیں جہاں \عددی{s!=1\cdot 2\cdot 3\cdots s} کے برابر ہے۔
\begin{align}\label{مساوات_اعدادی_نیوٹن_باہمی_تحریف_پ}
\binom{k}{0}=1,\quad \binom{k}{s}=\frac{k(k-1)(k-2)\cdots(k-s+1)}{s!}\quad (s\ge 0,\text{\RL{عدد صحیح}})
\end{align}
در حقیقت مساوات \حوالہ{مساوات_اعدادی_نیوٹن_باہمی_تحریف_الف} میں \عددی{r=k} پر کرنے سے مساوات \حوالہ{مساوات_اعدادی_نیوٹن_باہمی_تحریف_الف} کا دایاں ہاتھ اور  مساوات \حوالہ{مساوات_اعدادی_نیوٹن_باہمی_تحریف_ب} بالکل ایک جیسے ہوں گے۔مساوات \حوالہ{مساوات_اعدادی_نیوٹن_باہمی_تحریف_ب} کو الکراجی ماخوذ سے ثابت کرتے ہیں۔

%===============
\ابتدا{ثبوت}
 \عددی{k=0} کے لئے مساوات \حوالہ{مساوات_اعدادی_نیوٹن_باہمی_تحریف_ب} درست ہے۔فرض کریں کہ یہ  \عددی{k=q} کے لئے بھی درست ہے۔تب مساوات \حوالہ{مساوات_اعدادی_نیوٹن_باہمی_تحریف_ب} میں \عددی{k=q} استعمال کر کے، \عددی{\Delta} کی اطلاق سے درج ذیل لکھا جا سکتا ہے۔
\begin{multline*}
f_{q+1}=f_q+\Delta f_q\\
=\binom{q}{0}f_0+\binom{q}{1} \Delta f_0+\binom{q}{2}\Delta^2 f_0+\cdots +\binom{q}{q}\Delta^q f_0\\
+\binom{q}{0}\Delta f_0+\binom{q}{1}\Delta^2 f_0+\binom{q}{2}\Delta^3 f_0+\cdots +\binom{q}{q}\Delta^{q+1} f_0
\end{multline*}
اس کلیہ میں \عددی{\Delta^s f_0} کا عددی سر (مساوات \حوالہ{مساوات_اعدادی_نیوٹن_باہمی_تحریف_پ})
\begin{align*}
\binom{q}{s}+\binom{q}{s-1}=\binom{q+1}{s}
\end{align*}
ہے جو \عددی{k=q+1} کے لئے مساوات \حوالہ{مساوات_اعدادی_نیوٹن_باہمی_تحریف_ب} دیتا ہے۔یوں الکراجی ماخوذ کے ذریعہ ثبوت مکمل ہوتا ہے۔
\انتہا{ثبوت}
%===========================

مساوات \حوالہ{مساوات_اعدادی_نیوٹن_باہمی_تحریف_الف} کی طرح ایسا کلیہ جو پیچھے فرق پر مبنی ہو،  \اصطلاح{پیچھے فرق، باہمی تحریف کلیہ نیوٹن}\فرہنگ{نیوٹن!پیچھے فرق، باہمی تحریف کلیہ}\حاشیہب{Newton's backward-difference interpolation formula}\فرہنگ{Newton!backward-difference interpolation formula}
\begin{multline}\label{مساوات_اعدادی_نیوٹن_باہمی_تحریف_پیچھے}
f(x)\approx p_n(x)=f_0+r\nabla f_0+\frac{r(r+1)}{2!}\nabla^2 f_0+\cdots\\
\quad +\frac{r(r+1)\cdots (r+n-1)}{n!}\nabla ^n f_0
\end{multline}
ہے جہاں مساوات \حوالہ{مساوات_اعدادی_نیوٹن_باہمی_تحریف_الف} کی طرح \عددی{x=x_0+rh,\,\,r=\tfrac{x-x_0}{h},\,\,0\le r\le n} ہیں۔

باہمی تحریف کے کلیات اور استعمال پر کثیر مواد پایا جاتا ہے۔مثال کے طور پر  صرف جفت درجہ فرق پر مبنی کلیات پائے جاتے ہیں۔اس طرز کا ایک انتہائی اہم اور  سادہ ترین  \اصطلاح{کلیہ ایورٹ}\فرہنگ{باہمی تحریف!کلیہ ایورٹ}\حاشیہب{Everett formula}\فرہنگ{interpolation!Everett formula} درج ذیل ہے۔
\begin{align}\label{مساوات_اعدادی_کلیہ_ایورٹ}
f(x)\approx (1-r)f_0+rf_1+\frac{(2-r)(1-r)(-r)}{3!}\delta^2f_0+\frac{(r+1)r(r-1)}{3!}\delta^2f_1
\end{align}   
جہاں \عددی{r=\tfrac{x-x_0}{h},\,\,0\le r\le 1} ہیں۔

%=====================
\ابتدا{مثال}\شناخت{مثال_اعدادی_کلیہ_ایورٹ_الف}\quad \موٹا{کلیہ ایورٹ کا استعمال}\\
تفاعل \عددی{e^{1.24}} کی قیمت مساوات \حوالہ{مساوات_اعدادی_کلیہ_ایورٹ} میں دیے گئے کلیہ ایورٹ اور درج ذیل جدول سے حاصل کریں۔
\begin{align*}
\begin{array}{c|cc}
x&e^x&\delta^2\\
\hline
1.2&3.3201&333\\
1.3&3.6693&367
\end{array}
\end{align*}
اب \عددی{r=\tfrac{0.04}{0.1}=0.4} ہے  لہٰذا مساوات \حوالہ{مساوات_اعدادی_کلیہ_ایورٹ} درج ذیل دے گی
\begin{multline*}
e^{1.24}\approx 0.6\cdot 3.3201+0.4\cdot 3.6693+\frac{1.6\cdot 0.6\cdot(-0.4)}{6}\cdot 0.0333\\
+\frac{1.4\cdot 0.4\cdot(-0.6)}{6}\cdot 0.0367\\
=3.4598-0.0021-0.0021=3.4556
\end{multline*}
جو چار ملحوظ ہندسوں تک درست جواب ہے۔دھیان رہے کہ خطی باہمی تحریف \عددی{3.4598} دیتی ہے جو صرف دو ملحوظ ہندسوں تک درست جواب ہے۔ (آپ \عددی{e^{1.1}=3.0042} اور \عددی{e^{1.4}=4.0552} استعمال کرتے ہوئے دوسرے فرق کی جانچ پڑتال کر سکتے ہیں۔)
\انتہا{مثال}
%=======================

\اصطلاح{عمومی کلیہ ایورٹ}\فرہنگ{ایورٹ!عمومی کلیہ}\حاشیہب{Everett formula}\فرہنگ{Everett formula} درج ذیل ہے
\begin{multline}
f(x)=qf_0+rf_1+\binom{q+1}{3}\delta^2 f_0+\binom{r+1}{3}\delta^2 f_1\\
+\binom{q+2}{5}\delta^4 f_0+\binom{r+2}{5}\delta^4 f_1+\cdots
\end{multline}
جہاں \عددی{r=\tfrac{x-x_0}{h},\,\,0\le r\le 1} اور \عددی{q=1-r} ہیں۔اس کلیہ میں \عددی{\delta^4f_0} اور \عددی{\delta^2f_0} کے عددی سروں کی نسبت
\begin{align*}
\frac{\binom{q+2}{5}}{\binom{q+1}{3}}=\frac{q^2-4}{20}
\end{align*}
ہے۔اسی طرح \عددی{\delta^4f_1} اور \عددی{\delta^2f_1} کے عددی سروں کی نسبت \عددی{\tfrac{r^2-4}{20}}۔یہ دونوں نسبت وقفہ \عددی{0} تا \عددی{1} میں بہت کم تبدیل ہوتے ہیں۔یوں اگر ان کی جگہ ان کی کوئی موزوں اوسط قیمت \عددی{\mu} منتخب کی جائے  تب \اصطلاح{تبدیل شدہ دوسرے فرق}\فرہنگ{دوسرا فرق!تبدیل شدہ}\حاشیہب{modified second differences}\فرہنگ{second differences!modofoed}
\begin{align}
\delta_m^2f=\delta^2f+\mu\delta^4f,\quad \mu=-0.18393
\end{align}
 استعمال کرتے ہوئے  چوتھی فرق کے اثر کو مساوات \حوالہ{مساوات_اعدادی_کلیہ_ایورٹ} میں سمویا جا سکتا ہے، جہاں \عددی{\mu} کی دی گئی قیمت ایک موزوں قیمت ہے۔

ہم بغیر ثبوت پیش کیے بتلانا چاہتے ہیں کہ اگر \عددی{x_0,x_1,\cdots,x_n} کے آپس میں فاصلے اختیاری ہوں تب \عددی{n} درجی کثیر رکنی جو \عددی{(x_0,f_0)}،\نقطے،\عددی{(x_n,f_n)} سے گزرتا ہو، جہاں \عددی{f_j=f(x_j)} ہے،   \اصطلاح{منقسم فرق باہمی تحریف کلیہ نیوٹن}\فرہنگ{نیوٹن!باہمی تحریف،منقسم فرق کلیہ}\حاشیہب{Newton's divided difference interpolation formula}\فرہنگ{Newton!divided difference interpolation formula}
\begin{multline}\label{مساوات_اعدادی_نیوٹن_منقسم_فرق_الف}
f(x)\approx f_0+(x-x_0)f[x_0,x_1]+(x-x_0)(x-x_1)f[x_0,x_1,x_2]+\cdots\\
+(x-x_0)\cdots (x-x_{n-1})f[x_0,\cdots,x_n]
\end{multline}
 کا دایاں ہاتھ ہو گا جہاں \اصطلاح{منقسم فرق}\فرہنگ{فرق!منقسم}\حاشیہب{divided difference}\فرہنگ{difference!divided}  درج ذیل  دہرانے کے تعلقات دیتے ہیں۔
\begin{multline}\label{مساوات_اعدادی_نیوٹن_منقسم_فرق_ب}
f[x_0,x_1]=\frac{f(x_1)-f(x_0)}{x_1-x_0},\quad f[x_0,x_1,x_2]=\frac{f[x_1,x_2]-f[x_0,x_1]}{x_2-x_0},\cdots\\
f[x_0,\cdots,x_k]=\frac{f[x_1,\cdots,x_k]-f[x_0,\cdots,x_{k-1}]}{x_k-x_0}
\end{multline}
اگر \عددی{x_k=x_0+kh} (یکساں فاصلے) ہو تب \عددی{f[x_0,\cdots,x_k]=\tfrac{\Delta^k f_0}{k!h^k}} ہو گا اور  مساوات \حوالہ{مساوات_اعدادی_نیوٹن_منقسم_فرق_الف} سے  مساوات \حوالہ{مساوات_اعدادی_نیوٹن_باہمی_تحریف_الف} حاصل ہو گی۔


باہمی تحریف کی مختلف تراکیب  فرق میں ہم فرق معلوم کرتے ہیں جس کو جدول کی درستگی کے لئے بھی استعمال کیا جا سکتا ہے۔البتہ کس درجہ کی باہمی تحریف استعمال کی جائے،  عموماً اس سوال کا جدول میں جواب نہیں دیا جاتا ہے۔\اصطلاح{لیگرینج باہمی تحریف}\فرہنگ{لیگرینج!باہمی تحریف}\فرہنگ{تحریف!لیگرینج، باہمی}\حاشیہب{Lagrangian interpolation}\فرہنگ{interpolation!Lagragian} کی ترکیب \اصطلاح{لیگرینج باہمی تحریف} کے کلیہ
\begin{align}\label{مساوات_اعدادی_لیگرینج_الف}
f(x)\approx \Lagrange_n(x)=\sum\limits_{k=0}^{n}\frac{l_k(x)}{l_k(x_k)}f_k
\end{align}
پر مبنی ہے جہاں ضروری نہیں ہے کہ \عددی{x_0,\cdots,x_n} برابر فاصلوں پر ہوں اور
\begin{gather}
\begin{aligned}\label{مساوات_اعدادی_لیگرینج_ب}
l_0(x)&=(x-x_1)(x-x_2)\cdots(x-x_n)\\
l_k(x)&=(x-x_0)\cdots(x-x_{k-1})(x-x_{k+1})\cdots(x-x_n),\quad 0<k<n\\
l_n(x)&=(x-x_0)(x-x_1)\cdots(x-x_{n-1})
\end{aligned}
\end{gather}
ہیں۔ مساوات \حوالہ{مساوات_اعدادی_لیگرینج_الف} کو \اصطلاح{\عددی{n+1} نقطوں کا کلیہ لیگرینج}\فرہنگ{لیگرینج!کلیہ} کہتے ہیں۔چونکہ مساوات \حوالہ{مساوات_اعدادی_لیگرینج_ب} سے \عددی{j\ne k} کی صورت میں  \عددی{l_k(x_j)=0} اور \عددی{x=x_k} کی صورت میں \عددی{\tfrac{l_k(x)}{l_k(x_k)}=f_k} حاصل ہوتے ہیں  لہٰذا \عددی{\Lagrange_n(x_k)=f_k} ہو گا۔اس ترکیب میں فرق حاصل کرنے کی ضرورت نہیں ہے اور ہم مختلف \عددی{f_k} کے اثرات کو سیدھ و سیدھ دیکھ سکتے ہیں۔ ہاں اب حساب زیادہ مشکل ضرور ہو گا اور جدول میں غلطی کی جانچ پڑتال ممکن نہیں ہو گی۔اس لئے ضروری ہے کہ یہ ترکیب صرف مستند جدول پر لاگو کیا جائے۔

%========================
\ابتدا{مثال}\شناخت{مثال_اعدادی_لیگرینج_الف}\quad \موٹا{لیگرینج کلیہ باہمی تحریف کا استعمال}\\
\عددی{\ln 9.2} کی قیمت مساوات \حوالہ{مساوات_اعدادی_لیگرینج_الف} اور درج ذیل قیمتوں کی مدد سے تلاش کریں۔
\begin{align*}
\begin{array}{ccccc}
x&9.0&9.5&10.0&11.0\\
\ln x&\num{2.19722}&\num{2.25129}&\num{2.30259}&\num{2.39790}
\end{array}
\end{align*}
ہمارے پاس \عددی{l_0(x)=(x-9.5)(x-10)(x-11)}، \عددی{l_1(x)=(x-9)(x-10)(x-11)}، \نقطے ہیں۔یوں 
\begin{multline*}
\ln 9.2=\frac{-0.43200}{-1.00000}\cdot 2.19722+\frac{0.28800}{0.37500}\cdot 2.25129\\
+\frac{0.10800}{-0.50000}\cdot 2.30259+\frac{0.04800}{3.00000}\cdot 2.39790=2.21920
\end{multline*}
ہو گا جو پانچ ملحوظ ہندسوں تک درست جواب ہے۔
\انتہا{مثال}
%============================

\حصہء{سوالات}
%===================
\ابتدا{سوال}\quad
دکھائیں کہ مساوات \حوالہ{مساوات_اعدادی_دو_درجی_باہمی_تحریف} میں دیا گیا قطع مکافی نقطہ \عددی{(x_0,f_0)}، \عددی{(x_1,f_1)}، \عددی{(x_2,f_2)} سے گزرتا ہے۔
\انتہا{سوال}
%======================
جدول \حوالہ{جدول_اعدادی_سائن} کو سوال \حوالہ{سوال_اعدادی_سائن_تلاش_ب} تا سوال \حوالہ{سوال_اعدادی_سائن_تلاش_الف} میں استعمال کریں۔

%=============================
\ابتدا{سوال}\شناخت{سوال_اعدادی_سائن_تلاش_الف}\quad
\عددی{\sin 0.26} کی قیمت خطی باہمی تحریف (مساوات \حوالہ{مساوات_اعدادی_خطی_باہمی_تحریف}) سے تلاش کریں۔دکھائیں کہ اعشاریہ کے بعد پہلے دو ہندسے بالکل ٹھیک ٹھیک ہیں۔
\begin{table}
\caption{جدول برائے سوال \حوالہ{سوال_اعدادی_سائن_تلاش_الف} تا سوال \حوالہ{سوال_اعدادی_سائن_تلاش_ب}}
\label{جدول_اعدادی_سائن}
\centering
\begin{otherlanguage}{english}
\begin{tabular}{CCCCC}
x&\sin x&\phantom{xx}&\text{\RL{\urdufont{پہلا فرق}}}&\text{\RL{\urdufont{دوسرا فرق}}}\\
\Tstrut
0.0&\num{0.00000}&&&\\
&&&\num{19867}&\\
0.2&\num{0.19867}&&&\num{-792}\\
&&&\num{19075}&\\
0.4&\num{0.38942}&&&\num{-1553}\\
&&&\num{17522}&\\
0.6&\num{0.56464}&&&\num{-2250}\\
&&&\num{15272}&\\
0.8&\num{0.71736}&&&\num{-2861}\\
&&&\num{12411}&\\
1.0&\num{0.84147}&&&
\end{tabular}
\end{otherlanguage}
\end{table}
\انتہا{سوال}
%========================
\ابتدا{سوال}\شناخت{سوال_اعدادی_سائن_تلاش_پ}\quad
\عددی{\sin 0.26} کی قیمت دو درجی باہمی تحریف یعنی مساوات \حوالہ{مساوات_اعدادی_دو_درجی_باہمی_تحریف} کی مدد سے حاصل کریں۔دکھائیں پہلے تین ملحوظ ہندسے بالکل درست ہیں۔\\
جواب:\quad
$\num{0.25753}$
\انتہا{سوال}
%========================
\ابتدا{سوال}\شناخت{سوال_اعدادی_سائن_تلاش_ت}\quad
جدول \حوالہ{جدول_اعدادی_سائن} میں تیسرے فرق اور چوتھے فرق شامل کرتے ہوئے \عددی{\sin 0.26} کی قیمت مساوات \حوالہ{مساوات_اعدادی_نیوٹن_باہمی_تحریف_الف} کی مدد سے (الف) \عددی{n=3}  اور (ب) \عددی{n=4} لیتے ہوئے حاصل کریں۔نتائج کا موازنہ پانچ ملحوظ ہندسوں تک درست جواب \عددی{\sin 0.26=\num{0.25708}} کے ساتھ کریں۔آپ دیکھیں گے کہ \عددی{n=3} سے تین ملحوظ ہندسوں تک اور \عددی{n=4} سے پانچ ملحوظ ہندسوں تک درست جواب حاصل ہو گا۔
\انتہا{سوال}
%========================
\ابتدا{سوال}\quad
\عددی{\sin 0.26} کو مساوات \حوالہ{مساوات_اعدادی_نیوٹن_باہمی_تحریف_پیچھے} کی مدد سے (الف) \عددی{n=1} لے کر اور (ب) \عددی{n=2} لے کر حاصل کریں۔آپ دیکھیں گے کہ دونوں صورتوں میں پہلے دو ملحوظ ہندسے درست ہوں گے۔یوں موجودہ نتیجہ سوال \حوالہ{سوال_اعدادی_سائن_تلاش_الف} کے نتیجہ سے کم درست ہے۔ کیوں؟ \\
جوابات:\quad
(الف) \عددی{\num{0.25827}}، (ب) \عددی{\num{0.25827}}
\انتہا{سوال}
%=========================
\ابتدا{سوال}\quad
جدول \حوالہ{جدول_اعدادی_سائن} کو وسیع کرتے ہوئے  (الف) \عددی{n=3} لے کر، (ب) \عددی{n=4} لے کر اور (پ) \عددی{n=5} لے کر   مساوات \حوالہ{مساوات_اعدادی_نیوٹن_باہمی_تحریف_پیچھے} کی مدد سے \عددی{\sin 0.26} کی قیمت تلاش کریں۔آپ کو \عددی{x=-0.6,-0.4,-0.2,0,0.2} پر \عددی{\sin x} کی قیمتیں درکار ہوں گی اور مطابقتی فرق  درکار ہوں گے۔سائن تفاعل کی کون سی خاصیت اس وسعت کو آسان بناتی ہے۔موجودہ نتائج سوال \حوالہ{سوال_اعدادی_سائن_تلاش_ت} کے نتائج سے کیوں کم ٹھیک ہیں؟\\
جوابات:\quad
(الف) \عددی{\num{0.25709}}، (ب) \عددی{\num{0.25705}} اور (پ) \عددی{\num{0.25708}}؛ جواب (پ) پانچ ملحوظ ہندسوں تک درست ہے۔ 
\انتہا{سوال}
%===========================
\ابتدا{سوال}\شناخت{سوال_اعدادی_سائن_تلاش_ب}\quad
دکھائیں کہ بہت کم محنت کے ساتھ کلیہ ایورٹ (مساوات \حوالہ{مساوات_اعدادی_کلیہ_ایورٹ}) استعمال کرتے ہوئے \عددی{\sin 0.26=\num{0.25707}} حاصل ہوتا ہے۔
\انتہا{سوال}
%===========================
\ابتدا{سوال}\quad
مثال \حوالہ{مثال_اعدادی_کلیہ_ایورٹ_الف} میں کی گئی حساب کی تصدیق کریں۔
\انتہا{سوال}
%==========================
\ابتدا{سوال}\quad
\عددی{f(2.0)=\num{1.414214}}، \عددی{f(2.3)=\num{1.516575}} اور \عددی{f(2.6)=\num{1.612452}} استعمال کرتے ہوئے  تفاعل \عددی{f(x)=\sqrt{x}}  کی دو درجی باہمی تحریف کریں۔نتائج کا جدول \حوالہ{جدول_اعدادی_فرق_ت}  کے ساتھ موازنہ کریں۔ \\
جوابات:\quad
$f(x)\approx \num{0.566106}+\num{0.496098}x-\num{0.036022}x^2$
\انتہا{سوال}
%========================
\ابتدا{سوال}\quad
درج ذیل دکھائیں۔
\begin{align*}
\Delta^k f_n=\binom{k}{0}f_{n+k}-\binom{k}{1}f_{n+k-1}+-\cdots+(-1)^k\binom{k}{k}f_n
\end{align*}
\انتہا{سوال}
%====================
\ابتدا{سوال}\quad
\عددی{f(1)=2}، \عددی{f(2)=11} اور \عددی{f(4)=77} استعمال کرتے ہوئے 
عمومی کلیہ لیگرینج (مساوات \حوالہ{مساوات_اعدادی_لیگرینج_الف}) سے  \عددی{f(3)} تلاش کریں۔ \\
جواب:\quad
$8x^2-15x+9,\quad 36$
\انتہا{سوال}
%====================
\ابتدا{سوال}\شناخت{سوال_اعدادی_لیگرینج_درکار_الف}\quad
\عددی{\ln 8.5=\num{2.14007}} لیں جبکہ \عددی{\ln 9.0}، \عددی{\ln 9.5}، \عددی{\ln 10} کی قیمتیں مثال \حوالہ{مثال_اعدادی_لیگرینج_الف} میں دی گئی ہیں۔ \عددی{\ln 9.2} کو (الف) \عددی{n=3}  اور \عددی{x_0=8.8} لیتے ہوئے مساوات \حوالہ{مساوات_اعدادی_نیوٹن_باہمی_تحریف_الف}  سے حاصل کریں؛ (ب) \عددی{n=3} اور \عددی{x_0=10} لیتے ہوئے مساوات \حوالہ{مساوات_اعدادی_نیوٹن_باہمی_تحریف_پیچھے} سے حاصل کریں۔
\انتہا{سوال}
%=======================
\ابتدا{سوال}\quad
\عددی{\ln 8.5=\num{2.14007}} لیں جبکہ \عددی{\ln 9}، \عددی{\ln 10} اور \عددی{\ln 11} مثال \حوالہ{مثال_اعدادی_لیگرینج_الف} میں دی گئی ہیں۔اب \عددی{n=3} لیتے ہوئے مساوات \حوالہ{مساوات_اعدادی_لیگرینج_الف} سے \عددی{\ln 9.2} کی قیمت تلاش کریں۔حاصل جواب کا مثال \حوالہ{مثال_اعدادی_لیگرینج_الف} کے نتیجہ سے موازنہ کریں۔\\
جواب:\quad
$\num{2.21921}$
جو کم درست ہے چونکہ آخری ہندسہ میں \عددی{1} اکائی کا خلل ہے۔
\انتہا{سوال}
%=========================
\ابتدا{سوال}\شناخت{سوال_اعدادی_لیگرینج_درکار_ب}\quad
سوال \حوالہ{سوال_اعدادی_لیگرینج_درکار_الف} میں دی گئی مواد استعمال کرتے ہوئے \عددی{\ln 9.2} کی قیمت  (الف) مساوات \حوالہ{مساوات_اعدادی_کلیہ_ایورٹ} استعمال کرتے ہوئے، (ب) \عددی{n=3} لیتے ہوئے مساوات \حوالہ{مساوات_اعدادی_لیگرینج_الف} استعمال کرتے ہوئے تلاش کریں۔ 
\انتہا{سوال}
%====================
\ابتدا{سوال}\شناخت{سوال_اعدادی_لیگرینج_درکار_پ}\quad
فرض کریں کہ \عددی{x_1=x_0+h}، \عددی{x_2=x_0+2h}، \عددی{x_3=x_0+3h} ہیں اور \عددی{r=\tfrac{x-x_0}{h}} استعمال کرتے ہوئے دکھائیں کہ \عددی{n=3} کے لئے  ہم مساوات \حوالہ{مساوات_اعدادی_لیگرینج_الف}  کو درج ذیل لکھا جا سکتا ہے۔
\begin{align*}
f(x)\approx -\binom{r-1}{3}f_0+\frac{r(r-2)(r-3)}{2}f_1-\frac{r(r-1)(r-3)}{2}f_2+\binom{r}{3}f_3
\end{align*}
\انتہا{سوال}
%=======================
\ابتدا{سوال}\quad
سوال \حوالہ{سوال_اعدادی_لیگرینج_درکار_پ} کا کلیہ استعمال کرتے ہوئے سوال \حوالہ{سوال_اعدادی_لیگرینج_درکار_ب}-ب کا نتیجہ دوبارہ حاصل کریں۔  
\انتہا{سوال}
%====================
\ابتدا{سوال}\quad \موٹا{(فرق کی جانچ پڑتال)}\quad
دکھائیں کہ قطار میں دیے گئے اندراجات کا مجموعہ گزشتہ قطر کی آخری اور پہلی اندراج کے فرق کے برابر ہو گا۔اس جزوی پرکھ کی  جدول \حوالہ{جدول_اعدادی_فرق_ب} پر اطلاق کریں۔
\انتہا{سوال}
%=======================

\حصہ{لچکدار منحنیات}
ٹکڑوں میں مشابہ کثیر رکنی کو لچکدار منحنی کہتے ہیں۔اس کا مطلب ہے کہ وقفہ \عددی{a\le x\le b} پر ہم دیے گیے تفاعل \عددی{f(x)} کا مشابہ تفاعل \عددی{g(x)} حاصل کرنا چاہتے ہیں۔ظاہر ہے کہ ہم چاہیں گے کہ مشابہ تفاعل اصل تفاعل کے قریب سے قریب تر نمائندگی کرے۔ہم \عددی{g(x)} کو حاصل کرنے کی خاطر وقفہ \عددی{a\le x\le b} کو چھوٹے خانوں (ٹکڑوں) 
\begin{align}\label{مساوات_اعدادی_لچکدار_منحنی_ٹکڑے_الف}
a=x_0<x_1<\cdots<x_n=b
\end{align}
 میں تقسیم کرتے ہیں جہاں خانوں کے  سروں  کو \اصطلاح{جوڑ}\فرہنگ{جوڑ}\حاشیہب{nodes}\فرہنگ{nodes} کہا جاتا ہے۔ہر خانے  پر \عددی{g(x)} کو ایک ایسی  کثیر رکنی  سے ظاہر کیا جاتا ہے  کہ خانے کی سروں پر \عددی{g(x)} بار بار قابل تفرق ہو۔یوں پورے وقفہ \عددی{a\le x\le b} پر تفاعل \عددی{f(x)} کو مشابہ کثیر رکنی سے ظاہر کرنے کی بجائے ہم اس کو \عددی{n} عدد کثیر رکنی سے ظاہر کرتے ہیں۔یوں حاصل مشابہ \عددی{g(x)}  باہمی تحریف میں بہتر ثابت ہوتا ہے۔ مثال کے طور پر وقفہ \عددی{a\le x\le b} کے ہر ایک خانے میں کثیر رکنی سے \عددی{g(x)} کا  ارتعاشی کم ہو گا۔ یوں حاصل تفاعل \عددی{g(x)} کو \اصطلاح{لچکدار منحنیات}\فرہنگ{لچکدار منحنی}\حاشیہب{splines or flexible curves}\فرہنگ{splines}\فرہنگ{curve!flexible}\فرہنگ{flexible curve} کہتے ہیں۔

ہم ہر خانے  کا مشابہ خطی تفاعل استعمال کر سکتے ہیں لیکن ایسا تفاعل خانہ کی جوڑوں پر غیر استمراری ہو گا۔ایسا تفاعل جو وقفہ \عددی{a\le x\le b} کے ہر نقطہ پر کئی بار قابل تفرق ہو  بہتر ثابت ہوتا ہے۔

ہم کعبی لچکدار منحنیات پر غور کرتے ہیں جو عملی استعمال کے نقطہ نظر سے غالباً اہم ترین ہیں۔تعریف کی رو سے وقفہ \عددی{a\le x\le b} پر مساوات \حوالہ{مساوات_اعدادی_لچکدار_منحنی_ٹکڑے_الف} میں دیے گئے خانوں کے لحاظ سے  \اصطلاح{کعبی لچکدار منحنی}\فرہنگ{لچکدار!کعبی منحنی}\فرہنگ{منحنی!کعبی لچکدار}\حاشیہب{cubic spline}\فرہنگ{spline!cubic} \عددی{g(x)}  سے مراد استمراری تفاعل \عددی{g(x)} ہے جس کے  استمراری ایک درجی اور دو درجی تفرق پورے وقفہ پر پائے جاتے ہوں اور جس کو ہر خانہ پر ایسی  کثیر رکنی سے ظاہر کیا گیا ہو جس کا درجہ تین سے زیادہ نہ ہو۔یوں ہر خانہ میں \عددی{g(x)} کو ایک کعبی کثیر رکنی سے ظاہر کیا جائے گا۔

اگر وقفہ \عددی{a\le x\le b} پر تفاعل \عددی{f(x)} دیا گیا ہو اور اس وقفہ کے خانے (مساوات \حوالہ{مساوات_اعدادی_لچکدار_منحنی_ٹکڑے_الف}) منتخب کیے گئے ہوں تب، گزشتہ حصہ کی طرح،   \عددی{f(x)} کی مشابہ کعبی لچکدار منحنی \عددی{g(x)} درج ذیل کو مطمئن کرتے ہوئے حاصل ہو گی۔
\begin{align}\label{مساوات_اعدادی_لچکدار_منحنی_ٹکڑے_ب}
g(x_0)=f(x_0),\quad g(x_1)=f(x_1),\cdots \quad, g(x_n)=f(x_n)
\end{align}
ہم فرض کرتے ہیں کہ ایسا کعبی لچکدار منحنی \عددی{g(x)} پایا جاتا ہے جو مساوات \حوالہ{مساوات_اعدادی_لچکدار_منحنی_ٹکڑے_ب} کو مطمئن کرتا  ہو۔اب اگر \عددی{g(x)} درج ذیل بھی شرائط 
\begin{align}\label{مساوات_اعدادی_لچکدار_منحنی_ٹکڑے_پ}
g'(x_0)=k_0,\quad g'(x_n)=k_n
\end{align}
(جہاں \عددی{k_0} اور \عددی{k_n} دیے گئی عدد ہیں) پر بھی پورا اترتا ہو تب \عددی{g(x)} یکتا ہو گا۔درج ذیل مسئلہ لچکدار منحنی کی موجودگی اور یکتائی کو بیان کرتا ہے۔

%=====================
\ابتدا{مسئلہ}\شناخت{مسئلہ_اعدادی_لچکدار_منحنی_وجودیت_یکتائی}\quad \موٹا{کعبی لچکدار منحنیات}\\
فرض کریں کہ وقفہ \عددی{a\le x\le b} پر تفاعل \عددی{f(x)} دیا گیا ہے اور اس وقفہ کے خانے مساوات \حوالہ{مساوات_اعدادی_لچکدار_منحنی_ٹکڑے_الف} میں دیے گئے ہوں اور فرض کریں کہ \عددی{k_0} اور \عددی{k_n} کوئی دو عدد ہوں۔تب مساوات \حوالہ{مساوات_اعدادی_لچکدار_منحنی_ٹکڑے_الف} کے لحاظ سے ایسا صرف اور صرف ایک کعبی لچکدار منحنی \عددی{g(x)} موجود ہو گا جو مساوات \حوالہ{مساوات_اعدادی_لچکدار_منحنی_ٹکڑے_ب} اور مساوات \حوالہ{مساوات_اعدادی_لچکدار_منحنی_ٹکڑے_پ} کو مطمئن کرتا ہو۔
\انتہا{مسئلہ}
%========================
\ابتدا{ثبوت}\quad
تعریف کی رو سے ہر خانہ \عددی{I_j} میں، جس کو \عددی{x_j\le x\le x_{j+1}} ظاہر کرتا ہے، لچکدار منحنی \عددی{g(x)}  اور کعبی کثیر رکنی \عددی{p_j(x)} ایک جیسے ہوں گے اور درج ذیل کو مطمئن کریں گے۔
\begin{align}\label{مساوات_اعدادی_لچکدار_منحنی_ٹکڑے_ت}
p_j(x_j)=f(x_j),\quad p_j(x_{j+1})=f(x_{j+1})
\end{align}
ہم \عددی{\tfrac{1}{x_{j+1}-x_j}=c_j} اور
\begin{align}\label{مساوات_اعدادی_لچکدار_منحنی_ٹکڑے_ٹ}
p_j'(x_j)=k_j,\quad p_j'(x_{j+1})=k_{j+1}
\end{align}
لکھتے ہیں جہاں \عددی{a_0} اور \عددی{a_n} دیے گئے ہیں جبکہ \عددی{k_1,\cdots,k_{n-1}} بعد میں حاصل کیے جائیں گے۔\عددی{p_j(x)} کو مساوات \حوالہ{مساوات_اعدادی_لچکدار_منحنی_ٹکڑے_ت} اور مساوات \حوالہ{مساوات_اعدادی_لچکدار_منحنی_ٹکڑے_ٹ} میں دیے چار شرائط کو مطمئن کرنا ہو گا۔سیدھے حساب سے ہم تصدیق کر سکتے ہیں کہ ایسا کعبی کثیر رکنی \عددی{p_j(x)} جو ان شرائط کو مطمئن کرتا ہو درج ذیل ہے۔
\begin{gather}
\begin{aligned}\label{مساوات_اعدادی_لچکدار_منحنی_ٹکڑے_ث}
p_j(x)=&f(x_j)c_j^2(x-x_{j+1})^2[1+2c_j(x-x_j)]\\
&+f(x_{j+1})c_j^2(x-x_j)^2[1-2c_j(x-x_{j+1})]\\
&+k_jc_j^2(x-x_j)(x-x_{j+1})^2\\
&+k_{j+1}c_j^2(x-x_j)^2(x-x_{j+1})
\end{aligned}
\end{gather}
اس کا دو درجی تفرق درج ذیل دیگا۔
\begin{align}
p_j''&=-6c_j^2f(x_j)+6c_j^2f(x_{j+1})-4c_jk_j-2c_jk_{j+1}\label{مساوات_اعدادی_دو_درجی_تفرق_الف}\\
p_j''(x_{j+1})&=-6c_j^2f(x_j)+6c_j^2f(x_{j+1})+2c_jk_j+4c_jk_{j+1}\label{مساوات_اعدادی_دو_درجی_تفرق_ب}
\end{align}
تعریف کی رو سے \عددی{g(x)} کی استمراری دو درجی تفرق پائے جاتے ہیں۔اس سے درج ذیل شرط حاصل ہوتا ہے۔
\begin{align*}
p''_{j-1}(x_j)=p''_j(x_j)\quad \quad \quad j=1,2,\cdots,n-1
\end{align*}
مساوات \حوالہ{مساوات_اعدادی_دو_درجی_تفرق_ب} میں \عددی{j} کی جگہ \عددی{j-1}  اور مساوات \حوالہ{مساوات_اعدادی_دو_درجی_تفرق_الف}  استعمال کرتے ہوئے ہم دیکھتے ہیں کہ یہ \عددی{n-1} عدد مساوات درج ذیل صورت اختیار کرتے ہیں
\begin{align}\label{مساوات_اعدادی_دو_درجی_تفرق_پ}
c_{j-1}k_{j-1}+2(c_{j-1}+c_j)k_j+c_jk_{j+1}=3[c^2_{j-1}\nabla f_j+c_j^2\nabla f_{j+1}]
\end{align} 
جہاں \عددی{\nabla f_j=f(x_j)-f(x_{j-1})} اور \عددی{\nabla f_{j+1}=f(x_{j+1})-f(x_{j})} ہیں جبکہ \عددی{j=1,\cdots, n-1} ہے۔ اس \عددی{n-1} عدد مساوات کے نظام کا حل \عددی{k_1,\cdots,k_{n-1}} یکتا ہو گا چونکہ اس نظام کے تمام عددی سر غیر منفی ہیں اور مرکزی وتر پر ہر جزو، مطابقتی صف کے باقی  اجزاء کے مجموعہ سے زیادہ  ہے  لہٰذا عددی سر قالب صفر نہیں ہو سکتا ہے۔اس طرح ہم جوڑ پر \عددی{g(x)} کی یک درجی تفرق کے یکتا \عددی{k_1,\cdots,k_{n-1}} حاصل کرتے ہیں۔اس طرح ثبوت مکمل ہوتا ہے۔
\انتہا{ثبوت}
%==========================
آئیں اس مسئلے کو ایک مثال کی مدد سے دیکھیں۔

%==================
\ابتدا{مثال}\شناخت{مثال_اعدادی_کعبی_لچکدار_منحنی_الف}\quad \موٹا{مشابہ لچکدار منحنی}\\
وقفہ \عددی{-1\le x\le 1} پر \عددی{x_0=-1}، \عددی{x_1=0} اور \عددی{x_2=1} لیتے ہوئے  \عددی{f(x)=x^4} کی ایسی مشابہ کعبی لچکدار منحنی تلاش کریں جو مساوات \حوالہ{مساوات_اعدادی_لچکدار_منحنی_ٹکڑے_ب} کو مطمئن کرتی ہو اور \عددی{g'(-1)=f'(-1)}، \عددی{g'(1)=f'(1)} ہوں۔\\
حل:\quad ہمیں درج ذیل کے عددی سر تلاش کرنے ہوں گے۔
\begin{align*}
p_0(x)&=a_3x^3+a_2x^2+a_1x+a_0\\
p_1(x)&=b_3x^3+b_2x^2+b_1x+b_0
\end{align*}
\عددی{p_0(0)=p_1(0)=f(0)=0} سے \عددی{a_0=b_0=0} ملتے ہیں جبکہ \عددی{p'_0(0)=p_1'(0)} سے \عددی{a_1=b_1} اور \عددی{p_0''(0)=p_1''(0)} سے \عددی{a_2=b_2} حاصل ہوتے ہیں۔یوں
\begin{align*}
p_0(x)&=a_3x^3+a_2x^2+a_1x\\
p_1(x)&=b_3x^3+a_2x^2+a_1x
\end{align*}
ہو گا۔باقی چار عددی سروں  کو باقی چار شرائط سے حاصل کرتے ہیں۔
\begin{gather}
\begin{aligned}\label{مساوات_اعدادی_دو_درجی_تفرق_ت}
p_0(-1)&=-a_3+a_2-a_1=f(-1)=1\\
p_1(1)&=b_3+a_2+a_1=f(1)=1\\
p_0'(-1)&=3a_3-2a_2+a_1=f'(-1)=-4\\
p_1'(1)&=3b_3+2a_2+a_1=f'(1)=4
\end{aligned}
\end{gather}
اس نظام کا حل \عددی{a_1=0}، \عددی{a_2=-1}، \عددی{a_3=-2}، \عددی{b_3=2} ہے۔یوں درکار لچکدار منحنی
\begin{align}\label{مساوات_اعدادی_دو_درجی_تفرق_ٹ}
g(x)=
\begin{cases}
-2x^3-x^2&-1\le x\le 0\\
\phantom{-}2x^3-x^2&\phantom{-}0\le x\le 1
\end{cases}
\end{align}
ہو گی (شکل \حوالہ{شکل_مثال_اعدادی_کعبی_لچکدار_منحنی_الف} میں نقطہ دار منحنی)۔ 
\begin{figure}
\centering
\begin{tikzpicture}
\begin{axis}[height=4cm,small,axis lines*=middle,xtick={-1,-0.5,0.5,1},xticklabels={$-1$,,,$1$},ytick={0.5,1},yticklabels={,$1$},xlabel={$x$},ylabel=\empty,x label style={at={(current axis.right of origin)},anchor=west}]
\addplot[domain=-1:1] {x^4};
\addplot[dashed,domain=-1:0] {-2*x^3-x^2};
\addplot[dashed,domain=0:1] {2*x^3-x^2};
\end{axis}
\end{tikzpicture}
\caption{تفاعل \عددی{f(x)} اور (نقطہ دار) کعبی لچکدار منحنی \عددی{g(x)} (مثال \حوالہ{مثال_اعدادی_کعبی_لچکدار_منحنی_الف})}
\label{شکل_مثال_اعدادی_کعبی_لچکدار_منحنی_الف}
\end{figure}
\انتہا{مثال}
%===========================
لچکدار منحنیات کی ایک دلچسپ کمتر خوبی ہے جس کو اب اخذ کرتے ہیں۔ فرض کریں کہ مسئلہ \حوالہ{مسئلہ_اعدادی_لچکدار_منحنی_وجودیت_یکتائی} میں وقفہ \عددی{a\le x\le b} پر \عددی{f(x)} استمراری ہے اور اس وقفہ پر \عددی{f(x)} کے یک درجی اور  دو درجی استمراری تفرق پائے جاتے ہیں۔فرض کریں کہ مساوات \حوالہ{مساوات_اعدادی_لچکدار_منحنی_ٹکڑے_پ} کی صورت درج ذیل ہے (مثال \حوالہ{مثال_اعدادی_کعبی_لچکدار_منحنی_الف} کی طرح)۔
\begin{align}\label{مساوات_اعدادی_کمتر_خوبی_الف}
g'(a)=f'(a),\quad g'(b)=f'(b)
\end{align} 
تب \عددی{a} اور \عددی{b} پر \عددی{f'-g'} صفر ہو گا۔تکمل بالحصص سے
\begin{align*}
\int_a^b g''(x)[f''(x)-g''(x)]\dif x=-\int_a^b g'''(x)[f'(x)-g'(x)]\dif x
\end{align*}
حاصل ہو گا۔چونکہ وقفہ کے ہر چھوٹے حصے پر \عددی{g'''(x)} مستقل ہے لہٰذا دائیں ہاتھ تکمل کو کسی ایک ٹکڑے پر حاصل کرتے ہوئے  \عددی{c[f(x)-g(x)]}  ملتا ہے جہاں \عددی{c} مستقل ہے اور تکمل کی یہ قیمت ٹکڑے کے سروں پر حاصل کی جائے گی جو مساوات \حوالہ{مساوات_اعدادی_لچکدار_منحنی_ٹکڑے_ب} کی بنا صفر حاصل ہو گی۔چونکہ ہر ٹکڑے پر تکمل صفر ہے لہٰذا پورے وقفے پر تکمل صفر ہو گا۔اس طرح درج ذیل ثابت ہوتا ہے۔
\begin{align*}
\int_a^b f''(x)g''(x)\dif x=\int_a^b g''(x)^2\dif x
\end{align*}
نتیجتاً
\begin{align*}
\int_a^b[f''(x)-g''(x)]^2\dif x&=\int_a^b f''(x)^2\dif x-2\int_a^b f''(x)g''(x)\dif x+\int_a^b g''(x)^2\dif x\\
&=\int_a^b f''(x)^2\dif x-\int_a^b g''(x)^2\dif x
\end{align*}
ہو گا۔بائیں ہاتھ متکمل غیر منفی ہے لہٰذا تکمل بھی غیر منفی ہو گا جس سے درج ذیل ملتا ہے۔
\begin{align}\label{مساوات_اعدادی_کمتر_خوبی_ب}
\int_a^b f''(x)^2\dif x\ge \int_a^b g''(x)^2\dif x
\end{align}

اس نتیجہ کو درج ذیل مسئلہ پیش کرتا ہے۔

%=================
\ابتدا{مسئلہ}\quad \موٹا{کعبی لچکدار منحنی کی کمتر خاصیت}\\
فرض کریں کہ وقفہ \عددی{a\le x\le b} پر تفاعل \عددی{f(x)} اور اس کے یک درجی اور دو درجی تفرق استمراری ہوں۔فرض کریں کہ اس وقفہ کے  خانوں  (مساوات \حوالہ{مساوات_اعدادی_لچکدار_منحنی_ٹکڑے_الف}) کے لحاظ سے  \عددی{g(x)} مطابقتی کعبی لچکدار منحنی ہو جو  مساوات \حوالہ{مساوات_اعدادی_لچکدار_منحنی_ٹکڑے_ب} اور مساوات \حوالہ{مساوات_اعدادی_کمتر_خوبی_الف} کو مطمئن کرتی ہو۔ تب \عددی{f(x)} اور \عددی{g(x)} مساوات \حوالہ{مساوات_اعدادی_کمتر_خوبی_ب} کو مطمئن کریں گے جس میں  علامت مساوات \عددی{(=)} اس صورت کو ظاہر کرتی ہے جب \عددی{f(x)} کعبی لچکدار منحنی \عددی{g(x)} ہو۔
\انتہا{مسئلہ}
%=========================

\حصہء{سوالات}

%=======================
\ابتدا{سوال}\quad
تصدیق کریں کہ مساوات \حوالہ{مساوات_اعدادی_لچکدار_منحنی_ٹکڑے_ث} میں دیا گیا \عددی{p_j(x)} مساوات \حوالہ{مساوات_اعدادی_لچکدار_منحنی_ٹکڑے_ت} اور مساوات \حوالہ{مساوات_اعدادی_لچکدار_منحنی_ٹکڑے_ٹ} کو مطمئن کرتا ہے۔
\انتہا{سوال}
%==========================
\ابتدا{سوال}\quad
مساوات \حوالہ{مساوات_اعدادی_دو_درجی_تفرق_الف} اور مساوات \حوالہ{مساوات_اعدادی_دو_درجی_تفرق_ب} کو مساوات \حوالہ{مساوات_اعدادی_لچکدار_منحنی_ٹکڑے_ث} سے اخذ کریں۔
\انتہا{سوال}
%=====================
\ابتدا{سوال}\quad
مثال \حوالہ{مثال_اعدادی_کعبی_لچکدار_منحنی_الف} پر غور کریں۔دکھائیں کہ مثال میں دی گئی شرائط کے تحت مساوات \حوالہ{مساوات_اعدادی_لچکدار_منحنی_ٹکڑے_ث} درج ذیل دے گی
\begin{align*}
p_0(x)&=-2x^3-x^2+k_1x(x+1)^2\\
p_1(x)&=\phantom{-}2x^3-x^2+k_1x(x-1)^2
\end{align*}
جبکہ مساوات \حوالہ{مساوات_اعدادی_دو_درجی_تفرق_پ} سے \عددی{k_1=0} حاصل ہو گا اور یوں مساوات \حوالہ{مساوات_اعدادی_دو_درجی_تفرق_ٹ} حاصل ہو گی۔
\انتہا{سوال}
%===========================
\ابتدا{سوال}\quad
مثال \حوالہ{مثال_اعدادی_کعبی_لچکدار_منحنی_الف} میں کعبی لچکدار منحنی کا موازنہ پورے وقفہ پر دو درجی مشابہ کثیر رکنی \عددی{p(x)} کے ساتھ کریں۔ \عددی{f(x)} سے \عددی{g(x)} اور \عددی{p(x)} کی زیادہ سے زیادہ انحراف کتنی ہیں۔
\انتہا{سوال}
%===========================
\ابتدا{سوال}\quad
مساوات \حوالہ{مساوات_اعدادی_دو_درجی_تفرق_ت} میں دیے گئے نظام کا حل تلاش کریں۔
\انتہا{سوال}
%=========================
\ابتدا{سوال}\quad
دکھائیں کہ وقفہ کے خانوں کے لحاظ سے کعبی لچکدار منحنیات سمتی فضا (حصہ \حوالہ{حصہ_الجبرا_سمتیات_فضا_تابعیت}) بناتے ہیں۔
\انتہا{سوال}
%========================
\ابتدا{سوال}\quad
دکھائیں کہ وقفہ کے  دیے گئے خانوں  (مساوات \حوالہ{مساوات_اعدادی_لچکدار_منحنی_ٹکڑے_الف}) کے لحاظ سے  \عددی{n+1} یکتا کعبی لچکدار منحنیات \عددی{g_0(x),\cdots,g_n(x)} موجود ہوں گی جو \عددی{g'_j(a)=g_j'(b)=0} اور
\begin{align*}
g_j(x_k)=\delta_{jk}=
\begin{cases}
0& j\ne k\\
1&j=k
\end{cases}
\end{align*}
کو مطمئن کریں گی۔\\
جواب:\quad 
مسئلہ \حوالہ{مسئلہ_اعدادی_لچکدار_منحنی_وجودیت_یکتائی} سے ایسا اخذ ہوتا ہے۔
\انتہا{سوال}
%===========================
\ابتدا{سوال}\quad
دکھائیں کہ اگر ایک لچکدار منحنی تین بار قابل تفرق ہو تب یہ ضرور کثیر رکنی ہو گا۔
\انتہا{سوال}
%========================
\ابتدا{سوال}\quad
ایسا ممکن ہے کہ  وقفہ \عددی{a\le x\le b} کی دو قریبی خانوں کی لچکدار منحنیات ایک جیسی ہوں۔اس طرز کی لچکدار منحنیات دیکھنے کی خاطر وقفہ 
\عددی{-\tfrac{\pi}{2}\le x\le \tfrac{\pi}{2}} کی  خانوں \عددی{x_0=-\tfrac{\pi}{2}}، \عددی{x_1=0}، \عددی{x_2=\tfrac{\pi}{2}} پر \عددی{f(x)=\sin x} کی ایسی لچکدار منحنیات تلاش کریں جو \عددی{g'(-\tfrac{\pi}{2})=f'(-\tfrac{\pi}{2})} اور \عددی{g'(\tfrac{\pi}{2})=f'(\tfrac{\pi}{2})} کو مطمئن کرتی ہوں۔\\
جواب:\quad
$g(x)=-\tfrac{4}{\pi^3}x^3+\tfrac{3}{\pi}x$
\انتہا{سوال}
%==============================
\ابتدا{سوال}\quad
مساوات \حوالہ{مساوات_اعدادی_کمتر_خوبی_ب} کی جیومیٹریائی مطلب کچھ یوں ہے۔یہ مساوات کہتی ہے کہ لچکدار منحنی، مربع انحنا کے تکمل کی قیمت کو کم کرنے کی کوشش کرتی ہے۔اس پر بحث کریں۔ 
\انتہا{سوال}
%=====================

\حصہ{اعدادی تکمل اور تفرق}
\اصطلاح{اعدادی تکمل}\فرہنگ{اعدادی!تکمل}\فرہنگ{تکمل!اعدادی}\حاشیہب{numerical integration}\فرہنگ{integration!numerical}\فرہنگ{numerical!integration} سے مراد قطعی تکمل
\begin{align}
J=\int_a^b f(x)\dif x
\end{align}
کی اعدادی قیمت  کی تلاش ہے جہاں \عددی{a} اور \عددی{b} دیے گئے ہوں گے اور \عددی{f} دیا گیا تفاعل یا تفاعل کی قیمتوں کا جدول ہو گا۔

ہم جانتے ہیں کہ اگر ہم ایسا قابل تفرق تفاعل \عددی{F} تلاش کر سکیں جس کا تفرق \عددی{f} ہو تب \عددی{J} کو درج ذیل کلیہ سے حاصل کیا جا سکتا ہے۔
\begin{align*}
J=\int_a^b f(x)\dif x=F(b)-F(a)\quad \quad \quad [F'(x)=f(x)]
\end{align*}

انجینئری میں عموماً ایسے تکمل پائے جاتے ہیں جن کا متکمل جدول کی صورت میں ہو گا یا تکمل کو متناہی تعداد کے بنیادی تفاعل کی صورت میں ظاہر کرنا نا ممکن ہو گا اور یا \عددی{F} کی صریح صورت پیچیدہ اور غیر مفید ثابت ہو گی۔ ایسی صورتوں میں اعدادی تکمل کارآمد ثابت ہوتا ہے۔

چونکہ  وقفہ \عددی{a\le x\le b} میں تفاعل \عددی{f(x)} کے نیچے خطہ \عددی{R} کا رقبہ \عددی{J} ہے (شکل \حوالہ{شکل_اعدادی_قطعی_تکمل_معنی}) لہٰذا ہم گتے سے \عددی{R} کی شکل کاٹ کر، گتے کی اس ٹکڑے کے وزن کو اکائی رقبہ گتے کی وزن سے تقسیم کرتے ہوئے \عددی{R} کا رقبہ دریافت کر سکتے ہیں۔ہم \اصطلاح{کاغذ ترسیم}\فرہنگ{کاغذ ترسیم}\حاشیہب{graph paper}\فرہنگ{graph paper}  پر \عددی{R} کی شکل بنا کر ڈبے گن  کر بھی \عددی{R} کا رقبہ دریافت کر سکتے ہیں۔رقبہ کی بہتر ناپ کے لئے  \اصطلاح{سطح پیما}\فرہنگ{سطح پیما}\فرہنگ{سطح!پیما}\حاشیہب{planimeter}\فرہنگ{planimeter} کا استعمال ضروری ہو گا۔

\begin{figure}
\centering
\begin{tikzpicture}
\draw(0,0)--(4,0)node[right]{$x$};
\draw(0,0)--(0,2.5)node[left]{$y$};
\draw(-0.5,1) to [out=-20,in=180](0.5,0.5) to [out=0,in=180] coordinate[pos=0.25](kA)(3,2.2) to [out=0,in=160] coordinate[pos=0.5](kB)(4,1.75)node[right]{$y=f(x)$};
\draw(kA)--($(0,0)!(kA)!(4,0)$)coordinate(kC)node[below]{$a$};
\draw(kB)--($(0,0)!(kB)!(4,0)$)node[below]{$b$};
\draw($(kB)!0.5!(kC)$)node[]{$R$};
\end{tikzpicture}
\caption{قطعی تکمل کی جیومیٹریائی معنی}
\label{شکل_اعدادی_قطعی_تکمل_معنی}
\end{figure}

متکمل کو کثیر رکنی سے ظاہر کرتے ہوئے اعدادی تراکیب بنائے جا سکتے ہیں۔سادہ ترین کلیہ اخذ کرنے کی خاطر ہم تکمل کے وقفہ کو \عددی{h=\tfrac{b-a}{n}}  لمبائی کے  \عددی{n} عدد برابر ٹکڑوں میں تقسیم کرتے ہیں اور ہر ٹکڑے پر تفاعل کو مستقل تفاعل \عددی{f(x^*_j)} سے ظاہر کرتے ہیں جہاں \عددی{x^*_j} ٹکڑے کا وسطی نقطہ ہے (شکل \حوالہ{شکل_اعدادی_تکمل_مستطیل_ذوزنقہ}-الف)۔یوں \عددی{f} کو \اصطلاح{سیڑھی تفاعل}\فرہنگ{سیڑھی!تفاعل}\حاشیہب{step function}\فرہنگ{step!function} (ٹکڑوں میں مستقل تفاعل) ظاہر کرے گی۔شکل \حوالہ{شکل_اعدادی_تکمل_مستطیل_ذوزنقہ}-الف کے \عددی{n} مستطیلوں کے انفرادی رقبے  \عددی{hf(x_1^*), \cdots , hf(x_n^*)} ہیں جن کا مجموعہ اعدادی تکمل کا \اصطلاح{مستطیل قاعدہ}\فرہنگ{اعدادی!تکمل، مستطیل قاعدہ}\فرہنگ{تکمل!اعدادی، مستطیل قاعدہ}\حاشیہب{rectangular rule}\فرہنگ{numerical!integration, rectangular rule}\فرہنگ{integration!numerical,rectangular rule}
\begin{align}\label{مساوات_اعدادی_تکمل_مستطیل_قاعدہ}
J=\int_a^b f(x)\dif x\approx h[f(x_1^*)+f(x_2^*)+\cdots+f(x_n^*)],\quad \big(h=\frac{b-a}{n}\big)
\end{align}
 دیتی ہے۔
%
\begin{figure}
\centering
\begin{subfigure}{0.5\textwidth}
\centering
\begin{tikzpicture}
\pgfmathsetmacro{\s}{0.5}
\pgfmathsetmacro{\ss}{2.5}
\pgfmathsetmacro{\k}{0.75}
\draw(0,0)--(4.2,0)node[right]{$x$};
\draw(0,0)--(0,2)node[right]{$y$};
\draw[name path=kcurve](0.2,0.2) .. controls (2,4) and (3,-1.5) .. (4.2,2)node[left]{$y=f(x)$};
%
\path[name path=kA](\s,0)--++(0,2);
\path[name path=kB](\s+\k/2,0)--++(0,2);
\path[name path=kC](\s+\k,0)--++(0,2);
\path[name path=kD,name intersections={of={kB and kcurve}}] (intersection-1)coordinate(pA)++(-0.75,0)--++(1,0);
\draw[name intersections={of=kD and kA}](intersection-1)coordinate(pB)--($(0,0)!(intersection-1)!(4,0)$)node[below]{$a$};
\draw[dashed](intersection-1)--++(\k,0);
\draw(pA)node[ocirc]{};
\path[name path=kkA](\s+\k,0)--++(0,2);
\path[name path=kkB](\s+\k+\k/2,0)--++(0,2);
\path[name path=kkC](\s+\k+\k,0)--++(0,2);
\path[name path=kD,name intersections={of={kkB and kcurve}}] (intersection-1)coordinate(ppA)++(-0.75,0)--++(1,0);
\draw[name intersections={of=kD and kkA}](intersection-1)coordinate(pB)--($(0,0)!(intersection-1)!(4,0)$);
\draw[dashed](intersection-1)--++(\k,0)coordinate(ppR);
\draw(ppR)--($(0,0)!(ppR)!(4,0)$);
\draw(ppA)node[ocirc]{};
\path[name path=kkA](\ss+\k,0)--++(0,2);
\path[name path=kkB](\ss+\k+\k/2,0)--++(0,2);
\path[name path=kkC](\ss+\k+\k,0)--++(0,2);
\path[name path=kD,name intersections={of={kkB and kcurve}}] (intersection-1)coordinate(ppA)++(-0.75,0)--++(1,0);
\draw[name intersections={of=kD and kkA}](intersection-1)coordinate(pB)--($(0,0)!(intersection-1)!(4,0)$);
\draw[dashed](intersection-1)--++(\k,0)coordinate(ppR);
\draw(ppR)--($(0,0)!(ppR)!(4,0)$)node[below]{$b$};
\draw(ppA)node[ocirc]{};
\draw(\s+\k/2,0)node[below]{$x_1^*$}--++(0,0.1);
\draw(\s+\k+\k/2,0)node[below]{$x_2^*$}--++(0,0.1);
\draw(\ss+\k+\k/2,0)node[below]{$x_n^*$}--++(0,0.1);
\foreach \x in {2.2,2.5,2.75}{\draw(\x,0.5)node[circ,fill=black]{};}
\end{tikzpicture}
\caption*{(الف) مستطیل قاعدہ}
\end{subfigure}%
\begin{subfigure}{0.5\textwidth}
\centering
\begin{tikzpicture}
\pgfmathsetmacro{\s}{0.5}
\pgfmathsetmacro{\ss}{2.5}
\pgfmathsetmacro{\k}{0.75}
\draw(0,0)--(4.2,0)node[right]{$x$};
\draw(0,0)--(0,2)node[right]{$y$};
\draw[name path=kcurve](0.2,0.2) .. controls (2,4) and (3,-1.5) .. (4.2,2)node[left]{$y=f(x)$};
%
\path[name path=kA](\s,0)--++(0,2);
\path[name path=kB](\s+\k,0)--++(0,2);
\path[name path=kC](\s+\k+\k,0)--++(0,2);
\draw[name intersections={of=kA and kcurve}] (intersection-1)coordinate(pA)--($(0,0)!(intersection-1)!(4,0)$)node[below]{$a$};
\draw[name intersections={of=kB and kcurve}] (intersection-1)coordinate(pB)--($(0,0)!(intersection-1)!(4,0)$)node[below]{$x_1$};
\draw[name intersections={of=kC and kcurve}] (intersection-1)coordinate(pC)--($(0,0)!(intersection-1)!(4,0)$)node[below]{$x_2$};
\draw[dashed] (pA) node[ocirc,solid]{}--(pB)node[ocirc,solid]{}--(pC)node[ocirc,solid]{};
%
\path[name path=kA](\ss+\k,0)--++(0,2);
\path[name path=kB](\ss+\k+\k,0)--++(0,2);
\draw[name intersections={of=kA and kcurve}] (intersection-1)coordinate(pA)--($(0,0)!(intersection-1)!(4,0)$)node[below]{$x_{n-1}$};
\draw[name intersections={of=kB and kcurve}] (intersection-1)coordinate(pB)--($(0,0)!(intersection-1)!(4,0)$)node[below]{$b$};
%\draw[name intersections={of=kC and kcurve}] (intersection-1)coordinate(pC)--($(0,0)!(intersection-1)!(4,0)$)node[below]{$x_2$};
\draw[dashed] (pA) node[ocirc,solid]{}--(pB)node[ocirc,solid]{};
%
\foreach \x in {2.2,2.5,2.75}{\draw(\x,0.5)node[circ,fill=black]{};}
\end{tikzpicture}
\caption*{(ب) ذوزنقہ قاعدہ}
\end{subfigure}%
\caption{اعدادی تکمل}
\label{شکل_اعدادی_تکمل_مستطیل_ذوزنقہ}
\end{figure}

تفاعل \عددی{f} کو ٹکڑوں میں خطی قطعات (شکل \حوالہ{شکل_اعدادی_تکمل_مستطیل_ذوزنقہ}-ب) سے ظاہر کرنے سے اعدادی تکمل کا \اصطلاح{ذوزنقہ قاعدہ}\فرہنگ{اعدادی!تکمل، ذوزنقہ قاعدہ}\فرہنگ{تکمل!اعدادی، ذوزنقہ قاعدہ}\حاشیہب{trapezoidal rule}\فرہنگ{numerical!integration, trapezoidal rule}\فرہنگ{integration!numerical, trapezoidal rule}
\begin{align}\label{مساوات_اعدادی_تکمل_ذوزنقہ_قاعدہ}
J=\int_a^b f(x)\dif x\approx h[\tfrac{1}{2}f(a)+f(x_1)+f(x+2)+\cdots+f(x_{n-1})+\tfrac{1}{2}f(b)]
\end{align}
حاصل ہو گا جہاں \عددی{h=\tfrac{b-a}{n}} ہے اور \عددی{x_0\,(=a),x_1,x_2,c\dots,x_{n-1},x_n\,(=b)} وہی نقطے ہیں جو مساوات \حوالہ{مساوات_اعدادی_تکمل_ذوزنقہ_قاعدہ} میں استعمال کیے گئے ہیں۔یوں \عددی{x_j=x_0+jh} ہو گا۔شکل \حوالہ{شکل_اعدادی_تکمل_مستطیل_ذوزنقہ}-ب کے \عددی{n} ذوزنقہ کے انفرادی رقبے
\begin{align*}
\tfrac{1}{2}[f(a)+f(x_1)]h,\quad \tfrac{1}{2}[f(x_1)+f(x_2)]h,\quad \cdots, \tfrac{1}{2}[f(x_{n-1})+f(b)]h
\end{align*}
ہیں جن کا مجموعہ مساوات \حوالہ{مساوات_اعدادی_تکمل_ذوزنقہ_قاعدہ} کا دایاں ہاتھ دے گا۔

\عددی{J^*} میں خلل (حصہ \حوالہ{حصہ_اعدادی_خلل_غلطیاں_کمپیوٹر}) \عددی{\epsilon} درج ذیل ہو گا۔
\begin{align*}
\epsilon=J^*-J
\end{align*}
اگر \عددی{f(x)} خطی تفاعل ہو تب \عددی{\epsilon=0} ہو گا اور تمام \عددی{x} کے لئے  \عددی{f'} مستقل اور \عددی{f''} صفر ہو گا۔عین ممکن ہے کہ کسی عمومی تفاعل \عددی{f} (جس کا استمراری دو درجی تفرق پایا جاتا ہو)  کی صورت میں ہم  \اصطلاح{حد خلل}\فرہنگ{حد!خلل}\فرہنگ{خلل!حد}\حاشیہب{error bound}\فرہنگ{error!bound} (یعنی خلل \عددی{\epsilon} کی حد) تلاش کر سکیں جو \عددی{f''} پر منحصر ہو۔ اس خاطر ہم \عددی{b} کی جگہ متغیر \عددی{t} لیتے ہوئے  مساوات \حوالہ{مساوات_اعدادی_تکمل_ذوزنقہ_قاعدہ} کا اطلاق \عددی{n=1} کے لئے کرتے ہیں۔تب مطابقتی خلل
\begin{align*}
\epsilon(t)=\frac{t-a}{2}[f(a)+f(t)]-\int_a^t f(x)\dif x
\end{align*}
ہو گا۔ہم دیکھتے ہیں کہ \عددی{\epsilon(a)=0} ہے جو ایک غیر دلچسپ نتیجہ ہے۔تفرق لینے سے
\begin{align*}
\epsilon'(t)=\frac{1}{2}[f(a)+f(t)]+\frac{t-a}{2}f'(t)-f(t)
\end{align*}
حاصل ہو گا۔ہم دیکھتے ہیں کہ \عددی{\epsilon'(a)=0} ہے۔مزید ایک بار تفرق لینے سے
\begin{align*}
e''(t)=\frac{1}{2}(t-a)f''(t)
\end{align*}
حاصل ہو گا جس میں وقفہ \عددی{a\le x\le b} پر \عددی{f''} کی کم سے کم قیمت \عددی{M_2^*}  اور زیادہ سے زیادہ قیمت \عددی{M_2}  پر کرنے سے وقفے پر تمام \عددی{t} کے لئے  حد خلل
\begin{align*}
\frac{1}{2}(t-a)M_2^*\le \epsilon''(t)\le \frac{1}{2}(t-a)M_2
\end{align*}
حاصل ہوتا ہے۔\عددی{a} تا \عددی{t} تکمل لینے سے
\begin{align*}
\frac{1}{4}(t-a)^2M_2^*\le \epsilon'(t)-\epsilon'(a)\le \frac{1}{4}(t-a)^2M_2
\end{align*}
حاصل ہو گا جس میں \عددی{\epsilon'(a)=0} اور \عددی{\epsilon(a)=0} پر کرتے ہوئے دوبارہ تکمل لے کر \عددی{t=a+h} لکھتے ہوئے درج ذیل حاصل ہو گا۔
\begin{align*}
\frac{1}{12}h^3M_2^*\le \epsilon(a+h)\le \frac{1}{2}h^3M_2
\end{align*}
باقی \عددی{n-1} ٹکڑوں کی خلل کے لئے اسی طرح کے \عددی{n-1} عدد مطابقتی عدم مساوات  حاصل ہوں گے۔ان تمام \عددی{n} عدم مساوات کا مجموعہ \عددی{a} تا \عددی{b} تکمل کے لئے خلل \عددی{\epsilon} کی عدم مساوات دے گا۔چونکہ \عددی{h=\tfrac{b-a}{n}} ہے لہٰذا ہمیں
\begin{align}
KM_2^*\le \epsilon\le KM_2,\quad \quad [K=\tfrac{(b-a)^3}{12n^2}]
\end{align}
حاصل ہوتا ہے جہاں تکمل کے  وقفہ پر \عددی{f''} کی کم سے کم قیمت \عددی{M_2^*} اور زیادہ سے زیادہ قیمت \عددی{M_2} ہے۔

%===============
\ابتدا{مثال}\quad \موٹا{ذوزنقہ قاعدہ۔ تخمینہ خلل}\\

\انتہا{مثال}
%====================
