\باب{جزوی تفرقی مساوات}
مختلف طبعی اور جیومیٹریائی مسائل جہاں دو یا دو سے زیادہ متغیرات پر مبنی تفاعل پایا جاتے ہوں، جزوی تفرقی مساوات کو جنم دیتے ہیں۔یہ متغیرات وقت اور خلا کے محدد ہو سکتے ہیں۔اس باب میں انجینئری نقطہ نظر سے اہم مسائل پر غور کیا جائے گا۔ان مساوات کو طبعی نظام کی نمونہ کے طور پر حاصل کرنے کے بعد ابتدائی قیمت اور سرحدی قیمت مسائل حل کرنے کی تراکیب پر غور کیا جائے گا، یعنی ان مساوات کو دی گئی طبعی شرائط کے مطابق حل کیا جائے گا۔

ہم دیکھیں گے کہ جزوی تفرقی مساوات کو لاپلاس بدل کی مدد سے حل کیا جا سکتا ہے۔

\حصہ{بنیادی تصورات}
دو یا دو سے زیادہ غیر تابع متغیرات کی نا معلوم تفاعل اور اس کی ایک یا ایک سے زیادہ تفرقات پر مبنی مساوات کو \اصطلاح{جزوی تفرقی مساوات}\فرہنگ{تفرقی!جزوی مساوات}\فرہنگ{جزوی!تفرقی مساوات}\حاشیہب{partial differential equation}\فرہنگ{differential!partial equation} کہتے ہیں۔ بلند تر تفرق کا درجہ مساوت کا \اصطلاح{درجہ}\فرہنگ{درجہ!جزوی تفرقی مساوات}\فرہنگ{جزوی!درجہ مساوات}\حاشیہب{order}\فرہنگ{order!partial differential equation} کہلاتا ہے۔ 

سادہ تفرقی مساوات کی طرح اگر جزوی تفرقی مساوات میں تابع متغیر (نا معلوم تفاعل) اور اس کے تفرق کی طاقت اکائی ہو تب  یہ تفرقی مساوات \اصطلاح{خطی}\فرہنگ{خطی}\حاشیہب{linear}\فرہنگ{linear} کہلائے گی۔اگر مساوات کا ہر رکن تابع متغیرہ یا تابع متغیرہ کی تفرقات میں سے کوئی ایک تفرق ہو تب اس کو \اصطلاح{ہم جنسی}\فرہنگ{ہم جنسی!جزوی تفرقی مساوات}\حاشیہب{homogeneous}\فرہنگ{homogeneous!partial differential equation} کہیں گے ورنہ یہ \اصطلاح{غیر ہم جنسی}\فرہنگ{غیر ہم جنسی!جزوی تفرقی مساوات}\حاشیہب{non homogeneous}\فرہنگ{non homogeneous!partial differential equation} کہلائے گی۔   

%===============
\ابتدا{مثال}\quad اہم خطی دو درجی جزوی تفرقی مساوات\\
\begin{align}
&\frac{\partial^{\,2}u}{\partial t^2}=c^2\frac{\partial^{\,2}u}{\partial x^2}\label{مساوات_جزوی_الف}\quad \quad \text{\RL{یک بعدی موج کی مساوات}}\\
&\frac{\partial u}{\partial t}=c\frac{\partial^{\,2}u}{\partial x^2}\label{مساوات_جزوی_ب}\quad \quad \text{\RL{یک بعدی حراری مساوات}}\\
&\frac{\partial^{\,2}u}{\partial x^2}+\frac{\partial^{\,2}u}{\partial y^2}=0\label{مساوات_جزوی_پ}\quad \quad \quad \text{\RL{دو بعدی لاپلاس مساوات}}\\
&\frac{\partial^{\,2}u}{\partial x^2}+\frac{\partial^{\,2}u}{\partial y^2}=f(x,y)\label{مساوات_جزوی_ت}\quad \quad \quad \text{\RL{دو بعدی پوئسن مساوات}}\\
&\frac{\partial^{\,2}u}{\partial x^2}+\frac{\partial^{\,2}u}{\partial y^2}+\frac{\partial^{\,2}u}{\partial z^2}=0\label{مساوات_جزوی_ٹ}\quad \quad \quad \text{\RL{تین بعدی لاپلاس مساوات}}
\end{align}
یہاں \عددی{c} مستقل ہے، \عددی{t} وقت کو ظاہر کرتی ہے جبکہ \عددی{x}، \عددی{y}، \عددی{z} کارتیسی محدد ہیں۔مساوات \حوالہ{مساوات_جزوی_ت} میں اگر \عددی{f(x,y)\ne 0} ہو تب یہ غیر ہم جنسی ہو گی۔باقی تمام مساوات ہم جنسی ہیں۔
\انتہا{مثال}
%===========================

فضا میں غیر تابع متغیرہ کی کسی خطہ  \عددی{R} میں جزوی تفرقی مساوات کے \اصطلاح{حل} سے مراد ایسا تفاعل ہے جو خود اور جس کے وہ تمام تفرقات جو اس مساوات میں پائے جاتے ہوں کسی ایسے خطے میں موجود ہوں  جس کا \عددی{R} حصہ ہو اور یہ تمام مل کر پورے خطہ \عددی{R} میں اس مساوات کو مطمئن کرتے ہوں۔(عموماً \عددی{R} کی سرحد پر اس تفاعل کا استمراری ہونا اور درکار تفرقات کا خطہ کے اندرون معین ہونے کے ساتھ ساتھ خطہ کے اندرون مساوات کو مطمئن کرنا درکار ہو گا۔)

عموماً جزوی تفرقی مساوات کے تمام حل کی تعداد بہت زیادہ ہو گی۔ مثلاً جیسا آپ تصدیق کر سکتے ہیں کہ تفاعل
\begin{align}\label{مساوات_جزوی_مثال_تفاعل}
u=x^2-y^2,\quad u=e^x\cos y,\quad u=\ln(x^2+y^2)
\end{align} 
جو ایک دوسرے سے بالکل مختلف ہیں مساوات \حوالہ{مساوات_جزوی_پ} کے حل ہیں۔ہم بعد میں دیکھیں گے کہ جزوی تفرقی مساوات کا یکتا حل حاصل کرنے کی خاطر مزید معلومات درکار ہو گی جو طبعی حالت سے حاصل ہو گی۔مثال کے طور پر کبھی کبھار سرحد کے کسی حصے پر درکار حل کی قیمت معلوم ہو گی (\اصطلاح{سرحدی شرائط}\فرہنگ{سرحدی شرائط}\حاشیہب{boundary conditions}\فرہنگ{boundary conditions}) جب کہ بعض اوقات ابتدائی لمحہ \عددی{t=0} پر حل کی قیمت معلوم ہو گی (\اصطلاح{ابتدائی شرائط}\فرہنگ{ابتدائی شرائط}\حاشیہب{initial conditions}\فرہنگ{initial conditions})۔ 

ہم جانتے ہیں کہ اگر سادہ تفرقی مساوات خطی اور ہم جنسی ہو تب اس کی معلوم حل سے مزید حل بذریعہ خطی میل حاصل کیے جا سکتے ہیں۔ جزوی تفرقی مساوات کے لئے بھی ایسا کرنا ممکن ہے جیسا درج ذیل مسئلہ کہتا ہے۔

%=========================
\ابتدا{مسئلہ}\شناخت{مسئلہ_جزوی_بنیادی}\quad بنیادی مسئلہ\\
اگر کسی خطہ \عددی{R} میں  خطی ہم جنسی جزوی تفرقی مساوات کے دو حل \عددی{u_1} اور \عددی{u_2} ہوں تب
\begin{align*}
u=c_1u_1+c_2u_2
\end{align*} 
جہاں \عددی{c_1} اور \عددی{c_2} کوئی مستقل ہیں، بھی اس خطے میں اس مساوات کا حل ہو گا۔
\انتہا{مسئلہ}
%====================

اس مسئلے کا ثبوت نہایت آسان اور مسئلہ \حوالہ{مسئلہ_دو_درجی_خطی_میل} کی ثبوت سے ملتا جلتا ہے لہٰذا یہ آپ پر چھوڑا جاتا ہے۔

%================
\حصہء{سوالات}

%==================
\ابتدا{سوال}\quad
مسئلہ \حوالہ{مسئلہ_جزوی_بنیادی} کو دو اور تین متغیرات کی دو درجی جزوی تفرقی مساوات کے لئے ثابت کریں۔
\انتہا{سوال}
%======================
\ابتدا{سوال}\quad تصدیق کریں کہ مساوات \حوالہ{مساوات_جزوی_مثال_تفاعل} میں دیے گئے تمام تفاعل مساوات \حوالہ{مساوات_جزوی_پ} کے حل ہیں۔

\انتہا{سوال}
%====================
