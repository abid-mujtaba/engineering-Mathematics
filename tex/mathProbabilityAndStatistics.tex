\باب{امکانیات اور شماریات}
بڑے پیمانے پر مصنوعات کی پیداوار اور تجرباتی مواد کے تجزیہ  کے لئے حسابی شماریات بہت اہم ہے۔ اس باب کی شروع میں مواد کا جدول اور ترسیم سے اظہار پر غور کیا جائے گا۔چونکہ شماریات کی بنیاد حسابی امکانیات ہے لہٰذا  اس کے بعد حسابی امکانیات کے بنیادی تصورات اور اصولوں پر غور کیا جائے گا۔باب کا باقی حصہ شماریات کے اہم ترین تراکیب پر مشتمل ہے۔

\حصہ{حسابی شماریات کی نوعیت اور اس کا مقصد}
انجینئری شماریات میں ہمیں ایسے تجربات کی بناوٹ اور تشخیص سے غرض ہو گا جو عملی مسائل کے بارے میں معلومات فراہم کر سکے، مثلاً، خام مال  یا تیار کردہ مصنوعات کے معیار کی جانچ پڑتال، مشین اور آلات یا مصنوعات کی تیاری میں استعمال تراکیب کا آپس میں موازنہ، مزدور کی پیداوار، صارفین کا نئی مصنوعات کے لئے رد عمل،  مختلف حالات میں کیمیائی عمل سے حاصل پیداوار، خام لوہا کی کثافت اور اس میں لوہے کی مقدار کا تعلق،  مختلف درجہ حرارت پر ایئر کنڈشنر  نظام کی کارکردگی، فولاد میں کاربن کی مقدار اور فولاد کی \اصطلاح{راک ویل}\فرہنگ{راک ویل}\حاشیہب{Rockwell}\فرہنگ{Rockwell} سختی کا تعلق، وغیرہ وغیرہ۔

مثال کے طور پر، بڑے پیمانے پر (پیچ، بلب، موبائل فون وغیرہ کی) پیداوار کے عمل میں عموماً  \اصطلاح{بے عیب}\فرہنگ{بے عیب}\حاشیہب{nondefective}\فرہنگ{nondefective} اجزاء، جو درکار خواص کے معیار پر پورا اترے ہیں،  اور  \اصطلاح{عیب دار}\فرہنگ{عیب دار}\حاشیہب{defective}\فرہنگ{defective} اجزاء، جو درکار خواص کے معیار پر پورا نہیں اترتے ہیں، پائے جائیں گے۔  درکار خواص میں دھرا کا قطر، بلب کی کم سے کم \اصطلاح{عرصہ زندگی}\فرہنگ{عرصہ زندگی}\حاشیہب{lifetime}\فرہنگ{lifetime}،برقیاتی مصنوعات میں استعمال برقی مزاحمت کی قیمت کے حدود، کتاب میں استعمال کاغذ کی موٹائی، خود کار بھری گئی بوتل میں دوائی کی کم سے کم مقدار، برقی سوئچ کا زیادہ سے زیادہ دورانیہ ردعمل، اور کپڑے کی کم سے کم  مضبوطی شامل ہیں۔

مصنوعات کی معیار میں فرق متعدد وجوہات (مثلاً خام مال ، خود کار مشین کی کارکردگی، کاریگر  کی کاریگری) کی بنا ممکن ہے جن کو قبل از وقت جاننا ممکن نہیں ہے لہٰذا انہیں \اصطلاح{بے ترتیب تبدیلیاں}\فرہنگ{بے ترتیب تبدیلی}\حاشیہب{random variation}\فرہنگ{random variation} تصور کیا جات ہے۔پیداوار کے تراکیب کی کارکردگی اور متذکرہ بالا دیگر مثالوں میں بھی صورت حال ایسا ہی ہو گا۔ 

ہر ایک پیدا کردہ رکن کو پرکھنے کے لئے عموماً بہت وقت درکار  ہو  گا اور ایسا کرنا خاصہ مہنگا ہو گا۔اگر پرکھنے کے دوران رکن ضائع ہوتا ہو تب ہر رکن کو پرکھنا ممکن نہیں ہو گا۔اسی لئے تمام ارکان کو پرکھنے کی بجائے چند ارکان کو بطور \اصطلاح{نمونہ}\فرہنگ{نمونہ}\حاشیہب{sample}\فرہنگ{sample} پرکھا جاتا ہے اور اس نمونہ کے نتائج سے تمام ارکان\اصطلاح{کل تعداد}\فرہنگ{کل تعداد}\حاشیہب{totality}\فرہنگ{totality}) کے بارے میں رائے بنائی جاتی ہے۔ اگر \عددی{\num{10000}} پیچوں کی ڈھیر  سے \عددی{100} پیچوں کے نمونہ کو پرکھا جائے اور اس میں \عددی{5} پیچ عیب دار نکلیں تب ہم کہہ سکتے ہیں کہ اس ڈھیر میں \عددی{\SI{5}{\percent}} پیچ عیب دار ہوں گے، پس اتنا ضروری ہے کہ نمونہ کو \اصطلاح{بے قاعدگی}\فرہنگ{بے قاعدگی}\حاشیہب{at random}\فرہنگ{random!at} سے چننا جائے یعنی ڈھیر میں موجود ہر پیچ کا بطور نمونہ منتخب ہونے کا \اصطلاح{امکان}\فرہنگ{امکان}\حاشیہب{chance}\فرہنگ{chance} ایک جیسا ہو۔ظاہر ہے کہ ایسی رائے مکمل طور پر درست نہیں ہو سکتی ہے اور یہ کہنا کہ ٹھیک \عددی{\SI{5}{\percent}} پیچ عیب دار ہوں گے عموماً  درست نہیں ہو گا لیکن عام طور عملی زندگی میں اتنی درست رائے (یا نتیجہ)  کی ضرورت پیش نہیں آئے گی۔جتنے زیادہ ارکان کو پرکھا جائے ہمیں نتائج پر اتنا زیادہ اعتماد ہوتا ہے۔حسابی امکانیات کا نظریہ ان خیالات کو ٹھوس شکل دیتا ہے اور نتائج پر کتنا اعتبار کیا جائے، اس کی ناپ بھی پیش کرتا ہے۔یوں شماریات کی بنیاد  نظریہ امکانیات ہے۔

اسی طرح خام لوہا میں لوہے کی فی صد مقدار \عددی{\mu} جاننے کی خاطر ہم  بے قاعدگی سے  \عددی{n} تعداد کے  نمونے لیتے ہوئے ان میں لوہے کی فی صد مقدار تجرباتی طور دریافت  کریں گے۔  ان \عددی{n}  نمونوں کے تجرباتی نتائج \عددی{x_1,\cdots,x_n} کی اوسط \عددی{\bar{x}=\tfrac{x_1+\cdots+x_n}{n}} لوہے کی فی صد مقدار \عددی{\mu} کی تخمین ہو گی۔

مختلف نوعیت کے مسائل کے لئے مختلف تراکیب اور تکنیک  درکار ہوں گے البتہ مسئلے کی تشکیل سے حل تک کے قدم عموماً ایک جیسے ہوتے ہیں۔انہیں یہاں پیش کرتے ہیں۔
\begin{itemize}
\item{مسئلے کی تشکیل۔}
مسئلے کو ٹھیک ٹھیک بیان کرنا اور تفتیشی عمل کے حدود تعین کرنا ضروری ہے تا کہ شماریاتی تفتیش کی لاگت، تفتیش کار کی مہارت اور دستیاب سہولیات کو مد نظر رکھتے ہوئے مخصوص  وقت میں قابل استعمال نتائج حاصل ہوں۔اسی قدم میں واضح تصورات سے \اصطلاح{حسابی نمونہ}\فرہنگ{حسابی نمونہ}\فرہنگ{نمونہ!حسابی}\حاشیہب{mathematical model}\فرہنگ{model!mathematical}  کی تخلیق\حاشیہد{لفظ "نمونہ" اور لفظ "حسابی نمونہ" علیحدہ معنی رکھتے ہیں۔اسی لئے حسابی نمونہ کو بطور اصطلاح لیتے ہوئے پورا لکھا جائے گا یعنی "حسابی نمونہ"۔} بھی شامل ہے۔ (مثال کے طور پر ہم نے تعین کرنا ہو گا کہ عیب دار رکن سے کیا مراد ہے۔)
\item{تجربہ کی تخلیق۔}
آخری مرحلے میں استعمال ہونے والی شماریاتی ترکیب کا انتخاب، نمونہ کی جسامت (جتنے ارکان کا تجزیہ  یا ان پر تجربہ کیا جائے گا، وغیرہ) اور طبعی تراکیب اور تکنیک جو بروئے کار لائے جائیں گے کا انتخاب اس قدم میں کیا جائے گا۔کم سے کم وقت اور لاگت کے ساتھ زیادہ سے زیادہ معلومات حاصل کرنا مقصد ہے۔
\item{تجربہ یا مواد جمع کرنے کا عمل۔}
اس قدم میں قواعد پر سختی سے عمل کرنا ضروری ہے۔
\item{جدول بندی۔}
اس قدم میں تجرباتی نتائج کو واضح اور سادہ جدول کی شکل میں لکھا جاتا ہے اور ساتھ ہی انہیں ترسیم کیا جا سکتا ہے یا انہیں \اصطلاح{ڈبہ ترسیم}\فرہنگ{ترسیم!ڈبہ}\فرہنگ{ڈبہ!ترسیم}\حاشیہب{bar graph}\فرہنگ{bar graph} کی صورت میں دکھایا جا سکتا ہے۔ اس قدم میں نمونہ کی اوسط اور قیمتوں میں پھیل کے تخمین کا حساب بھی کیا جاتا ہے۔
\item{شماریاتی رائے زنی۔}
اس قدم میں کوئی مخصوص شماریاتی ترکیب کو نمونہ سے حاصل  نتائج پر لاگو کرتے ہوئے نا معلوم خواص کے بارے میں رائے قائم کی جاتی ہے تا کہ ہم مطلوبہ جواب حاصل کر سکیں۔  
\end{itemize}

%===================
\حصہ{نمونہ کا اظہار بذریعہ جدول اور ڈبہ ترسیم}
