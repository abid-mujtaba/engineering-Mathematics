\باب{نظامِ تفرقی مساوات}
گزشتہ باب میں آپ نے بلند درجی سادہ تفرقی مساوات کو حل کرنا سیکھا۔اس باب میں سادہ تفرقی مساوات حل کرنے کا نیا طریقہ دکھایا جائے گا جس میں \عددی{n} درجی سادہ تفرقی مساوات سے \عددی{n} عدد درجہ اول سادہ تفرقی مساوات کا نظام حاصل کیا جائے گا۔اس نظام کو حل کرنا بھی سکھایا جائے گا۔تفرقی مساوات کے نظام کو قالب اور سمتیہ کی صورت میں لکھنا زیادہ مفید ثابت ہوتا ہے لہٰذا حصہ \حوالہ{حصہ_نظام_قالب} میں قالب اور سمتیہ کے بنیادی حقائق پر غور کیا جائے گا۔

اسی باب میں تفرقی مساوات کے نظام کو حل کرنے کی بجائے تمام مساوات کی مجموعی طرز عمل پر غور کیا جائے گا جس سے نظام کے حل کی \اصطلاح{توازن}\فرہنگ{توازن}\حاشیہب{stability}\فرہنگ{stability} کے بارے میں معلومات حاصل ہوتی ہے۔انجینئری میں متوازن نظام  اہمیت رکھتے ہیں۔متوازن نظام میں کسی لمحے پر معمولی تبدیلی، بعد کے لمحات پر معمولی تبدیلی ہی پیدا کرتی ہے۔اس ترکیب سے مساوات کا اصل حل دریافت نہیں ہوتا لہٰذا اس کو \اصطلاح{کیفی ترکیب}\فرہنگ{کیفی ترکیب}\فرہنگ{ترکیب!کیفی}\حاشیہب{qualitative method}\فرہنگ{qualitative method} کہتے ہیں۔جس ترکیب سے نظام کا اصل حل حاصل ہوتا ہو اس کو \اصطلاح{مقداری ترکیب}\فرہنگ{مقداری ترکیب}\فرہنگ{ترکیب!مقداری}\حاشیہب{quantitative method}\فرہنگ{quantitative method} کہتے ہیں۔

 
\حصہ{قالب اور سمتیہ کے بنیادی حقائق}\شناخت{حصہ_نظام_قالب}
تفرقی مساوات کے نظام پر غور کے دوران قالب اور سمتیات استعمال کئے جائیں گے۔

دو عدد خطی سادہ تفرقی مساوات کے نظام
\begin{gather}
\begin{aligned}\label{مساوات_نظام_دو_مساوات}
y_1' &=a_{11}y_1+a_{12}y_2\\
y_2' &=a_{21}y_1+a_{22}y_2
\end{aligned}
\end{gather}
 میں دو عدد نا معلوم تفاعل \عددی{y_1(t)} اور \عددی{y_2(t)} پائے جاتے ہیں۔ان مساوات میں دائیں جانب اضافی تفاعل \عددی{g_1(t)} اور \عددی{g_2(t)} بھی موجود ہو سکتے ہیں۔اسی طرح \عددی{n} عدد درجہ اول سادہ تفرقی مساوات پر مبنی نظام
\begin{gather}
\begin{aligned}\label{مساوات_نظام_متعدد_مساوات}
y_1' &=a_{11}y_1+a_{12}y_2+\cdots+a_{1n}y_n\\
y_2' &=a_{21}y_1+a_{22}y_2+\cdots a_{2n}y_n\\
\vdots &\\
y_n'&=a_{n1}y_1+a_{n2}y_2+\cdots+a_{nn}y_n
\end{aligned}
\end{gather}
 میں \عددی{y_1(t)} تا \عددی{y_n(t)} نا معلوم تفاعل پائے جائیں گے۔درج بالا ہر مساوات میں دائیں جانب اضافی تفاعل بھی پائے جا سکتے ہیں۔
%====================
\جزوحصہء{تکنیکی اصطلاحات}
\موٹا{قالب}۔ \quad نظام \حوالہ{مساوات_نظام_دو_مساوات} کے عددی سر (جو مستقل یا متغیرات ممکن ہیں) کو \عددی{2 \times 2} \اصطلاح{قالب}\فرہنگ{قالب}\حاشیہب{matrix}\فرہنگ{matrix} \bM{A} کی صورت میں لکھا جا سکتا ہے۔
\begin{align}\label{مساوات_نظام_دو_جمع_دو_قالب}
\bM{A}=[a_{jk}]=
\begin{bmatrix}
a_{11} & a_{12}\\
a_{21} & a_{22}
\end{bmatrix}
\end{align}
اسی طرح نظام \حوالہ{مساوات_نظام_متعدد_مساوات} کے عددی سر کو \عددی{n \times n} قالب کی صورت میں لکھا جا سکتا ہے۔
\begin{align}\label{مساوات_نظام_متعدد_جمع_متعدد_قالب}
\bM{A}=[a_{jk}]=
\begin{bmatrix}
a_{11} & a_{12} &\cdots & a_{1n}\\
a_{21} & a_{22}& \cdots & a_{2n}\\
& \vdots &  &\\
a_{n1} & a_{n2}& \cdots & a_{nn}
\end{bmatrix}
\end{align}
قالب میں درج \عددی{a_{11}}، \عددی{a_{12}}، \عددی{a_{21}} وغیرہ کو \اصطلاح{اندراجات}\فرہنگ{اندراجات}\حاشیہب{entry}\فرہنگ{entry} کہتے ہیں۔ افقی لکیروں کو \اصطلاح{صف}\فرہنگ{صف}\حاشیہب{row}\فرہنگ{row} جبکہ عمودی لکیروں کو \اصطلاح{قطار}\فرہنگ{قطار}\حاشیہب{column}\فرہنگ{column} کہتے ہیں۔قالب \حوالہ{مساوات_نظام_دو_جمع_دو_قالب} میں پہلا صف \عددی{[a_{11} \,\,\,  a_{12}]} جبکہ دوسرا صف \عددی{[a_{21} \,\,\,  a_{22}]} ہے۔اسی طرح پہلا قطار درج ذیل ہے۔
\begin{align*}
\begin{bmatrix}
a_{11}\\
a_{21}
\end{bmatrix}
\end{align*}
اندراجات کی علامتی اظہار میں دو گنا زیر نوشت کا پہلا عدد صف کو ظاہر کرتا ہے جبکہ دوسرا عدد قطار کو ظاہر کرتا ہے۔یوں \عددی{a_{21}} دوسری صف اور پہلی قطار کا اندراج ہے۔اسی طرح قالب \حوالہ{مساوات_نظام_دو_جمع_دو_قالب} کا \اصطلاح{مرکزی وتر}\فرہنگ{مرکزی وتر}\فرہنگ{وتر!مرکزی}\حاشیہب{main diagonal}\فرہنگ{diagonal!main} \عددی{a_{11}} اور \عددی{a_{22}} پر مبنی ہے جبکہ قالب \حوالہ{مساوات_نظام_متعدد_جمع_متعدد_قالب} کا مرکزی وتر \عددی{a_{11}}، \عددی{a_{22}}، \نقطے، \عددی{a_{nn}} پر مبنی ہے۔ہمیں یہاں صرف \اصطلاح{مربع قالب}\فرہنگ{مربع قالب}\فرہنگ{قالب!مربع}\حاشیہب{square matrix}\فرہنگ{matrix!square} درکار ہوں گے۔مربع قالب سے مراد ایسی قالب ہے جس میں صفوں کی تعداد قطاروں کی تعداد کے برابر ہو۔ قالب \حوالہ{مساوات_نظام_دو_جمع_دو_قالب} اور قالب \حوالہ{مساوات_نظام_متعدد_جمع_متعدد_قالب} مربع قالب ہیں۔

\موٹا{سمتیہ}۔\quad ایک قطار اور \عددی{n} اندراج کا \اصطلاح{سمتیہ قطار}\فرہنگ{سمتیہ!قطار}\فرہنگ{قطار!سمتیہ}\حاشیہب{column vector}\فرہنگ{vector!column} درج ذیل ہے۔
\begin{align*}
\bM{x}=
\begin{bmatrix}
x_1\\
x_2\\
x_3\\
\vdots\\
x_n
\end{bmatrix}
\end{align*}
اسی طرح ایک صف اور \عددی{n} اندراج کا \اصطلاح{سمتیہ صف}\فرہنگ{سمتیہ صف}\فرہنگ{سمتیہ!صف}\حاشیہب{row vector}\فرہنگ{vector!row} درج ذیل ہے۔
\begin{align*}
\bM{v}=
\begin{bmatrix}
v_1 & v_2& v_3& \cdots & v_n
\end{bmatrix}
\end{align*}
%=========================
\جزوحصہء{قالب اور سمتیات کے ساتھ حساب}
\موٹا{برابری مساوات}۔\quad دو عدد \عددی{n \times n} قالب صرف اور صرف اس صورت برابر ہوں گے جب ان کے تمام نظیری اندراجات برابر ہوں۔ظاہر ہے کہ دو قالب کی برابری کے لئے لازم ہے کہ ان میں صفوں کی تعداد یکساں ہو اور ان میں قطاروں کی تعداد یکساں ہو۔یوں \عددی{n=2} کی صورت میں 
\begin{align*}
\bM{A}=
\begin{bmatrix}
a_{11} & a_{12}\\
a_{21} & a_{22}
\end{bmatrix} \quad \text{اور} \quad
\bM{B}=
\begin{bmatrix}
b_{11} & b_{12}\\
b_{21} & b_{22}
\end{bmatrix}
\end{align*}
صرف اور صرف اس صورت برابر \عددی{(\bM{A}=\bM{B})} ہوں گے جب
\begin{align*}
a_{11}&=b_{11}, \quad a_{12}=b_{12}\\
a_{21}&=b_{21}, \quad a_{22}=b_{22}
\end{align*}
ہوں۔دو عدد سمتیہ صف (یا دو عدد سمتیہ قطار) صرف اور صرف اس صورت برابر ہوں گے جب دونوں میں اندراجات کی تعداد  \عددی{n} برابر ہو اور  ان کے تمام نظیری  اندراجات برابر ہوں ۔یوں 
\begin{align*}
\bM{v}=
\begin{bmatrix}
v_1\\
v_2
\end{bmatrix} \quad \text{اور} \quad
\bM{x}=
\begin{bmatrix}
x_1\\
x_2
\end{bmatrix}
\end{align*} 
کی صورت میں \عددی{\bM{v}=\bM{x}} صرف اور صرف تب ہو گا جب
\begin{align*}
v_1=x_1\quad \text{اور} \quad v_2=x_2
\end{align*}
ہوں۔

\موٹا{مجموعہ} حاصل کرنے کی خاطر دونوں قالب کے نظیری اندراج کا مجموعہ لیا جاتا ہے۔دونوں قالب یکساں \عددی{m \times n} ہونا لازم ہے۔اسی طرح دونوں سمتیہ صف (یا دونوں سمتیہ قطار) میں برابر ارکان ہونا لازم ہے۔یوں \عددی{2 \times 2} قالب کا مجموعہ درج ذیل ہو گا۔
\begin{align}
\bM{A}+\bM{B}=
\begin{bmatrix}
a_{11}+b_{11} & a_{12}+b_{12}\\
a_{21}+b_{21}& a_{22}+b_{bb}
\end{bmatrix}, \quad \bM{v}+\bM{x}=
\begin{bmatrix}
v_1+x_1\\
v_2+x_2
\end{bmatrix}
\end{align} 
\موٹا{غیر سمتی ضرب} یعنی مستقل عدد  \عددی{c} سے ضرب

