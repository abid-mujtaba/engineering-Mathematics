\باب{خطی الجبرا کے اعدادی تراکیب}
اس باب میں ہم خطی الجبرائی مساوات کے نظام کے حل، مناسب سیدھی لکیروں کا حصول اور قالبی امتیازی اقدار کے حصول کے  اہم ترین تراکیب پر غور کریں گے۔یہ تراکیب اور اس سے ملتے جلتے تراکیب عملاً انتہائی اہم ثابت ہوتے ہیں جو انجینئری یا دیگر شعبوں (مثلاً شماریات) کے مسائل حل کرنے میں کام آتے ہیں۔

\حصہ{خطی مساوات کا نظام۔ گاوسی اسقاط، معکوس قالب}
\عددی{n} نا معلوم متغیرات  \عددی{x_1,\cdots,x_n} کے \عددی{m} خطی مساوات کے نظام (یا \عددی{m}  ہمزاد خطی مساوات) سے مراد درج ذیل روپ کی مساوات
\begin{gather}
\begin{aligned}\label{مساوات_خطی_اعدادی_قالبی_نظام_الف}
a_{11}x_1+\cdots+a_{1n}x_n&=b_1\\
a_{21}x_1+\cdots+a_{2n}x_n&=b_2\\
&\vdots\\
a_{m1}x_1+\cdots+a_{mn}x_n&=b_m
\end{aligned}
\end{gather}  
کا سلسلہ ہے  جہاں عددی سر \عددی{a_{jk}} اور \عددی{b_j} معلوم اعداد ہیں۔تمام \عددی{b_j} صفر ہونے کی صورت میں یہ نظام \اصطلاح{متجانس}\فرہنگ{متجانس}\حاشیہب{homogeneous}\فرہنگ{homogeneous} کہلاتا ہے ورنہ اس کو \اصطلاح{غیر متجانس}\فرہنگ{متجانس!غیر}\حاشیہب{nonhomogeneous}\فرہنگ{nonhomogeneous} کہتے ہیں۔اگر آپ قالبی ضرب (حصہ \حوالہ{حصہ_الجبرا_قالبی_ضرب}) سے آشنا ہوں  تب آپ دیکھ سکتے ہیں کہ نظام \حوالہ{مساوات_خطی_اعدادی_قالبی_نظام_الف} کو ایک سمتی مساوات
\begin{align}\label{مساوات_خطی_اعدادی_قالبی_نظام_ب}
\bM{A}\bM{x}=\bM{b}
\end{align}
لکھا جا سکتا ہے جہاں \اصطلاح{عددی سر قالب}\فرہنگ{عددی سر!قالب} \عددی{\bM{A}=[a_{ik}]} درج ذیل \عددی{m\times n} قالب ہے
\begin{align*}
\bM{A}=
\begin{bmatrix}
a_{11}&a_{12}\cdots a_{1n}\\
a_{21}&a_{22}\cdots a_{2n}\\
\vdots\\
a_{m1}&a_{m2}\cdots a_{mn}
\end{bmatrix},
\quad
\bM{x}=
\begin{bmatrix}
x_1\\
x_2\\
\vdots\\
x_n
\end{bmatrix},
\quad
\bM{b}=
\begin{bmatrix}
b_1\\
b_2\\
\vdots\\
b_m
\end{bmatrix}
\end{align*}
جبکہ \عددی{\bM{x}} اور \عددی{\bM{b}} سمتیہ قطار ہیں۔نظام \حوالہ{مساوات_خطی_اعدادی_قالبی_نظام_الف} کے حل سے مراد اعداد \عددی{x_1,\cdots,x_n} کا سلسلہ ہے جو ان تمام \عددی{m} مساوات کو مطمئن کرتے ہیں اور نظام \حوالہ{مساوات_خطی_اعدادی_قالبی_نظام_الف} کے حل سمتیہ سے مراد سمتیہ \عددی{\bM{x}} ہے جس کے اجزاء  نظام \حوالہ{مساوات_خطی_اعدادی_قالبی_نظام_الف} کے حل ہیں۔

زیادہ تعداد کی مساوات کے نظام کا حل بذریعہ قاعدہ کریمر (حصہ \حوالہ{حصہ_الجبرا_قاعدہ_کریمر})  قابل عمل نہیں ہے۔زیادہ بہتر ترکیب \اصطلاح{گاوسی اسقاط}\فرہنگ{گاوسی اسقاط} ہے جس کو ایک مثال کی مدد سے سمجھتے ہیں۔

%===============
\ابتدا{مثال}\شناخت{مثال_خطی_اعدادی_گاوسی_اسقاط}\quad \موٹا{گاوسی اسقاط}\\
درج ذیل نظام کو حل کریں۔
\begin{align*}
2w+x+2y+z&=6\\
6w-6x+6y+12z&=36\\
4w+3x+3y-3z&=-1\\
2w+2x-y+z&=10
\end{align*}
حل:\quad
\موٹا{پہلا قدم:} ہم پہلی مساوات کے  مضرب  کو باقی مساوات سے منفی کرتے ہوئے ان سے \عددی{w} حذف کرتے ہوئے درج ذیل حاصل کرتے ہیں۔
\begin{alignat*}{4}
-9x&{}&+9z&=18\\
x&-y&-5z&=-13\\
x&-3y&{}&=4
\end{alignat*}
\موٹا{دوسرا قدم:} ان میں پہلی مساوات کے مضرب باقی مساوات سے منفی کرتے ہوئے ان سے \عددی{x} حذف کرتے ہوئے درج ذیل حاصل کرتے ہیں۔
\begin{align*}
-y-4z&=-11\\
-3y+z&=6
\end{align*}
\موٹا{تیسرا قدم:} ان میں پہلی مساوات کے مضرب کو باقی مساوات سے منفی کرتے ہوئے ان سے \عددی{y} حذف کرتے ہوئے درج ذیل حاصل کرتے ہیں۔
\begin{align*}
13z=39
\end{align*}
\موٹا{آخری قدم:} ہم اب واپس پر کرتے ہوئے تمام نا معلوم متغیرات حاصل کرتے ہیں۔
\begin{gather}
\begin{aligned}
13z&=39\\
-y-4\cdot3&=-11\\
-9x\phantom{+4y}+9\cdot 3&=18\\
2w+1+2\cdot(-1)+3&=6
\end{aligned}
\quad 
\begin{aligned}
z&=3\\
y&=-1\\
x&=1\\
w&=2
\end{aligned}
\end{gather} 
\انتہا{مثال}
%=======================

مثال \حوالہ{مثال_خطی_اعدادی_گاوسی_اسقاط} میں \عددی{a_{11}\ne 0} تھا۔اگر ایسا نہ ہوتا تب ہم باقی مساوات سے \عددی{w} حذف کرنے میں نا کام ہوتے۔یوں \عددی{a_{11}=0} کی صورت میں نظام میں مساوات کی ترتیب بدلی جائے گی  تا کہ نظام میں پہلی مساوات کا پہلا عددی سر غیر صفر ہو (اور ہو سکتا ہے کہ نا معلوم متغیرات کی ترتیب بھی بدلنی پڑے)۔باقی قدم پر بھی ایسا ہی کرنا پڑ سکتا ہے۔اس طرح درج ذیل ترکیب حاصل ہوتی ہے جس کی اطلاق کے بعد حاصل قیمتیں پر کرتے ہوئے تمام متغیرات حاصل کیے جاتے ہیں۔


\noindent\makebox[\linewidth]{\rule{\textwidth}{0.4pt}}
\موٹا{الخوارزمی: گاوسی اسقاط}\فرہنگ{الخوارزمی!گاوسی اسقاط}\حاشیہد{algorithm}\فرہنگ{algorithm}\\
مساوات \حوالہ{مساوات_خطی_اعدادی_قالبی_نظام_ب} میں \عددی{m=n} کی صورت میں \عددی{n\times n} قالب \عددی{\bM{A}} کے ساتھ بطور آخری صف \عددی{\bM{b}} شامل کرتے ہوئے \عددی{n\times (n+1)} قالب \عددی{\bM{B}=[b_{jk}]} حاصل ہو گا جس کے لئے گاوسی اسقاط کی \اصطلاح{الکراجی}\فرہنگ{الکراجی}\حاشیہب{algorithm}\فرہنگ{algorithm} درج ذیل ہے۔

\عددی{k=1} تا \عددی{k=n-1} \اصطلاح{کے لئے}\فرہنگ{کے لئے}  کریں:\\
ایسا کم تر \عددی{j\ge k} تلاش کریں کہ \عددی{b_{jk}\ne 0} ہو۔\\
\اصطلاح{اگر}\فرہنگ{اگر} ایسا کوئی \عددی{j} نہیں پایا جاتا ہو \اصطلاح{تب}\فرہنگ{تب} بتائیں کہ \عددی{\bM{A}} نادر ہے اور حساب \اصطلاح{روک}\فرہنگ{روک} دیں،\\
\اصطلاح{ورنہ}\فرہنگ{ورنہ} \عددی{\bM{B}} کے صف \عددی{j} اور صف \عددی{k} کے اجزاء کا آپس میں تبادلہ  کرتے ہوئے چلتے رہیں۔\\
\عددی{j=k+1} تا \عددی{j=n} \اصطلاح{کے لئے} کریں:\\
\عددی{q:\tfrac{b_{jk}}{b_{kk}}}\\
\عددی{p=k+1} تا \عددی{p=n+1} \اصطلاح{کے لئے} کریں:\\
\عددی{b_{jp}:b_{jp}-qb_{kp}}\\
\اصطلاح{اگر} \عددی{b_{nn}=0} ہو \اصطلاح{تب} بتائیں کہ \عددی{\bM{A}} نادر ہے اور حساب \اصطلاح{روک} دیں۔\\
\noindent\makebox[\linewidth]{\rule{\textwidth}{0.4pt}}
%=========================================

ہر قدم پر پہلی مساوات کے پہلی متغیر کے عددی سر کو \اصطلاح{چول عددی سر}\فرہنگ{چول!عددی سر}\حاشیہب{pivotal coefficient}\فرہنگ{pivotal coefficient} کہتے ہیں جس کا غیر صفر ہونا ضروری ہے۔اگر چول عددی سر کی قیمت کم ہو تب ہمیں مطابقتی مساوات کا بڑا مضرب باقی مساوات سے منفی کرنا ہو گا جس سے  تعداد ہندسہ خلل بڑھتے ہوئے  نتائج متاثر کرے گا۔اس سے بچنے کی ترکیب سمجھنے سے پہلے آئیں ایک مثال سے ایسا ہوتے دیکھیں۔

%=================
\ابتدا{مثال}\شناخت{مثال_خطی_اعدادی_چول}\quad \موٹا{کم چول عددی سر سے پیدا مشکلات}\\
درج ذیل نظام 
\begin{align*}
0.0004x_1+1.402x_2&=1.406\\
0.4003x_1-1.502x_2&=2.501
\end{align*}
کا حل \عددی{x_1=10}، \عددی{x_2=1} ہے۔ہم چار ہندسی غیر مقررہ نقطہ نظام استعمال کرتے ہوئے اس کو گاوسی اسقاط سے حل کرتے ہیں۔

(الف)  پہلی مساوات کو مساوات چول لیتے ہوئے ہم اس کو \عددی{q=\tfrac{0.4003}{0.0004}=1001} سے ضرب دے کر دوسری مساوات سے منفی کر کے
\begin{align*}
-1405x_2=-1404
\end{align*}
حاصل کرتے ہیں۔یوں \عددی{x_2=\tfrac{-1404}{-1405}=0.9993} ہو گا اور یوں پہلی مساوات سے \عددی{x_1=10} کی بجائے 
\begin{align*}
x_1=\frac{1}{0.0004}(1.406-1.402\cdot 0.9993)=\frac{0.005}{0.0004}=12.5
\end{align*}
حاصل ہو گا۔اس ناکامی کی وجہ \عددی{\abs{a_{12}}} کے لحاظ سے  \عددی{\abs{a_{11}}} کی کم قیمت ہے جو \عددی{x_2} میں تعداد ہندسہ خلل کی قلیل قیمت سے \عددی{x_1} کی قیمت میں بہت زیادہ خلل پیدا کرتا ہے۔\\
(ب) آئیں اب دوسری مساوات کو چول مساوات لے کر اس کو \عددی{\tfrac{0.0004}{0.4003}=0.0009993} سے ضرب دے کر پہلی مساوات سے منفی کرتے ہوئے
\begin{align*}
1.404x_2=1.404
\end{align*}
حاصل کرتے ہیں۔یوں \عددی{x_2=1} حاصل ہو گا جس کو دوسری مساوات میں پر کرتے ہوئے \عددی{x_1=10} ملتا ہے۔یہاں \عددی{\abs{a_{22}}} کے لحاظ سے \عددی{\abs{a_{21}}} بہت کم نہیں ہے لہٰذا \عددی{x_2} میں معمولی تعداد ہندسہ خلل \عددی{x_1} کی قیمت میں بڑا خلل پیدا نہیں کرتا ہے۔یہی ہماری کامیابی کی  وجہ ہے۔یقیناً \عددی{x_2=1.002} کی صورت میں بھی دوسری مساوات سے \عددی{x_1=\tfrac{2.501+1.505}{0.4003}=10.01} حاصل ہوتا جو بہت بہتر نتیجہ  ہے۔
\انتہا{مثال}
%======================

وہ مساوات جس کے \عددی{x_1} کا عددی سر باقی مساواتوں کے \عددی{x_1} کے عددی سر سے بڑا ہو کو پہلی مساوات منتخب کرتے ہوئے اور اسی طرح دوسری قدم پر \عددی{x_2} کے لحاظ سے مساوات منتخب کرتے ہوئے نظام میں پہلی، دوسری، تیسری،\نقطے مساوات منتخب کی جا سکتی ہے۔اس عمل کو \اصطلاح{جزوی چول}\فرہنگ{چول!جزوی}\حاشیہب{partial pivoting}\فرہنگ{pivot!partial pivoting} کہتے ہیں۔ \اصطلاح{مکمل چول}\فرہنگ{چول!مکمل}\حاشیہب{total pivoting}\فرہنگ{pivot!total pivoting} میں ہم  پورے نظام میں سب سے بڑے حتمی عددی سر کو چول عددی سر لیتے ہوئے باقی مساوات میں سے اس کا مطابقتی متغیر حذف کرتے ہیں۔اگلی قدم میں اسی ترکیب کو دہراتے ہیں اور اسی طرح آخر تک چلتے ہیں۔عملاً مکمل چول کی ترکیب زیادہ مہنگی ثابت ہوتی ہے لہٰذا جزوی چول کی ترکیب ہی استعمال کی جاتی ہے۔

ہم پوری مساوات کو بڑی عدد سے ضرب دے کر کسی بھی عددی سر کی قیمت بڑھا سکتے ہیں لیکن ایسا کرنے سے نتائج پر کوئی اثر نہیں پڑتا ہے۔مساوات کو جزو ضربی سے ضرب دینے کو  \اصطلاح{تبدیلی پیما صف}\فرہنگ{تبدیلی پیما!صف}\حاشیہب{scaling}\فرہنگ{scaling} کہتے ہیں۔عملاً ہم \عددی{10} (یا کمپیوٹر کی اساس \عددی{\beta}) کی طاقت سے مساوات کو ضرب دے کر عددی سر کی سب سے بڑی حتمی قیمت کو \عددی{0.1} اور \عددی{1} (یعنی \عددی{\beta^{-1}} اور \عددی{1}) کے بیچ لاتے ہیں۔

عملاً ہم تبدیل پیما جزوی چول استعمال کرتے ہیں یعنی حذف کی \عددی{k} ویں قدم (جہاں \عددی{k=1,2,\cdots} ہو گا) میں ہم باقی میسر \عددی{n-k} مساواتوں میں سے اس کو مساوات چول منتخب کرتے ہیں جس کے متغیر \عددی{x_k} کے عددی سر اور اس مساوات میں سب سے بڑی حتمی قیمت کے عددی سر کے حاصل تقسیم کی حتمی قیمت سب سے زیادہ ہو۔

گاوسی اسقاط میں پیدا ہونے والے خلل پر اس کتاب میں غور نہیں کیا جائے گا۔   

\جزوحصہء{ترکیب گاوس میں ترمیم}
ترکیب گاوسی کے کئی ترامیم ممکن ہیں۔ہم \اصطلاح{شولسکی}\فرہنگ{شولسکی}\حاشیہد{فرانسیسی ریاضی دان اندرِ لوئی شولسکی [1875-1918]} کے ایک قاعدہ پر مبنی ترمیم پیش کرتے ہیں۔ شولسکی\حاشیہب{Cholesky}\فرہنگ{Cholesky} کا قاعدہ کہتا ہے  کہ حتمی مثبت چکور قالب \عددی{\bM{A}} کو 
\begin{align}\label{مساوات_خطی_اعدادی_شولسکی_الف}
\bM{A}=\bM{L}\bM{U}
\end{align} 
لکھا جا سکتا ہے جہاں \عددی{\bM{L}} اور \عددی{\bM{U}} بالترتیب نچلا تکونی قالب\فرہنگ{تکونی قالب!نچلا} اور بالائی تکونی قالب\فرہنگ{تکونی قالب!بالائی} ہیں۔\عددی{\bM{L}} اور \عددی{\bM{U}} عملاً یکتا ہوں گے۔ہم مساوات کو حل کیے بغیر \عددی{\bM{L}} اور \عددی{\bM{U}} کو حاصل کر سکتے ہیں (نیچے مثال دیکھیں)۔\عددی{n} متغیرات کے \عددی{n} مساوات کا نظام \عددی{\bM{A}\bM{x}=\bM{b}} حل کرنے  کے لئے ہم  مساوات \حوالہ{مساوات_خطی_اعدادی_شولسکی_الف} کا سہارا لیتے ہوئے  نظام کو
\begin{align*}
\bM{L}\bM{U}\bM{x}=\bM{b}
\end{align*}
لکھتے ہیں۔اس کو بائیں طرف \عددی{\bM{L}^{-1}} سے ضرب دے کر
\begin{align}\label{مساوات_خطی_اعدادی_شولسکی_ب}
\bM{U}\bM{x}=\bM{z}\quad \quad \bM{z}=\bM{L}^{-1}\bM{b}
\end{align}
حاصل ہو گا جو اس نظام کی تکونی صورت ہے۔ہم پہلے \عددی{\bM{z}}  کو درج ذیل تعلق
\begin{align}\label{مساوات_خطی_اعدادی_شولسکی_پ}
\bM{L}\bM{z}=\bM{b}
\end{align}
سے حاصل کر کے بعد میں 
\begin{align}\label{مساوات_خطی_اعدادی_شولسکی_ت}
\bM{U}\bM{x}=\bM{z}
\end{align}
سے \عددی{\bM{x}} حاصل کریں گے۔بہت سی اہم مسائل میں \عددی{\bM{A}} تشاکل قالب ہو گا جس کی بنا \عددی{\bM{U}=\bM{L}^T} ہو گا (درج ذیل مثال دیکھیں)۔

%=====================
\ابتدا{مثال}\quad \موٹا{ترکیب شولسکی}\\
آپ تسلی کر سکتے ہیں کہ نظام
\begin{align*}
x+2y+3z&=14\\
2x+3y+4z&=20\\
3x+4y+z&=14
\end{align*}
کا حل \عددی{x=1}، \عددی{y=2}، \عددی{z=3} ہے۔ہم اس حل کو ترکیب شولسکی سے حاصل کرتے ہیں۔عددی سر قالب تشاکلی ہے لہٰذا  \عددی{\bM{U}=\bM{L}^T} ہو گا۔ ہم ضرب قالب کی تعریف استعمال کرتے ہوئے
\begin{align*}
\begin{bmatrix}
1&2&3\\
2&3&4\\
3&4&1
\end{bmatrix}=
\begin{bmatrix}
a_{11}&0&0\\
a_{12}&a_{22}&0\\
a_{13}&a_{23}&a_{33}
\end{bmatrix}
\begin{bmatrix}
a_{11}&a_{12}&0\\
0&a_{22}&a_{23}\\
0&0&a_{33}
\end{bmatrix}
\end{align*}
کے دونوں اطراف مطابقتی اجزاء کو برابر پر کرتے ہوئے  \عددی{\bM{U}} کے اجزاء حاصل کرتے ہیں۔ایسا کرنے سے ہمیں  بالترتیب \عددی{a^2_{11}=1} مثلاً \عددی{a_{11}=1} جس سے \عددی{a_{11}a_{12}=a_{12}=2}، \عددی{a_{11}a_{13}=a_{13}=3}، \عددی{a^2_{12}+a^2_{22}=4+a^2_{22}=3} مثلاً
 \عددی{a_{22}=i\,(\sqrt{-1})} اور اس سے
\begin{align*}
a_{12}a_{13}+a_{22}a_{23}=6+ia_{23}=4,\quad a_{23}=i2
\end{align*}
اور آخر میں
\begin{align*}
a^2_{13}+a^2_{23}+a^2_{33}=9-4+a^2_{33}=1
\end{align*}
سے مثلاً \عددی{a_{33}=i2} حاصل ہو گا۔یوں مساوات \حوالہ{مساوات_خطی_اعدادی_شولسکی_پ}
\begin{align*}
\begin{bmatrix}
1&0&0\\
2&i&0\\
3&i2&i2
\end{bmatrix}
\begin{bmatrix}
z_1\\
z_2\\
z_3
\end{bmatrix}=
\begin{bmatrix}
14\\
20\\
14
\end{bmatrix}\quad\implies
\begin{bmatrix}
z_1\\
z_2\\
z_3
\end{bmatrix}=
\begin{bmatrix}
14\\
i8\\
i6
\end{bmatrix}
\end{align*}
دے گا۔آخر میں ہم مساوات \حوالہ{مساوات_خطی_اعدادی_شولسکی_ت} حل کرتے ہیں یعنی:
\begin{align*}
\begin{bmatrix}
1&2&3\\
0&i&i2\\
0&0&i2
\end{bmatrix}
\begin{bmatrix}
x_1\\
x_2\\
x_3
\end{bmatrix}=
\begin{bmatrix}
14\\
i8\\
i6
\end{bmatrix}\quad\implies
\begin{bmatrix}
x_1\\
x_2\\
x_3
\end{bmatrix}=
\begin{bmatrix}
1\\
2\\
3
\end{bmatrix}
\end{align*}

\انتہا{مثال}
%===========================

گاوسی اسقاط کی دوسری ترمیم کو \اصطلاح{گاوس جارڈن اسقاط}\فرہنگ{گاوس جارڈن اسقاط}\فرہنگ{Gauss-Jordan elimination} کہتے ہیں۔اس ترکیب میں قالب کو "تکونی صورت" کی بجائے مزید چال چلتے ہوئے  "وتری صورت" میں تبدیل کرتے ہوئے قیمتوں کے واپس پر کرنے کے عمل سے چھٹکارا حاصل کیا جاتا ہے۔ان اضافی چال کی بنا مساوات کا نظام حل کرنے میں کوئی آسانی پیدا نہیں ہوتی ہے۔البتہ معکوس قالب حاصل کرنے میں صورت حال مختلف ہے جہاں ترکیب گاوس اور ترکیب گاوس جارڈن دونوں میں \عددی{n^3} ضرب درکار ہیں۔

%==============
\جزوحصہء{معکوس قالب} 
غیر نادر چکور قالب \عددی{\bM{A}} کا معکوس اب اصولی طور پر \عددی{n} عدد نظام
\begin{align}\label{مساوات_خطی_اعدادی_شولسکی_ٹ}
\bM{A}\bM{x}=\bM{b}_j\quad \quad \quad (j=1,\cdots,n)
\end{align}
کے حل سے حاصل کیا جا سکتا ہے جہاں \عددی{n\times n} اکائی قالب کا \عددی{j} واں قطار \عددی{\bM{b}_j} ہے۔

البتہ اکائی قالب \عددی{\bM{I}} پر ترکیب گاوس جارڈن کی طرح عمل کرتے ہوئے \عددی{\bM{A}} کی تخفیف سے  \عددی{\bM{I}} حاصل کرتے ہوئے \عددی{\bM{A}^{-1}} کے حصول کو ترجیح دی جاتی ہے۔

%===================
\حصہء{سوالات}
سوال \حوالہ{سوال_خطی_اعدادی_گاوسی_اسقاط_الف} تا سوال \حوالہ{سوال_خطی_اعدادی_گاوسی_اسقاط_ب} کو گاوسی اسقاط سے حل کریں۔
%======================
\ابتدا{سوال}\شناخت{سوال_خطی_اعدادی_گاوسی_اسقاط_الف}\quad
\begin{align*}
2x+3y&=7\\
x-y&=1
\end{align*}
جوابات:\quad
$x=2,\,\,y=1$
\انتہا{سوال}
%============================
\ابتدا{سوال}\quad
\begin{align*}
-2x+y&=5\\
x+2y&=0
\end{align*}
جوابات:\quad
$x=-2,\,\,y=1$
\انتہا{سوال}
%============================
\ابتدا{سوال}\quad
\begin{align*}
-3x-y&=-3\\
5x+2y&=6
\end{align*}
جوابات:\quad
$x=0,\,\,y=3$
\انتہا{سوال}
%============================
\ابتدا{سوال}\شناخت{سوال_خطی_اعدادی_گاوسی_اسقاط_پ}\quad
\begin{align*}
x-y+z&=2\\
2x+y-3z&=-3\\
3x+2y+z&=7
\end{align*}
جوابات:\quad
$x=-1,\,\,y=1,\,\,z=2$
\انتہا{سوال}
%============================
\ابتدا{سوال}\شناخت{سوال_خطی_اعدادی_گاوسی_اسقاط_ت}\quad
\begin{align*}
x+y+z&=-2\\
-2x+y-3z&=13\\
-3x+2y-z&=10
\end{align*}
جوابات:\quad
$x=-1,\,\,y=2,\,\,z=-3$
\انتہا{سوال}
%============================
\ابتدا{سوال}\quad
\begin{align*}
2x-y+4z&=2\\
x+y-3z&=11\\
-3x+y-z&=-3
\end{align*}
جوابات:\quad
$x=4,\,\,y=10,\,\,z=1$
\انتہا{سوال}
%============================
\ابتدا{سوال}\quad
\begin{align*}
x-2y+z&=1\\
3x-2y-z&=-1
\end{align*}
جوابات:\quad
$x=y,\,\,z=y+1$
\انتہا{سوال}
%============================
\ابتدا{سوال}\quad
\begin{align*}
x-2y+z&=0\\
2x-2z&=-4
\end{align*}
جوابات:\quad
$x=y-1,\,\,z=y+1$
\انتہا{سوال}
%============================
\ابتدا{سوال}\quad
\begin{align*}
4x-3y+3z&=0\\
8x+7y-7z&=0
\end{align*}
جوابات:\quad
$x=0,\,\,z=y$
\انتہا{سوال}
%============================
\ابتدا{سوال}\quad
\begin{align*}
2w-4x+3y-z&=3\\
w-2x+5y-3z&=0\\
3w-6x-y-z&=0
\end{align*}
جوابات:\quad
$w=2x+1,y=1,z=2$
\انتہا{سوال}
%============================
\ابتدا{سوال}\شناخت{سوال_خطی_اعدادی_گاوسی_اسقاط_ب}\quad
\begin{align*}
3w-x+8y-2z&=-2\\
-w+2x-13y+3z&=3\\
4w+3x-9y+z&=1
\end{align*}
جوابات:\quad
$w=0,\,\, x=2y,\,\,z=3y+1$
\انتہا{سوال}
%============================
\ابتدا{سوال}\quad \موٹا{(تعداد قدم)} کسی بھی اعدادی ترکیب کی کارکردگی کی ناپ اس ترکیب سے حل نکالنے کے لئے درکار  کل حسابی اعمال کی تعداد ہے۔ دکھائیں کہ \عددی{m=n} کی صورت میں، واپس پر کرنے کے عمل کے علاوہ، مساوات \حوالہ{مساوات_خطی_اعدادی_قالبی_نظام_الف} کو گاوسی اسقاط سے حل کرنے کے لئے \عددی{\tfrac{1}{2}n(n-1)} تقسیم، \عددی{\tfrac{1}{3}n(n^2-1)} ضرب اور \عددی{\tfrac{1}{3}n(n^2-1)} جمع حاصل کرنے ہوں گے۔ یوں بڑی \عددی{n} کی صورت میں ہم کہہ سکتے ہیں کہ  \عددی{\tfrac{n^3}{3}}  ضرب اور جمع  درکار ہوں گے۔تقسیم کی تعداد کم ہونے کی بنا رد کی جا سکتی ہے۔
\انتہا{سوال}
%=============================
\ابتدا{سوال}\quad
دکھائیں کہ \عددی{m=n} کی صورت میں گاوسی اسقاط سے مساوات \حوالہ{مساوات_خطی_اعدادی_قالبی_نظام_الف} حل کرنے کے دوران واپس پر کرنے کے عمل میں  \عددی{\tfrac{1}{2}n(n-1)} ضرب، \عددی{\tfrac{1}{2}n(n-1)} جمع اور \عددی{n} تقسیم درکار ہوں گے۔
\انتہا{سوال}
%=====================
\ابتدا{سوال}\quad
قلم و کاغذ سے حل کرتے ہوئے ہم عموماً صرف عددی سر لکھ کر ان پر حسابی عمل کرتے ہیں۔یوں مثال \حوالہ{مثال_خطی_اعدادی_گاوسی_اسقاط} میں پہلے قدم کو درج ذیل لکھا جا سکتا ہے جہاں \عددی{S_1} سے مراد پہلی صف ہے۔یوں \عددی{S_2-3S_1} سے مراد دوسری صف سے پہلی صف کی تین گنا کی تفریق ہے۔
\begin{align*}
\centering
\begin{array}{rrrrrrl}
2&1&2&1&6&12& S_1\\
0&-9&0&9&18&18&S_2-3S_1\\
0&1&-1&-5&-13&-18&S_3-2S_1\\
0&1&-3&0&4&2&S_4-S_1
\end{array}
\end{align*} 
سوال \حوالہ{سوال_خطی_اعدادی_گاوسی_اسقاط_پ} میں اس طرح تمام قدم لکھیں۔
\انتہا{سوال}
%=====================
\ابتدا{سوال}\quad \موٹا{(گاوس جارڈن اسقاط)} 
مثال \حوالہ{مثال_خطی_اعدادی_گاوسی_اسقاط} میں گاوسی اسقاط درج ذیل دیتا ہے۔
\begin{alignat*}{5}
&2w&+x&+2y&+z&=\phantom{-0}6 \tag*{(الف)}\\
&{}&-9x&{}&+9z&=\phantom{-}18\tag*{(ب)}\\
&{}&{}&-y&-4z&=-11\tag*{(پ)}\\
&{}&{}&{}&13z&=\phantom{-}39\tag*{(ت)}
\end{alignat*}{5}
  گاوس جارڈن اسقاط میں ہم  (ب) استعمال کرتے ہوئے (الف) سے  \عددی{x} حذف  کرتے ہیں۔اس کے بعد (پ) کی مدد سے (الف) اور (ب) سے \عددی{y} حذف کرتے ہیں[(ب) سے حذف کی یہاں ضرورت نہیں ہے] اور آخر میں (ت) کی مدد سے (الف)، (ب)، (پ) سے \عددی{z}  حذف کرتے ہیں۔دکھائیں کہ ایسا کرنے سے درج ذیل حاصل ہوتا ہے۔
\begin{alignat*}{5}
&2w&{}&{}&{}&=4\\
&{}&-9x&{}&{}&=-9\\
&{}&{}&-y&{}&=1\\
&{}&{}&{}&13z&=39
\end{alignat*}
ان مساوات کو حل کرتے ہوئے  \عددی{w=2}، \عددی{x=1}، \عددی{y=-1} اور \عددی{z=3} حاصل کریں۔
\انتہا{سوال}
%========================
\ابتدا{سوال}\quad
گاوس جارڈن اسقاط سے سوال \حوالہ{سوال_خطی_اعدادی_گاوسی_اسقاط_ت} حل کریں۔
\انتہا{سوال}
%========================
\ابتدا{سوال}\quad
درج ذیل نظام پر مثال \حوالہ{مثال_خطی_اعدادی_چول} کی طرح بحث کریں۔
\begin{align*}
0.0003x_1+3.0000x_2&=2.0001\\
1.0000x_1+1.0000x_2&=1.0000
\end{align*}
\انتہا{سوال}
%===========================

\حصہ{خطی مساوات کا نظام: حل بذریعہ اعادہ}
گزشتہ حصہ میں گاوسی اسقاط پر غور کیا گیا جو خطی مساوات کے نظام کو حل کرنے کی \اصطلاح{بلا واسطہ تراکیب} میں سے ایک ہے۔ان تراکیب میں ہم پہلے سے بتا سکتے ہیں کہ حل حاصل کرنے کی خاطر کتنی حساب درکار ہو گی۔اس کے برعکس \اصطلاح{بالواسطہ ترکیب} یا \اصطلاح{اعادہ}\فرہنگ{ترکیب اعادہ}\حاشیہب{iterative method}\فرہنگ{iterative method} میں ہم تخمینی قیمت سے شروع کر کے، بار بار حساب دہراتے ہوئے، حل کی بہتر سے بہتر  تخمین کی طرف بڑھتے ہیں۔یوں جتنی زیادہ درستگی درکار ہو اتنا زیادہ حساب درکار ہو گا۔

اعادہ کی تراکیب ہم اس صورت استعمال کرتے ہیں جب ارتکاز کی شرح زیادہ ہو اور یوں بلا واسطہ تراکیب سے زیادہ جلدی حل حاصل ہو۔عملی استعمال کی ایک اہم ترکیب اعادہ کو \اصطلاح{گاوس زائڈل اعادہ}\فرہنگ{اعادہ!گاوس زائڈل}\حاشیہب{Gauss-Seidel iteration}\فرہنگ{iteration!Gauss-Seidel} کہتے ہیں۔  جس کو ہم ایک مثال کی مدد سے سمجھتے ہیں۔درج ذیل نظام پر غور کریں۔
\begin{gather}
\begin{alignedat} {5}\label{مساوات_خطی_اعدادی_مثال_الف}
&w&-0.25x&-0.25y&{}&=50\\
&-0.25w&+x&{}&-0.25z&=50\\
&-0.25w&{}&+y&-0.25z&=25\\
&{}&-0.25x&-0.25y&+z&=25
\end{alignedat}
\end{gather}
(اس قسم کے نظام جزوی تفرقی مساوات کے حل اور لچکدار منحنی کی  باہمی تحریف کے دوران پیش آتے ہیں۔) ہم اس نظام کو درج ذیل صورت میں 
\begin{gather}
\begin{alignedat}{6}\label{مساوات_خطی_اعدادی_مثال_ب}
w&={}&0.25x&+0.25y&{}&+50\\
x&=0.25w&{}&{}&+0.25z&+50\\
y&=0.25w&{}&{}&+0.25z&+25\\
z&=&{}&0.25x&+0.25y+{}&+25
\end{alignedat}
\end{gather}
لکھ کر انہیں اعادہ میں استعمال کرتے ہیں یعنی ہم تمام متغیرات کی تخمینی قیمتوں مثلاً \عددی{w_0=100}، \عددی{x_0=100}، \عددی{y_0=100}، \عددی{z_0=100} سے ابتدا کرتے ہوئے بہتر تخمین 
\begin{gather}
\begin{alignedat}{7}\label{مساوات_خطی_اعدادی_مثال_پ}
w_1&={}&0.25x_0&+0.25y_0&{}&+50&=100.00\\
x_1&=0.25w_1&{}&{}&+0.25z_0&+50&=100.00\\
y_1&=0.25w_1&{}&{}&+0.25z_0&+25&=75.00\\
z_1&=&{}&{}0.25x_1&+0.25y_1+{}&+25&=68.75
\end{alignedat}
\end{gather}
حاصل کرتے ہیں۔مساوات \حوالہ{مساوات_خطی_اعدادی_مثال_ب} کے دائیں ہاتھ تازہ ترین قیمتیں پر کرتے ہوئے مساوات \حوالہ{مساوات_خطی_اعدادی_مثال_پ} حاصل کی گئی ہیں۔ہر مرتبہ متغیر کی تازہ ترین قیمت استعمال کی جاتی ہے۔یوں دوسری مساوات میں  میں \عددی{w_0} کی بجائے (\عددی{w} کی تازہ ترین قیمت) \عددی{w_1} استعمال کی جائے گی۔اسی طرح آخری مساوات میں \عددی{x_1} اور \عددی{y_1} استعمال کیے گئے ہیں۔اگلے قدم میں مزید بہتر نتائج حاصل کرتے ہیں۔
\begin{alignat*}{7}
w_2&={}&0.25x_1&+0.25y_1&{}&+50&=93.75\\
x_2&=0.25w_2&{}&{}&+0.25z_1&+50&=90.62\\
y_2&=0.25w_2&{}&{}&+0.25z_1&+25&=65.62\\
z_2&=&{}&{}0.25x_2&+0.25y_2+{}&+25&=64.06
\end{alignat*}
آپ تسلی کر سکتے ہیں کہ درست حل \عددی{w=x=87.5}، \عددی{y=z=62.5} ہے۔

ہم ثبوت پیش کیے بغیر بتانا چاہتے ہیں کہ ترکیب گاوس زائڈل  ہر ابتدائی تخمینی قیمتوں کے لئے صرف اور صرف اس صورت مرتکز ہو گا جب  \اصطلاح{قالب اعادہ}\فرہنگ{قالب!اعادہ}\فرہنگ{اعادہ!قالب}\حاشیہب{iteration matrix}\فرہنگ{iteration!matrix} \عددی{\bM{C}} (مساوات \حوالہ{مساوات_خطی_اعدادی_مثال_ٹ} دیکھیں) کے ہر امتیازی قدر کی  حتمی قیمت \عددی{1} سے کم ہو اور ارتکاز کی شرح  \اصطلاح{رداس طیف} (یعنی ان حتمی قیمتوں میں سب سے زیادہ قیمت) پر منحصر ہے۔قالب \عددی{\bM{C}} کو اب حاصل کرتے ہیں۔فرض کریں کہ درج ذیل \عددی{n} خطی مساوات کا نظام ہے
\begin{align*}
\bM{A}\bM{x}=\bM{b}
\end{align*}
جہاں سمتیہ قطار \عددی{\bM{x}}   کے اجزاء نا معلوم متغیرات \عددی{x_1,\cdots,x_n} ہیں۔فرض کریں کہ ابتدائی تخمین \عددی{\bM{x}_{(0)}} کے لحاظ سے
 \عددی{\bM{x}_{(0)},\bM{x}_{(1)},\cdots} گاوس زائڈل اعادہ سے یک بعد دیگرے حاصل تخمینی نتائج کی ترتیب ہے۔ اگر یہ ترتیب نظام کے حل کو مرتکز ہو تب ہم کہتے ہیں کہ یہ ترکیب   \عددی{\bM{x}_{(0)}} کے لحاظ سے \ترچھا{مرتکز} ہے۔

ہم فرض کرتے ہیں کہ \عددی{j=1,\cdots,n} کے لئے \عددی{a_{jj}=1} ہے (نظام کی ایسی صورت حاصل کرنے کی خاطر ہم مساواتوں کو یوں ترتیب دیتے ہیں کہ تمام وتری جزو غیر صفر ہوں اور وتری جزو سے مطابقتی مساوات  تقسیم کرتے ہیں)۔ہم اب \عددی{\bM{A}=\bM{I}+\tilde{\bM{L}}+\tilde{\bM{U}}} لکھ سکتے ہیں جہاں \عددی{\tilde{\bM{U}}} اور \عددی{\tilde{\bM{L}}} بالترتیب بالائی تکونی قالب اور نچلا تکونی قالب ہیں جن کے مرکزی وتر کے اجزاء صفر ہیں جبکہ  \عددی{\bM{I}} اکائی قالب ہے جو \عددی{n} صف پر مشتمل ہے۔\عددی{\bM{A}} کی اس صورت کو \عددی{\bM{A}\bM{x}=\bM{b}} میں پر کرتے ہوئے \عددی{(\bM{I}+\tilde{\bM{L}}+\tilde{\bM{U}})\bM{x}=\bM{b}} حاصل ہو گا۔روایتی طور پر \عددی{\tilde{\bM{U}}=\bM{U}} اور \عددی{\tilde{\bM{L}}=\bM{L}} لکھا جاتا ہے۔یوں
\begin{align*}
(\bM{I}-\bM{L}-\bM{U})\bM{x}=\bM{b}\quad \implies \quad (\bM{I}-\bM{L})\bM{x}=\bM{b}+\bM{U}\bM{x}
\end{align*}
ہو گا جس سے کلیہ گاوس زائڈل
\begin{align}\label{مساوات_خطی_اعدادی_مثال_ت}
(\bM{I}-\bM{L})\bM{x}_{(m+1)}=\bM{b}+\bM{U}\bM{x}_{(m)}\quad \quad \quad (m=0,1,\cdots)
\end{align}
اخذ ہوتا ہے۔درحقیقت \عددی{\bM{U}} بالائی تکونی قالب ہے جس کے غیر صفر اجزاء ان مقامات کے مطابقتی ہیں جن کی تازہ ترین تخمینی قیمتیں ابھی حاصل نہیں کی گئی ہیں۔اس کے برعکس \عددی{\bM{L}} نچلا تکونی قالب ہے جس کے غیر صفر اجزاء ان مقامات کے مطابقتی ہیں جن کی تازہ ترین تخمینی قیمتیں \عددی{\bM{x}_{(m+1)}} ہم حاصل کر چکے ہیں۔مساوات \حوالہ{مساوات_خطی_اعدادی_مثال_ت} کو \عددی{\bM{x}_{(m+1)}} کے لئے حل کرتے ہوئے
\begin{align}\label{مساوات_خطی_اعدادی_مثال_ٹ}
\bM{x}_{(m+1)}=(\bM{I}-\bM{L})^{-1}\bM{b}+\bM{C}\bM{x}_{(m)},\quad \quad \bM{C}=(\bM{I}-\bM{L})^{-1}\bM{U}
\end{align}
حاصل ہو گا۔اعادہ گاوس زائڈل  کی ارتکاز ، قالب اعادہ \عددی{\bM{C}} کی امتیازی اقدار کی مشروط ہے۔

ہم \عددی{\bM{x}_{(m)}=[x_j^{(m)}]} لکھ کر اعادہ گاوسی زائڈل کو درج ذیل بیان کر سکتے ہیں۔\\

\noindent\makebox[\linewidth]{\rule{\textwidth}{0.4pt}}
\موٹا{الخوارزمی:اعادہ گاوس زائڈل}\\
نظام \عددی{\bM{A}\bM{x}=\bM{b}} جہاں \عددی{n\times n} قالب \عددی{\bM{A}=[a_{jk}]} میں \عددی{j=1,\cdots,n} کے لئے \عددی{a_{jj}\ne 0} ہے، دیا گیا ہے۔\\
منتخب کریں کوئی \عددی{\bM{x}_{(0)}}\\
حاصل کریں \عددی{v_{jk}=-\tfrac{a_{jk}}{a_{jj}}} جب \عددی{j\ne k} ہو؛ \عددی{j,k=1,\cdots,n}\\
حاصل کریں \عددی{\tilde{b}_j=\tfrac{b_j}{a_{jj}}}\\
\عددی{m} کے لئے \عددی{0} تا اختتام کریں:
\عددی{j=1,\cdots, n} کے لئے کریں\\
$x_j^{(m+1)}:=\sum\limits_{k=1}^{j-1}v_{jk}x_k^{(m+1)}+\sum\limits_{k=j+1}^{n}v_{jk}x_k^{(m)}+\tilde{b}_j$\\
اختتام کی تصدیق کریں۔\\
\noindent\makebox[\linewidth]{\rule{\textwidth}{0.4pt}}

یہاں اختتام کی تصدیق سے مراد ایسی صورت ہے جہاں مطلوبہ درستگی حاصل ہو جائے، یا قدموں کی درکار تعداد پوری ہو جائے یا مزید لاگو شرائط مطمئن ہوں۔

\جزوحصہء{اعادہ یعقوبی}
اعادہ گاوس زائڈل \اصطلاح{مسلسل اصلاح}\فرہنگ{اصلاح!مسلسل}\فرہنگ{اعادہ!مسلسل اصلاح} کی ترکیب ہے جس میں تازہ ترین نئی تخمینی قیمتیں استعمال کی جاتی ہیں۔اگر  نئی قیمتوں کو صرف اس وقت حساب کے لئے استعمال کیا جائے جب تمام متغیرات کی نئی قیمتیں حاصل کر لی جائیں تب \اصطلاح{بیک وقت اصلاح}\فرہنگ{اصلاح!بیک وقت}\فرہنگ{اعادہ!بیک وقت اصلاح} کی ترکیب حاصل ہو گی۔\اصطلاح{اعادہ یعقوبی}\فرہنگ{اعادہ!یعقوبی} اس قسم کی ایک ترکیب ہے۔ یہ ترکیب اعادہ گاوس زائڈل کی طرح ہے پس اس میں نئی قیمتیں صرف اس صورت پر کی جاتی ہیں جب تمام متغیرات کی قیمتیں حاصل کر لی جائیں۔یوں \عددی{\bM{A}\bM{x}=\bM{b}} کو \عددی{\bM{x}=\bM{b}+(\bM{I}-\bM{A})\bM{x}} صورت میں لکھ کر اعادہ یعقوبی کی قالبی اظہار
\begin{align}\label{مساوات_خطی_اعدادی_مثال_ث}
\bM{x}_{(m+1)}=\bM{b}+(\bM{I}-\bM{A})\bM{x}_{(m)}
\end{align}
ہو گی۔ یہ ترکیب زیادہ تر نظریاتی اہمیت رکھتی ہے۔یہ \عددی{\bM{x}_{(0)}} کی ہر منتخب قیمت کے لئے صرف اور صرف اس صورت مرتکز ہو گی جب  \عددی{\bM{I}-\bM{A}}  کا رداس طیف \عددی{1} سے کم ہو؛ یہاں بھی \عددی{j=1,\cdots,n} کے لئے \عددی{a_{jj}=1} فرض کیا جاتا ہے۔

نظام \عددی{\bM{A}\bM{x}=\bM{b}} کی صورت میں ہم  \اصطلاح{بقیہ}\فرہنگ{بقیہ}\حاشیہب{residual}\فرہنگ{residual} \عددی{\bM{r}} متعارف کر سکتے ہیں جس کی تعریف
\begin{align*}
\bM{r}=\bM{A}\bM{x}-\bM{b}
\end{align*}
ہے۔ظاہر ہے کہ \عددی{\bM{r}=\bM{0}} صرف اور صرف اس صورت ہو گا جب \عددی{\bM{x}} نظام کا حل ہو۔یوں تخمینی حل کی صورت میں \عددی{\bM{r}\ne \bM{0}} ہو گا۔اعادہ گاوس زائڈل میں ہم ہر منزل پر تخمینی حل کے ایک جزو میں ترمیم یا نرمی  کرتے ہوئے \عددی{\bM{r}} کے ایک جزو  گھٹا کر صفر کرتے ہیں۔یوں اعادہ گاوس زائڈل ان تراکیب میں سے ایک ہے جنہیں \اصطلاح{تراکیب آرام}\فرہنگ{ترکیب!آرام}\فرہنگ{آرام!ترکیب}\حاشیہب{relaxation methods}\فرہنگ{method!relaxation}\فرہنگ{relaxation methods} کہتے ہیں۔ 

غیر نادر چکور قالب کا معکوس بھی اعادہ کے ذریعہ حاصل کیا جا سکتا ہے۔آئیں اس ترکیب کو دیکھیں۔عدد \عددی{a} کے معکوس \عددی{x} سے مراد ایسا عددی ہے جو \عددی{ax=1} کو مطمئن کرتا ہو۔ترکیب نیوٹن کو تفاعل \عددی{f(x)=x^{-1}-a} پر لاگو کرتے ہوئے تقسیم کے عمل کے بغیر \عددی{x} حاصل کیا جا سکتا ہے۔چونکہ \عددی{f'(x)=-\tfrac{1}{x^2}} ہے لہٰذا اعادہ نیوٹن
\begin{align*}
x_{m+1}=x_m-(x_m^{-1}-a)(-x_m^2)=x_m(2-ax_m)
\end{align*}
ہو گا۔اس کو دیکھ کر ہم \عددی{\bM{A}} کے معکوس \عددی{\bM{X}=\bM{A}^{-1}} کے لئے درج ذیل کلیہ لکھتے ہیں۔
\begin{align}\label{مساوات_خطی_اعدادی_مثال_ج}
\bM{X}_{(m+1)}=\bM{X}_{(m)}(2\bM{I}-\bM{A}\bM{X}_{(m)})
\end{align}
یہ عمل صرف اور صرف اس صورت مرتکز ہو گا (یعنی \عددی{m\to \infty} کرنے سے \عددی{\bM{A}^{-1}} دے گا) جب \عددی{\bM{X}_{(0)}} کی ایسی قیمت منتخب کی جائے کہ \عددی{\bM{I}-\bM{A}\bM{X}_{(0)}} کے ہر امتیازی قدر  کی حتمی قیمت \عددی{1} سے کم ہو۔یہ ترکیب اس صورت موزوں ثابت ہوتی ہے جب پیش آنے والے ضرب آسان ہوں (مثلاً جب \عددی{\bM{A}} میں بہت سارے صفر ہوں)۔عملاً \عددی{\bM{X}_{(0)}}  کی موزوں قیمت منتخب کرنا اگر نا ممکن نہیں تو مشکل ضرور  ثابت ہوتا ہے۔اسی لئے کسی دوسرے ترکیب سے حاصل معکوس کو اس ترکیب سے صرف زیادہ درست بنایا جاتا ہے۔ 

%=========================
\حصہء{سوالات}
%=================
سوال \حوالہ{سوال_خطی_اعدادی_گاوس_زائڈل_الف} تا سوال \حوالہ{سوال_خطی_اعدادی_گاوس_زائڈل_ب} کو اعادہ گاوس زائڈل سے حل کریں۔ابتدائی قیمتیں \عددی{1,1,1} لیں۔ تین قدم تک چلیں۔

%===============
\ابتدا{سوال}\شناخت{سوال_خطی_اعدادی_گاوس_زائڈل_الف}\quad
\begin{alignat*}{4}
10x&+y&+z&=6\\
x&+10y&+z&=6\\
x&+y&+10z&=6
\end{alignat*}
جواب:\quad
درست حل \عددی{0.5,0.5,0.5} ہے۔
\انتہا{سوال}
%======================
\ابتدا{سوال}
\begin{alignat*}{3}
4x&+y&{}&=-8\\
{}&4y&+z&=2\\
&{}&{}2z&=2
\end{alignat*}
\انتہا{سوال}
%====================
\ابتدا{سوال}\quad
\begin{align*}
10x-y-z&=13\\
x+10y+z&=36\\
-x-y+10z&=35
\end{align*}
جواب:\quad
درست حل \عددی{2,3,4} ہے۔
\انتہا{سوال}
%======================
\ابتدا{سوال}\شناخت{سوال_خطی_اعدادی_گاوس_زائڈل_ب}\quad
\begin{align*}
4x+2y+z&=14\\
x+5y-z&=10\\
x+y+8z&=20
\end{align*}
\انتہا{سوال}
%===================
\ابتدا{سوال}\quad
(الف) \عددی{0,0,0} اور (ب) \عددی{10,10,10}  سے ابتدا کرتے ہوئے سوال \حوالہ{سوال_خطی_اعدادی_گاوس_زائڈل_الف} کے نظام کو اعادہ گاوس زائڈل سے حل کریں۔تین قدم تک چلیں۔ 
\انتہا{سوال}
%=======================
\ابتدا{سوال}\quad
\عددی{1,1,1} سے ابتدا کرتے ہوئے سوال \حوالہ{سوال_خطی_اعدادی_گاوس_زائڈل_الف} کے نظام  کو تین قدم تک اعادہ گاوس زائڈل اور اعادہ یعقوبی سے حل کریں۔ نتائج کا آپس میں موازنہ کریں۔
\انتہا{سوال}
%===========================
\ابتدا{سوال}\quad
مساوات \حوالہ{مساوات_خطی_اعدادی_مثال_الف} میں دی گئی نظام کا حل کتاب میں دیا گیا ہے۔اس حل کی تمام قدموں  کی تصدیق کریں۔اس نظام کو گاوسی اسقاط سے حل کریں۔
\انتہا{سوال}
%==========================
\ابتدا{سوال}\quad
کتاب میں مساوات \حوالہ{مساوات_خطی_اعدادی_مثال_الف} کے اعادہ  گاوس زائڈل   کے مزید دو قدم چلیں۔ 
\انتہا{سوال}
%=======================
\ابتدا{سوال}\quad
مساوات \حوالہ{مساوات_خطی_اعدادی_مثال_الف}  کے نظام کے لئے مساوات  \حوالہ{مساوات_خطی_اعدادی_مثال_ٹ} کی مدد سے \عددی{\bM{C}} تلاش کریں۔\\
جواب:
\begin{align*}
\begin{bmatrix}
0&0.25&0.25&0\\
0&0.0625&0.0625&0.25\\
0&0.0625&0.0625&0.25\\
0&0.03125&0.03125&0.125
\end{bmatrix}
\end{align*}
\انتہا{سوال}
%=========================
\ابتدا{سوال}\quad
\عددی{w_0=100}، \عددی{x_0=100}، \عددی{y_0=100}، \عددی{z_0=100} سے ابتدا کرتے ہوئے اعادہ یعقوبی سے مساوات \حوالہ{مساوات_خطی_اعدادی_مثال_الف} کے نظام  کا حل دو قدم تک حاصل کریں۔کتاب میں دیے گئے حل کے ساتھ موازنہ کریں۔
\انتہا{سوال}
%======================
\ابتدا{سوال}\quad
\عددی{0,0,0} سے ابتدا کرتے ہوئے دکھائیں کہ درج ذیل نظام کے لئے اعادہ گاوس زائڈل مرتکز ہے جبکہ اعادہ یعقوبی منفرج ہے۔
\begin{align*}
2x+y+z&=4\\
x+2y+z&=4\\
x+y+2z&=4
\end{align*}
جواب:\quad
اعادہ یعقوبی \عددی{0,0,0} کے بعد \عددی{2,2,2} اور اس کے بعد \عددی{0,0,0}، \نقطے دیتا ہے۔اعادہ گاوس زائڈل کی اعادہ قالب \عددی{\bM{C}} کے تمام جزو کی حتمی قیمت \عددی{1} سے کم ہے لہٰذا یہ اعادہ مرتکز ہو گا۔یہاں \عددی{\bM{C}} درج ذیل ہے۔
\begin{align*}
\bM{C}=
\begin{bmatrix}
0&-0.5&-0.5\\
0&0.25&-0.25\\
0&0.125&0.375
\end{bmatrix}
\end{align*}
\انتہا{سوال}
%=====================
\ابتدا{سوال}\quad
عین ممکن ہے کہ ہم سوچیں کہ اعادہ یعقوبی سے  اعادہ گاوس زائڈل بہتر ہے۔حقیقت میں ان اعادہ کا آپس میں موازنہ کرنا ممکن نہیں ہے۔اس حیران کن حقیقت کو دیکھنے کی خاطر درج ذیل نظام کو دونوں اعادہ سے حل کریں۔آپ دیکھیں گے کہ اعادہ یعقوبی مرتکز ہو گا جبکہ اعادہ گاوس زائڈل منفرج ہو گا۔(اشارہ۔ امتیازی اقدار کا سہارا لیں)
\begin{alignat*}{3}
x&{}&+z&=2\\
-x&+y&{}&=0\\
x&+2y&-3z&=0
\end{alignat*}  
\انتہا{سوال}
%======================
\ابتدا{سوال}\quad
قالب \عددی{\bM{A}} کے  تخمینی معکوس \عددی{\bM{X}_{(0)}} پر غور کریں جہاں
\begin{align*}
\bM{X}_{(0)}&=\begin{bmatrix} 0.5&-0.1&0.4\\0&0.2&0\\-0.4&0.3&-1.5 \end{bmatrix} & \bM{A}&=
\begin{bmatrix}
3&0&1\\0&5&0\\-1&1&-1
\end{bmatrix}
\end{align*}
ہیں۔مساوت \حوالہ{مساوات_خطی_اعدادی_مثال_ج} کی مدد سے \عددی{\bM{X}_{(1)}} حاصل کریں۔\عددی{\bM{A}^{-1}} تلاش کرتے ہوئے دکھائیں کہ \عددی{\bM{X}_{(0)}} کا ہر جزو \عددی{\bM{A}^{-1}} کے مطابقتی جزو سے زیادہ سے زیادہ \عددی{0.1} انحراف کرتا ہے جبکہ \عددی{\M{X}_{(1)}} کا مطابقتی جزو  \عددی{0.03} انحراف کرتا ہے۔\\
جواب:
\begin{align*}
\bM{X}_{(1)}&=
\begin{bmatrix} 0.49&-0.1&0.51\\0&0.2&0\\-0.51&0.3&-1.47 \end{bmatrix} &
\bM{A}^{-1}&=\begin{bmatrix} 0.5&-0.1&0.5\\ 0&0.2&0\\ -0.5&0.3&-1.5 \end{bmatrix}
\end{align*}
\انتہا{سوال}
%==========================
\ابتدا{سوال}\quad
درج ذیل \عددی{\bM{X}_{(0)}} اور \عددی{\bM{A}}  کے لئے مساوت \حوالہ{مساوات_خطی_اعدادی_مثال_ج} کے ارتکاز کی تصدیق کرتے ہوئے  دو قدم چل کر درست حل کے ساتھ موازنہ کریں۔
\begin{align*}
\bM{X}_{(0)}&=\begin{bmatrix} 2.9& -0.9\\-4.9&1.9 \end{bmatrix} &
\bM{A}&=\begin{bmatrix} 2&1\\ 5&3 \end{bmatrix}
\end{align*}
\انتہا{سوال}
%======================
\ابتدا{سوال}\quad
\عددی{\bM{X}_{(m)}=\bM{A}^{-1}} سے مساوت \حوالہ{مساوات_خطی_اعدادی_مثال_ج} کے ذریعہ \عددی{\bM{X}_{(m+1)}=\bM{A}^{-1}} حاصل کریں۔
\انتہا{سوال}
%===============
\ابتدا{سوال}\quad
دکھائیں کہ مساوات \حوالہ{مساوات_خطی_اعدادی_مثال_الف} میں \عددی{w} اور \عددی{x} آپس میں بدلنے اور \عددی{y} اور \عددی{z} کو آپس میں بدلنے سے نظام میں کوئی تبدیلی پیدا نہیں ہوتی ہے۔اس حقیقت کو استعمال کرتے ہوئے اس نظام کو گھٹا کر دو نا معلوم متغیرات کی دو مساوات کا نظام حاصل کریں۔
\انتہا{سوال}
%=====================

\حصہ{خطی مساوات کا نظام:بد خوئی}
