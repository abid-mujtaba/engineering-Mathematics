\باب{خطی الجبرا۔سمتیات}
خطی الجبرا وسیع مضمون ہے جس میں قالب اور سمتیات، مقطع قالب، خطی مساوات کے نظام، سمتی فضا اور  خطی تبادلہ، آئگنی قیمت مسائل، اور دیگر موضوعات شامل ہیں۔اس کا استعمال انجینئری، طبیعیات، جیومیٹری، کمپیوٹر سائنس، معاشیات اور دیگر میدانوں میں پایا جاتا ہے۔

متعدد اعداد و شمار یا متعدد تفاعل کو مربوط طریقے سے \اصطلاح{قالب}\فرہنگ{قالب}\حاشیہب{matrices}\فرہنگ{matrix} اور \اصطلاح{سمتیات}\فرہنگ{سمتیہ}\حاشیہب{vectors}\فرہنگ{vector} کی مدد سے ظاہر کیا جاتا ہے۔ قالب اور سمتیات ہی خطی الجبرا کی زبان ہیں۔
%===================

\حصہ{قالب اور سمتیات۔مجموعہ اور غیر سمتی ضرب}
مستطیلی ترتیب وار فہرست کو \اصطلاح{قالب} کہتے ہیں۔درج ذیل قالب کی مثال ہیں۔قالب میں درج اعداد یا تفاعل کو قالب کے \اصطلاح{اندراجات} یا قالب کے \اصطلاح{ارکان}\فرہنگ{ارکان}\حاشیہب{elements}\فرہنگ{elements} کہتے ہیں۔ 
\begin{gather}
\begin{aligned}\label{مساوات_قالب_عمومی_قالب_الف}
\begin{bmatrix}
0.1& -2 & 1.2\\
-6 & 0 & 23
\end{bmatrix}, \quad
\begin{bmatrix}
a_{11}& a_{12} & a_{13}\\
a_{21} & a_{22} & a_{23}\\
a_{31} & a_{32} & a_{33}
\end{bmatrix}, \quad
\begin{bmatrix}
\ln x& -e^x\\
e^{3x}& 3.2x^2
\end{bmatrix},\\
\begin{bmatrix}
a_{1} & a_{2} & a_{3}
\end{bmatrix},\quad 
\begin{bmatrix}
3.22\\
-\tfrac{4}{5}
\end{bmatrix}
\end{aligned}
\end{gather}
بالائی بائیں ہاتھ قالب کے ارکان \عددی{0.1}، \عددی{-2}، \عددی{1.2}، \عددی{-6}، \عددی{0} اور \عددی{23} ہیں۔اس قالب کے دو \اصطلاح{صف}\فرہنگ{صف}\حاشیہب{rows}\فرہنگ{rows} اور تین \اصطلاح{قطار}\فرہنگ{قطار}\حاشیہب{columns}\فرہنگ{columns} ہیں۔افقی اندراجات کی لکیر کو صف اور عمودی اندراجات کی لکیر کو قطار کہتے ہیں۔بالائی درمیانی قالب میں \عددی{3} صف اور \عددی{3} قطار پائے جاتے ہیں۔ایسا قالب جس میں صفوں کی تعداد، قطاروں کی تعداد کے برابر ہو \اصطلاح{مربع قالب}\فرہنگ{قالب!مربع}\حاشیہب{square matrix}\فرہنگ{matrix!square}  کہلاتا ہے۔یوں بالائی دائیں ہاتھ قالب بھی مربع قالب ہے۔بالائی درمیانی قالب میں ارکان کو \عددی{a_{mn}} سے ظاہر کیا گیا ہے جہاں دو عدد اشاریہ \عددی{m} اور \عددی{n} بالترتیب اس صف اور قطار کو ظاہر کرتے ہیں جہاں یہ رکن پایا جاتا ہو۔قالب میں اندراجات کے مقام کی وضاحت اسی معیاری ترکیب سے کی جاتی ہے۔ یوں \عددی{a_{23}} رکن دوسرے صف اور تیسرے قطار میں پایا جاتا ہے۔

ایسا قالب جو صرف ایک عدد صف یا صرف ایک عدد قطار پر مشتمل ہو، \اصطلاح{سمتیہ}\فرہنگ{سمتیہ}\حاشیہب{vector}\فرہنگ{vector} کہلاتا ہے۔یوں نچلے دائیں ہاتھ دو ارکان پر مشتمل \اصطلاح{سمتیہ قطار}\فرہنگ{سمتیہ!قطار}\فرہنگ{قطار!سمتیہ}\حاشیہب{column vector}\فرہنگ{column!vector}\فرہنگ{vector!column} پایا جاتا ہے جبکہ نچلے بائیں ہاتھ \اصطلاح{سمتیہ صف}\فرہنگ{سمتیہ!صف}\فرہنگ{صف!سمتیہ}\حاشیہب{row vector}\فرہنگ{row!vector}\فرہنگ{vector!row} پایا جاتا ہے۔چونکہ سمتیہ قطار میں کوئی صف نہیں پایا جاتا لہٰذا اس میں ارکان کے مقام کو صرف ایک عدد اشاریہ سے ظاہر کیا جاتا ہے۔اسی طرح سمتیہ صف میں بھی ارکان کا مقام صرف ایک عدد اشاریہ سے ظاہر کیا جاتا ہے۔یوں سمتیہ قطار میں \عددی{a_1=3.22} اور \عددی{a_2=-\tfrac{4}{5}} ہیں۔

عملی استعمال میں مواد کے ذخیرہ اور اس پر عمل کرنے میں قالب کار آمد ثابت ہوتے ہیں۔درج ذیل مثال دیکھیں
%==============
\ابتدا{مثال}\quad خطی نظام\\
درج ذیل \اصطلاح{خطی} نظام میں \عددی{x_1}، \عددی{x_2} اور \عددی{x_3} نا معلوم متغیرات ہیں۔ 
\begin{align*}
2x_1+3x_2+2x_3&=0\\
3x_1-2x_2+4x_3&=15\\
5x_1\phantom{+3x_2}+3x_3&=11
\end{align*}
آئیں درج بالا نظام میں \عددی{x_1}، \عددی{x_2} اور \عددی{x_3} کے عددی سروں سے  \اصطلاح{عددی سر قالب}\فرہنگ{عددی سر قالب}\فرہنگ{قالب!عددی سر}\حاشیہب{coefficient matrix}\فرہنگ{coefficient!matrix}\فرہنگ{matrix!coefficient} \عددی{\bM{A}} لکھیں۔ \عددی{\bM{A}} قالب میں ہر رکن کا مقام عین خطی مساوات کے مطابق ہو گا۔
\begin{align*}
\bM{A}=
\begin{bmatrix*}[r]
2&3&2\\
3&-2&3\\
5&0&3
\end{bmatrix*}
\end{align*} 
چونکہ تیسری مساوات میں \عددی{x_2} نہیں پایا جاتا لہٰذا اس کا عددی سر صفر کے برابر ہو گا اور یوں \عددی{\bM{A}} میں \عددی{a_{32}=0} درج کیا گیا ہے۔عددی سر قالب \عددی{\bM{A}} میں مساوات کے دائیں ہاتھ کی معلومات کا اضافہ کرنے سے \اصطلاح{افزودہ قالب}\فرہنگ{افزودہ قالب}\فرہنگ{قالب!افزودہ}\حاشیہب{augmented matrix}\فرہنگ{augmented matrix}\فرہنگ{matrix!augmented} \عددی{\tilde{\bM{A}}} ملتا ہے۔
\begin{align*}
\tilde{\bM{A}}=
\begin{bmatrix*}[r]
2&3&2& 0\\
3&-2&3& 15\\
5&0&3&11
\end{bmatrix*}
\end{align*}
چونکہ افزودہ قالب \عددی{\tilde{\bM{A}}} سے تینوں مساوات لکھے جا سکتے ہیں لہٰذا دیے گئے خطی نظام کو \عددی{\tilde{\bM{A}}} مکمل طور ظاہر کرتا ہے۔یوں ہم \عددی{\tilde{\bM{A}}} کو حل کرتے ہوئے نا معلوم متغیرات \عددی{x_1}، \عددی{x_2} اور \عددی{x_3} حاصل کر سکتے ہیں۔ایسا کرنا جلد سمجھایا جائے گا۔فی الحال تسلی کر لیں کہ اس نظام کا حل \عددی{x_1=1}، \عددی{x_2=-2} اور \عددی{x_3=2} ہے۔

نا معلوم متغیرات کو \عددی{x_1}، \عددی{x_2} اور \عددی{x_3} سے ظاہر کرنے کی بجائے دیگر علامتوں سے ظاہر کیا جا سکتا ہے مثلاً \عددی{x}، \عددی{y} اور \عددی{z}۔
\انتہا{مثال}
%======================= 
\ابتدا{مثال}\quad فروخت کھاتا\\

\begin{equation*}
\begingroup % keep the change local
\setlength\arraycolsep{8pt}
\begin{matrix}
 \bM{A}
 =
 \begin{bmatrix}
 \bovermat{جمع}{32} & \covermat{جمعرات}{23} & \bovermat{بدھ}{13}& \bovermat{منگل}{18}& \bovermat{پیر}{11}&  \bovermat{اتوار}{19}& \bovermat{ہفتہ}{20} \\
10& 12 & 14& 5 & 0 & 17 & 25\\
29& 16 & 32& 18 & 9 & 14 & 17
  \end{bmatrix}
  \begin{aligned}
  \begin{matrix}
  \text{ الف}  \\
   \text{ب} \\
   \text{پ}  \\
  \end{matrix}
 \end{aligned}
 \end{matrix}
\endgroup
 \end{equation*}
ایک دکان کی تین اشیاء کی ہفتہ وار فروخت درج بالا قالب میں دی گئی ہے۔ہر ہفتے کی فروخت کو اسی طرح قالبوں میں لکھا جا سکتا ہے۔مہینے کے آخر میں تمام قالبوں کے مطابقتی ارکان کا مجموعہ لینے سے ہر دن، تینوں اشیاء کی کل فروخت کی فہرست حاصل ہو گی۔ 
\انتہا{مثال}
%============================

\جزوحصہء{عمومی تصورات اور علامت نویسی}
آئیں اب تک پیش کیے گئے تصورات کو با ضابطہ دستوری صورت دیں۔ ہم موٹی لکھائی میں لاطینی حروف تہجی  کے بڑے حروف سے قالب کو ظاہر کریں گے مثلاً \عددی{\bM{A}}، \عددی{\bM{B}}، \عددی{\bM{C}}، \نقطے، اور یا اس کو  چکور قوسین میں عمومی رکن سے ظاہر کریں گے مثلاً \عددی{\bM{A}=[a_{jk}]} وغیرہ۔ایسا قالب جس میں \عددی{m} صف اور \عددی{n} قطار ہوں، \عددی{m\times n} (اس کو \عددی{m} ضرب \عددی{n} پڑھیں) قالب کہلاتا ہے (پہلے صف اور بعد میں قطار آئے گا) اور \عددی{m\times n} قالب کی \اصطلاح{جسامت}\فرہنگ{جسامت}\حاشیہب{size}\فرہنگ{size} کہلاتی ہے۔یوں \عددی{m\times n} قالب درج ذیل صورت کا ہو گا۔
\begin{align}\label{مساوات_قالب_عمومی_قالب_ب}
\bM{A}=[a_{jk}]=
\begin{bmatrix}
a_{11}& a_{12}& \cdots &a_{1n}\\
a_{21}& a_{22}& \cdots &a_{2n}\\
\vdots&&\\
a_{m1}& a_{m2}&\cdots&a_{mn}
\end{bmatrix}
\end{align}
مساوات \حوالہ{مساوات_قالب_عمومی_قالب_الف} میں بالائی بائیں قالب \عددی{2\times 3} جسامت کا ہے جبکہ نچلا بایاں قالب \عددی{1\times 3} جسامت کا ہے۔

مساوات \حوالہ{مساوات_قالب_عمومی_قالب_ب} میں ہر رکن کو دو عدد اشاریہ سے پہچانا جاتا ہے جہاں پہلا اشاریہ صف اور دوسرا اشاریہ قطار ہے۔یوں \عددی{a_{23}} دوسرے صف اور تیسرے قطار پر موجود اندراج ہے۔

ایسا قالب جس میں \عددی{m=n} ہو \عددی{n\times n} \اصطلاح{چکور} قالب کہلاتا ہے۔ چکور قالب کا وہ وتر جس پر \عددی{a_{11}}، \عددی{a_{22}}، \نقطے، \عددی{a_{nn}} پائے جاتے ہیں، قالب کا \اصطلاح{مرکزی وتر}\فرہنگ{مرکزی وتر}\حاشیہب{main diagonal}\فرہنگ{main diagonal} کہلاتا ہے۔ مساوات \حوالہ{مساوات_قالب_عمومی_قالب_الف} میں ایک چکور قالب کے مرکزی وتر کے ارکان \عددی{a_{11}}، \عددی{a_{22}} اور \عددی{a_{33}} ہیں جبکہ دوسرے چکور قالب کے مرکزی وتر کے ارکان \عددی{\ln x} اور \عددی{3.2x^2} ہیں۔جیسا ہم دیکھیں گے، چکور قالب نہایت اہم ہیں۔

ایسا قالب جس میں \عددی{m \ne n} ہو \عددی{m\times n} \اصطلاح{مستطیل}\فرہنگ{مستطیل}\حاشیہب{rectangular matrix}\فرہنگ{rectangular matrix} قالب کہلاتا ہے۔مستطیل قالب کی ایک مخصوص قسم چکور قالب ہے۔
%====================

\جزوحصہء{سمتیات}
صرف ایک صف یا ایک قطار پر مبنی قالب کو \اصطلاح{سمتیہ} کہتے ہیں۔سمتیہ کے اندراج کو سمتیہ کے \اصطلاح{اجزاء}\فرہنگ{اجزاء}\حاشیہب{components}\فرہنگ{components} کہتے ہیں۔ ہم موٹی لکھائی میں لاطینی حروف تہجی  کے چھوٹے حروف سے سمتیہ کو ظاہر کریں گے مثلاً \عددی{\bM{a}}، \عددی{\bM{b}}، \عددی{\bM{c}}، \نقطے، اور یا اس کو   چکور قوسین میں عمومی رکن سے ظاہر کریں گے مثلاً \عددی{\bM{a}=[a_{j}]} وغیرہ۔ \اصطلاح{سمتیہ صف} کی مثالیں درج ذیل ہیں۔
\begin{align*}
\bM{a}=
\begin{bmatrix}
a_1& a_2 & \cdots & a_n
\end{bmatrix}, \quad 
\bM{b}=
\begin{bmatrix}
2& -3&0& 4.2& \tfrac{3}{5} 
\end{bmatrix}
\end{align*}
اسی طرح \اصطلاح{سمتیہ قطار} کی مثالیں درج ذیل ہیں۔
\begin{align*}
\bM{c}=
\begin{bmatrix}
c_1\\
c_2\\
\vdots\\
c_m
\end{bmatrix}, \quad \quad
\bM{d}=
\begin{bmatrix*}[r]
2\\
-1\\
2.3
\end{bmatrix*}
\end{align*}
%================

\جزوحصہء{مجموعہ اور غیر سمتی ضرب}
آئیں پہلے مساوات کا تصور جانتے ہیں۔

\ابتدا{تعریف}
دو قالب \عددی{\bM{A}} اور \عددی{\bM{B}} اس صورت مساوی ہوں گے جب دونوں قالب کی جسامت برابر ہو اور ان کے نظیری ارکان آپس میں برابر ہوں یعنی 
\عددی{a_{11}=b_{11}}، \عددی{a_{12}=b_{12}}، \نقطے ہوں۔غیر مساوی قالب \اصطلاح{مختلف}\فرہنگ{مختلف}\حاشیہب{different}\فرہنگ{different} کہلاتے ہیں۔یوں مختلف جسامت کے قالب ہر صورت مختلف ہوں گے۔مساوات کا تعلق \عددی{\bM{A}=\bM{B}} لکھا جاتا ہے۔
\انتہا{تعریف}
%=======================

\ابتدا{مثال}\quad قالبوں کی مساوات\\
اگر درج ذیل قالب مساوی ہوں
\begin{align*}
\bM{A}=
\begin{bmatrix}
a_{11}& a_{12}\\
a_{21}&a_{22}
\end{bmatrix}\quad \text{اور} \quad
\bM{B}=
\begin{bmatrix}
2&-3\\
0&3.2
\end{bmatrix}
\end{align*}
تب \عددی{a_{11}=2}، \عددی{a_{12}=-3}، \عددی{a_{21}=0} اور \عددی{a_{22}=3.2} ہوں گے اور ہم \عددی{\bM{A}=\bM{B}} لکھ سکتے ہیں۔درج ذیل تمام قالب آپس میں \اصطلاح{مختلف} ہیں۔
\begin{align*}
\begin{bmatrix}
2&7\\
5&1
\end{bmatrix}\quad 
\begin{bmatrix}
5&1\\
2&7
\end{bmatrix}\quad
\begin{bmatrix}
2&7\\
1&5
\end{bmatrix}\quad 
\begin{bmatrix}
2\\
1
\end{bmatrix}
\end{align*}
\انتہا{مثال}
%=============================
\ابتدا{تعریف}\quad قالبوں کا مجموعہ\\
دو یکساں جسامت کے قالب \عددی{\bM{A}=[a_{jk}]} اور \عددی{\bM{B}=[b_{jk}]} کا مجموعہ \عددی{\bM{A}+\bM{B}} لکھا جائے گا جس کے اندراجات \عددی{a_{jk}+b_{jk}} کو \عددی{\bM{A}} اور \عددی{\bM{B}} کے نظیری ارکان کے مجموعے سے حاصل کیا جائے گا۔دو مختلف جسامت کے قالبوں کا مجموعہ حاصل نہیں نا ممکن ہے۔
\انتہا{تعریف}
