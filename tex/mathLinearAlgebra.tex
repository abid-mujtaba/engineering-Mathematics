\باب{خطی الجبرا۔سمتیات}
خطی الجبرا وسیع مضمون ہے جس میں قالب اور سمتیات، مقطع قالب، خطی مساوات کے نظام، سمتی فضا اور  خطی تبادلہ، آئگنی قیمت مسائل، اور دیگر موضوعات شامل ہیں۔اس کا استعمال انجینئری، طبیعیات، جیومیٹری، کمپیوٹر سائنس، معاشیات اور دیگر میدانوں میں پایا جاتا ہے۔

متعدد اعداد و شمار یا متعدد تفاعل کو مربوط طریقے سے \اصطلاح{قالب}\فرہنگ{قالب}\حاشیہب{matrices}\فرہنگ{matrix} اور \اصطلاح{سمتیات}\فرہنگ{سمتیہ}\حاشیہب{vectors}\فرہنگ{vector} کی مدد سے ظاہر کیا جاتا ہے۔ قالب اور سمتیات ہی خطی الجبرا کی زبان ہیں۔
%===================

\حصہ{قالب اور سمتیات۔مجموعہ اور غیر سمتی ضرب}
مستطیلی ترتیب وار فہرست کو \اصطلاح{قالب} کہتے ہیں۔درج ذیل قالب کی مثال ہیں۔قالب میں درج اعداد یا تفاعل کو قالب کے \اصطلاح{اندراجات} یا قالب کے \اصطلاح{ارکان}\فرہنگ{ارکان}\حاشیہب{elements}\فرہنگ{elements} کہتے ہیں۔ 
\begin{gather}
\begin{aligned}\label{مساوات_قالب_عمومی_قالب_الف}
\begin{bmatrix}
0.1& -2 & 1.2\\
-6 & 0 & 23
\end{bmatrix}, \quad
\begin{bmatrix}
a_{11}& a_{12} & a_{13}\\
a_{21} & a_{22} & a_{23}\\
a_{31} & a_{32} & a_{33}
\end{bmatrix}, \quad
\begin{bmatrix}
\ln x& -e^x\\
e^{3x}& 3.2x^2
\end{bmatrix},\\
\begin{bmatrix}
a_{1} & a_{2} & a_{3}
\end{bmatrix},\quad 
\begin{bmatrix}
3.22\\
-\tfrac{4}{5}
\end{bmatrix}
\end{aligned}
\end{gather}
بالائی بائیں ہاتھ قالب کے ارکان \عددی{0.1}، \عددی{-2}، \عددی{1.2}، \عددی{-6}، \عددی{0} اور \عددی{23} ہیں۔اس قالب کے دو \اصطلاح{صف}\فرہنگ{صف}\حاشیہب{rows}\فرہنگ{rows} اور تین \اصطلاح{قطار}\فرہنگ{قطار}\حاشیہب{columns}\فرہنگ{columns} ہیں۔افقی اندراجات کی لکیر کو صف اور عمودی اندراجات کی لکیر کو قطار کہتے ہیں۔بالائی درمیانی قالب میں \عددی{3} صف اور \عددی{3} قطار پائے جاتے ہیں۔ایسا قالب جس میں صفوں کی تعداد، قطاروں کی تعداد کے برابر ہو \اصطلاح{مربع قالب}\فرہنگ{قالب!مربع}\حاشیہب{square matrix}\فرہنگ{matrix!square}  کہلاتا ہے۔یوں بالائی دائیں ہاتھ قالب بھی مربع قالب ہے۔بالائی درمیانی قالب میں ارکان کو \عددی{a_{mn}} سے ظاہر کیا گیا ہے جہاں دو عدد اشاریہ \عددی{m} اور \عددی{n} بالترتیب اس صف اور قطار کو ظاہر کرتے ہیں جہاں یہ رکن پایا جاتا ہو۔قالب میں اندراجات کے مقام کی وضاحت اسی معیاری ترکیب سے کی جاتی ہے۔ یوں \عددی{a_{23}} رکن دوسرے صف اور تیسرے قطار میں پایا جاتا ہے۔

ایسا قالب جو صرف ایک عدد صف یا صرف ایک عدد قطار پر مشتمل ہو، \اصطلاح{سمتیہ}\فرہنگ{سمتیہ}\حاشیہب{vector}\فرہنگ{vector} کہلاتا ہے۔یوں نچلے دائیں ہاتھ دو ارکان پر مشتمل \اصطلاح{سمتیہ قطار}\فرہنگ{سمتیہ!قطار}\فرہنگ{قطار!سمتیہ}\حاشیہب{column vector}\فرہنگ{column!vector}\فرہنگ{vector!column} پایا جاتا ہے جبکہ نچلے بائیں ہاتھ \اصطلاح{سمتیہ صف}\فرہنگ{سمتیہ!صف}\فرہنگ{صف!سمتیہ}\حاشیہب{row vector}\فرہنگ{row!vector}\فرہنگ{vector!row} پایا جاتا ہے۔چونکہ سمتیہ قطار میں کوئی صف نہیں پایا جاتا لہٰذا اس میں ارکان کے مقام کو صرف ایک عدد اشاریہ سے ظاہر کیا جاتا ہے۔اسی طرح سمتیہ صف میں بھی ارکان کا مقام صرف ایک عدد اشاریہ سے ظاہر کیا جاتا ہے۔یوں سمتیہ قطار میں \عددی{a_1=3.22} اور \عددی{a_2=-\tfrac{4}{5}} ہیں۔

عملی استعمال میں مواد کے ذخیرہ اور اس پر عمل کرنے میں قالب کار آمد ثابت ہوتے ہیں۔درج ذیل مثال دیکھیں
%==============
\ابتدا{مثال}\quad خطی نظام\\
درج ذیل \اصطلاح{خطی} نظام میں \عددی{x_1}، \عددی{x_2} اور \عددی{x_3} نا معلوم متغیرات ہیں۔ 
\begin{align*}
2x_1+3x_2+2x_3&=0\\
3x_1-2x_2+4x_3&=15\\
5x_1\phantom{+3x_2}+3x_3&=11
\end{align*}
آئیں درج بالا نظام میں \عددی{x_1}، \عددی{x_2} اور \عددی{x_3} کے عددی سروں سے  \اصطلاح{عددی سر قالب}\فرہنگ{عددی سر قالب}\فرہنگ{قالب!عددی سر}\حاشیہب{coefficient matrix}\فرہنگ{coefficient!matrix}\فرہنگ{matrix!coefficient} \عددی{\bM{A}} لکھیں۔ \عددی{\bM{A}} قالب میں ہر رکن کا مقام عین خطی مساوات کے مطابق ہو گا۔
\begin{align*}
\bM{A}=
\begin{bmatrix*}[r]
2&3&2\\
3&-2&3\\
5&0&3
\end{bmatrix*}
\end{align*} 
چونکہ تیسری مساوات میں \عددی{x_2} نہیں پایا جاتا لہٰذا اس کا عددی سر صفر کے برابر ہو گا اور یوں \عددی{\bM{A}} میں \عددی{a_{32}=0} درج کیا گیا ہے۔عددی سر قالب \عددی{\bM{A}} میں مساوات کے دائیں ہاتھ کی معلومات کا اضافہ کرنے سے \اصطلاح{افزودہ قالب}\فرہنگ{افزودہ قالب}\فرہنگ{قالب!افزودہ}\حاشیہب{augmented matrix}\فرہنگ{augmented matrix}\فرہنگ{matrix!augmented} \عددی{\tilde{\bM{A}}} ملتا ہے۔
\begin{align*}
\tilde{\bM{A}}=
\begin{bmatrix*}[r]
2&3&2& 0\\
3&-2&3& 15\\
5&0&3&11
\end{bmatrix*}
\end{align*}
چونکہ افزودہ قالب \عددی{\tilde{\bM{A}}} سے تینوں مساوات لکھے جا سکتے ہیں لہٰذا دیے گئے خطی نظام کو \عددی{\tilde{\bM{A}}} مکمل طور ظاہر کرتا ہے۔یوں ہم \عددی{\tilde{\bM{A}}} کو حل کرتے ہوئے نا معلوم متغیرات \عددی{x_1}، \عددی{x_2} اور \عددی{x_3} حاصل کر سکتے ہیں۔ایسا کرنا جلد سمجھایا جائے گا۔فی الحال تسلی کر لیں کہ اس نظام کا حل \عددی{x_1=1}، \عددی{x_2=-2} اور \عددی{x_3=2} ہے۔

نا معلوم متغیرات کو \عددی{x_1}، \عددی{x_2} اور \عددی{x_3} سے ظاہر کرنے کی بجائے دیگر علامتوں سے ظاہر کیا جا سکتا ہے مثلاً \عددی{x}، \عددی{y} اور \عددی{z}۔
\انتہا{مثال}
%======================= 
\ابتدا{مثال}\شناخت{مثال_الجبرا_فروخت}\quad فروخت کھاتا\\

\begin{equation*}
\begingroup % keep the change local
\setlength\arraycolsep{8pt}
\begin{matrix}
 \bM{A}
 =
 \begin{bmatrix}
 \bovermat{جمع}{32} & \covermat{جمعرات}{23} & \bovermat{بدھ}{13}& \bovermat{منگل}{18}& \bovermat{پیر}{11}&  \bovermat{اتوار}{19}& \bovermat{ہفتہ}{20} \\
10& 12 & 14& 5 & 0 & 17 & 25\\
29& 16 & 32& 18 & 9 & 14 & 17
  \end{bmatrix}
  \begin{aligned}
  \begin{matrix}
  \text{ الف}  \\
   \text{ب} \\
   \text{پ}  \\
  \end{matrix}
 \end{aligned}
 \end{matrix}
\endgroup
 \end{equation*}
ایک دکان کی تین اشیاء کی ہفتہ وار فروخت درج بالا قالب میں دی گئی ہے۔ہر ہفتے کی فروخت کو اسی طرح قالبوں میں لکھا جا سکتا ہے۔مہینے کے آخر میں تمام قالبوں کے مطابقتی ارکان کا مجموعہ لینے سے ہر دن، تینوں اشیاء کی کل فروخت کی فہرست حاصل ہو گی۔ 
\انتہا{مثال}
%============================

\جزوحصہء{عمومی تصورات اور علامت نویسی}
آئیں اب تک پیش کیے گئے تصورات کو با ضابطہ دستوری صورت دیں۔ ہم موٹی لکھائی میں لاطینی حروف تہجی  کے بڑے حروف سے قالب کو ظاہر کریں گے مثلاً \عددی{\bM{A}}، \عددی{\bM{B}}، \عددی{\bM{C}}، \نقطے، اور یا اس کو  چکور قوسین میں عمومی رکن سے ظاہر کریں گے مثلاً \عددی{\bM{A}=[a_{jk}]} وغیرہ۔ایسا قالب جس میں \عددی{m} صف اور \عددی{n} قطار ہوں، \عددی{m\times n} (اس کو \عددی{m} ضرب \عددی{n} پڑھیں) قالب کہلاتا ہے (پہلے صف اور بعد میں قطار آئے گا) اور \عددی{m\times n} قالب کی \اصطلاح{جسامت}\فرہنگ{جسامت}\حاشیہب{size}\فرہنگ{size} کہلاتی ہے۔یوں \عددی{m\times n} قالب درج ذیل صورت کا ہو گا۔
\begin{align}\label{مساوات_قالب_عمومی_قالب_ب}
\bM{A}=[a_{jk}]=
\begin{bmatrix}
a_{11}& a_{12}& \cdots &a_{1n}\\
a_{21}& a_{22}& \cdots &a_{2n}\\
\vdots&&\\
a_{m1}& a_{m2}&\cdots&a_{mn}
\end{bmatrix}
\end{align}
مساوات \حوالہ{مساوات_قالب_عمومی_قالب_الف} میں بالائی بائیں قالب \عددی{2\times 3} جسامت کا ہے جبکہ نچلا بایاں قالب \عددی{1\times 3} جسامت کا ہے۔

مساوات \حوالہ{مساوات_قالب_عمومی_قالب_ب} میں ہر رکن کو دو عدد اشاریہ سے پہچانا جاتا ہے جہاں پہلا اشاریہ صف اور دوسرا اشاریہ قطار ہے۔یوں \عددی{a_{23}} دوسرے صف اور تیسرے قطار پر موجود اندراج ہے۔

ایسا قالب جس میں \عددی{m=n} ہو \عددی{n\times n} \اصطلاح{چکور} قالب کہلاتا ہے۔ چکور قالب کا وہ وتر جس پر \عددی{a_{11}}، \عددی{a_{22}}، \نقطے، \عددی{a_{nn}} پائے جاتے ہیں، قالب کا \اصطلاح{مرکزی وتر}\فرہنگ{مرکزی وتر}\حاشیہب{main diagonal}\فرہنگ{main diagonal} کہلاتا ہے۔ مساوات \حوالہ{مساوات_قالب_عمومی_قالب_الف} میں ایک چکور قالب کے مرکزی وتر کے ارکان \عددی{a_{11}}، \عددی{a_{22}} اور \عددی{a_{33}} ہیں جبکہ دوسرے چکور قالب کے مرکزی وتر کے ارکان \عددی{\ln x} اور \عددی{3.2x^2} ہیں۔جیسا ہم دیکھیں گے، چکور قالب نہایت اہم ہیں۔

ایسا قالب جس میں \عددی{m \ne n} ہو \عددی{m\times n} \اصطلاح{مستطیل}\فرہنگ{مستطیل}\حاشیہب{rectangular matrix}\فرہنگ{rectangular matrix} قالب کہلاتا ہے۔مستطیل قالب کی ایک مخصوص قسم چکور قالب ہے۔
%====================

\جزوحصہء{سمتیات}
صرف ایک صف یا ایک قطار پر مبنی قالب کو \اصطلاح{سمتیہ} کہتے ہیں۔سمتیہ کے اندراج کو سمتیہ کے \اصطلاح{اجزاء}\فرہنگ{اجزاء}\حاشیہب{components}\فرہنگ{components} کہتے ہیں۔ ہم موٹی لکھائی میں لاطینی حروف تہجی  کے چھوٹے حروف سے سمتیہ کو ظاہر کریں گے مثلاً \عددی{\bM{a}}، \عددی{\bM{b}}، \عددی{\bM{c}}، \نقطے، اور یا اس کو   چکور قوسین میں عمومی رکن سے ظاہر کریں گے مثلاً \عددی{\bM{a}=[a_{j}]} وغیرہ۔ \اصطلاح{سمتیہ صف} کی مثالیں درج ذیل ہیں۔
\begin{align*}
\bM{a}=
\begin{bmatrix}
a_1& a_2 & \cdots & a_n
\end{bmatrix}, \quad 
\bM{b}=
\begin{bmatrix}
2& -3&0& 4.2& \tfrac{3}{5} 
\end{bmatrix}
\end{align*}
اسی طرح \اصطلاح{سمتیہ قطار} کی مثالیں درج ذیل ہیں۔
\begin{align*}
\bM{c}=
\begin{bmatrix}
c_1\\
c_2\\
\vdots\\
c_m
\end{bmatrix}, \quad \quad
\bM{d}=
\begin{bmatrix*}[r]
2\\
-1\\
2.3
\end{bmatrix*}
\end{align*}
%================

\جزوحصہء{مجموعہ اور غیر سمتی ضرب}
آئیں پہلے مساوات کا تصور جانتے ہیں۔

\ابتدا{تعریف}
دو قالب \عددی{\bM{A}} اور \عددی{\bM{B}} اس صورت مساوی ہوں گے جب دونوں قالب کی جسامت برابر ہو اور ان کے نظیری ارکان آپس میں برابر ہوں یعنی 
\عددی{a_{11}=b_{11}}، \عددی{a_{12}=b_{12}}، \نقطے ہوں۔غیر مساوی قالب \اصطلاح{مختلف}\فرہنگ{مختلف}\حاشیہب{different}\فرہنگ{different} کہلاتے ہیں۔یوں مختلف جسامت کے قالب ہر صورت مختلف ہوں گے۔مساوات کا تعلق \عددی{\bM{A}=\bM{B}} لکھا جاتا ہے۔
\انتہا{تعریف}
%=======================

\ابتدا{مثال}\quad قالبوں کی مساوات\\
اگر درج ذیل قالب مساوی ہوں
\begin{align*}
\bM{A}=
\begin{bmatrix}
a_{11}& a_{12}\\
a_{21}&a_{22}
\end{bmatrix}\quad \text{اور} \quad
\bM{B}=
\begin{bmatrix}
2&-3\\
0&3.2
\end{bmatrix}
\end{align*}
تب \عددی{a_{11}=2}، \عددی{a_{12}=-3}، \عددی{a_{21}=0} اور \عددی{a_{22}=3.2} ہوں گے اور ہم \عددی{\bM{A}=\bM{B}} لکھ سکتے ہیں۔درج ذیل تمام قالب آپس میں \اصطلاح{مختلف} ہیں۔
\begin{align*}
\begin{bmatrix}
2&7\\
5&1
\end{bmatrix}\quad 
\begin{bmatrix}
5&1\\
2&7
\end{bmatrix}\quad
\begin{bmatrix}
2&7\\
1&5
\end{bmatrix}\quad 
\begin{bmatrix}
2\\
1
\end{bmatrix}
\end{align*}
\انتہا{مثال}
%=============================
\ابتدا{تعریف}\quad قالبوں کا مجموعہ\\
دو یکساں جسامت کے قالب \عددی{\bM{A}=[a_{jk}]} اور \عددی{\bM{B}=[b_{jk}]} کا مجموعہ \عددی{\bM{A}+\bM{B}} لکھا جائے گا جس کے اندراجات \عددی{a_{jk}+b_{jk}} کو \عددی{\bM{A}} اور \عددی{\bM{B}} کے نظیری ارکان کے مجموعے سے حاصل کیا جائے گا۔دو مختلف جسامت کے قالبوں کا مجموعہ حاصل کرنا نا ممکن ہے۔
\انتہا{تعریف}
%=============================

\ابتدا{مثال}\شناخت{مثال_الجبرا_قالب_الف}
اگر 
\begin{align*}
\bM{A}=
\begin{bmatrix*}[r]
2&-1&3\\
1&0&-2\\
3&2&1
\end{bmatrix*}
,\quad 
\bM{B}=
\begin{bmatrix*}[r]
7&3&0\\
1&2&1\\
2&-1&3
\end{bmatrix*},\quad
\bM{a}=
\begin{bmatrix*}[r]
1\\
3\\
-2
\end{bmatrix*},\quad
\bM{b}=
\begin{bmatrix*}[r]
0\\
2&\\
1
\end{bmatrix*}
\end{align*}
ہوں تب \عددی{\bM{A}+\bM{B}}، \عددی{\bM{a}+\bM{b}} اور \عددی{0\bM{A}+\bM{b}} حاصل کریں۔

حل:چونکہ \عددی{\bM{A}} اور \عددی{\bM{B}} کی یکساں جسامت ہے لہٰذا انہیں جمع کیا جا سکتا ہے۔مجموعہ درج ذیل ہو گا۔
\begin{align*}
\bM{A}+\bM{B}=
\begin{bmatrix*}[r]
2+7&-1+3&3+0\\
1+1&0+2&-2+1\\
3+2&2-1&1+3
\end{bmatrix*}
=
\begin{bmatrix*}[r]
9&2&3\\
2&2&-1\\
5&1&4
\end{bmatrix*}
\end{align*}
اسی طرح چونکہ \عددی{\bM{a}} اور \عددی{\bM{b}} کی جسامت یکساں ہے لہٰذا انہیں جمع  کیا جا سکتا ہے۔ ان کا مجموعہ درج ذیل ہے۔
\begin{align*}
\bM{a}+\bM{b}=
\begin{bmatrix*}[r]
1+0\\
3+2\\
-2+1
\end{bmatrix*}=
\begin{bmatrix*}[r]
1\\
5\\
-1
\end{bmatrix*}
\end{align*}
چونکہ \عددی{\bM{A}} اور \عددی{\bM{b}} کی جسامت یکساں نہیں ہے لہٰذا \عددی{0\bM{A}+\bM{b}} حاصل نہیں کیا جا سکتا ہے۔
\انتہا{مثال}
%==================================
\ابتدا{تعریف}\quad غیر سمتی ضرب\\
کسی بھی \عددی{m\times n} قالب \عددی{\bM{A}=[a_{jk}]} اور کسی بھی غیر سمتی مقدار (عدد) \عددی{c} کا حاصل \اصطلاح{ضرب}\فرہنگ{ضرب!غیر سمتی} \عددی{c\bM{A}} لکھا جاتا ہے جو ایسا \عددی{m \times n} قالب \عددی{c\bM{A}=[ca_{jk}]} ہے جس کا ہر رکن \عددی{\bM{A}} کے نظیری رکن کو \عددی{c} سے ضرب دیتے حاصل کیا جاتا ہے۔
\انتہا{تعریف}
%=========================

ہم \عددی{(-1)\bM{A}} کو \عددی{-\bM{A}} لکھتے ہیں اور اس کو \عددی{\bM{A}} کا نفی کہتے ہیں۔اسی طرح \عددی{(-k)\bM{A}} کو \عددی{-k\bM{A}} لکھا جاتا ہے۔ \عددی{\bM{A}+(-\bM{B})} کو \عددی{\bM{A}-\bM{B}} لکھا جاتا ہے جو \عددی{\bM{A}} اور \عددی{\bM{B}} کا \اصطلاح{فرق}\فرہنگ{فرق}\حاشیہب{difference}\فرہنگ{difference} کہلاتا ہے (فرق صرف یکساں جسامت کے قالب کا حاصل کیا جا سکتا ہے)۔

%==================
\ابتدا{مثال}\quad غیر سمتی ضرب\\
اگر 
\begin{align*}
\bM{A}=
\begin{bmatrix*}[r]
1.2&3.3\\
0.6& -1.5\\
0\phantom{.0}&6.0
\end{bmatrix*}
\end{align*}
ہو تب درج ذیل لکھے جا سکتے ہیں۔
\begin{align*}
-\bM{A}
\begin{bmatrix*}[r]
-1.2&-3.3\\
-0.6& 1.5\\
0\phantom{.0}&-6.0
\end{bmatrix*},\quad
\frac{10}{3}\bM{A}=
\begin{bmatrix*}[r]
4&11\\
2&-5\\
0&20
\end{bmatrix*},\quad
0\bM{A}=
\begin{bmatrix*}[r]
0&0\\
0&0\\
0&0
\end{bmatrix*}
\end{align*}
اگر قالب \عددی{\bM{B}} میں مختلف اشیاء کی کلوگرام کمیت درج ہو تب \عددی{1000\bM{B}} قالب انہیں اشیاء کی کمیت گرام میں دے گا۔  
\انتہا{مثال}
%=======================

\جزوحصہء{مجموعہ قالب اور غیر سمتی ضرب کے قواعد}
مجموعہ اعداد کے قواعد سے  یکساں جسامت \عددی{m\times n} کے قالبوں کے مجموعے کے درج ذیل قاعدے حاصل ہوتے ہیں۔
\begin{gather}
\begin{aligned}
\bM{A}+\bM{B}&=\bM{B}+\bM{A}\\
(\bM{A}+\bM{B})+\bM{C}&=\bM{A}+(\bM{B}+\bM{C})\quad (\text{یعنی} \,\,\,\bM{A}+\bM{B}+\bM{C}) \\
\bM{A}+\bM{0}&=\bM{A}\\
\bM{A}-\bM{A}&=\bM{0}
\end{aligned}
\end{gather}
درج بالا موٹی لکھائی میں صفر \عددی{\bM{0}} ایسے \عددی{m\times n} \اصطلاح{صفر قالب}\فرہنگ{صفر!قالب}\فرہنگ{قالب!صفر}\حاشیہب{zero matrix}\فرہنگ{matrix!zero}\فرہنگ{zero!matrix} کو ظاہر کرتی ہے جس کے تمام ارکان صفر \عددی{0} کے برابر ہوں۔اگر \عددی{m=1} یا \عددی{n=1} ہو تب اس کو \اصطلاح{صفر سمتیہ}\فرہنگ{صفر!سمتیہ}\فرہنگ{سمتیہ!صفر}\حاشیہب{zero vector}\فرہنگ{vector!zero}\فرہنگ{zero!vector} کہیں گے۔

یوں مجموعہ قالب \اصطلاح{قانون تبادل}\فرہنگ{قانون!تبادل} اور \اصطلاح{قانون تلازم}\فرہنگ{قانون!تلازم}  پر پورا اترتا ہے۔

اسی طرح غیر سمتی ضرب درج ذیل قواعد پر پورا اترتا ہے۔
 \begin{gather}
\begin{aligned}
c(\bM{A}+\bM{B})&=c\bM{A}+c\bM{B}\\
(c+k)\bM{A}&=c\bM{A}+k\bM{B}\\
c(k\bM{A})&=(ck)\bM{A}\quad \quad (\text{یعنی} \,\,\, ck\bM{A})\\
1\bM{A}&=\bM{A}
\end{aligned}
\end{gather}
%=====================
\حصہء{سوالات}
سوال \حوالہ{سوال_الجبرا_عمومی_الف} تا سوال \حوالہ{سوال_الجبرا_عمومی_ب} عمومی سوالات ہیں۔
%=======================
\ابتدا{سوال}\شناخت{سوال_الجبرا_عمومی_الف}
 \عددی{\bM{A}=[a_{jk}]} لکھتے ہوئے مثال \حوالہ{مثال_الجبرا_فروخت} میں \عددی{[a_{12}]} اور \عددی{[a_{25}]} کیا ہیں۔

جوابات:\عددی{[a_{12}]=23} اور \عددی{[a_{25}]=0}
\انتہا{سوال}
%=======================
\ابتدا{سوال}
مثال \حوالہ{مثال_الجبرا_فروخت} میں دیے گئے قالب کی جسامت لکھیں۔

جواب:\عددی{3\times 7}
\انتہا{سوال}
%========================
\ابتدا{سوال}\شناخت{سوال_الجبرا_عمومی_ب}
مثال \حوالہ{مثال_الجبرا_قالب_الف} میں قالب \عددی{\bM{A}} کی مرکزی وتر لکھیں۔

جواب: \عددی{2}، \عددی{0} اور \عددی{1}
\انتہا{سوال}
%===========================
سوال \حوالہ{سوال_الجبرا_مجموعہ_الف} تا سوال \حوالہ{سوال_الجبرا_مجموعہ_ب} میں قالبوں کے مجموعے اور غیر سمتی ضرب حاصل کرنے ہوں گے۔ان سوالات میں درکار قالب درج ذیل ہیں۔
\begin{align*}
\bM{A}=
\begin{bmatrix*}[r]
1&0&2\\
3&-1&1\\
2&1&0
\end{bmatrix*},\quad \bM{B}=
\begin{bmatrix*}[r]
2&0&3\\
-1&2&3\\
0&4&1
\end{bmatrix*},\quad
\bM{C}=
\begin{bmatrix*}[r]
2&0\\
6&-2\\
4&2
\end{bmatrix*},\quad
\bM{D}=
\begin{bmatrix*}[r]
0&4\\
2&2\\
-1&3
\end{bmatrix*}\\
\bM{E}=
\begin{bmatrix*}[r]
4&0\\
12&-4\\
8&4
\end{bmatrix*},\quad
\bM{u}=
\begin{bmatrix*}[r]
2.2\\
1.0\\
0.0,
\end{bmatrix*}\quad
\bM{v}=
\begin{bmatrix*}[r]
1.1\\
0.5\\
0.0
\end{bmatrix*},\quad
\bM{w}=
\begin{bmatrix*}[r]
2.0\\
1.6\\
3.2
\end{bmatrix*}
\end{align*}
%==============================
\ابتدا{سوال}\شناخت{سوال_الجبرا_مجموعہ_الف}
\عددی{0.5\bM{A}}، \عددی{0.2\bM{B}}، \عددی{-2\bM{u}}

جوابات:
\begin{align*}
0.5\bM{A}=
\begin{bmatrix*}[r]
0.5&0&1.0\\
1.5&-0.5&0.5\\
1.0&0.5&0
\end{bmatrix*},\quad
0.2\bM{B}=
\begin{bmatrix*}[r]
0.4&0&0.6\\
-0.2&0.4&0.6\\
0&0.8&0.2
\end{bmatrix*},\quad
-2\bM{u}=
\begin{bmatrix*}[r]
-4.4\\
-2.0\\
0
\end{bmatrix*}
\end{align*}
\انتہا{سوال}
%========================
\ابتدا{سوال}\quad
$3\bM{A}+2\bM{B},\quad 2\bM{C}-\bM{E},\quad -3\bM{u}+\bM{v}-2\bM{w}$\\

جوابات:
\begin{align*}
\begin{bmatrix*}
7&0&12\\
7&1&9\\
6&11&2
\end{bmatrix*}, \quad 
\begin{bmatrix*}
0&0\\
0&0\\
0&0
\end{bmatrix*}, \quad 
\begin{bmatrix*}
-9.5\\
-5.7\\
-6.4
\end{bmatrix*}
\end{align*}
\انتہا{سوال}
%=============================
\ابتدا{سوال}\quad
$(3\cdot 6)\bM{B},\quad 6(3)\bM{B},\quad 5\bM{A}-3\bM{A}$\\
جوابات:
\begin{align*}
\begin{bmatrix*}[r]
18&0&36\\
54&-18&18\\
36&18&0
\end{bmatrix*}, \quad 
\begin{bmatrix*}[r]
18&0&36\\
54&-18&18\\
36&18&0
\end{bmatrix*},\quad
\begin{bmatrix*}[r]
2&0&4\\
6&-2&2\\
4&2&0
\end{bmatrix*}
\end{align*}
\انتہا{سوال}
%=================================
\ابتدا{سوال}\quad
$3(2\bM{C}+5\bM{D}),\quad 0.2(0.1\bM{E}-0.3\bM{D})$\\
جوابات:
\begin{align*}
\begin{bmatrix*}
12&60\\
66&18\\
9&57
\end{bmatrix*},\quad
\begin{bmatrix*}
0.08&-0.24\\
0.12&-0.2\\
0.22&-0.1
\end{bmatrix*}
\end{align*}
\انتہا{سوال}
%=========================
\ابتدا{سوال}\quad
$\bM{E}+(\bM{D}+\bM{C}),\quad (\bM{D}+\bM{E})+\bM{C},\quad \bM{A}+\bM{C},\quad 0\bM{B}+\bM{D}$\\
جوابات:چونکہ \عددی{\bM{A}} اور \عددی{\bM{C}} کی جسامت یکساں نہیں ہے لہٰذا انہیں جمع نہیں کیا جا سکتا ہے۔غیر یکساں جسامت کی بنا \عددی{0\bM{B}+\bM{D}} بھی حاصل نہیں کیا جا سکتا ہے۔ 
\begin{align*}
\bM{E}+(\bM{D}+\bM{C})=(\bM{D}+\bM{E})+\bM{C}=
\begin{bmatrix*}[r]
6&4\\
20&-4\\
11&9
\end{bmatrix*}
\end{align*}
\انتہا{سوال}
%=========================
\ابتدا{سوال}
\عددی{\bM{u}}، \عددی{\bM{v}} اور \عددی{\bM{w}} کو خلاء میں قوت کے اجزاء تصور کرتے ہوئے ان کے مجموعے سے کل قوت دریافت کریں۔

جواب:
\begin{align*}
\begin{bmatrix*}[r]
5.3\\
3.1\\
3.2
\end{bmatrix*}
\end{align*}
\انتہا{سوال}
%============================
\ابتدا{سوال}\شناخت{سوال_الجبرا_مجموعہ_ب}\quad متوازن صورت\\
تمام قوتوں کا مجموعہ صفر کے برابر ہونے کی صورت کو \اصطلاح{متوازن}\فرہنگ{متوازن}\حاشیہب{equilibrium}\فرہنگ{equilibrium} حال کہتے ہیں۔

ایسا قوت \عددی{\bM{x}} دریافت کریں کہ \عددی{\bM{u}}، \عددی{\bM{v}}، \عددی{\bM{w}} اور \عددی{\bM{x}} متوازن حال میں ہوں۔ 
\begin{align*}
\bM{x}=
\begin{bmatrix*}[r]
-5.3\\
-3.1\\
-3.2
\end{bmatrix*}
\end{align*}
\انتہا{سوال}
%==============================

\حصہ{قالبی ضرب}
قالبی ضرب سے مراد دو عدد قالبوں کا آپس میں ضرب ہے۔آپ سے گزارش ہے کہ  چند مثالیں حل کرتے ہوئے قالبی ضرب کو اچھی طرح سمجھیں۔قالبی ضرب کی تعریف درج ذیل ہے۔

%===============
\ابتدا{تعریف}\quad قالبی ضرب\\
\عددی{m\times n} قالب \عددی{\bM{A}=[a_{jk}]} اور \عددی{r\times p} قالب \عددی{\bM{B}=[b_{jk}]} کا (اسی ترتیب سے) حاصل ضرب \عددی{\bM{C}=\bM{A}\bM{B}} صرف \عددی{r=n} کی صورت میں ممکن ہو گا اور یہ \عددی{m\times p} قالب \عددی{\bM{C}=[c_{jk}]} ہو گا جس کے اندراجات درج ذیل ہوں گے۔
\begin{align}\label{مساوات_الجبرائی_قالبی_ضرب_الف}
c_{jk}=\sum_{l=1}^{n} a_{jl}b_{lk}=a_{j1}b_{1k}+a_{j2}b_{2k}+\cdots+a_{jn}b_{nk},\quad j=1,\cdots,m \quad k=1,\cdots,p
\end{align}
\انتہا{تعریف}
%========================

یوں پہلے جزو \عددی{\bM{A}} میں قطاروں کی تعداد \عددی{n} دوسرے جزو \عددی{\bM{B}} کی صفوں کی تعداد \عددی{r} کے برابر ہونا لازمی ہے۔مساوات \حوالہ{مساوات_الجبرائی_قالبی_ضرب_الف} میں \عددی{c_{jk}} کو \عددی{\bM{A}} کے \عددی{j} صف کے ہر رکن کو \عددی{\bM{B}} کے \عددی{k} قطار کے نظیری رکن سے ضرب دیتے ہوئے تمام \عددی{n} حاصل ضرب کا مجموعہ لینے سے حاصل کیا جاتا ہے۔ ہم کہتے ہیں \موٹا{صف ضرب قطار} سے قالبی ضرب حاصل کیا جاتا ہے۔ قالبی ضرب \عددی{n=3} کی صورت میں درج ذیل ہو گا
\begin{align*}
\begin{bmatrix}
a_{11}& a_{12} & a_{13}\\
a_{21}& a_{22} & a_{23}\\
a_{31}& a_{32} & a_{33}
\end{bmatrix}
\begin{bmatrix}
b_{11}& b_{12}\\
b_{21}& b_{22}\\
b_{31}& b_{32}
\end{bmatrix}=
\begin{bmatrix}
c_{11} & c_{12}\\
c_{21} & c_{22}\\
c_{31} & c_{32}\\
c_{41} & c_{42}
\end{bmatrix}
\end{align*}
جہاں \عددی{\bM{A}} کی پہلی صف کے ارکان کو \عددی{\bM{B}} کی پہلی قطار کے نظیری ارکان سے ضرب دیتے ہوئے تمام کا مجموعہ لینے سے \عددی{c_{11}} حاصل ہو گا۔اسی طرح \عددی{\bM{A}} کی پہلی صف کے ارکان کو \عددی{\bM{B}} کی دوسری قطار کے نظیری ارکان سے ضرب دیتے ہوئے تمام کا مجموعہ لینے سے \عددی{c_{12}} حاصل ہو گا اور  \عددی{\bM{A}} کی دوسری صف کے ارکان کو \عددی{\bM{B}} کی پہلی قطار کے نظیری ارکان سے ضرب دیتے ہوئے تمام کا مجموعہ لینے سے \عددی{c_{21}} حاصل ہو گا۔اس عمل کو درج ذیل لکھا جائے گا۔
\begin{align*}
c_{11}&=a_{11}b_{11}+a_{12}b_{21}+a_{13}b_{31}\\
c_{12}&=a_{11}b_{12}+a_{12}b_{22}+a_{13}b_{32}\\
c_{21}&=a_{21}b_{11}+a_{22}b_{21}+a_{23}b_{31}\\
\end{align*}
 
چونکہ سمتیہ درحقیقت قالب کی مخصوص صورت ہے لہٰذا قالب اور سمتیہ کا ضرب بھی بالکل اسی طرح حاصل کیا جائے گا۔قالبی ضرب کی چند مثالیں درج ذیل ہیں۔
%=====================
\ابتدا{مثال}\quad قالبی ضرب\\

\begin{align*}
\begin{bmatrix}
1&3\\
4&6\\
5&2
\end{bmatrix}
\begin{bmatrix}
9&7\\
8&10
\end{bmatrix}=
\begin{bmatrix}
1\cdot9+3\cdot 8& 1\cdot 7+3\cdot 10\\
4\cdot9+6\cdot 8& 4\cdot 7+6\cdot 10\\
5\cdot9+2\cdot 8& 5\cdot 7+2\cdot 10
\end{bmatrix}=
\begin{bmatrix}
33&37\\
84&88\\
61&55
\end{bmatrix}
\end{align*}
\انتہا{مثال}
%====================
\ابتدا{مثال}\quad قالب اور سمتیہ کا ضرب\\
\begin{align*}
\begin{bmatrix}
2&1\\
3&0
\end{bmatrix}
\begin{bmatrix}
4\\
5
\end{bmatrix}
=\begin{bmatrix}
2\cdot 4+1\cdot 5\\
3\cdot 4+0\cdot 5
\end{bmatrix}=
\begin{bmatrix}
13\\
12
\end{bmatrix}\quad \text{جبکہ} \quad 
\begin{bmatrix}
4\\
5
\end{bmatrix}
\begin{bmatrix}
2&1\\
3&0
\end{bmatrix}=\text{\RL{نا ممکن}}
\end{align*}
درج بالا میں قالب اور سمتیہ کی جگہ تبدیل کرنے سے پہلے جزو کی قطاروں اور دوسرے جزو کی صفوں کی تعداد یکساں نہیں رہتی لہٰذا ایسا ضرب نا ممکن ہے۔یوں ضروری نہیں ہے کہ \عددی{\bM{A}\bM{B}} اور \عددی{\bM{B}\bM{A}} برابر ہوں اور یہ کہ دونوں ضرب کا حصول ممکن ہو۔
\انتہا{مثال}
%=============================
\ابتدا{سوال}\quad 
\begin{align*}
\begin{bmatrix} 
2 & 1 & 3
\end{bmatrix}
\begin{bmatrix*}[r]
1\\
0\\
-2
\end{bmatrix*}=
\begin{bmatrix*}[r]
-4
\end{bmatrix*}, \quad
\begin{bmatrix*}[r]
1\\
0\\
-2
\end{bmatrix*}
\begin{bmatrix} 
2 & 1 & 3
\end{bmatrix}=
\begin{bmatrix*}[r]
2&1&3\\
0&0&0\\
-4&-2&-6
\end{bmatrix*}
\end{align*}
آپ نے دیکھا کہ سمتیات کی جگہ تبدیل کرنے سے حاصل ضرب تبدیل ہوتا ہے یعنی \موٹا{قالبی ضرب قانون تبادل  پر پورا نہیں اترتا}۔
\انتہا{سوال}
%==============================
\ابتدا{مثال}\quad قالبی ضرب قانون تبادل پر پورا نہیں اترتا لہٰذا عموماً \عددی{\bM{A}\bM{B} \ne \bM{B}\bM{A}} ہو گا\\
\begin{align*}
\begin{bmatrix*}[r]
1&1\\
200&200
\end{bmatrix*}
\begin{bmatrix*}[r]
-1&1\\
1&-1
\end{bmatrix*}=
\begin{bmatrix*}[r]
0&0\\
0&0
\end{bmatrix*},\quad 
\begin{bmatrix*}[r]
-1&1\\
1&-1
\end{bmatrix*}
\begin{bmatrix*}[r]
1&1\\
200&200
\end{bmatrix*}
=
\begin{bmatrix*}[r]
199&199\\
-199&-199
\end{bmatrix*}
\end{align*}

\انتہا{مثال}
%===============================
آپ نے دیکھا کہ قالبی ضرب میں اجزاء کی جگہ تبدیل نہیں کی جا سکتی ہے۔اس کے علاوہ قالبی ضرب، عام اعدادی ضرب کے درج ذیل قواعد پر پورا اترتا ہے۔
\begin{gather}
\begin{aligned}
\text{(الف)}\quad (k\bM{A})\bM{B}&=k(\bM{A}\bM{B})=\bM{A}(k\bM{B}) \quad (k\bM{A}\bM{B}\,\,\, \text{یا}\,\,\, \bM{A} k \bM{B})\\
\text{(ب)}\quad \bM{A}(\bM{B}\bM{C})&=(\bM{A}\bM{B})\bM{C} \quad (\text{یعنی} \,\,\, \bM{A}\bM{B}\bM{C})\\
\text{(پ)}\quad (\bM{A}+\bM{B})\bM{C}&=\bM{A}\bM{C}+\bM{B}\bM{C}\\
\text{(ت)}\quad \bM{C}(\bM{A}+\bM{B})&=\bM{C}\bM{A}+\bM{C}\bM{B}
\end{aligned}
\end{gather}
درج بالا میں \عددی{k} کوئی عدد ہے اور یہ قواعد اس صورت درست ہوں گے کہ بائیں ہاتھ کے قالب، قالبی ضرب کی تعریف پر پورا اترتے ہوں۔درج بالا میں مساوات-ب \اصطلاح{قانون تلازم}\حاشیہب{associative law} کہلاتا ہے جبکہ مساوات-پ اور مساوات-ت \اصطلاح{قانون تقسیم}\حاشیہب{distributive law} کہلاتا ہے۔

چونکہ قالبی ضرب صف ضرب قطار کو کہتے ہیں لہٰذا مساوات \حوالہ{مساوات_الجبرائی_قالبی_ضرب_الف} کو زیادہ خوش اسلوبی سے درج ذیل لکھا جا سکتا ہے
\begin{align}\label{مساوات_الجبرائی_قالبی_ضرب_ب}
c_{jk}=\bM{a}_j \bM{b}_k,\quad j=1,\cdots,m \quad k=1,\cdots,p
\end{align} 
جہاں \عددی{\bM{a}_j} قالب \عددی{\bM{A}} کا صف \عددی{j} اور \عددی{\bM{b}_k} قالب \عددی{\bM{B}} کا قطار \عددی{k} ہے۔ 
\begin{align*}
\bM{a}_j\bM{b}_k=
\begin{bmatrix}
a_{j1}& a_{j2}& \cdots & a_{jn}
\end{bmatrix}
\begin{bmatrix}
b_{1k}\\
b_{2k}\\
\vdots\\
b_{nk}
\end{bmatrix}=
\begin{bmatrix}
a_{j1}b_{1k}+a_{j2}b_{2k}+\cdots +a_{jn}b_{nk}
\end{bmatrix}
\end{align*}
%======================
\ابتدا{مثال}\quad صف اور قطار سمتیہ کی صورت میں ضرب ارکان\\
\عددی{3\times 3} قالب \عددی{\bM{A}=[a_{jk}]} اور \عددی{3\times 4} قالب \عددی{\bM{B}=[b_{jk}]} کو ضرب دینے سے درج لکھا جا سکتا ہے۔
\begin{align}\label{مساوات_الجبرا_ضرب_تین_چار}
\bM{A} \bM{B}=
\begin{bmatrix}
\bM{a}_1\bM{b}_1 & \bM{a}_1\bM{b}_2 & \bM{a}_1\bM{b}_3 & \bM{a}_1\bM{b}_4 \\
\bM{a}_2\bM{b}_1 & \bM{a}_2\bM{b}_2 & \bM{a}_2\bM{b}_3 & \bM{a}_2\bM{b}_4 \\
\bM{a}_3\bM{b}_1 & \bM{a}_3\bM{b}_2 & \bM{a}_3\bM{b}_3 & \bM{a}_3\bM{b}_4 
\end{bmatrix}
\end{align}
\انتہا{مثال}
%=================================
\ابتدا{مثال}
\عددی{3\times 3} قالب \عددی{\bM{A}=[a_{jk}]} اور \عددی{3\times 4} قالب \عددی{\bM{B}=[b_{jk}]} درج ذیل ہیں۔ مساوات \حوالہ{مساوات_الجبرا_ضرب_تین_چار}  سے \عددی{\bM{A}\bM{B}} حاصل کریں۔
\begin{align*}
\bM{A}=
\begin{bmatrix}
1&0&2\\
2&1&1\\
3&2&1
\end{bmatrix},\quad
\bM{B}=
\begin{bmatrix}
2&2&1&2\\
1&2&0&3\\
2&1&3&1
\end{bmatrix}
\end{align*}
حل: یہاں \عددی{\bM{a}_1=[1\quad 0\quad 2]}، \عددی{\bM{a}_2=[2\quad 1\quad 1]} اور \عددی{\bM{a}_3=[3\quad 2\quad 1]} ہیں۔یوں درج ذیل لکھا جا سکتا ہے۔
\begin{align*}
\bM{a}_1\bM{b}_1=
\begin{bmatrix}
1&0&2
\end{bmatrix}
\begin{bmatrix}
2\\
1\\
2
\end{bmatrix}=
2+0+4=6
\end{align*}
اسی طرح بقایا ارکان حاصل کرتے ہوئے درج ذیل ملتا ہے۔
\begin{align*}
\bM{A}\bM{B}=
\begin{bmatrix}
6&4&7&4\\
7&7&5&8\\
10&11&6&13
\end{bmatrix}
\end{align*}
\انتہا{مثال}
%===========================

\جزوحصہء{قالبی ضرب بذریعہ کمپیوٹر}
مساوات \حوالہ{مساوات_الجبرا_ضرب_تین_چار} کو ذرہ مختلف طریقے سے لکھتے ہیں۔\عددی{\bM{A}} کو جوں کا توں جبکہ \عددی{\bM{B}} کو سمتیہ قطار کی صورت میں  لکھتے ہوئے درج ذیل ملتا ہے۔
\begin{align}\label{مساوات_الجبرا_ضرب_بذریعہ_کمپیوٹر}
\bM{A}\bM{B}=\bM{A}
\begin{bmatrix}
\bM{b}_1 & \bM{b}_2 & \cdots & \bM{b}_p
\end{bmatrix}=
\begin{bmatrix}
\bM{A}\bM{b}_1 &\bM{A} \bM{b}_2 & \cdots & \bM{A}\bM{b}_p
\end{bmatrix}
\end{align}
متعدد متوازی جڑے کمپیوٹر کو علیحدہ علیحدہ \عددی{\bM{b}_1}، \عددی{\bM{b}_2}، \نقطے، \عددی{\bM{b}_p}  یا انہیں کئی کئی علیحدہ سمتیہ قطار فراہم کیے جاتے ہیں اور ساتھ ہی تمام کو \عددی{\bM{A}} بھی فراہم کیا جاتا ہے۔ یوں قالبی ضرب کے اجزاء \عددی{\bM{A}\bM{b}_1}، \عددی{\bM{A} \bM{b}_2 }، \نقطے، \عددی{\bM{A}\bM{b}_p} بہ یک وقت (نسبتاً بہت کم وقت میں) حاصل ہوتے ہیں۔
%================
\ابتدا{مثال}
درج ذیل کو مساوات \حوالہ{مساوات_الجبرا_ضرب_بذریعہ_کمپیوٹر} کی مدد سے حل کریں۔
\begin{align*}
\bM{A}\bM{B}=
\begin{bmatrix*}[r]
2&1\\
1&-1
\end{bmatrix*}
\begin{bmatrix*}[r]
1&0&2\\
2&1&3
\end{bmatrix*}=
\begin{bmatrix*}[r]
4&1&7\\
-1&-1&-1
\end{bmatrix*}
\end{align*}
حل: مساوات \حوالہ{مساوات_الجبرا_ضرب_بذریعہ_کمپیوٹر} سے قالبی ضرب کے قطار حاصل کرتے ہیں جنہیں ایک ہی قالب میں یکجا کرتے ہوئے درج بالا جواب ملتا ہے۔
\begin{align*}
\begin{bmatrix*}[r]
2&1\\
1&-1
\end{bmatrix*}
\begin{bmatrix*}
1\\
2
\end{bmatrix*}=
\begin{bmatrix*}[r]
4\\
-1
\end{bmatrix*}, \quad
\begin{bmatrix*}[r]
2&1\\
1&-1
\end{bmatrix*}
\begin{bmatrix*}
0\\
1
\end{bmatrix*}=
\begin{bmatrix*}[r]
1\\
-1
\end{bmatrix*},\quad
\begin{bmatrix*}[r]
2&1\\
1&-1
\end{bmatrix*}
\begin{bmatrix*}
2\\
3
\end{bmatrix*}=
\begin{bmatrix*}[r]
7\\
-1
\end{bmatrix*}
\end{align*}
\انتہا{مثال}
%==================
\جزوحصہء{خطی تبادل اور قالبی ضرب}
دو متغیرات پر مبنی خطی تبادل درج ذیل لکھا جاتا ہے
\begin{gather}
\begin{aligned}\label{مساوات_الجبرا_خطی_تعلق_الف}
y_1&=a_{11}x_1+a_{12}x_2\\
y_2&=a_{21}x_1+a_{22}x_2
\end{aligned}
\end{gather}
جس کو سمتیات کی مدد سے درج ذیل لکھا جا سکتا ہے۔
\begin{align}\label{مساوات_الجبرا_خطی_تعلق_ب}
\bM{y}=\begin{bmatrix} y_1\\y_2 \end{bmatrix}=\bM{A}\bM{x}=\begin{bmatrix} a_{11}& a_{12}\\a_{21}&a_{22} \end{bmatrix}\begin{bmatrix} x_1\\ x_2 \end{bmatrix}=\begin{bmatrix} a_{11}x_1 + a_{12}x_2\\  a_{21}x_1 + a_{22}x_2\end{bmatrix}
\end{align}
اب اگر \عددی{x_1 x_2} نظام ازخود \عددی{w_1 w_2} پر مبنی ہو یعنی
 \begin{gather}
\begin{aligned}\label{مساوات_الجبرا_خطی_تعلق_پ}
x_1&=b_{11}w_1+b_{12}w_2\\
x_2&=b_{21}w_1+b_{22}w_2
\end{aligned}
\end{gather}
یا
\begin{align}\label{مساوات_الجبرا_خطی_تعلق_ت}
\bM{x}=\begin{bmatrix} x_1\\x_2 \end{bmatrix}=\bM{B}\bM{w}=\begin{bmatrix} b_{11}& b_{12}\\b_{21}&b_{22} \end{bmatrix}\begin{bmatrix} w_1\\ w_2 \end{bmatrix}=\begin{bmatrix} b_{11}w_1 + b_{12}w_2\\  b_{21}w_1 + b_{22}w_2\end{bmatrix}
\end{align}
تب \عددی{y_1y_2} نظام بالواسطہ  \عددی{w_1 w_2} پر مبنی ہو گا۔آئیں اس تعلق کو جانیں۔

مساوات \حوالہ{مساوات_الجبرا_خطی_تعلق_الف} میں مساوات \حوالہ{مساوات_الجبرا_خطی_تعلق_پ} استعمال کرتے ہوئے
\begin{align*}
y_1&=a_{11}(b_{11}w_1+b_{12}w_2)+a_{12}(b_{21}w_1+b_{22}w_2)\\
&=(a_{11}b_{11}+a_{12}b_{21})w_1+(a_{11}b_{12}+a_{12}b_{22})w_2\\
y_2&=a_{21}(b_{11}w_1+b_{12}w_2)+a_{22}(b_{21}w_1+b_{22}w_2)\\
&=(a_{21}b_{11}+a_{22}b_{21})w_1+(a_{21}b_{12}+a_{22}b_{22})w_2
\end{align*}
یعنی
\begin{gather}
\begin{aligned}
y_1&=c_{11}w_1+c_{12}w_2\\
y_2&=c_{21}w_1+c_{22}w_2
\end{aligned}
\end{gather}
ملتا ہے جہاں
\begin{gather}
\begin{aligned}
c_{11}&=a_{11}b_{11}+a_{12}b_{21}, \quad c_{12}=a_{11}b_{12}+a_{12}b_{22}\\
c_{21}&=a_{21}b_{11}+a_{22}b_{21},\quad c_{22}=a_{21}b_{12}+a_{22}b_{22}
\end{aligned}
\end{gather}
لیا گیا ہے۔اس تعلق کو سمتیات کی مدد سے درج ذیل لکھا جا سکتا ہے۔
\begin{align}
\bM{y}=\bM{C}\bM{w}=\begin{bmatrix}c_{11} & c_{12}\\ c_{21}&c_{22}  \end{bmatrix}\begin{bmatrix}w_1\\w_2  \end{bmatrix}=
\begin{bmatrix} c_{11}w_1+c_{12}w_2\\ c_{21}w_1+c_{22}w_2 \end{bmatrix}
\end{align}
آئیں قالبی ضرب \عددی{\bM{A}\bM{B}} حاصل کرتے ہوئے ثابت کریں کہ \عددی{\bM{C}=\bM{A}\bM{B}} ہے۔
\begin{align}\label{مساوات_الجبرا_خطی_تعلق_ٹ}
\bM{A}\bM{B}=\begin{bmatrix}a_{11}& a_{12}\\a_{21}& a_{22}  \end{bmatrix}\begin{bmatrix}b_{11} & b_{12}\\ b_{21} & b_{22}  \end{bmatrix}
=\begin{bmatrix}a_{11}b_{11}+a_{12}b_{21} &a_{11}b_{12}+a_{12}b_{22}\\  a_{21}b_{11}+a_{22}b_{21} & a_{21}b_{12}+a_{22}b_{22} \end{bmatrix}=\bM{C}
\end{align}
بڑے جسامت کے قالبوں کے لئے بھی \عددی{\bM{C}=\bM{A}\bM{B}} بالکل اسی طرح ثابت کیا جاتا ہے۔یوں \عددی{y}، \عددی{x} اور \عددی{w} متغیرات کی تعداد بالترتیب \عددی{m}، \عددی{n} اور \عددی{p} کی صورت میں \عددی{\bM{A}}، \عددی{\bM{B}} اور \عددی{\bM{C}} قالبوں کی جسامت بالترتیب \عددی{m\times n}، \عددی{n\times p} اور \عددی{m\times p} ہو گی جہاں \عددی{\bM{C}=\bM{A}\bM{B}} ہے۔  قالبی ضرب (مساوات \حوالہ{مساوات_الجبرائی_قالبی_ضرب_الف})  کی تعریف مساوات \حوالہ{مساوات_الجبرا_خطی_تعلق_ٹ} کی بدولت ہے۔
%===================
\جزوحصہ{تبدیلی محل}
قالب کے صفوں کو بطور قطار (یعنی قطاروں کو بطور صف) لکھ کر \اصطلاح{تبدیل محل قالب}\فرہنگ{تبدیل محل قالب}\حاشیہب{transpose matrix}\فرہنگ{transpose matrix} حاصل ہوتا ہے اور اس عمل کو \عددی{تبدیلی محل}\فرہنگ{تبدیلی محل}\حاشیہب{transposition}\فرہنگ{transposition} کہتے ہیں۔سمتیہ کی تبدیلی محل بھی اسی طرح کی جاتی ہے۔اس طرح قالب کا صف، تبدیل محل قالب کا قطار ہو گا اور یونہی قالب کا قطار، تبدیل محل قالب کا صف ہو گا۔ چکور قالب کے ارکان کا مرکزی وتر میں "عکس" لینے سے بھی تبدیل محل قالب حاصل ہو گا۔مرکزی وتر کے دونوں اطراف یکساں مقامات پر ارکان کی آپس میں جگہ تبدیل کرنے سے ان کا "عکس" حاصل ہو گا۔یوں \عددی{a_{12}} اور \عددی{a_{21}} آپس میں جگہ تبدیل کریں گے، \عددی{a_{13}} اور \عددی{a_{31}} آپس میں جگہ تبدیل کریں گے، وغیرہ وغیرہ۔قالب \عددی{\bM{A}} سے حاصل تبدیل محل قالب کو \عددی{\bM{A}^T} سے ظاہر کیا جائے گا۔درج ذیل مثال دیکھیں۔
%=========================
\ابتدا{مثال}\quad تبدیل محل قالب\\
قالب \عددی{\bM{A}} کا تبدیل محل \عددی{\bM{A}^T} درج ذیل ہے۔
\begin{align*}
\bM{A}=\begin{bmatrix*}[r] 5&1&-2\\3&6&4  \end{bmatrix*}, \quad \bM{A}^T=\begin{bmatrix*}[r] 5&3\\1&6\\-2&4 \end{bmatrix*}
\end{align*}
درج بالا کو درج ذیل بھی لکھا جا سکتا ہے۔
\begin{align*}
\begin{bmatrix*}[r] 5&1&-2\\3&6&4  \end{bmatrix*}^T=\begin{bmatrix*}[r] 5&3\\1&6\\-2&4 \end{bmatrix*}
\end{align*}
چکور قالب اور اس کا تبدیل محل درج ذیل ہیں۔چکور قالب اور اس کے تبدیل محل قالب میں مرکزی وتر کے ارکان جگہ تبدیل نہیں کرتے ہیں۔
\begin{align*}
\begin{bmatrix*}[r] 5&-2&6\\7&1&0\\4&8&3 \end{bmatrix*}^T=\begin{bmatrix*}[r] 5&7&4\\-2&1&8\\6&0&3\end{bmatrix*}
\end{align*}
سمتیہ صف کا تبدیل محل، سمتیہ قطار ہو گا اور یونہی سمتیہ قطار کا تبدیل محل، سمتیہ صف ہو گا۔
\begin{align*}
\begin{bmatrix*}[r] 3&7&-1 \end{bmatrix*}^T=\begin{bmatrix*}[r] 3\\7\\-1 \end{bmatrix*}, \quad\quad \begin{bmatrix*}[r] 4\\5\\2\end{bmatrix*}^T=\begin{bmatrix*}[r] 4&5&2 \end{bmatrix*}
\end{align*}
تبدیل محل کا تبدیل محل اصل قالب ہو گا۔
\begin{align*}
\begin{bmatrix*}[r] 2&1&4 \end{bmatrix*}^T=\begin{bmatrix*}[r] 2\\1\\4\end{bmatrix*}, \quad\quad \begin{bmatrix*}[r]2\\1\\4\end{bmatrix*}^T=\begin{bmatrix*}[r] 2&1&4 \end{bmatrix*}
\end{align*}
\انتہا{مثال}
%=====================================
\ابتدا{تعریف}\quad قالب اور سمتیہ کا تبدیل محل\\
\عددی{m\times n} قالب \عددی{\bM{A}=[a_{jk}]} کا تبدیل محل \عددی{n\times m} قالب \عددی{\bM{A}^T} ہے جس کا پہلا صف، \عددی{\bM{A}} کا پہلا قطار، اس کا دوسرا صف \عددی{\bM{A}} کا دوسرا قطار، وغیرہ وغیرہ ہوں گے۔ یوں مساوات \حوالہ{مساوات_قالب_عمومی_قالب_ب} میں دیے گئے \عددی{\bM{A}} کا تبدیل محل \عددی{\bM{A}^T} درج ذیل ہو گا۔

\begin{align}
\bM{A}T=[a_{kj}]=
\begin{bmatrix}
a_{11}& a_{21}& \cdots &a_{m1}\\
a_{12}& a_{22}& \cdots &a_{m2}\\
\vdots&&\\
a_{1n}& a_{2n}&\cdots&a_{mn}
\end{bmatrix}
\end{align}
سمتیہ صف کا تبدیل محل سمتیہ قطار ہو گا جبکہ سمتیہ قطار کا تبدیل محل سمتیہ صف ہو گا۔
\انتہا{تعریف}
%=======================

بعض اوقات قالب اور بعض اوقات تبدیل محل کے ساتھ کام کرنا زیادہ آسان ثابت ہوتا ہے۔  \اصطلاح{تبدیلی محل} کے قواعد درج ذیل ہیں۔
\begin{gather}
\begin{aligned}\label{مساوات_الجبرا_تبدیل_محل_قواعد}
\text{(الف)} \quad\quad \left(\bM{A}^T\right)^T&=\bM{A}\\
\text{(ب)} \quad \left(\bM{A}+\bM{B}\right)^T&=\bM{A}^T+\bM{B}^T\\
\text{(پ)} \quad \quad \left(c\bM{A}\right)^T&=c\bM{A}^T\\
\text{(ت)} \quad \quad \left(\bM{A}\bM{B}\right)^T&=\bM{B}^T\bM{A}^T
\end{aligned}
\end{gather}
دھیان رہے کہ مساوات \حوالہ{مساوات_الجبرا_تبدیل_محل_قواعد}-ت میں دائیں ہاتھ قالبوں کی ترتیب بائیں ہاتھ کی ترتیب کے الٹ ہے۔سوال \حوالہ{سوال_الجبرا_قالب_عمل_ب} میں آپ کو درج بالا تعلقات ثابت کرنے کو کہا گیا ہے۔
%==========================
\ابتدا{مثال}
درج ذیل قالب کو استعمال کرتے ہوئے مساوات \حوالہ{مساوات_الجبرا_تبدیل_محل_قواعد}-ت ثابت کریں۔
\begin{align*}
\bM{A}=\begin{bmatrix}a_{11}& a_{12}\\ a_{21}&a_{22}  \end{bmatrix}, \quad \bM{B}=\begin{bmatrix} b_{11}& b_{12}\\ b_{21}& b_{22} \end{bmatrix}
\end{align*}
حل:پہلے مساوات \حوالہ{مساوات_الجبرا_تبدیل_محل_قواعد}-ت کا بایاں ہاتھ حاصل کرتے ہیں۔قالبی ضرب \عددی{\bM{A}\bM{B}} لینے کے بعد
\begin{align*}
\bM{A}\bM{B}=\begin{bmatrix}a_{11}& a_{12}\\ a_{21}&a_{22}  \end{bmatrix}\begin{bmatrix} b_{11}& b_{12}\\ b_{21}& b_{22} \end{bmatrix}=\begin{bmatrix}a_{11}b_{11}+a_{12}b_{21} &a_{11}b_{12}+a_{12}b_{22} \\ a_{21}b_{11}+a_{22}b_{21}& a_{21}b_{12}+a_{22}b_{22} \end{bmatrix}
\end{align*}
 اس کا تبدیل محل حاصل کرتے ہیں۔
\begin{align}\label{مساوات_الجبرا_تبدیل_محل_بائیں_ہاتھ}
(\bM{A}\bM{B})^T=\begin{bmatrix}a_{11}b_{11}+a_{12}b_{21} &a_{21}b_{11}+a_{22}b_{21} \\a_{11}b_{12}+a_{12}b_{22} & a_{21}b_{12}+a_{22}b_{22} \end{bmatrix}
\end{align}
آئیں اب مساوات \حوالہ{مساوات_الجبرا_تبدیل_محل_قواعد}-ت کا دایاں ہاتھ حاصل کرتے ہیں۔یوں \عددی{\bM{B}^T} اور \عددی{\bM{A}^T} حاصل کرنے کے بعد
\begin{align*}
\bM{B}^T=\begin{bmatrix} b_{11}& b_{21}\\ b_{12}& b_{22} \end{bmatrix}, \quad \bM{A}^T=\begin{bmatrix}a_{11}& a_{21}\\ a_{12}&a_{22}  \end{bmatrix}
\end{align*}
ان کا قالبی ضرب لیتے ہیں۔
\begin{align}\label{مساوات_الجبرا_تبدیل_محل_دائیں_ہاتھ}
\bM{B}^T\bM{A}^T=\begin{bmatrix} b_{11}& b_{21}\\ b_{12}& b_{22} \end{bmatrix}\begin{bmatrix}a_{11}& a_{21}\\ a_{12}&a_{22}  \end{bmatrix}=\begin{bmatrix}b_{11}a_{11}+b_{21}a_{12} &b_{11}a_{21}+b_{21}a_{22} \\b_{12}a_{11}+b_{22}a_{12} &b_{12} a_{21}+b_{22}a_{22} \end{bmatrix}
\end{align}
چونکہ \عددی{a_{11}b_{11}=b_{11}a_{11}}، \عددی{a_{12}b_{21}=b_{21}a_{12}}،\نقطے ہیں لہٰذا مساوات \حوالہ{مساوات_الجبرا_تبدیل_محل_بائیں_ہاتھ} اور مساوات \حوالہ{مساوات_الجبرا_تبدیل_محل_دائیں_ہاتھ} کے دائیں ہاتھ آپس میں برابر ہیں لہٰذا ان کے بائیں ہاتھ بھی آپس میں برابر ہوں گے۔اس طرح  مساوات \حوالہ{مساوات_الجبرا_تبدیل_محل_قواعد}-ت ثابت ہوا۔
\انتہا{مثال}
%=====================
\جزوحصہء{مخصوص قالب}
چند اقسام  کے قالب عملی استعمال کے لحاض سے زیادہ اہم ہیں۔ان پر غور کرتے ہیں۔

\جزوحصہء{تشاکلی قالب اور منحرف تشاکلی قالب}
ایسا چکور قالب جو اپنے تبدیل محل قالب کے برابر  \عددی{\bM{A}=\bM{A}^T} ہو \اصطلاح{تشاکلی}\فرہنگ{تشاکلی!قالب}\فرہنگ{قالب!تشاکلی}\حاشیہب{symmetric}\فرہنگ{symmetric!matrix} قالب کہلاتا ہے۔ایسا قالب جو اپنے تبدیل محل قالب کے \موٹا{نفی} کے برابر  \عددی{\bM{A}=-\bM{A}^T} ہو \اصطلاح{منحرف تشاکلی}\فرہنگ{تشاکلی!منحرف}\فرہنگ{قالب!منحرف تشاکلی}\حاشیہب{skew-symmetric}\فرہنگ{skew-symmetric} قالب کہلاتا ہے۔
\begin{gather}
\begin{aligned}
\text{\RL{تشاکلی}}\quad \bM{A}&=\bM{A}^T,\quad (a_{jk}=a_{kj})\\
\text{\RL{منحرف تشاکلی}}\quad \bM{A}&=-\bM{A}^T,\quad (a_{jk}=-a_{kj}\,\, \text{اور}\,\, a_{jj}=0)
\end{aligned}
\end{gather}
%==============================
\ابتدا{مثال}\quad تشاکلی اور منحرف تشاکلی قالب\\
\عددی{\bM{A}} تشاکلی قالب ہے، \عددی{\bM{B}} منحرف تشاکلی قالب ہے جبکہ \عددی{\bM{C}} نہ تشاکلی اور نہ منحرف تشاکلی ہے۔
\begin{align*}
\text{\RL{تشاکلی}}\quad \bM{A}&=\begin{bmatrix*}[r] 2&7&5\\7&1&-2\\5&-2&3 \end{bmatrix*}\\
\text{\RL{منحرف تشاکلی}}\quad \bM{B}&=\begin{bmatrix*}[r] 0&3&-1\\-3&0&2\\1&-2&0 \end{bmatrix*}\\
\bM{C}&=\begin{bmatrix} 1&3\\2&-4 \end{bmatrix}
\end{align*}
\انتہا{مثال}
%===========================
\جزوحصہء{تکونی قالب}
\اصطلاح{بالائی تکونی قالب}\فرہنگ{تکونی!بالائی قالب}\فرہنگ{قالب!بالائی تکونی}\حاشیہب{upper triangular matrix}\فرہنگ{triangular!upper matrix} اس چکور قالب کو کہتے ہیں جس میں غیر صفر مقدار صرف مرکزی وتر اور اس سے بالائی جانب پائے جاتے ہیں جبکہ مرکزی وتر سے نیچے کی طرف تمام ارکان صفر ہوں۔اسی طرح \اصطلاح{نچلا تکونی قالب}\فرہنگ{تکونی!نچلا قالب}\فرہنگ{قالب!نچلا تکونی}\حاشیہب{lower triangular matrix}\فرہنگ{triangular!lower matrix} اس چکور قالب کو کہتے ہیں جس میں غیر صفر مقدار صرف مرکزی وتر اور مرکزی وتر کے نیچے پائے جاتے ہیں جبکہ مرکزی وتر کے بالائی جانب تمام ارکان صفر کے برابر ہوں۔
%==========================
\ابتدا{مثال}\quad بالائی تکونی اور نچلا تکونی قالب\\
\begin{align*}
\text{\RL{بالائی تکونی قالب}}\quad \begin{bmatrix*}[r]3&1\\0&2 \end{bmatrix*}, \quad \begin{bmatrix*}[r]3&-7&2\\0&0&5\\0&0&1 \end{bmatrix*}, \quad \begin{bmatrix*}[r]2&1&0\\0&3&-4\\0&0&1 \end{bmatrix*}\\
\text{\RL{نچلا تکونی قالب}}\quad \begin{bmatrix*}[r]3&0&0\\0&7&0\\-1&2&0 \end{bmatrix*}, \quad \begin{bmatrix*}[r]0&0&0\\2&0&0\\0&0&0 \end{bmatrix*}
\end{align*}
\انتہا{مثال}
%=============================
\جزوحصہء{وتری قالب}
ایسا چکور قالب جس میں غیر صفر ارکان صرف مرکزی وتر پر پائے جاتے ہوں \اصطلاح{وتری قالب}\فرہنگ{وتری قالب}\فرہنگ{قالب!وتری}\حاشیہب{diagonal matrix}\فرہنگ{diagonal!matrix}\فرہنگ{matrix!diagonal} کہلاتا ہے۔مرکزی وتر سے ہٹ کر تمام ارکان صفر ہوں گے۔

اگر وتری قالب \bM{S} کے تمام ارکان یکساں، مثلاً \عددی{c} کے برابر ہوں، تب \bM{S} \اصطلاح{غیر سمتی قالب}\فرہنگ{غیر سمتی!قالب}\فرہنگ{قالب!غیر سمتی}\حاشیہب{scalar matrix}\فرہنگ{scalar!matrix}\فرہنگ{matrix!scalar} کہلائے گا۔ کسی بھی چکور قالب \bM{A} جس کی جسامت \bM{S} کی جسامت کے برابر ہو، کا \bM{S} کے ساتھ قالبی ضرب کا حاصل، غیر سمتی مقدار \عددی{c} اور \bM{A} کے حاصل ضرب کے برابر ہو گا۔ 
\begin{align}
\bM{A}\bM{S}=\bM{S}\bM{A}=c\bM{A}
\end{align}
ایسا غیر سمتی قالب جس کے ارکان اکائی \عددی{(1)} کے برابر ہوں \اصطلاح{اکائی قالب}\فرہنگ{اکائی!قالب}\فرہنگ{قالب!اکائی}\حاشیہب{unit matrix}\فرہنگ{unit matrix} کہلاتا ہے جسے \عددی{\bM{I}_n} یا \bM{I} سے ظاہر کیا جاتا ہے۔اکائی قالب کی صورت میں درج بالا مساوات درج ذیل صورت اختیار کرتی ہے۔
\begin{align}
\bM{A}\bM{I}=\bM{I}\bM{A}=\bM{A}
\end{align}
%===================
\ابتدا{مثال}\quad وتری قالب \bM{D}، غیر سمتی قالب \bM{S} اور اکائی قالب \bM{I} \\
\begin{align*}
\bM{D}=\begin{bmatrix*}[r] 3&0&0\\0&0&0\\0&0&-4 \end{bmatrix*}, \quad \bM{S}=\begin{bmatrix*}[r] 5&0&0\\0&5&0\\0&0&5 \end{bmatrix*},\quad \bM{I}=\begin{bmatrix*}[r] 1&0&0\\0&1&0\\0&0&1\end{bmatrix*}
\end{align*}

\انتہا{مثال}
%===========================
\ابتدا{مثال}\quad کارخانے کے اخراجات\\
ایک کارخانے میں تین اقسام کے کھلونے (الف، ب اور پ) تیار ہوتے ہیں۔ایک کھلونا تیار کرنے کے اخراجات قالب \bM{A} میں دیے گئے ہیں۔ قالب \bM{B} ایک ہفتے کی پیداوار دیتا ہے۔ جمع اور جمع رات کے دن تعطیل ہوتی ہے۔ایسا قالب \bM{C} حاصل کریں جو اس ایک ہفتے میں پیدا کیے گئے کھلونوں پر خرچ اخراجات پیش کرے۔
\newline
\begin{align*}
\bM{A}=\begin{bmatrix*}[r]
\bovermat{الف}{200}& \bovermat{ب}{100}& \bovermat{پ}{50} \\
15&12&10\\
5&4&2
  \end{bmatrix*}
\begin{matrix*}[r]\text{\RL{خام مال}} \\ \text{\RL{مزدوری}} \\ \text{\RL{اضافی اخراجات}}  \end{matrix*}
, \quad \bM{B}=\begin{bmatrix*}[r] 
\bovermat{بدھ}{13}& \bovermat{منگل}{18}& \bovermat{پیر}{11}&  \bovermat{اتوار}{19}& \bovermat{ہفتہ}{20} \\
2.0& 2.2 & 2.3 & 2.1 & 2.2\\
0.8& 0.9& 1.0& 1.1& 0.9
\end{bmatrix*}\begin{matrix}\text{\RL{الف}}\\ \text{\RL{ب}} \\ \text{\RL{پ}}  \end{matrix}\end{align*}
حل:
\begin{align*}
\bM{C}=\bM{A}\bM{B}=\begin{bmatrix*}[r]
\bovermat{بدھ}{2840.0}& \bovermat{منگل}{3865.0}& \bovermat{پیر}{2480.0}&  \bovermat{اتوار}{4065.0}& \bovermat{ہفتہ}{4265.0}\\
 227.0& 305.4&202.6&321.2&335.4\\
 74.6&100.6&66.2&105.6&110.6 \end{bmatrix*}\begin{matrix}\text{\RL{خام مال}} \\ \text{\RL{مزدوری}} \\ \text{\RL{اضافی اخراجات}}  \end{matrix}
\end{align*}
\انتہا{مثال}
%===========================
%==============================
\ابتدا{مثال}\quad امکانی شماریاتی قالب\\
ایک شہر کے رقبے کا استعمال \سن{2018} میں درج  ذیل ہے۔
\begin{align*}
\text{رہائشی}\, R=\SI{60}{\percent}, \quad \text{تجارتی}\, T=\SI{25}{\percent}, \quad \text{صنعتی}\,S=\SI{15}{\percent}
\end{align*} 
پانچ سالوں میں رقبے کا استعمال تبدیل ہو گا۔اس تبدیلی کو درج ذیل \اصطلاح{امکانی شماریاتی قالب}\فرہنگ{قالب!امکانی شماریاتی}\فرہنگ{امکانی شماریاتی قالب}\حاشیہب{stochastic matrix}\فرہنگ{matrix!stochastic}\فرہنگ{stochastic matrix} \عددی{\bM{A}} دیتا ہے جو سالہا سال اس شہر کے لئے قابل استعمال ہے۔  
\newline
\newline
\begin{align*}
\bM{A}=\begin{bmatrix*}[r]
\phantom{x}&\covermat{رہائشی سے}{0.8}&&&&\covermat{تجارتی سے}{0.1}&&&&\covermat{صنعتی سے}{0}\\
&0.2&&&&0.7&&&&0.1\\
&0&&&&0.2&&&&0.9
\end{bmatrix*}
\begin{matrix}
\text{\RL{رہائشی کو منتقل}}\\
\text{\RL{تجارتی کو منتقل}}\\
\text{\RL{صنعتی کو منتقل}}
\end{matrix}
\end{align*}
درج بالا امکانی شماریاتی قالب \عددی{\bM{A}} کے تمام ارکان مثبت ہیں جبکہ ہر قطار کے ارکان کا مجموعہ اکائی کے برابر ہے (چونکہ تمام ممکنہ امکانات کا مجموعہ اکائی کے برابر ہوتا ہے)۔پانچ سال بعد \سن{2023} میں رقبے کی تقسیم درج ذیل ہو گی۔
\begin{align*}
\bM{y}=
\begin{bmatrix}
0.8&0.1&0\\0.2&0.7&0.1\\0&0.2&0.9
\end{bmatrix}
\begin{bmatrix}
60\\25\\15  
\end{bmatrix}=
\begin{bmatrix}
0.8\cdot 60+0.1\cdot 25+0\cdot 15\\
0.2\cdot 60+0.7\cdot 25+0.1\cdot 15\\
0.6\cdot 60+0.2\cdot 25+0.9\cdot 15
\end{bmatrix}=
\begin{bmatrix}
50.5\\31.0\\18.5
\end{bmatrix}
\begin{matrix}
\text{رہائشی}\\
\text{تجارتی}\\
\text{صنعتی}
\end{matrix}
\end{align*}
اس عمل کو \عددی{\bM{A}} کی مدد سے سمجھتے ہیں۔ پانچ سالوں میں \عددی{0.8} امکان ہے کہ رہائشی رقبہ، رہائشی ہی رہے گا جبکہ \عددی{0.1} امکان ہے کہ تجارتی رقبے پر رہائش ہو گی اور \عددی{0} امکان ہے کہ  صنعتی رقبے پر رہائش ہو گی۔یوں \سن{2023} میں رہائشی رقبہ درج ذیل ہو گا۔
\begin{align*}
0.8\cdot 60+0.1\cdot 25+0\cdot 15=\SI{50.5}{\percent}
\end{align*}
اس پورے عمل کو درج ذیل لکھا جا سکتا  ہے
\begin{align*}
\bM{y}=\bM{A}\bM{x}=\bM{A}\begin{bmatrix} 60&25&15 \end{bmatrix}^T
\end{align*}
جہاں \عددی{\bM{x}} \اصطلاح{سمتیہ حال}\فرہنگ{سمتیہ!حال}\فرہنگ{حال!سمتیہ}\حاشیہب{state vector}\فرہنگ{state vector} ہے جو \سن{2018} میں رقبے کی تقسیم بیان کرتا ہے۔ اسی طرح \سن{2028} اور \سن{2033} میں صورت حال بالترتیب درج ذیل ہو گی۔
\begin{align*}
\bM{z}&=\bM{A}\bM{y}=\bM{A}(\bM{A}\bM{x})=\bM{A}^2\bM{x}=\begin{bmatrix*}[l]43.50\\33.65\\22.85  \end{bmatrix*}\\
\bM{u}&=\bM{A}\bM{z}=\bM{A}(\bM{A}^2\bM{x})=\bM{A}^3\bM{x}=\begin{bmatrix*}[l]38.165\\34.540\\27.295  \end{bmatrix*}
\end{align*}
یوں \سن{2033} میں \عددی{\SI{38.165}{\percent}} علاقہ رہائشی، \عددی{\SI{34.54}{\percent}} تجارتی اور \عددی{\SI{27.295}{\percent}} صنعتی ہو گا۔یاد رہے کہ رقبہ مستقل قیمت ہے۔
\انتہا{مثال}
%==================================

\حصہء{سوالات}
%=========================
\ابتدا{سوال}\quad چکور قالب\\
ایسا چکور قالب جو تشاکلی اور منحرف تشاکلی ہو، کی صورت کیا ہو گی۔

حل:صفر قالب
\انتہا{سوال}
%===============================
سوال \حوالہ{سوال_الجبرا_قالب_عمل_الف} تا سوال \حوالہ{سوال_الجبرا_قالب_عمل_ب} میں درج ذیل قالب استعمال کریں۔
\begin{align*}
\bM{A}=\begin{bmatrix*}[r]-3&2&4\\ 0&1&2\\2&3&5  \end{bmatrix*}, \quad \bM{B}=\begin{bmatrix*}[r] 3&4&0\\ -4&-1&0\\0&0&2 \end{bmatrix*}, \quad \bM{C}=\begin{bmatrix*}[r] 3&0\\1&2\\2&-1 \end{bmatrix*}\\
\bM{a}=\begin{bmatrix*}[r] 2&-1&0 \end{bmatrix*}, \quad \bM{b}=\begin{bmatrix*}[r] 1\\3\\-2 \end{bmatrix*}
\end{align*}
%==============================
\ابتدا{سوال}\شناخت{سوال_الجبرا_قالب_عمل_الف}\quad
$\bM{A}^T, \bM{B}^T,\bM{a}^T, \bM{b}^T $\\
جوابات:
$\bM{A}^T=\begin{bmatrix*}[r] -3&0&2\\2&1&3\\4&2&5 \end{bmatrix*},\, \bM{B}^T=\begin{bmatrix*}[r] 3&-4&0\\4&-1&0\\0&0&2 \end{bmatrix*},\, \bM{a}^T=\begin{bmatrix*}[r] 2\\-1\\0 \end{bmatrix*},\, \bM{b}^T=\begin{bmatrix*}[r] 1&3&-2 \end{bmatrix*}$
\انتہا{سوال}
%==============================
\ابتدا{سوال}\quad
$\bM{A}\bM{B}, \bM{B}\bM{A}$\\
جوابات:
$\bM{A}\bM{B}=\begin{bmatrix*}[r] -17&-14&8\\-4&-1&4\\-6&5&10 \end{bmatrix*},\quad \bM{B}\bM{A}=\begin{bmatrix*}[r] -9&10&20\\12&-9&-18\\4&6&10\end{bmatrix*}$
\انتہا{سوال}
%===============================
\ابتدا{سوال}\quad
$(\bM{A}\bM{B})^T, \bM{B}^T\bM{A}^T, \bM{A}^T\bM{B}^T$\\
جوابات:
$(\bM{A}\bM{B})^T=\bM{B}^T\bM{A}^T=\begin{bmatrix*}[r] -17&-4&-6\\-14&-1&5\\8&4&10 \end{bmatrix*},\, \bM{A}^T\bM{B}^T=\begin{bmatrix*}[r] -9&12&4\\10&-9&6\\20&-18&10 \end{bmatrix*}$
\انتہا{سوال}
%==============================
\ابتدا{سوال}\quad
$\bM{A}\bM{A}^T, \bM{A}^2$\\
جوابات:
$\bM{A}\bM{A}^T=\begin{bmatrix*}[r] 29&10&20\\10&5&13\\20&13&38 \end{bmatrix*}, \, \bM{A}^2=\begin{bmatrix*}[r] 17&8&12\\4&7&12\\4&22&39 \end{bmatrix*}$
\انتہا{سوال}
%============================
\ابتدا{سوال}\quad
$\bM{B}\bM{B}^T, \bM{B}^2$\\
جوابات:
$\bM{B}\bM{B}^T=\begin{bmatrix*}[r] 25&-16&0\\-16&17&0\\0&0&4 \end{bmatrix*}, \, \bM{B}^2=\begin{bmatrix*}[r]-7&8&0\\-8&-15&0\\0&0&4  \end{bmatrix*}$
\انتہا{سوال}
%==============================
\ابتدا{سوال}\quad
$\bM{C}\bM{C}^T, \bM{B}\bM{C}$\\
جوابات:
$\bM{C}\bM{C}^T=\begin{bmatrix*}[r]9&3&6\\3&5&0\\6&0&5  \end{bmatrix*},\, \bM{B}\bM{C}=\begin{bmatrix*}[r]13&8\\-13&-2\\4&-2  \end{bmatrix*}$
\انتہا{سوال}
%======================
\ابتدا{سوال}\quad
$2\bM{A}-3\bM{B}, (2\bM{A}-3\bM{B})^T, 2\bM{A}^T-3\bM{B}^T$\\
جوابات:
$2\bM{A}-3\bM{B}=\begin{bmatrix*}[r] -15&-8&8\\12&5&4\\4&6&4 \end{bmatrix*},(2\bM{A}-3\bM{B})^T=2\bM{A}^T-3\bM{B}^T=\begin{bmatrix*}[r] -15&12&4\\-8&5&6\\8&4&4 \end{bmatrix*}$
\انتہا{سوال}
%=======================
\ابتدا{سوال}\quad
$\bM{B}\bM{a}, \bM{B}\bM{a}^T, \bM{B}\bM{b},\bM{B}\bM{b}^T $\\
جوابات:
$\bM{B}\bM{a}=\bM{B}\bM{a}^T=\begin{bmatrix*}[r]2\\-7\\0 \end{bmatrix*}, \,\bM{B}\bM{b}^T=\bM{B}\bM{b}=\begin{bmatrix*}[r] 15\\-7\\-4 \end{bmatrix*}$
\انتہا{سوال}
%===========================
\ابتدا{سوال}\quad
$\bM{A}\bM{a},\bM{A}\bM{a}^T, \bM{A}\bM{b},\bM{A}\bM{b}^T$\\
جوابات:
$\bM{A}\bM{a}=\bM{A}\bM{a}^T=\begin{bmatrix*}[r]-8\\-1\\1 \end{bmatrix*}, \bM{A}\bM{b}=\bM{A}\bM{b}^T=\begin{bmatrix*}[r]-5\\-1\\1 \end{bmatrix*}$
\انتہا{سوال}
%=============================
\ابتدا{سوال}\quad
$(\bM{A}\bM{b})^T, \bM{b}^T\bM{A}^T$\\
جوابات:
$(\bM{A}\bM{b})^T=\bM{b}^T\bM{A}^T=\begin{bmatrix*}[r] -5&-1&1 \end{bmatrix*}$

\انتہا{سوال}
%================================
\ابتدا{سوال}\quad
$\bM{A}\bM{B}\bM{C}, \bM{A}\bM{B}\bM{a}, \bM{A}\bM{B}\bM{b}$\\
جوابات:
$\begin{bmatrix*}[r] -49&-36\\-5&-6\\7&0  \end{bmatrix*}, \, \begin{bmatrix*}[r] -20\\-7\\-17 \end{bmatrix*},\, \begin{bmatrix*}[r] -75\\-15\\-11 \end{bmatrix*}$
\انتہا{سوال}
%===============================
\ابتدا{سوال}\quad
$\bM{a}\bM{b}, \bM{b}\bM{a}, \bM{a}\bM{B},\bM{B}\bM{b}$\\
جوابات:
$\begin{bmatrix*}[r] -1\end{bmatrix*}, \,\begin{bmatrix*}[r]2&-1&0\\6&-3&0\\-4&2&0  \end{bmatrix*},\,\begin{bmatrix*}[r]10&9&0  \end{bmatrix*},\, \begin{bmatrix*}[r]15\\-7\\-4 \end{bmatrix*}$ \\
\انتہا{سوال}
%===================================
\ابتدا{سوال}\شناخت{سوال_الجبرا_قالب_عمل_ب}\quad
$\bM{a}+\bM{b}, \bM{a}^T+\bM{b}, \bM{a}+\bM{b}^T$\\
جوابات:\عددی{\bM{a}+\bM{b}} ممکن نہیں ہے اور
$\bM{a}^T+\bM{b}=\begin{bmatrix*}[r] 3\\2\\-2 \end{bmatrix*}, \bM{a}+\bM{b}^T=\begin{bmatrix*}[r]3&2&-2  \end{bmatrix*}$ 
\انتہا{سوال}
%=======================
\ابتدا{سوال}
\عددی{\bM{A}\bM{B}} کو سوال \حوالہ{سوال_الجبرا_قالب_عمل_الف} میں حاصل کیا گیا ہے۔اسی کو دوبارہ \عددی{\bM{A}} کے قطار اور \عددی{\bM{B}} کے صف استعمال کرتے ہوئے دوبارہ حاصل کریں۔
\انتہا{سوال}
%===========================
\ابتدا{سوال}\شناخت{سوال_الجبرا_قالب_عمل_پ}
مساوات \حوالہ{مساوات_الجبرا_تبدیل_محل_قواعد} کو عمومی \عددی{2\times 2} قالب کے لئے ثابت کریں۔
\انتہا{سوال}
%============================
\ابتدا{سوال}\quad قانون تبادل\\
ایسا \عددی{2\times 2} قالب \عددی{\bM{B}} دریافت کریں کہ \عددی{\bM{A}\bM{B}=\bM{B}\bM{A}} ہو جہاں
 \عددی{\bM{A}=\begin{bmatrix} 2&3\\3&4 \end{bmatrix}} ہے۔

جواب:\عددی{\bM{B}=\begin{bmatrix} c&0\\0&c \end{bmatrix}}
\انتہا{سوال}
%======================
\ابتدا{سوال}
ثابت کریں  کہ کسی بھی چکور قالب \عددی{\bM{C}} کے لئے \عددی{\tfrac{1}{2}(\bM{C}+\bM{C}^T)} تشاکلی ہے جبکہ \عددی{\tfrac{1}{2}(\bM{C}-\bM{C}^T)} منحرف تشاکلی ہیں۔
\انتہا{سوال}
%=========================
\ابتدا{سوال}
درج بالا سوال کے تحت \عددی{\bM{T}=\tfrac{1}{2}(\bM{C}+\bM{C}^T)} اور \عددی{\bM{M}=\tfrac{1}{2}(\bM{C}-\bM{C}^T)} لکھا جا سکتا ہے جہاں \عددی{\bM{T}} تشاکلی اور \عددی{\bM{M}} منحرف تشاکلی قالب ہیں۔کسی بھی قالب کو تشاکل قالب اور منحرف تشاکلی قالب کا مجموعہ لکھا جا سکتا ہے۔یوں سوال \حوالہ{سوال_الجبرا_قالب_عمل_الف} تا سوال \حوالہ{سوال_الجبرا_قالب_عمل_ب} میں استعمال کیے گئے \عددی{\bM{A}} کو تشاکلی اور منحرف تشاکلی قالب کا مجموعہ لکھا جا سکتا ہے۔ان قالبوں کو دریافت کریں۔

جوابات:
$\bM{T}=\begin{bmatrix*}[r] -3&1&3\\1&1&2.5\\3&2.5&5 \end{bmatrix*}, \bM{M}=\begin{bmatrix*}[r] 0&1&1\\-1&0&-0.5\\-1&0.5&0 \end{bmatrix*}$
\انتہا{سوال}
%===========================
\ابتدا{سوال}\quad قابل تبادل\\
ثابت کریں کہ تشاکلی \عددی{\bM{A}} اور تشاکلی \عددی{\bM{B}} کا قالبی ضرب \عددی{\bM{A}\bM{B}} اس صورت تشاکلی ہو گا جب \عددی{\bM{A}} اور \عددی{\bM{B}}  آپس میں (ضرب میں) \اصطلاح{قابل تبادل}\فرہنگ{قابل تبادل}\حاشیہب{commutative}\فرہنگ{commutative} ہوں یعنی جب \عددی{\bM{A}\bM{B}=\bM{B}\bM{A}} ہو۔

جواب:
$\bM{A}\bM{B}=(\bM{A}\bM{B})^T=\bM{B}^T\bM{A}^T=\bM{B}\bM{A}$
\انتہا{سوال}
%================================
\ابتدا{سوال}
کن صورتوں میں منحرف تشاکلی قالبوں کا قالبی ضرب منحرف تشاکلی قالب دے گا؟

جواب:
$\bM{A}\bM{B}=-\bM{B}\bM{A}$
\انتہا{سوال}
%=======================================
\ابتدا{سوال}\quad امکانی شماریاتی عمل\\
ایک مشین اگر آج ٹھیک ہو تب \عددی{0.9} امکان ہے کہ وہ ایک دن بعد (کل) بھی ٹھیک ہو گا۔یوں \عددی{0.1} امکان ہے کہ وہ کل خراب ہو گا۔اسی طرح اگر مشین آج خراب ہو تب \عددی{0.4} امکان ہے کہ وہ کل بھی خراب ہو گا۔یوں \عددی{0.6} امکان ہے کہ وہ کل ٹھیک ہو گا۔آج ٹھیک اور خراب کو بالترتیب \عددی{t} اور \عددی{k} سے ظاہر کریں جبکہ ایک دن بعد انہیں \عددی{T} اور \عددی{K} سے ظاہر کریں۔ اس پیش گوئی سے امکانی شماریاتی قالب \عددی{\bM{A}} لکھیں۔ اگر آج مشین ٹھیک ہو تب دو دن بعد (پرسوں) مشین ٹھیک ہونے کا کتنا فی صد امکان ہے۔ 

جوابات: دو دن بعد \عددی{\SI{87}{\percent}} امکان ہے کہ مشین ٹھیک ہو گی۔ \quad 
$\bM{A}=
\begin{bmatrix}  
\bovermat{t}{0.9} & \bovermat{k}{0.6}\\
0.1&0.4
\end{bmatrix}
\begin{matrix} \text{T}\\ \text{K} \end{matrix}$
\انتہا{سوال}
%==========================================
\ابتدا{سوال}\quad امکانی شماریاتی عمل\\
ایک شہر کی آبادی \عددی{\num{20000}} ہے۔ایک بینک میں آج کھاتے دار کا \عددی{\SI{90}{\percent}} امکان ہے کہ وہ اگلے سال بھی اسی بینک کا کھاتے دار ہو گا جبکہ یہاں کھاتا نہ رکھنے  والے کا \عددی{\SI{1}{\percent}} امکان ہے کہ وہ اگلے سال یہاں کا کھاتا دار ہو گا۔اگر آج \عددی{1000} افراد اس بینک کے کھاتے دار ہوں تب ایک سال، دو سال اور تین سال بعد کتنے افراد یہاں کے کھاتے دار ہوں گے؟

جوابات:\عددی{1090}، \عددی{1170}، \عددی{1241}
\انتہا{سوال}
%===========================================
\ابتدا{سوال}
ایک کارخانہ لاہور، پشاور اور کراچی میں تین اشیاء الف، ب اور پ فروخت کرتا ہے۔فی کلوگرام منافع بالترتیب \عددی{8}، \عددی{10} اور \عددی{6} روپیہ ہے۔ ایک دن کی فروخت درج ذیل ہے۔
\begin{align*}
\begin{bmatrix}
\bovermat{الف}{2000}&\bovermat{ب}{3000}&\bovermat{پ}{1800}\\
2200&2800&1500\\
3200&4200&2700
\end{bmatrix}
\begin{matrix} \text{لاہور}\\ \text{پشاور} \\ \text{کراچی}  \end{matrix}
\end{align*} 
ایسا "سمتیہ منافع" \عددی{\bM{m}} دریافت کریں کہ \عددی{\bM{y}=\bM{A}\bM{m}} ہر شہر میں روزانہ کمائی دے۔

جواب:\عددی{\bM{m}=\begin{bmatrix}8&10&6  \end{bmatrix}^T} 
\انتہا{سوال}
%===================================
\ابتدا{سوال}\quad خطی تبادلہ۔ گھومنا\\
کارتیسی محدد کی \عددی{xy} سطح پر گھڑی کے سوئیوں کے گھومنے کی الٹ رخ گھومنے کو \عددی{\bM{y}=\bM{A}\bM{x}} ظاہر کرتی ہے جہاں \عددی{\bM{y}}، \عددی{\bM{A}} اور \عددی{\bM{x}} درج ذیل ہیں۔
\begin{align*}
\bM{A}=\begin{bmatrix*}[r] \cos \theta&-\sin \theta\\ \sin \theta & \cos \theta \end{bmatrix*}, \quad \bM{y}=\begin{bmatrix} y_1\\ y_2 \end{bmatrix},\quad \bM{x}=\begin{bmatrix}x_1 \\ x_2  \end{bmatrix}
\end{align*}
ثابت کریں کہ \عددی{\bM{y}=\bM{A}\bM{x}} کسی بھی سطح پر \عددی{x_1x_2} کارتیسی محدد کے نظام کو، مرکز کے گرد،  گھڑی کی الٹ رخ،  \عددی{\theta} زاویہ گھما کر نیا کارتیسی محدد \عددی{y_1y_2} دیتا ہے۔
\انتہا{سوال}
%==============================
\ابتدا{سوال}\quad خطی تبادلہ۔ گھومنا\\
درج بالا سوال میں \عددی{} زاویہ گھومنا دیکھا گیا۔ثابت کریں کہ درج ذیل قالب، مرکز کے گرد، گھڑی کی الٹ رخ،  \عددی{n\theta} زاویہ گھومنے کو ظاہر کرتا ہے۔ 
\begin{align*}
\bM{A^n}=\begin{bmatrix*}[r] \cos n\theta&-\sin n\theta\\ \sin n\theta & \cos n\theta \end{bmatrix*}
\end{align*}
\انتہا{سوال}
%==============================
\ابتدا{سوال}\quad خطی تبادلہ۔ گھومنا\\
درج بالا دو سوالات کو دیکھیں۔درج ذیل قالب، مرکز کے گرد، گھڑی کی الٹ رخ، \عددی{\alpha} اور \عددی{\beta} زاویہ گھومنے کو ظاہر کرتے ہیں۔
\begin{align*}
\bM{A}=\begin{bmatrix*}[r] \cos \alpha&-\sin \alpha\\ \sin \alpha & \cos \alpha \end{bmatrix*}, \quad \bM{B}=\begin{bmatrix*}[r] \cos \beta&-\sin \beta\\ \sin \beta & \cos \beta \end{bmatrix*}
\end{align*}
یوں باری باری  \عددی{\alpha} اور \عددی{\beta} گھومنے کو \عددی{\bM{A}\bM{B}} ظاہر کرے گا۔یوں درج ذیل ثابت کریں۔ 
\begin{align*}
\begin{bmatrix*}[r] \cos \alpha&-\sin \alpha\\ \sin \alpha & \cos \alpha \end{bmatrix*}\begin{bmatrix*}[r] \cos \beta&-\sin \beta\\ \sin \beta & \cos \beta \end{bmatrix*}=\begin{bmatrix*}[r] \cos (\alpha+\beta)&-\sin (\alpha+\beta)\\ \sin (\alpha+\beta) & \cos (\alpha+\beta) \end{bmatrix*}
\end{align*}
\انتہا{سوال}
%================================
\ابتدا{سوال}\quad خطی تبادلہ۔ گھومنا\\
خلا میں گھومنا \عددی{\bM{y}=\bM{A}\bM{x}} دیتا ہے جہاں \عددی{\bM{x}=\begin{bmatrix} x_1&x_2&x_3  \end{bmatrix}^T}، 
 \عددی{\bM{y}=\begin{bmatrix} y_1&y_2&y_3  \end{bmatrix}^T} ہیں جبکہ \عددی{\bM{A}} درج ذیل ہو سکتے ہیں۔
\begin{align*}
\begin{bmatrix*}[r] 1&0&0\\ 0& \cos \theta&-\sin \theta\\ 0&\sin \theta & \cos \theta \end{bmatrix*}, \quad \begin{bmatrix*}[r] \cos \phi& 0&-\sin \phi\\ 0&1&0\\ \sin \phi &0& \cos \phi \end{bmatrix*}, \quad \begin{bmatrix*}[r] \cos \gamma&-\sin \gamma&0\\ \sin \gamma & \cos \gamma &0\\0&0&1\end{bmatrix*}
\end{align*}
کیا آپ ذہن میں اس عمل کو دیکھ پاتے ہیں؟
\انتہا{سوال}
%================================

\حصہ{خطی مساوات کے نظام۔ گاوسی حدف}
