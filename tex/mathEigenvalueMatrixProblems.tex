\باب{خطی الجبرا: آئگنی قدر مسائل قالب}
آئگنی قدر مسائل درج ذیل سمتی مساوات پر مبنی ہیں جہاں \عددی{\bM{A}} چکور قالب، \عددی{\bM{x}} نا معلوم سمتیہ اور \عددی{\lambda} نا معلوم غیر سمتیہ ہے۔
\begin{align}\label{مساوات_آئگنی_مسئلہ_الف}
\bM{A}\bM{x}=\lambda \bM{x}
\end{align}
آئگنی قدر مسائل میں ہمیں وہ \عددی{\lambda} اور \عددی{\bM{x}} درکار ہیں جو درج بالا مساوات پر پورا اترتے ہوں۔ \عددی{\lambda} کی ہر قیمت کے لئے 
 \عددی{\bM{x}=\bM{0}} مساوات \حوالہ{مساوات_آئگنی_مسئلہ_الف} کا غیر اہم صفر حل ہے۔ ہم اس غیر اہم صفر حل میں دلچسپی نہیں رکھتے ہیں لہٰذا ہم غیر صفر حل \عددی{\bM{x} \ne \bM{0}} جاننا چاہیں گے۔

\عددی{\lambda} کی وہ قیمتیں جو مساوات \حوالہ{مساوات_آئگنی_مسئلہ_الف} پر پورا اترتے ہیں \عددی{\bM{A}} کے \اصطلاح{آئگنی اقدار}\فرہنگ{آئگنی اقدار}\حاشیہب{eigenvalues}\فرہنگ{eigenvalues} کہلاتے ہیں اور وہ \عددی{\bM{x}} جو مساوات \حوالہ{مساوات_آئگنی_مسئلہ_الف} پر پورا اترتے ہیں \عددی{\bM{A}} کے \اصطلاح{آئگنی سمتیات}\فرہنگ{آئگنی سمتیات}\حاشیہب{eigenfunctions}\فرہنگ{eigenfunctions} کہلاتے ہیں۔

اس معصوم نظر آنے والا سمتی مساوات کے اندر حیران کن تفصیل چھپی ہے۔آئگنی قدر مسائل انجینئری، طبیعیات، ریاضی، حیاتیات، ماحولیاتی سائنس، شہری منصوبہ بندی،  معاشیات، نفسیات اور دیگر شعبوں میں عموماً درپیش آتے ہیں۔آپ کو یقیناً ان سے زندگی میں واسطہ پڑے گا۔

%=============
\حصہ{آئگنی قدر مسائل قالب۔ آئگنی اقدار اور آئگنی سمتیات کا حصول}
درج ذیل پر غور کریں جہاں غیر صفر سمتیہ اور چکور قالب کے ضرب دکھائے گئے ہیں۔
\begin{align*}
\begin{bmatrix}6&3\\4&7  \end{bmatrix}\begin{bmatrix} 5\\1 \end{bmatrix}=\begin{bmatrix}33\\27  \end{bmatrix},\quad \begin{bmatrix}6&3\\4&7  \end{bmatrix}\begin{bmatrix} 3\\4 \end{bmatrix}=\begin{bmatrix}30\\40  \end{bmatrix}
\end{align*} 
بائیں ہاتھ کی ضرب میں ہمیں مکمل طور پر نیا سمتیہ حاصل ہوتا ہے جس کی لمبائی اور سمت ابتدائی سمتیہ کی لمبائی اور سمت  سے مختلف ہیں۔عموماً  سمتیہ کو چکور قالب سے ضرب دینے  سے مکمل طور پر مختلف سمتیہ حاصل ہوتا ہے۔دائیں ہاتھ کی ضرب میں حاصل سمتیہ کو درج ذیل لکھا جا سکتا ہے 
\begin{align*}
\begin{bmatrix}30\\40  \end{bmatrix}=10\begin{bmatrix}3\\4  \end{bmatrix}
\end{align*}
یعنی حاصل سمتیہ اور ابتدائی سمتیہ کی سمتیں ایک جیسی ہیں جبکہ حاصل سمتیہ کی لمبائی ابتدائی سمتیہ کی لمبائی کے دس گنّا ہے جس کو \عددی{\lambda=10} لکھا جائے گا۔چکور قالب \عددی{\bM{A}} کے لحاض سے  ایسے \عددی{\lambda} اور غیر صفر سمتیات کا حصول اس باب کا مرکزی مضمون ہے۔  

آئیں درج بالا مشاہدے کو دستوری شکل دیں۔فرض کریں کہ \عددی{\bM{A}=[a_{jk}]} غیر صفر \عددی{n\times n} جسامت کا چکور قالب ہے۔اب درج ذیل سمتی مساوات پر غور کریں۔
\begin{align}\label{مساوات_آئگنی_مسئلہ_ب}
\bM{A}\bM{x}=\lambda \bM{x}
\end{align}
ان \عددی{\lambda} اور غیر صفر \عددی{\bM{x}} کے حصول کے مسئلے کو، جو مساوات \حوالہ{مساوات_آئگنی_مسئلہ_ب} پر پورا اترے ہوں،  \اصطلاح{آئگنی قدر مسئلہ}\فرہنگ{آئگنی قدر مسئلہ}\فرہنگ{مسئلہ!آئگنی قدر} کہتے ہیں۔

یہاں توجہ دیں کہ \عددی{\bM{A}} دیا گیا چکور قالب ہے جبکہ \عددی{\lambda} نا معلوم غیر سمتیہ اور \عددی{\bM{x}} نا معلوم سمتیہ ہے۔ہم وہ \عددی{\lambda} اور \عددی{\bM{x}} حاصل کرنا چاہتے ہیں جو مساوات \حوالہ{مساوات_آئگنی_مسئلہ_ب} پر پورا اترتے ہوں۔ جیومیٹریائی طور پر ہم وہ سمتیات \عددی{\bM{x}} حاصل کرنا چاہتے ہیں جنہیں \عددی{\bM{A}} سے ضرب دینا ایسا ہی ہے جیسے ان سمتیوں کو غیر سمتی \عددی{\lambda} سے ضرب دیا جائے یعنی کہ \عددی{\bM{A}\bM{x}} اور \عددی{\bM{x}} راست تناسب ہوں۔یوں مثبت \عددی{\lambda} کی صورت میں ابتدائی اور حاصل سمتیات کی سمتیں ایک جیسی ہوں گی جبکہ منفی \عددی{\lambda} کی صورت میں ان کی سمتیں آپس میں الٹ ہوں گی۔ (باب کی شروع میں سادہ مثال سے اس کی وضاحت کی گئی ہے۔)

 \عددی{\lambda} کی وہ مخصوص قیمت جس کے لئے مساوات \حوالہ{مساوات_آئگنی_مسئلہ_ب} کے غیر صفر \عددی{\bM{x}\ne \bM{0}} حل موجود ہوں  \عددی{\bM{A}} کی \اصطلاح{آئگنی قدر}\فرہنگ{آئگنی!قدر}\حاشیہب{eigenvalue}\فرہنگ{eigenvalue} کہلاتی ہے اور مطابقتی سمتیات \عددی{\bM{x}}،  اس \عددی{\lambda} کے لحاض سے قالب \عددی{\bM{A}} کے  \اصطلاح{آئگنی سمتیات}\فرہنگ{آئگنی!سمتیات}\حاشیہب{eigenvectors}\فرہنگ{eigenvectors} یا \اصطلاح{امتیازی سمتیات}\فرہنگ{امتیازی!سمتیات}\حاشیہب{characteristic vectors}\فرہنگ{characteristic vectors} کہلاتے ہیں۔\عددی{\bM{A}} کے تمام آئگنی اقدار کو \عددی{\bM{A}} کا \اصطلاح{طیف}\فرہنگ{طیف}\حاشیہب{spectrum}\فرہنگ{spectrum} کہتے ہیں۔طیف میں کم سے کم ایک عدد آئگنی قدر اور زیادہ سے زیادہ \عددی{n} مختلف آئگنی اقدار ہو سکتے ہیں۔ آئگنی اقدار کی سب سے  زیادہ حتمی قیمت کو \عددی{\bM{A}} کا \اصطلاح{رداس طیف}\فرہنگ{رداس!طیف}\فرہنگ{طیف!رداس}\حاشیہب{spectral radius}\فرہنگ{spectral radius} کہتے ہیں۔

آئگنی قدر مسئلے کا حل چند مثالوں کی مدد سے سیکھتے ہیں۔
%=================
\ابتدا{مثال}\quad آئگنی اقدار اور آئگنی سمتیات کا حصول\\
درج ذیل قالب کے آئگنی اقدار اور آئگنی سمتیات دریافت کریں۔
\begin{align*}
\bM{A}=\begin{bmatrix*}[r] -5&2\\2&-2 \end{bmatrix*}
\end{align*}
\انتہا{مثال}
%=====================
