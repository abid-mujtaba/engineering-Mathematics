\باب{خطی الجبرا: آئگنی قدر مسائل قالب}
آئگنی قدر مسائل درج ذیل سمتی مساوات پر مبنی ہیں جہاں \عددی{\bM{A}} چکور قالب، \عددی{\bM{x}} نا معلوم سمتیہ اور \عددی{\lambda} نا معلوم غیر سمتیہ ہے۔
\begin{align}\label{مساوات_آئگنی_مسئلہ_الف}
\bM{A}\bM{x}=\lambda \bM{x}
\end{align}
آئگنی قدر مسائل میں ہمیں وہ \عددی{\lambda} اور \عددی{\bM{x}} درکار ہیں جو درج بالا مساوات پر پورا اترتے ہوں۔ \عددی{\lambda} کی ہر قیمت کے لئے 
 \عددی{\bM{x}=\bM{0}} مساوات \حوالہ{مساوات_آئگنی_مسئلہ_الف} کا غیر اہم صفر حل ہے۔ ہم اس غیر اہم صفر حل میں دلچسپی نہیں رکھتے ہیں لہٰذا ہم غیر صفر حل \عددی{\bM{x} \ne \bM{0}} جاننا چاہیں گے۔

\عددی{\lambda} کی وہ قیمتیں جو مساوات \حوالہ{مساوات_آئگنی_مسئلہ_الف} پر پورا اترتے ہیں \عددی{\bM{A}} کے \اصطلاح{آئگنی اقدار}\فرہنگ{آئگنی اقدار}\حاشیہب{eigenvalues}\فرہنگ{eigenvalues} کہلاتے ہیں اور وہ \عددی{\bM{x}} جو مساوات \حوالہ{مساوات_آئگنی_مسئلہ_الف} پر پورا اترتے ہیں \عددی{\bM{A}} کے \اصطلاح{آئگنی سمتیات}\فرہنگ{آئگنی سمتیات}\حاشیہب{eigenfunctions}\فرہنگ{eigenfunctions} کہلاتے ہیں۔

اس معصوم نظر آنے والا سمتی مساوات کے اندر حیران کن تفصیل چھپی ہے۔آئگنی قدر مسائل انجینئری، طبیعیات، ریاضی، حیاتیات، ماحولیاتی سائنس، شہری منصوبہ بندی،  معاشیات، نفسیات اور دیگر شعبوں میں عموماً درپیش آتے ہیں۔آپ کو یقیناً ان سے زندگی میں واسطہ پڑے گا۔

%=============
\حصہ{آئگنی قدر مسائل قالب۔ آئگنی اقدار اور آئگنی سمتیات کا حصول}\شناخت{حصہ_آئگنی_اقدار_اور_سمتیات}
درج ذیل پر غور کریں جہاں غیر صفر سمتیہ اور چکور قالب کے ضرب دکھائے گئے ہیں۔
\begin{align}\label{مساوات_آئگنی_پہلی_مثال}
\begin{bmatrix}6&3\\4&7  \end{bmatrix}\begin{bmatrix} 5\\1 \end{bmatrix}=\begin{bmatrix}33\\27  \end{bmatrix},\quad \begin{bmatrix}6&3\\4&7  \end{bmatrix}\begin{bmatrix} 3\\4 \end{bmatrix}=\begin{bmatrix}30\\40  \end{bmatrix}
\end{align} 
بائیں ہاتھ کی ضرب میں ہمیں مکمل طور پر نیا سمتیہ حاصل ہوتا ہے جس کی لمبائی اور سمت ابتدائی سمتیہ کی لمبائی اور سمت  سے مختلف ہیں۔عموماً  سمتیہ کو چکور قالب سے ضرب دینے  سے مکمل طور پر مختلف سمتیہ حاصل ہوتا ہے۔دائیں ہاتھ کی ضرب میں حاصل سمتیہ کو درج ذیل لکھا جا سکتا ہے 
\begin{align*}
\begin{bmatrix}30\\40  \end{bmatrix}=10\begin{bmatrix}3\\4  \end{bmatrix}
\end{align*}
یعنی حاصل سمتیہ اور ابتدائی سمتیہ کی سمتیں ایک جیسی ہیں جبکہ حاصل سمتیہ کی لمبائی ابتدائی سمتیہ کی لمبائی کے دس گنّا ہے جس کو \عددی{\lambda=10} لکھا جائے گا۔چکور قالب \عددی{\bM{A}} کے لحاض سے  ایسے \عددی{\lambda} اور غیر صفر سمتیات کا حصول اس باب کا مرکزی مضمون ہے۔  

آئیں درج بالا مشاہدے کو دستوری شکل دیں۔فرض کریں کہ \عددی{\bM{A}=[a_{jk}]} غیر صفر \عددی{n\times n} جسامت کا چکور قالب ہے۔اب درج ذیل سمتی مساوات پر غور کریں۔
\begin{align}\label{مساوات_آئگنی_مسئلہ_ب}
\bM{A}\bM{x}=\lambda \bM{x}
\end{align}
ان \عددی{\lambda} اور غیر صفر \عددی{\bM{x}} کے حصول کے مسئلے کو، جو مساوات \حوالہ{مساوات_آئگنی_مسئلہ_ب} پر پورا اترے ہوں،  \اصطلاح{آئگنی قدر مسئلہ}\فرہنگ{آئگنی قدر مسئلہ}\فرہنگ{مسئلہ!آئگنی قدر} کہتے ہیں۔

یہاں توجہ دیں کہ \عددی{\bM{A}} دیا گیا چکور قالب ہے جبکہ \عددی{\lambda} نا معلوم غیر سمتیہ اور \عددی{\bM{x}} نا معلوم سمتیہ ہے۔ہم وہ \عددی{\lambda} اور \عددی{\bM{x}} حاصل کرنا چاہتے ہیں جو مساوات \حوالہ{مساوات_آئگنی_مسئلہ_ب} پر پورا اترتے ہوں۔ جیومیٹریائی طور پر ہم وہ سمتیات \عددی{\bM{x}} حاصل کرنا چاہتے ہیں جنہیں \عددی{\bM{A}} سے ضرب دینا ایسا ہی ہے جیسے ان سمتیوں کو غیر سمتی \عددی{\lambda} سے ضرب دیا جائے یعنی کہ \عددی{\bM{A}\bM{x}} اور \عددی{\bM{x}} راست تناسب ہوں۔یوں مثبت \عددی{\lambda} کی صورت میں ابتدائی اور حاصل سمتیات کی سمتیں ایک جیسی ہوں گی جبکہ منفی \عددی{\lambda} کی صورت میں ان کی سمتیں آپس میں الٹ ہوں گی۔ (باب کی شروع میں سادہ مثال سے اس کی وضاحت کی گئی ہے۔)

 \عددی{\lambda} کی وہ مخصوص قیمت جس کے لئے مساوات \حوالہ{مساوات_آئگنی_مسئلہ_ب} کے غیر صفر \عددی{\bM{x}\ne \bM{0}} حل موجود ہوں  \عددی{\bM{A}} کی \اصطلاح{آئگنی قدر}\فرہنگ{آئگنی!قدر}\حاشیہب{eigenvalue}\فرہنگ{eigenvalue} کہلاتی ہے اور مطابقتی سمتیات \عددی{\bM{x}}،  اس \عددی{\lambda} کے لحاض سے قالب \عددی{\bM{A}} کے  \اصطلاح{آئگنی سمتیات}\فرہنگ{آئگنی!سمتیات}\حاشیہب{eigenvectors}\فرہنگ{eigenvectors} یا \اصطلاح{امتیازی سمتیات}\فرہنگ{امتیازی!سمتیات}\حاشیہب{characteristic vectors}\فرہنگ{characteristic vectors} کہلاتے ہیں۔\عددی{\bM{A}} کے تمام آئگنی اقدار کو \عددی{\bM{A}} کا \اصطلاح{طیف}\فرہنگ{طیف}\حاشیہب{spectrum}\فرہنگ{spectrum} کہتے ہیں۔طیف میں کم سے کم ایک عدد آئگنی قدر اور زیادہ سے زیادہ \عددی{n} مختلف آئگنی اقدار ہو سکتے ہیں۔ آئگنی اقدار کی سب سے  زیادہ حتمی قیمت کو \عددی{\bM{A}} کا \اصطلاح{رداس طیف}\فرہنگ{رداس!طیف}\فرہنگ{طیف!رداس}\حاشیہب{spectral radius}\فرہنگ{spectral radius} کہتے ہیں۔

آئگنی قدر مسئلے کا حل چند مثالوں کی مدد سے سیکھتے ہیں۔
%=================
\ابتدا{مثال}\شناخت{مثال_آئگنی_اقدار_الف}\quad آئگنی اقدار اور آئگنی سمتیات کا حصول\\
درج ذیل قالب کے آئگنی اقدار اور آئگنی سمتیات قدم بہ قدم دریافت کرتے ہیں۔
\begin{align*}
\bM{A}=\begin{bmatrix*}[r] -5&2\\2&-2 \end{bmatrix*}
\end{align*}
پہلے آئگنی اقدار دریافت کیے جاتے ہیں۔مساوات \حوالہ{مساوات_آئگنی_مسئلہ_ب} درج ذیل ہو گا۔
\begin{gather*}
\bM{A}\bM{x}=\begin{bmatrix*}[r] -5&2\\2&-2 \end{bmatrix*}\begin{bmatrix} x_1\\x_2 \end{bmatrix}=\lambda \begin{bmatrix} x_1\\x_2 \end{bmatrix};\quad \implies \quad \begin{aligned}-5x_1+2x_2&=\lambda x_1\\ 2x_1-2x_2&=\lambda x_2\end{aligned}
\end{gather*}
تمام اجزاء کو ایک طرف منتقل کرتے ہوئے
\begin{gather}
\begin{aligned}\label{مساوات_آئگنی_مثال_الف}
(-5-\lambda)x_1+2x_2&=0\\
2x_2+(-2-\lambda)x_2&=0
\end{aligned}
\end{gather}
قالبی صورت میں لکھتے ہیں۔
\begin{align*}
(\bM{A}-\lambda \bM{I})\bM{x}=\bM{0}
\end{align*}
مسئلہ \حوالہ{مساوات_الجبرا_مسئلہ_کریمر_عمومی} کے تحت اس متجانس نظام کا غیر صفر اہم حل \عددی{\bM{x} \ne \bM{0}} (قالب \عددی{\bM{A}} کا آئگنی سمتیہ جس کی ہمیں تلاش ہے)  اس صورت ممکن ہو گا جب عددی سر قالب کا مقطع صفر کے برابر ہو گا۔
\begin{align*}
D(\lambda)=\begin{vmatrix} -5-\lambda& 2\\ 2 & -2-\lambda \end{vmatrix}=(-5-\lambda)(-2-\lambda)-4=\lambda^2+7\lambda+6=0
\end{align*}
ہم \عددی{D(\lambda)} کو \عددی{\bM{A}} کی \اصطلاح{امتیازی مقطع} جبکہ اس کی پھیلی  ہوئی صورت کو \اصطلاح{امتیازی کثیر رکنی} اور \عددی{D(\lambda)=0} کو \اصطلاح{امتیازی مساوات} کہتے ہیں۔اس دو درجی الجبرائی مساوات کے حل \عددی{\lambda_1=-1} اور \عددی{\lambda_2=-6} ہیں جو \عددی{\bM{A}} کے آئگنی اقدار ہیں۔

\عددی{\lambda_1=-1} کا مطابقتی آئگنی سمتیہ مساوات \حوالہ{مساوات_آئگنی_مثال_الف} میں \عددی{\lambda=\lambda_1=-1} پر کرتے ہوئے حاصل کرتے ہیں۔
\begin{gather*}
\begin{aligned}
[-5-(-1)]x_1+2x_2&=0\\
2x_2+[-2-(-1)]x_2&=0
\end{aligned}\quad \implies \quad \begin{aligned} -4x_1+2x_2&=0\\2x_2-x_2&=0 \end{aligned}
\end{gather*}
ان میں سے کسی بھی مساوات کو حل کرتے ہوئے  \عددی{x_2=2x_1} ملتا ہے جس کو استعمال کرتے ہوئے متعدد متوازی آئگنی سمتیات حاصل کیے جا سکتے ہیں۔یوں \عددی{x_1} (یا \عددی{x_2}) کی کوئی بھی قیمت چن کر\عددی{x_2} (\عددی{x_1}) حاصل کرتے ہوئے آئگنی سمتیہ حاصل ہو گا۔ہم \عددی{x_1=1} چن کر \عددی{x_2=2} حاصل کرتے ہیں اور یوں \عددی{\bM{x}_1=[1 \quad 2]^T} ہو گا۔اس جواب کی تصدیق کرتے ہیں۔
\begin{align*}
\bM{A}\bM{x}_1=\begin{bmatrix*}[r] -5&2\\2&-2 \end{bmatrix*}\begin{vmatrix} 1\\2 \end{vmatrix}=\begin{bmatrix}-1\\-2  \end{bmatrix}=(-1)\bM{x}_1=\lambda_1\bM{x}_1
\end{align*}
\عددی{\lambda_2=-6} کا مطابقتی آئگنی سمتیہ مساوات \حوالہ{مساوات_آئگنی_مثال_الف} میں \عددی{\lambda=\lambda_1=-6} پر کرتے ہوئے حاصل کرتے ہیں۔
\begin{gather*}
\begin{aligned}
[-5-(-6)]x_1+2x_2&=0\\
2x_2+[-2-(-6)]x_2&=0
\end{aligned}\quad \implies \quad \begin{aligned} x_1+2x_2&=0\\2x_2+4x_2&=0 \end{aligned}
\end{gather*}
ان میں سے کسی بھی مساوات کو حل کرتے ہوئے \عددی{x_2=-\tfrac{1}{2}x_1} ملتا ہے۔یوں \عددی{x_1=2} چنتے ہوئے  \عددی{x_2=-1} ملتا ہے لہٰذا \عددی{\lambda_2=-6} کا مطابقتی آئگنی سمتیہ \عددی{\bM{x}_2=[2\quad -1]^T} ہو گا۔اس کی تصدیق کرتے ہیں۔
\begin{align*}
\bM{A}\bM{x}_2=\begin{bmatrix*}[r] -5&2\\2&-2 \end{bmatrix*}\begin{vmatrix} 2\\-1 \end{vmatrix}=\begin{bmatrix}-12\\6  \end{bmatrix}=(-6)\bM{x}_2=\lambda_2\bM{x}_2
\end{align*}
آپ حصہ \حوالہ{حصہ_آئگنی_اقدار_اور_سمتیات} کے آغاز میں مساوات \حوالہ{مساوات_آئگنی_پہلی_مثال} میں دیے گئے مثال کو حل کرتے ہوئے  آئگنی اقدار \عددی{10}، \عددی{3} اور  مطابقتی آئگنی سمتیات \عددی{[3\quad 4]^T}، \عددی{[-1\quad 1 ]^T} حاصل کریں۔
\انتہا{مثال}
%=====================

درج بالا مثال میں استعمال کی گئی ترکیب کی عمومی صورت پیش کرتے ہیں۔ مساوات \حوالہ{مساوات_آئگنی_مسئلہ_ب} کو اجزاء کی صورت میں درج ذیل لکھا جا سکتا ہے۔
\begin{gather}
\begin{aligned}\label{مساوات_آئگنی_متجانس_نظام_الف}
a_{11}x_1+\cdots +a_{1n}x_n&=\lambda x_1\\
a_{21}x_1+\cdots +a_{2n}x_n&=\lambda x_2\\
\vdots&\\
a_{n1}x_1+\cdots +a_{nn}x_n&=\lambda x_n\\
\end{aligned}
\end{gather}
تمام اجزاء کو بائیں ہاتھ منتقل کرتے ہیں۔
\begin{gather}
\begin{aligned}\label{مساوات_آئگنی_متجانس_نظام_ب}
(a_{11}-\lambda)x_1+a_{12}x_2+\cdots +a_{1n}x_n&=0\\
a_{21}x_1+(a_{22}-\lambda)x_2+\cdots +a_{2n}x_n&=0\\
\vdots&\\
a_{n1}x_1+a_{n2}x_2+\cdots +(a_{nn}-\lambda)x_n&=0\\
\end{aligned}
\end{gather}
اس کو قالب کی صورت میں لکھتے ہیں۔
\begin{align}\label{مساوات_آئگنی_متجانس_نظام_پ}
(\bM{A}-\lambda \bM{I})\bM{x}=0
\end{align}
مسئلہ کریمر (مسئلہ \حوالہ{مساوات_الجبرا_مسئلہ_کریمر_عمومی}) کے تحت درج بالا متجانس نظام کا غیر صفر حل صرف اور صرف اس صورت ممکن ہو گا جب  اس کے عددی سر قالب کا مقطع صفر کے برابر ہو:
\begin{align}\label{مساوات_آئگنی_متجانس_نظام_ت}
D(\lambda)=\begin{vmatrix} a_{11}-\lambda & a_{12}&\cdots & a_{1n} \\ a_{21}& a_{22}-\lambda&\cdots & a_{2n}\\ \vdots \\ a_{n1}& a_{n2}&\cdots & a_{nn}-\lambda \end{vmatrix}=0
\end{align} 
\عددی{\bM{A}-\lambda\bM{I}} کو \عددی{\bM{A}} کا \اصطلاح{امتیازی قالب}\فرہنگ{امتیازی!قالب}\فرہنگ{قالب!امتیازی}  جبکہ \عددی{D(\lambda)}  کو \عددی{\bM{A}} کا \اصطلاح{امتیازی مقطع}\فرہنگ{امتیازی!مقطع}\فرہنگ{مقطع!امتیازی} کہتے ہیں۔ مساوات \حوالہ{مساوات_آئگنی_متجانس_نظام_ت} کو \عددی{\bM{A}} کی \اصطلاح{امتیازی مساوات}\فرہنگ{امتیازی!مساوات}\فرہنگ{مساوات!امتیازی} کہتے ہیں۔مساوات \حوالہ{مساوات_آئگنی_متجانس_نظام_ت} کو پھیلا کر \عددی{\bM{A}} کی امتیازی کثیر رکنی حاصل ہو گی۔

مساوات \حوالہ{مساوات_آئگنی_متجانس_نظام_ت} کو پھیلا کر حاصل کثیر رکنی میں \عددی{\lambda^n} بلند تر طاقت ہے لہٰذا اس سے زیادہ سے زیادہ \عددی{n} مختلف آئگنی اقدار حاصل ہو سکتے ہیں۔ 
%========================

\ابتدا{مسئلہ}\شناخت{مسئلہ_آئگنی_اقدار}\quad آئگنی اقدار\\
چکور قالب \عددی{\bM{A}} کے آئگنی اقدار \عددی{\bM{A}} کے امتیازی مساوات \حوالہ{مساوات_آئگنی_متجانس_نظام_ت} سے حاصل ہوں گے۔\\
یوں \عددی{n\times n} قالب کی کم سے کم ایک عدد آئگنی قدر اور زیادہ سے زیادہ \عددی{n} مختلف  آئگنی اقدار ہو سکتے ہیں۔
\انتہا{مسئلہ}
%=============================

\عددی{n} کی بڑی قیمت کی صورت میں آئگنی اقدار عموماً ترکیب نیوٹن یا کسی اور اعدادی ترکیب سے حاصل کئے جائیں گے۔

آئگنی اقدار پہلے حاصل کیے جاتے ہیں۔باری باری ان آئگنی قدر کو مساوات \حوالہ{مساوات_آئگنی_متجانس_نظام_ب} کے نظام میں پر کرتے ہوئے مطابقتی آئگنی سمتیہ ( گاوسی اسقاط کی مدد سے) حاصل کیا جاتا ہے۔

آئگنی سمتیات درج ذیل خصوصیات رکھتے ہیں۔
%=================

\ابتدا{مسئلہ}\شناخت{مسئلہ_آئگنی_سمتیہ_فضا}\quad آئگنی سمتیات اور آئگنی فضا\\
اگر قالب \عددی{\bM{A}} کے کسی ایک آئگنی قدر \عددی{\lambda} کے مطابقتی آئگنی سمتیات \عددی{\bM{w}} اور \عددی{\bM{x}} ہوں تب \عددی{\bM{w}+\bM{x}} (بشرطیکہ \عددی{\bM{w} \ne -\bM{x}} ہو) اور \عددی{k\bM{x}} جہاں \عددی{k\ne 0} ہے بھی اس \عددی{\lambda} کے مطابقتی بھی آئگنی سمتیات ہوں گے۔

یوں کسی ایک آئگنی قدر کے مطابقتی آئگنی سمتیات اور \عددی{\bM{0}} سمتیہ مل کر فضا بناتے ہیں جس کو اس \عددی{\lambda} کے لئے \عددی{\bM{A}} کی مطابقتی  \اصطلاح{آئگنی فضا} کہتے ہیں۔
\انتہا{مسئلہ}
%============================

\ابتدا{ثبوت}
\عددی{\bM{A}\bM{w}=\lambda \bM{w}} اور \عددی{\bM{A}\bM{x}=\lambda \bM{x}} سے مراد درج ذیل ہے
\begin{align*}
\bM{A}(\bM{w}+\bM{x})=\bM{A}\bM{w}+\bM{A}\bM{x}=\lambda \bM{w}+\lambda\bM{x}=\lambda(\bM{w}+\bM{x})
\end{align*}
اور \عددی{\bM{A}(k\bM{w})=k(\bM{A}\bM{w})=k(\lambda \bM{w})=\lambda(k\bM{w})} ہے لہٰذا \عددی{\bM{A}(k\bM{w}+l\bM{x})=\lambda(k\bM{w}+l\bM{x})} ہو گا۔
\انتہا{ثبوت}
%===============================

آئگنی سمتیہ کو معیار سے تقسیم کرتے ہوئے \اصطلاح{معیاری آئگنی سمتیہ}  یعنی \اصطلاح{اکائی آئگنی سمتیہ} حاصل کیا جا سکتا ہے۔مثلاً مثال \حوالہ{مثال_آئگنی_اقدار_الف} میں \عددی{\bM{x}_1=[1 \quad 2]^T} کی لمبائی \عددی{\norm{\bM{x}_1}=\sqrt{1^2+2^2}=\sqrt{5}} ہے جس سے معیاری آئگنی سمتیہ (اکائی آئگنی سمتیہ) \عددی{[\tfrac{1}{\sqrt{5}}\quad \tfrac{2}{\sqrt{5}}]^T} حاصل ہوتا ہے۔ 
%===============================

\ابتدا{مثال}\quad متعدد آئگنی سمتیات\\
درج ذیل قالب کے آئگنی اقدار اور آئگنی سمتیات دریافت کریں۔
\begin{align*}
\bM{A}=\begin{bmatrix*}[r]-2&2&-3\\2&1&-6\\-1&-2&0  \end{bmatrix*}
\end{align*}
حل:اس قالب کی امتیازی مساوات درج ذیل ہے
\begin{align*}
-\lambda^3-\lambda^2+21\lambda+45=0
\end{align*} 
جس سے \عددی{\bM{A}} کے جذر \عددی{\lambda_1=5} اور \عددی{\lambda_2=\lambda_3=-3} ملتے ہیں۔(بلند درجی مساوات کا خط کھینچ کر اس کے جذر با آسانی حاصل کیے جاتے ہیں)۔نظام \عددی{(\bM{A}-\lambda\bM{I})\bM{x}=\bM{0}} میں \عددی{\lambda=\lambda_1=5} پر کرتے ہوئے درج ذیل مطابقتی  امتیازی قالب ملتا  ہے جس کی تخفیف شدہ صورت گاوسی اسقاط کی مدد سے حاصل کی گئی ہے
\begin{align*}
\bM{A}-\lambda\bM{I}=\bM{A}-5\bM{I}=\begin{bmatrix*}[r] -7&2&-3\\2&-4&-6\\-1&-2&-5 \end{bmatrix*}\quad \stackrel{\text{\RL{گاوسی اسقاط}}}{\implies} \quad \begin{bmatrix*}[r]  -7&2&-3\\0&-\frac{24}{7}&-\frac{48}{7}\\0&0&0\end{bmatrix*}
\end{align*}
جس کا درجہ دو (\عددی{2}) ہے۔یوں \عددی{-\tfrac{24}{7}x_2-\tfrac{48}{7}x_3=0} میں \عددی{x_3=-1} چنتے ہوئے \عددی{x_2=2} حاصل ہوتا ہے۔ ان قیمتوں کو  \عددی{-7x_1+2x_2-3x_3=0} میں پر کرتے ہوئے \عددی{x_1=1} ملتا ہے۔یوں \عددی{\bM{x}_1=[1\,\,\,2\,\,\,-1]^T} قالب \عددی{\bM{A}} کا آئگنی قدر  \عددی{\lambda=5} کا مطابقتی آئگنی سمتیہ   ہے۔

\عددی{\lambda=-3} سے درج ذیل امتیازی قالب ملتا ہے  جس کی تخفیف شدہ صورت  گاوسی اسقاط کی مدد سے حاصل کی گئی ہے۔ 
\begin{align*}
\bM{A}-\lambda \bM{I}=\bM{A}+3\bM{I}=\begin{bmatrix*}[r] 1&2&-3\\2&4&-6\\-1&-2&3 \end{bmatrix*}\quad \stackrel{\text{\RL{گاسقی اسقاط}}}{\implies} \quad \begin{bmatrix*}[r] 1&2&3\\0&0&0\\0&0&0 \end{bmatrix*}
\end{align*}
\عددی{x_1+2x_2-3x_3=0} سے \عددی{x_1=-2x_2+3x_3} لکھا جا سکتا ہے۔\عددی{x_2=1} چنتے ہوئے \عددی{x_3=0} ملتا ہے  جبکہ \عددی{x_2=0} چنتے ہوئے \عددی{x_3=1} ملتا ہے۔اس طرح \عددی{\lambda=-3} کے مطابقتی درج ذیل دو مختلف آئگنی سمتیات حاصل ہوتے ہیں۔
\begin{align*}
\bM{x}_2=\begin{bmatrix*}[r] -2\\1\\0 \end{bmatrix*}, \quad \bM{x}_3=\begin{bmatrix*}[r] 3\\0\\1 \end{bmatrix*}
\end{align*}
\انتہا{مثال}
%====================================
