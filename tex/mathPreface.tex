\باب{دیباچہ}
انجینئری حساب دو جلدوں پر مشتمل ہے۔جلد اول میں تقریباً \عددی{1351} سوالات بمع جوابات اور \عددی{221} اشکال  پائے جاتے ہیں۔

اس کتاب کے پہلے چار ابواب میں بالترتیب ایک درجی سادہ تفرقی مساوات، دو درجی سادہ تفرقی مساوات، بلند درجی سادہ تفرقی مساوات  اور سادہ تفرقی مساوات کے نظام پر بحث  کی گئی ہے۔ سادہ تفرقی مساوات عملی انجینئری میں نہایت اہم کردار ادا کرتے ہیں۔ 

اس  کے بعد ایک باب طاقتی تسلسل اور ایک باب لاپلاس بدل پر  غور کرتا ہے جہاں سادہ تفرقی مساوات کے حل حاصل کرنا سکھایا گیا ہے۔ 

خطی الجبرا پر تین ابواب ہیں۔پہلا باب میں سمتیات پر غور کیا گیا ہے جبکہ دوسرے باب میں قالب اور تیسرے باب میں امتیازی قدر مسائل قالب پر غور کیا گیا ہے۔

آخری باب سمتی میدان اور ان کے خواص پر غور کرتا ہے۔ 

کتاب کے آخر میں فرہنگ دیا گیا ہے۔کتاب میں کسی بھی موضوع تک جلد پہنچنے کے لئے فرہنگ کو استعمال کریں۔اردو کے علاوہ انگریزی زبان میں بھی فرہنگ دیا گیا ہے۔

یہ کتاب \تحریر{Ubuntu} استعمال کرتے ہوئے \تحریر{XeLatex} میں تشکیل دی گئی جبکہ سوالات کے جوابات \تحریر{wxMaxima} کی مدد سے حاصل کئے گئے ہیں۔

یہ کتاب درج ذیل کتاب کو سامنے رکھتے ہوئے لکھی گئی ہے

{
\begin{otherlanguage}{english}
Advanced Engineering Mathematics by Erwin Kreyszig
\end{otherlanguage}
}

جبکہ اردو اصطلاحات چننے میں درج ذیل لغت سے استفادہ  کیا گیا۔
{
\begin{otherlanguage}{english}
\begin{itemize}
\item
http:/\!\!/www.urduenglishdictionary.org
\item
http:/\!\!/www.nlpd.gov.pk/lughat/
\end{itemize}
\end{otherlanguage}
}
آپ سے گزارش ہے کہ اس کتاب کو زیادہ سے زیادہ طلبہ و طالبات تک پہنچائیں اور کتاب میں غلطیوں کی نشاندہی میرے  برقی پتہ پر کریں۔میری تمام کتابوں کی مکمل \تحریر{XeLatex} معلومات

{
\begin{otherlanguage}{english}
https:/\!\!/www.github.com/khalidyousafzai
\end{otherlanguage}
}

سے حاصل کی جا سکتی ہیں جنہیں آپ مکمل اختیار کے ساتھ استعمال کر سکتے ہیں۔میں امید کرتا ہوں کہ طلبہ و طالبات اس کتاب سے استفادہ ہوں گے۔
\vspace{5mm}

{\raggedleft{
خالد خان یوسفزئی

18 مئی 2018}}


