\باب{سمتی تکملی علم الاحصاء۔ تکمل کے مسئلے}
تکمل سے آپ بخوبی واقف ہیں جس کو \اصطلاح{سمتی تکملی علم الاحصاء}\حاشیہب{vector calculus} وسعت دیتا ہے۔یوں منحنی پر تکمل، جسے \اصطلاح{خطی تکمل}\حاشیہب{line integral} کہتے ہیں، سطح پر تکمل جسے \اصطلاح{سطحی تکمل}\حاشیہب{surface integral} کہتے ہیں اور حجم پر تمکل جسے \اصطلاح{حجمی تکمل}\حاشیہب{volume integral} کہتے ہیں،  حاصل کیا جا سکتا ہے۔مزید ایک قسم کی تکمل کا دوسری قسم کی تکمل میں تبادلہ کیا جا سکتا ہے۔ایسا کرنے سے بعض اوقات نسبتاً آسان تکمل حاصل ہوتا ہے۔یوں سطح میں  \اصطلاح{مسئلہ گرین}\حاشیہب{Green's theorem} کی مدد سے خطی تکمل کو دو درجی تکمل میں یا دو درجی تکمل کو خطی تکمل میں تبدیل کیا جا سکتا ہے۔ \اصطلاح{گاوسی مسئلہ ارتکاز}\حاشیہب{Gauss's convergence theorem} کی مدد سے حجمی تکمل کو سطحی تکمل یا سطحی تکمل کو حجمی تکمل میں تبدیل کیا جاتا ہے۔\اصطلاح{مسئلہ سٹوکس}\حاشیہب{Stoke's theorem} کی مدد سے تین درجی تکمل کو خطی تکمل یا خطی تکمل کو تین درجی تکمل میں تبدیل کیا جا سکتا ہے۔

سمتی تکملی علم الاحصاء کا انجینئری، طبیعیات، ٹھوس میکانیات، سیالی میکانیات اور دیگر میدان میں اہم کردار پایا جاتا ہے۔

%===============
\حصہ{خطی تکمل}
 درج ذیل قطعی تکمل
\begin{align}\label{مساوات_سمتی_تکمل_قطعی}
\int_a^b f(x)\dif x
\end{align}
تفاعل \عددی{f(x)} کا \عددی{x} محور پر \عددی{x=a} تا \عددی{x=b} ایک درجی تکمل ہے۔\اصطلاح{خطی تکمل}\فرہنگ{خطی!تکمل}\فرہنگ{تکمل!خطی}  سے مراد تفاعل، جسے \اصطلاح{متکمل}\فرہنگ{متکمل}\حاشیہب{integrand}\فرہنگ{integrand} کہتے ہیں، کا فضا یا سطح میں منحنی  \عددی{C} پر تکمل ہے۔  

یوں \عددی{C} کو منحنی مقدار معلوم صورت (حصہ \حوالہ{حصہ_الاحصاء_منحنی}) میں لکھنا ہو گا۔
\begin{align}\label{مساوات_سمتی_تکمل_راہ_الف}
\bM{r}(t)=[x(t),y(t),z(t)]=x(t)\bM{i}+y\bM{j}+z\bM{k}\quad \quad \quad (a\le t\le b)
\end{align}
\عددی{C} کو \اصطلاح{تکمل کی راہ}\فرہنگ{راہ!تکمل}\حاشیہب{path of integration}\فرہنگ{integration!path} کہتے ہیں۔شکل \حوالہ{شکل_سمتی_تکمل_سمت_بند_منحنی}-الف میں راہ \عددی{A} سے ابتدا ہو کر \عددی{B} پر اختتام پذیر ہوتی ہے لہٰذا  \عددی{A:\bM{r}(a)} ابتدائی نقطہ اور \عددی{B:\bM{r}(b)} اختتامی نقطہ ہو گا اور یوں \عددی{C} \اصطلاح{سمت بند منحنی}\فرہنگ{سمت بند!منحنی} ہو گی۔\عددی{A} سے \عددی{B} جانب سمت جو بڑھتی \عددی{t} کو ظاہر کرتی ہے کو \عددی{C} کی \اصطلاح{مثبت دائری سمت} یا \اصطلاح{مثبت سمت}\فرہنگ{سمت!مثبت}\فرہنگ{مثبت!سمت} کہتے ہیں جس کو تیر کی نشان سے ظاہر کیا جاتا ہے۔اب جیسا شکل \حوالہ{شکل_سمتی_تکمل_سمت_بند_منحنی}-ب میں دکھایا گیا ہے، \عددی{A} اور \عددی{B} ہم مکانی ہو سکتے ہیں۔ایسی صورت میں \عددی{C} \اصطلاح{بند راہ}\فرہنگ{بند!راہ}\فرہنگ{راہ!بند}\حاشیہب{closed path}\فرہنگ{closed!path}\فرہنگ{path!closed} کہلاتی ہے۔
%
\begin{figure}
\centering
\begin{subfigure}{0.5\textwidth}
\centering
\begin{tikzpicture}
\draw[->-=0.5](0,0)node[ocirc]{}node[right]{$A$} to [out=135,in=0]++(-1,2)node[ocirc]{}node[left]{$B$};
\draw(-0.3,1)node[right]{$C$};
\end{tikzpicture}
\caption*{(الف)}
\end{subfigure}%
\begin{subfigure}{0.5\textwidth}
\centering
\begin{tikzpicture}
\draw[->-=0.2](0,0) to [out=0,in=-90]++(1,1) to [out=90,in=-90]++(0.5,1)node[left]{$C$} to [out=90,in=0]++(-0.5,0.5) to [out=180,in=90]++(-1.5,-1)node[ocirc]{}node[left]{$A$}node[right]{$B$} to [out=-90,in=180] (0,0);
\end{tikzpicture}
\caption*{(ب)}
\end{subfigure}%
\caption{سمت بند منحنی}
\label{شکل_سمتی_تکمل_سمت_بند_منحنی}
\end{figure}

اگر \عددی{C} کا مماس ہر نقطے پر  انفرادی ہو جس کی سمت  \عددی{C} پر چلنے سے  استمراری تبدیل ہوتی ہو تب \عددی{C} \اصطلاح{ہموار}\فرہنگ{ہموار}\حاشیہب{smooth}\فرہنگ{smooth} کہلائے گی۔یاد رہے کہ مساوات \حوالہ{مساوات_سمتی_تکمل_راہ_الف} میں دیا گیا \عددی{\bM{r}(t)} قابل تفرق ہے اور اس کا تفرق
 \عددی{\bM{r}'=\tfrac{\dif \bM{r}}{\dif t}}  استمراری ہے جو \عددی{C} کے ہر نقطے پر غیر صفر سمتیہ ہے۔

\جزوحصہء{عمومی مفروضہ}
اس کتاب میں فرض کیا جائے گا کہ خطی تکمل کی ہر راہ  \اصطلاح{ٹکڑوں میں ہموار}\فرہنگ{ہموار!ٹکڑوں میں}\حاشیہب{piecewise smooth}\فرہنگ{piecewise smooth}\فرہنگ{smooth!piecewise} ہے، یعنی کہ راہ کو محدود تعداد کی ہموار ٹکڑوں میں تقسیم کیا جا سکتا ہے۔

\جزوحصہء{خطی تکمل کی تعریف اور اس کا حصول}
راہ \عددی{C:\bM{r}(t)} پر سمتی تفاعل \عددی{\bM{F}(\bM{r})} کی سمتی تکمل کی تعریف درج ذیل ہے
\begin{align}\label{مساوات_سمتی_تکمل_سمتی_تفاعل_تکمل_الف}
\int_C \bM{F}(\bM{r})\cdot \dif \bM{r}=\int_a^b \bM{F}(\bM{r}(t))\cdot \bM{r}'(t)\dif t\quad \quad \quad (\bM{r}'=\frac{\dif \bM{r}}{\dif t})
\end{align}
 جہاں \عددی{C} کی مقدار معلوم صورت \عددی{\bM{r}(t)} مساوات \حوالہ{مساوات_سمتی_تکمل_راہ_الف} دیتی ہے۔ضرب نقطہ (اندرونی ضرب) پر حصہ \حوالہ{حصہ_سمتیہ_اندرونی_ضرب_فضا} میں غور کیا گیا ہے۔مساوات \حوالہ{مساوات_سمتی_تکمل_سمتی_تفاعل_تکمل_الف} کو اجزاء کی صورت میں لکھتے ہیں جہاں حصہ \حوالہ{حصہ_الاحصاء_لمبائی_قوس} کی طرح \عددی{\dif \bM{r}=[\dif x, \quad \dif y, \quad \dif z]} اور \عددی{'=\tfrac{\dif}{\dif t}} لکھا جائے گا۔
\begin{gather}
\begin{aligned}\label{مساوات_سمتی_تکمل_سمتی_تفاعل_تکمل_ب}
\int_C \bM{F}(\bM{r}(t))\cdot \dif \bM{r}&=\int_C(F_1\dif x+F_2\dif y+F_3\dif z)\\
&=\int_C (F_1x'+F_2y'+F_3z')\dif t
\end{aligned}
\end{gather}
بند راہ کی صورت میں ہم \عددی{\int_C} کی بجائے \عددی{\oint_C} لکھتے ہیں۔یاد رہے کہ اندرونی ضرب کی بنا  مساوات \حوالہ{مساوات_سمتی_تکمل_سمتی_تفاعل_تکمل_ب} کا متکمل  غیر سمتی مقدار ہو گا۔درحقیقت \عددی{\tfrac{\bM{F}\cdot \bM{r}'}{\abs{\bM{r}'}}} تفاعل \عددی{\bM{F}} کا مماسی جزو ہے (کسی بھی سمت میں تفاعل کا جزو  اندرونی ضرب کی مدد سے حاصل کیا جا سکتا ہے)۔

مساوات \حوالہ{مساوات_سمتی_تکمل_سمتی_تفاعل_تکمل_ب}  قطعی تکمل  ہے جہاں محور \عددی{t} کی مثبت سمت میں (یعنی بڑھتے \عددی{t} کی سمت میں) تفاعل کا متغیر \عددی{t}  وقفہ \عددی{a\le t\le b} پر حرکت کرتا ہے۔استمراری \عددی{\bM{F}} اور ٹکڑوں میں ہموار \عددی{C} کی صورت میں، چونکہ \عددی{\bM{F}\cdot \dif \bM{r}} ٹکڑوں میں ہموار ہو گا لہٰذا، یہ تکمل موجود ہو گا۔

میکانیات میں راہ \عددی{C} پر چلتے ہوئے قوت \عددی{\bM{F}} سے سر زد  \اصطلاح{کام}\فرہنگ{کام}\حاشیہب{work}\فرہنگ{work} مساوات \حوالہ{مساوات_سمتی_تکمل_سمتی_تفاعل_تکمل_ب} دیتی ہے۔یوں مساوات \حوالہ{مساوات_سمتی_تکمل_سمتی_تفاعل_تکمل_ب} کو \اصطلاح{تکمل کام}\فرہنگ{تکمل!کام}\فرہنگ{کام تکمل}\حاشیہب{work integral}\فرہنگ{work integral} کہتے ہیں۔دیگر خطی تکمل پر اسی حصے میں غور کیا جائے گا۔

%=================
\ابتدا{مثال}\شناخت{مثال_سمتی_تکمل_راہ_الف}\quad سطح میں خطی تکمل\\
سمتی تفاعل \عددی{\bM{F}=x\bM{i}+xy\bM{j}} کا شکل \حوالہ{شکل_مثال_سمتی_تکمل_راہ_الف}-الف میں دکھائی گئی راہ پر، گھڑی کی سوئیوں کے گھومنے کی الٹ رخ،  \عددی{t=0} تا \عددی{t=\tfrac{\pi}{2}} سمتی تکمل (مساوات \حوالہ{مساوات_سمتی_تکمل_سمتی_تفاعل_تکمل_ب}) حاصل کریں۔
\begin{figure}
\centering
\begin{subfigure}{0.5\textwidth}
\begin{tikzpicture}
\draw(0,0)--++(2,0)node[right]{$x$};
\draw(0,0)--++(0,2)node[left]{$y$};
%
\draw[->-=0.5]([shift={(0:1.5)}]0,0)node[ocirc]{}node[below]{$1$}node[above left]{$A$} arc (0:90:1.5)node[ocirc]{}node[left]{$B$};
\draw(45:1.5)node[above right]{$C$};
\end{tikzpicture}
\caption*{(الف) سطح میں تکمل کی راہ (مثال \حوالہ{مثال_سمتی_تکمل_راہ_الف})}
\end{subfigure}%
\begin{subfigure}{0.5\textwidth}
\centering
\begin{tikzpicture}[x={(-0.5cm,-0.5cm)},y={(1cm,0cm)},z={(0cm,1cm)}]
\draw(0,0,0)--++(2,0,0)node[left]{$x$};
\draw(0,0,0)--++(0,2,0)node[right]{$y$};
\draw(0,0,0)--++(0,0,2)node[left]{$z$};
%
\draw[->-=0.5,domain=0:360,samples=200] plot ({cos (\x)},{sin(\x)},{\x/360});
%\draw[dashed,domain=0:360,samples=200] plot ({cos (\x)},{sin(\x)},{0});
\draw[dashed] (1,0,0)node[ocirc,solid]{}node[below right]{$A$}node[left]{$1$}--(1,0,1)node[ocirc,solid]{}node[above]{$B$};
\end{tikzpicture}
\caption*{(ب) فضا میں خطی تکمل کی  راہ (مثال \حوالہ{مثال_سمتی_تکمل_فضا_میں_راہ})}
\end{subfigure}%
\caption{سطح میں راہ اور فضا میں راہ۔}
\label{شکل_مثال_سمتی_تکمل_راہ_الف}
\end{figure}

حل:راہ \عددی{C} کی مقدار معلوم مساوات درج ذیل ہے۔
\begin{align*}
\bM{r}(t)=\cos t\bM{i}+\sin t\bM{j}\quad \quad 0\le t \le \frac{\pi}{2}
\end{align*} 
اس راہ پر سمتی تفاعل درج ذیل ہو گا۔
\begin{align*}
\bM{F}(\bM{r}(t))=x(t)\bM{i}+x(t)y(t)\bM{j}=\cos t \bM{i}+\cos t \sin t \bM{j}
\end{align*}
مساوات \حوالہ{مساوات_سمتی_تکمل_سمتی_تفاعل_تکمل_ب} میں \عددی{\bM{r}'(t)=-\sin t\bM{i}+\cos t\bM{j}} پر کرتے ہوئے تکمل لے کر جواب حاصل کرتے ہیں۔
\begin{align*}
\int_C \bM{F}(\bM{r})\cdot \dif \bM{r}&=\int_0^{\frac{\pi}{2}}[\cos t \bM{i}+\cos t \sin t \bM{j}]\cdot [-\sin t\bM{i}+\cos t\bM{j}] \dif t\\
&=\int_0^{\frac{\pi}{2}}(-\cos t\sin t+\cos^2 t\sin t)\dif t\\
&=\left. \frac{\cos^2 t}{2}-\frac{\cos^3 t}{3}\right|_0^{\frac{\pi}{2}}=\frac{1}{6}
\end{align*}
\انتہا{مثال}
%============================
\ابتدا{مثال}\شناخت{مثال_سمتی_تکمل_فضا_میں_راہ}\quad فضا میں راہ پر خطی تکمل ہو بہو سطح میں راہ پر خطی تکمل کی طرح حاصل کیا جاتا ہے\\
تفاعل \عددی{\bM{F}(\bM{r})=y\bM{i}+z\bM{j}+x\bM{k}} کا شکل \حوالہ{شکل_مثال_سمتی_تکمل_راہ_الف}-ب کی پیچ دار راہ پر  ابتدائی نقطہ \عددی{(1,0,0)}تا  اختتامی نقطہ \عددی{(0,0,2\pi)} خطی تکمل حاصل کریں۔

حل:پیچ دار راہ کی مساوات 
\begin{align*}
\bM{r}(t)=\cos t\bM{i}+\sin t\bM{j}+t\bM{k}\quad \quad 0\le t \le 2\pi
\end{align*}
ہے لہٰذا \عددی{\bM{r}'=-\sin t\bM{i}+\cos \bM{j}+\bM{k}} ہو گا۔اس راہ پر چلتے ہوئے \عددی{\bM{F}} میں \عددی{x}، \عددی{y}، \عددی{z} کی قیمتیں راہ سے ہٹ کر نہیں ہو سکتی ہیں لہٰذا راہ پر تفاعل درج ذیل ہو گا۔
\begin{align*}
\bM{F}(\bM{r}(t))=\sin t \bM{i}+t\bM{j}+\cos t\bM{k} \quad \quad 0\le t \le 2\pi
\end{align*}
یوں مساوات \حوالہ{مساوات_سمتی_تکمل_سمتی_تفاعل_تکمل_ب} سے درج ذیل حاصل ہو گا۔
\begin{align*}
\int_C\bM{F}(\bM{r}(t))\cdot \dif \bM{r}(t)=\int_0^{2\pi} [\sin t \bM{i}+t\bM{j}+\cos t\bM{k}] \cdot [-\sin t\bM{i}+\cos \bM{j}+\bM{k}]\dif t=-\pi
\end{align*}
\انتہا{مثال}
%=========================

\جزوحصہء{خطی تکمل کی خواص}
قطعی تکمل کی خواص سے  خطی تکمل کی درج ذیل مطابقتی خواص حاصل ہوتی ہیں
\begin{align}
\int_C k\bM{F}\cdot \dif \bM{r}&=k\int_C \bM{F}\cdot \dif \bM{r}\quad \quad \quad (k \, \text{مستقل}) \label {مساوات_سمتی_تکمل_خطی_تکمل_خواص_الف}\\
\int_C(\bM{F}+\bM{G})\cdot \dif \bM{r}&=\int_C \bM{F}\cdot \dif \bM{r}+\int_C \bM{G}\cdot \dif \bM{r} \label {مساوات_سمتی_تکمل_خطی_تکمل_خواص_ب}\\
\int_C k\bM{F}\cdot \dif \bM{r}&\int_{C_1} k\bM{F}\cdot \dif \bM{r}+\int_{C_2} k\bM{F}\cdot \dif \bM{r}\label {مساوات_سمتی_تکمل_خطی_تکمل_خواص_پ}
\end{align}
جہاں مساوات \حوالہ{مساوات_سمتی_تکمل_خطی_تکمل_خواص_پ} میں راہ \عددی{C} کو دو ایسے ٹکڑوں \عددی{C_1} اور \عددی{C_2} میں تقسیم کیا گیا ہے جن کی سمت بندی \عددی{C} کی سمت بندی  کے عین مطابق ہے (شکل \حوالہ{شکل_سمتی_تکمل_راہ_تقسیم_خواص})۔مساوات \حوالہ{مساوات_سمتی_تکمل_خطی_تکمل_خواص_ب} میں تینوں تکمل کی راہ کی سمت بندی ایک دوسرے جیسی ہے۔اگر سمت بندی الٹ کر دی جائے تب تکمل کی قیمت \عددی{-1} سے ضرب ہو گی۔البتہ مثبت سمت محفوض رہنے کی صورت میں درج ذیل ہو گا۔

\begin{figure}
\centering
\begin{tikzpicture}
\draw[->-=0.25,->-=0.75](0,0)node[ocirc]{}node[right]{$A$} to [out=120,in=-45] ++(-2,1)node[ocirc]{}node[left]{$B$};
\draw(-0.75,0.6)--++(60:-0.2);
\draw(-0.2,0.6)node{$C_1$};
\draw(-1.4,0.4)node{$C_2$};
\end{tikzpicture}
\caption{تکمل کی راہ کو ٹکڑوں میں تقسیم کیا جا سکتا ہے (مساوات \حوالہ{مساوات_سمتی_تکمل_خطی_تکمل_خواص_پ})۔}
\label{شکل_سمتی_تکمل_راہ_تقسیم_خواص}
\end{figure} 

%=============================
\ابتدا{مسئلہ}\quad سمت راہ محفوض رکھتے مقدار معلوم تبادل\\
راہ  کی ایسی تمام مقدار معلوم صورتیں جو \عددی{C} کی مثبت سمت محفوض رکھتی ہوں کے خطی تکمل کی قیمت یکساں ہو گی۔     
\انتہا{مسئلہ}
%============================

\ابتدا{ثبوت}

\انتہا{ثبوت}
%======================
