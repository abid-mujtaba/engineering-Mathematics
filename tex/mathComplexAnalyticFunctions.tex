\باب{مخلوط اعداد۔ مخلوط  تحلیلی تفاعل}
انجینئری کے کئی مسائل مخلوط تجزیہ سے با آسانی حل ہو پاتے ہیں۔ان مسئلوں کو دو بڑے گروہوں میں تقسیم کیا جا سکتا ہے۔پہلی گروہ میں سادہ مسائل شامل ہیں جنہیں حل کرنے کے لئے کالج میں سیکھی گئی مخلوط اعداد کی الجبرا کافی ہے۔ برقی ادوار اور میکانی ارتعاش  کے کئی مسائل اس نوعیت کے ہیں۔ دوسری گروہ کے لئے مخلوط تحلیلی تفاعل کا نظریہ اور اس میں استعمال کیے جانے والے انتہائی طاقتور اور شائستہ تراکیب  تفصیلاً جاننا ضروری ہے۔ نظریہ حرارت، حرکیات سیال اور برقی سکون کے مسائل اس نوعیت کے ہیں۔

اس باب کے علاوہ اگلے کئی ابواب میں مخلوط تحلیلی تفاعل کے نظریہ کی بیشتر حصوں اور ان تفاعل کی استعمال  پر غور کیا جائے گا۔ ہم دیکھیں گے کہ انجینئری حساب میں ان تفاعل کی اہمیت درج ذیل تین وجوہات کی بنا ہے۔\\
\begin{description}
\item{(الف)}
تحلیلی تفاعل کے حقیقی اور خیالی اجزاء، دو متغیرات کی لاپلاس مساوات کا حل ہوتے ہیں۔ یوں دو ابعادی مخفی قوہ مسائل پر تحلیلی تفاعل کے لئے بنائے گئے تراکیب کی مدد سے غور کیا جا سکتا ہے۔
\item{(ب)}
مختلف مسائل میں درپیش کئی پیچیدہ حقیقی اور مخلوط تکملات کو مخلوط تکمل کی تراکیب سے حاصل کرنا ممکن ہے۔
\item{(پ)}
انجینئری حساب میں پائے جانے والے غیر بنیادی تفاعل کا بیشتر حصہ تحلیلی تفاعل پر مشتمل ہے۔مخلوط غیر تابع متغیرات کے لئے ان تفاعل کے مشاہدہ  سے تفاعل کی خواص کی مفصل اور گہری  سمجھ پیدا ہوتی ہے۔
\end{description}

موجودہ باب میں ہم مخلوط اعداد اور تحلیلی تفاعل اور ان ان کے عمومی خواص پر غور کریں گے۔باب کا دوسرا حصہ اہم ترین بنیادی مخلوط تفاعل کے لئے مختص ہے۔

\حصہ{مخلوط اعداد}
تاریخی طور پر دیکھا گیا کہ کئی مساوات مثلاً 
\begin{align*}
x^2+4=0,\quad x^2+2x+5=0
\end{align*}
 کو کوئی بھی حقیقی عدد مطمئن نہیں کرتا ہے۔مخلوط اعداد کا آغاز یہیں سے ہوا۔\حاشیہد{اس مقصد کے لئے مخلوط اعداد سب سے پہلے اطالوی ریاضی دان جرولامو کردانو [1501-1576] نے استعمال کیے جنہوں نے کعبی مساوات کے حل کا کلیہ دریافت کیا۔مخلوط اعداد کی منظم  اور عام استعمال کی بنیاد جرمنی کے ریاضی دان یوہان کارل فرڈرش گاوس نے ڈالی۔}

%===========================
\ابتدا{تعریف}
حقیقی اعداد \عددی{x} اور \عددی{y} کی مرتب جوڑی \عددی{(x,y)} کو \اصطلاح{مخلوط عدد}\فرہنگ{مخلوط عدد}\حاشیہب{complex number}\فرہنگ{complex number} \عددی{z} کہتے ہیں جو درج ذیل لکھا جاتا ہے۔
\begin{align*}
z=(x,y)
\end{align*}
ہم \عددی{x} کو \عددی{z} کا \اصطلاح{حقیقی حصہ}\فرہنگ{حقیقی!حصہ}\حاشیہب{real part}\فرہنگ{real!part} اور \عددی{y} کو \عددی{z} کا \اصطلاح{خیالی حصہ}\فرہنگ{خیالی!حصہ}\حاشیہب{imaginary part}\فرہنگ{imaginary!part} کہتے ہیں جنہیں ہم درج ذیل لکھتے ہیں۔
\begin{align*}
x=z\,\text{حقیقی},\quad y=z\,\text{خیالی}
\end{align*}
\انتہا{تعریف}
%===============================
یوں 
$-7=(-7,2)\,\text{حقیقی}$
اور 
$2=(-7,2)\,\text{خیالی}$
ہوں گے۔مزید دو مخلوط اعداد \عددی{z_1=(x_1,y_1)} اور \عددی{z_2=(x_2,y_2)} کی برابری کی تعریف ہم یوں کرتے ہیں کہ یہ مخلوط اعداد صرف اور صرف اس صورت \موٹا{برابر}\فرہنگ{مخلوط!برابری} ہوں گے جب ان کے حقیقی حصے آپس میں برابر ہوں اور ان کے خیالی حصے آپس میں برابر ہوں۔ 

مخلوط اعداد \عددی{z_1=(x_1,y_1)} اور \عددی{z_2=(x_2,y_2)} کا  \موٹا{مجموعہ} درج ذیل قاعدہ دیتا ہے
\begin{align}\label{مساوات_مخلوط_مجموعہ_اعداد}
z_1+z_2=(x_1,y_1)+(x_2,y_2)=(x_1+x_2,y_1+y_2)
\end{align}
جبکہ ان کا حاصل ضرب درج ذیل قاعدہ دے گا۔
\begin{align}\label{مساوات_مخلوط_ضرب_اعداد}
z_1z_2=(x_1,y_1)(x_2,y_2)=(x_1x_2-y_1y_2,x_1y_2+x_2y_1)
\end{align}
ان اعمال ریاضی پر مزید بحث آگے کی جائے گی۔ 

\موٹا{روپ \عددی{z=x+iy} میں خیالی اعداد کا اظہار}\\
ایسا مخلوط عدد جس کا خیالی حصہ صفر کی روپ \عددی{(x,0)} ہو گی۔ اس طرز کے مخلوط اعداد کے لئے حقیقی اعداد کی طرح 
\begin{align*}
(x_1,0)+(x_2,0)&=(x_1+x_2,0)\\
(x_1,0)(x_2,0)&=(x_1x_2,0)
\end{align*}
لکھا جا سکتا ہے لہٰذا  \عددی{(x,0)} کو حقیقی عدد تصور کیا جا سکتا ہے۔یوں حقیقی عددی نظام کی توسیعی حالت مخلوط عددی نظام ہے۔مزید درج ذیل مخلوط عدد
\begin{align*}
i=(0,1)
\end{align*}
 کو \اصطلاح{خیالی اکائی}\فرہنگ{خیالی!اکائی}\فرہنگ{اکائی!خیالی}\حاشیہب{imaginary unit}\فرہنگ{imaginary!unit} کہتے ہیں۔ مساوات \حوالہ{مساوات_مخلوط_ضرب_اعداد} کے تحت ہر حقیقی \عددی{y} کے لئے درج ذیل لکھا جا سکتا ہے
\begin{align*}
iy=(0,1)(y,0)=(0,y)
\end{align*}
جبکہ مساوات \حوالہ{مساوات_مخلوط_مجموعہ_اعداد} کے تحت
\begin{align*}
(x,y)=(x,0)+(0,y)
\end{align*}
ہو گا۔یوں \عددی{(x,0)} کے لئے \عددی{x} اور \عددی{(0,y)=iy} استعمال کرتے ہوئے 
\begin{align*}
z=x+iy
\end{align*}
لکھا جا سکتا ہے۔مخلوط اعداد کو عموماً اسی روپ میں لکھا جاتا ہے۔خیالی اکائی \عددی{i} کی ایک اہم خاصیت 
\begin{align}\label{مساوات_مخلوط_مربع_خیالی_اکائی}
i^2=-1
\end{align}
کو مساوات \حوالہ{مساوات_مخلوط_ضرب_اعداد} سے حاصل کیا جا سکتا ہے یعنی:  \عددی{i^2=(0,1)(0,1)=(-1,0)=-1}  

\جزوحصہء{مخلوط سطح}
مخلوط اعداد کو سطح پر ظاہر کیا جا سکتا ہے۔ایسا کرنا نہایت مفید ثابت ہوتا ہے۔ہم دو عدد آپس میں عمودی محور چنتے ہیں۔افقی \عددی{x} محور کو \اصطلاح{حقیقی محور}\فرہنگ{حقیقی!محور}\حاشیہب{real axis}\فرہنگ{real!axis} جبکہ انتصابی \عددی{y} محور کو \اصطلاح{خیالی محور}\فرہنگ{خیالی!محور}\حاشیہب{imaginary axis}\فرہنگ{imaginary!axis} تصور کیا جاتا ہے۔ دونوں محوروں پر یکساں اکائی لمبائی استعمال کی جاتی ہے (شکل \حوالہ{شکل_مخلوط_سطح_تعریف}-الف)۔اس کو کارتیسی محددی نظام\فرہنگ{کارتیسی!نظام محدد} کہتے ہیں۔ہم اب مخلوط عدد \عددی{z=(x,y)=x+iy} کو اس سطح پر بطور نقطہ \عددی{P} ظاہر کرتے ہیں جس کے محدد \عددی{x} اور \عددی{y} ہوں گے۔ایسی \عددی{xy} سطح جس پر اس طرح مخلوط اعداد ظاہر کیے جاتے  ہیں \اصطلاح{مخلوط سطح}\فرہنگ{مخلوط!سطح}\حاشیہب{complex plane}\فرہنگ{complex!plane} یا \ترچھا{نقشہ اغگن}\فرہنگ{اغگن!نقشہ}\فرہنگ{نقشہ!اغگن}\حاشیہب{Argand diagram}\فرہنگ{Argand diagram} کہلاتی\حاشیہد{فرانسیسی ریاضی دان ژاں غوبغ اغگن [1768-1822]} ہے۔
\begin{figure}
\centering
\begin{subfigure}{0.5\textwidth}
\centering
\begin{tikzpicture}
\draw(0,0)--++(3,0)node[below left]{$x$}node[right]{\RL{حقیقی محور}};
\draw(0,0)--++(0,2)node[below left]{$y$}node[above]{\RL{خیالی محور}};
\draw(0.5,0)node[below]{$1$}--++(0,0.1);
\draw(0,0.5)node[left]{$1$}--++(0.1,0);
\draw(0,0)node[ocirc]{}--++(2,1.5)node[ocirc]{}node[right]{$z=x+iy$}node[above]{$P$};
\end{tikzpicture}
\caption*{(الف) مخلوط سطح}
\end{subfigure}%
\begin{subfigure}{0.5\textwidth}
\centering
\begin{tikzpicture}
\draw(0,0)--++(4,0)node[below]{$x$};
\draw(0,1)node[right]{$y$}--(0,-2.2);
\foreach \x in {1,2,3,4,5,6,7} {\draw (\x*0.5,0)--++(0,0.1);}
\foreach \y in {1,-1,-2,-3,-4}{\draw (0,\y*0.5)--++(0.1,0);}
\draw(3.5,0)node[below]{$7$};
\draw(0,-1.5)node[left]{$-3$};
\draw(0,0)node[ocirc]{}--(6*0.5,-3*0.5)node[right]{$6-i3$};
\draw[dashed](3,-1.5)--(3,0);
\draw[dashed](3,-1.5)--(0,-1.5);
\draw(3,-1.5)node[ocirc]{};
\end{tikzpicture}
\caption*{(ب) مخلوط سطح پر $\, 6-i3\,$ کا اظہار}
\end{subfigure}%
\caption{مخلوط سطح اور مخلوط سطح پر مخلوط عدد کا اظہار}
\label{شکل_مخلوط_سطح_تعریف}
\end{figure}

"مخلوط سطح میں مخلوط عدد \عددی{z}" کہنے کی بجائے ہم  "مخلوط سطح میں نقطہ \عددی{z}" کہیں گے۔ اس سے کوئی غلط فہمی پیدا نہیں ہوتی ہے۔

\جزوحصہء{ریاضی اعمال}
ہم اب مخلوط عدد کی روپ \عددی{z=x+iy} اور مخلوط سطح کو استعمال کرتے ہیں۔

\موٹا{جمع۔}\quad مساوات \حوالہ{مساوات_مخلوط_مجموعہ_اعداد} میں دیا گیا مجموعہ \عددی{z_1+z_2} اب
\begin{align}\label{مساوات_مخلوط_عمل_جمع}
z_1+z_2=(x_1+x_2)+i(y_1+y_2)
\end{align}
لکھا جا سکتا ہے۔آپ دیکھ سکتے ہیں کہ مخلوط اعداد کی جمع، میکانیات میں قوتوں کا مجموعہ حاصل کرنے کے \ترچھا{متوازی الاضلاع قاعدہ} کے مطابق ہے (شکل \حوالہ{شکل_مخلوط_اعداد_تفریق}-الف)۔
\begin{figure}
\centering
\begin{subfigure}{0.5\textwidth}
\centering
\begin{tikzpicture}
\draw(0,0)--(3.5,0)node[below]{$x$};
\draw(0,0)--++(0,2)node[right]{$y$};
\draw(0,0)--++(20:2)coordinate(kA);
\draw(0,0)--++(70:1)coordinate(kB);
\draw[dashed](kA)--++(70:1)coordinate(kC);
\draw[dashed](kB)node[ocirc,solid]{}node[above]{$z_2$}--++(20:2)node[right]{$z_1+z_2$};
\draw(20:2)node[ocirc]{}node[right]{$z_1$};
\draw(0,0)--(kC)node[ocirc,solid]{};
\end{tikzpicture}
\caption*{(الف) مخلوط اعداد کی جمع}
\end{subfigure}%
\begin{subfigure}{0.5\textwidth}
\centering
\begin{tikzpicture}
\draw(0,0)--(3.5,0)node[below]{$x$};
\draw(0,0)--++(0,2)node[right]{$y$};
\draw(0,0)--++(20:2)coordinate(kA)node[right]{$z_1$};
\draw(0,0)--++(70:1)node[ocirc]{}node[right]{$z_2$}coordinate(kB);
\draw[dashed](0,0)--++(70:-1)coordinate(kBB)node[below]{$-z_2$};
\draw[dashed](kBB)--++(20:2)coordinate(kC);
\draw[dashed](kA)--(kC);
\draw(kBB)node[ocirc]{};
\draw(0,0)--(kC)node[ocirc]{}node[below right]{$z_1-z_2$};
\draw(kA)node[ocirc]{};
\end{tikzpicture}
\caption*{(ب) مخلوط اعداد کی تفریق}
\end{subfigure}%
\caption{مخلوط اعداد کی جمع اور تفرق}
\label{شکل_مخلوط_اعداد_تفریق}
\end{figure}

\موٹا{تفریق۔} یہ جمع کا الٹ عمل ہے۔فرق \عددی{z_1-z_2} ایسے مخلوط عدد \عددی{z} کے برابر ہو گا کہ \عددی{z_1=z+z_2} ہو۔یوں (شکل \حوالہ{شکل_مخلوط_اعداد_تفریق}-ب) درج ذیل ہو گا۔
\begin{align}\label{مساوات_مخلوط_عمل_تفریق}
z_1-z_2=(x_1-x_2)+i(y_1-y_2)
\end{align}

\موٹا{ضرب۔} مساوات \حوالہ{مساوات_مخلوط_ضرب_اعداد} میں دی گئی ضرب \عددی{z_1z_2} کو اب درج ذیل لکھا جا سکتا ہے۔
\begin{align}\label{مساوات_مخلوط_عمل_ضرب}
z_1z_2=(x_1+iy_1)(x_2+iy_2)=(x_1x_2-y_1y_2)+i(x_1y_2+x_2y_1)
\end{align}
چونکہ یہ نتیجہ حقیقی اعداد کی حساب کے قوانین اور مساوات \حوالہ{مساوات_مخلوط_مربع_خیالی_اکائی} یعنی \عددی{i^2=ii=-1} کی استعمال سے حاصل ہوتا ہے لہٰذا اس کو یاد رکھنا آسان ہے۔

\موٹا{تقسیم۔} یہ ضرب کا الٹ عمل ہے۔یوں حاصل تقسیم \عددی{z=\tfrac{z_1}{z_2}} ایسا مخلوط عدد \عددی{z=x+iy} ہو گا جو درج ذیل کو مطمئن کرتا ہو۔
\begin{align}\label{مساوات_مخلوط_عمل_تقسیم}
z_1=zz_2=(x+iy)(x_2+iy_2)\quad \quad \quad (z_2\ne 0)
\end{align} 
ہم \عددی{z_2\ne 0} کی صورت میں حاصل تقسیم \عددی{z=x+iy=\tfrac{z_1}{z_2}} کی درج ذیل صورت حاصل کرتے ہیں۔
\begin{align}\label{مساوات_مخلوط_عمل_تقسیم_کلیات_الف}
x=\frac{x_1x_2+y_1y_2}{x_2^2+y_2^2},\quad y=\frac{x_2y_1-x_1y_2}{x_2^2+y_2^2}\quad \quad (z_2\ne 0)
\end{align}
عملاً مساوات \حوالہ{مساوات_مخلوط_عمل_تقسیم_کلیات_الف} کو حاصل کرنے کے لئے  ہم \عددی{\tfrac{z_1}{z_2}} کی شمار کنندہ اور نسب نما کو \عددی{x_2-iy_2} سے ضرب دے کر سادہ صورت حاصل کرتے ہیں یعنی:
\begin{align}\label{مساوات_مخلوط_عمل_تقسیم_کلیات_ب}
z=\frac{x_1+y_1}{x_2+iy_2}=\frac{(x_1+iy_1)(x_2-iy_2)}{(x_2+iy_2)(x_2-iy_2)}=\frac{x_1x_2+y_1y_2}{x_2^2+y_2^2}+i\,\frac{x_2y_1-x_1y_2}{x_2^2+y_2^2}
\end{align}
مثال کے طور پر اگر \عددی{z_1=3-i2} اور \عددی{z_2=-4+i} ہوں تب
\begin{align*}
\frac{3-i2}{-4+i}=\frac{(3-i2)(-4-i)}{(-4+i)(-4-i)}=\frac{-12-i3+i8-2}{16+1}=-\frac{14}{17}+i\frac{5}{17}
\end{align*}
ہو گا جس کی درستگی آپ درج ذیل طریقہ سے ثابت کر سکتے ہیں۔
\begin{align*}
zz_2=\big(-\frac{14}{17}+i\frac{5}{17}\big)(-4+i)=\frac{56}{17}-i\frac{14}{17}-i\frac{20}{17}-\frac{5}{17}=3-i2
\end{align*} 
مساوات \حوالہ{مساوات_مخلوط_عمل_تقسیم_کلیات_الف} کا ثبوت کچھ یوں ہے۔مساوات \حوالہ{مساوات_مخلوط_عمل_ضرب} سے ہم دیکھتے ہیں کہ مساوات \حوالہ{مساوات_مخلوط_عمل_تقسیم} کو درج ذیل لکھا جا سکتا ہے۔
\begin{align*}
x_1+iy_1=(x_2x-y_2y)+i(y_2x+x_2y)
\end{align*}
مخلوط اعداد کی برابری کی تعریف کے مطابق دونوں مخلوط اعداد کے حقیقی حصے آپس میں برابر ہوں گے اور ان کے خیالی حصے آپس میں برابر ہوں گے یعنی:
\begin{align*}
x_1&=x_2x-y_2y\\
y_1&=y_2x+x_2y
\end{align*}
یہ دو دو خطی مساوات کا نظام ہے جس کے نا معلوم متغیرات \عددی{x} اور \عددی{y} ہیں۔یہ فرض کرتے ہوئے کہ \عددی{x_2} اور \عددی{y_2} بیک وقت صفر نہیں ہیں (جس کو مختصراً \عددی{z\ne 0} لکھا جاتا ہے) ہمیں مساوات \حوالہ{مساوات_مخلوط_عمل_تقسیم_کلیات_الف} میں دیا گیا یکتا حل حاصل ہوتا ہے۔

%===================
\ابتدا{مثال}\quad \موٹا{جمع، تفریق، ضرب، تقسیم}\\
فرض کریں کہ \عددی{z_1=3-i2} اور \عددی{z_2=-4+i} ہیں۔ تب
\begin{align*}
z_1+z_2=-1-i, z_1-z_2=7-i3, z_1z_2=-10+i11 
\end{align*}
اور جیسے ہم حاصل کر سکے ہیں \عددی{\tfrac{z_1}{z_2}=-\tfrac{14}{17}+i\tfrac{5}{17}} ہو گا۔
\انتہا{مثال}
%=======================

\جزوحصہء{ریاضی اعمال کے خواص}
حقیقی اعداد کے  قواعد سے مخلوط اعداد \عددی{z_1} اور \عددی{z_2} کے لئے درج ذیل قواعد حاصل ہوتے ہیں
\begin{gather}
\left. \begin{aligned}\nonumber
 z_1+z_2&=z_2+z_1\\
z_1z_2&=z_2z_1
\end{aligned} \right \}\quad \text{\RL{(قانون تبادل)}}\\
\left. \begin{aligned}
(z_1+z_2)+z_3&=z_1+(z_2+z_3)\\
(z_1z_2)z_3&=z_1(z_2z_3)
\end{aligned}\right \}\quad \text{\RL{قانون تلازم}}\label{مساوات_مخلوط_قواعد}\\
\begin{aligned}\nonumber
z_1(z_2+z_3)&=z_1z_2+z_1z_3 \quad\quad \text{\RL{(قانون جزئیتی تقسیم)}}\\
0+z&=z+0=z\\
z+(-z)&=(-z)+z=0\\
z\cdot 1=z
\end{aligned}
\end{gather}  
جہاں \عددی{0=(0,0)} اور \عددی{-z=-x-iy} ہیں۔

\جزوحصہء{جوڑی دار مخلوط اعداد}
فرض کریں کہ \عددی{z=x+iy} کوئی مخلوط عدد ہے۔تب \عددی{x-iy} کو \عددی{z} کا جوڑی دار مخلوط کہا جائے گا اور \عددی{z} کے جوڑی دار مخلوط کو \عددی{\bar{z}} سے ظاہر کیا جائے گا۔یوں
\begin{align*}
z=x+iy,\quad \bar{z}=x-iy
\end{align*}
ہوں گا۔مثلاً \عددی{z=5+i2} کا جوڑی دار مخلوط \عددی{\bar{z}=5-i2} ہے (شکل \حوالہ{شکل_مخلوط_جوڑی_دار})۔مزید جمع اور تفریق سے
\begin{align*}
z+\bar{z}=2x,\quad z-\bar{z}=i2y
\end{align*}
حاصل ہوتے ہیں جو درج ذیل اہم کلیات کا سبب بنتے ہیں۔
\begin{align}
\frac{1}{2}(z+\bar{z})=z\,\text{حقیقی}, \quad \frac{1}{i2}(z-\bar{z})=z\,\text{خیالی}
\end{align}
حقیقی عدد \عددی{z=x} کا مخلوط جوڑی دار عدد \عددی{\bar{z}=z} ہو گا جبکہ \عددی{z=0+iy=iy} کا جوڑی دار مخلوط عدد \عددی{\bar{z}=-z} ہو گا۔اس طرح کا عدد جس کا حقیقی حصہ صفر ہو \اصطلاح{خالص خیالی عدد}\فرہنگ{خالص خیالی عدد}\فرہنگ{خیالی!خالص خیالی عدد}\حاشیہب{pure imaginary number}\فرہنگ{imaginary!pure imaginary} کہلاتا ہے جو خیالی محدد پر کسی نقطہ کو ظاہر کرتا ہے۔
%
\begin{figure}
\centering
\begin{tikzpicture}
\draw(0,0)--(3,0)node[right]{$x$};
\draw(0,-1)--(0,1)node[right]{$y$};
\draw[thick](0,0)--(2,0.75)node[ocirc]{}node[right,font=\normalsize]{$z=x+iy=5+i2$};
\draw[thick](0,0)--(2,-0.75)node[ocirc]{}node[right,font=\normalsize]{$\bar{z}=x-iy=5-i2$};
\draw(2,0)node[below]{$5$}--++(0,0.1);
\draw(0,0.75)node[left]{$2$}--++(0.1,0);
\draw(0,-0.75)node[left]{$-2$}--++(0.1,0);
\end{tikzpicture}
\caption{جوڑی دار مخلوط اعداد}
\label{شکل_مخلوط_جوڑی_دار}
\end{figure}

اس کے علاوہ درج ذیل تعلق بھی لکھے جا سکتے ہیں۔
\begin{gather}
\begin{aligned}\label{مساوات_مخلوط_تعلقات}
\overline{(z_1+z_2)}=\bar{z}_1+\bar{z}_2,\quad \overline{(z_1-z_2)}=\bar{z}_1+\bar{z}_2\\
\overline{(z_1z_2)}=\bar{z}_1\bar{z}_2,\quad \overline{\big(\frac{z_1}{z_2}\big)}=\frac{\bar{z}_1}{\bar{z}_2}
\end{aligned}
\end{gather}

%=======================
\حصہء{سوالات}

%=====================
\ابتدا{سوال}\quad \موٹا{خیالی اکائی کے طاقت}\\
درج ذیل ثابت کریں۔
\begin{gather}
 \begin{aligned}
i^2=-1,\quad i^3=-i,\quad i^4=1, \quad i^5=i,\cdots\\
\frac{1}{i}=-i, \quad \frac{1}{i^2}=-1,\quad \frac{1}{i^3}=i,\cdots
\end{aligned}
\end{gather}
\انتہا{سوال}
%=======================
فرض کریں کہ \عددی{z_1=4+i3} اور \عددی{z_2=2-i5} ہیں۔سوال \حوالہ{سوال_مخلوط_جمع_منفی_الف} تا سوال \حوالہ{سوال_مخلوط_جمع_منفی_ب} کو حل کرتے ہوئے \عددی{x+iy} روپ میں لکھیں۔

%=========================
\ابتدا{سوال}\شناخت{سوال_مخلوط_جمع_منفی_الف}\quad
$(z_1-z_2)^2$\\
جواب:\quad
$-60+i32$
\انتہا{سوال}
%================================
\ابتدا{سوال}\quad
$\tfrac{z_1}{z_2}$\\
جواب:\quad
$-\tfrac{7}{29}+i\tfrac{26}{29}$
\انتہا{سوال}
%================================
\ابتدا{سوال}\quad
$\tfrac{1}{z^2_1}$\\
جواب:\quad
$\tfrac{7}{625}-i\tfrac{24}{625}$
\انتہا{سوال}
%================================
\ابتدا{سوال}\شناخت{سوال_مخلوط_جمع_منفی_ب}\quad
$\tfrac{2z_1}{3z_2}$\\
جواب:\quad
$-\tfrac{14}{87}+i\tfrac{52}{87}$
\انتہا{سوال}
%================================
سوال \حوالہ{سوال_مخلوط_حقیقی_خیالی_الف} تا سوال \حوالہ{سوال_مخلوط_حقیقی_خیالی_ب} کو حل کریں جہاں \عددی{z=x+iy} ہے۔

%===================
\ابتدا{سوال}\شناخت{سوال_مخلوط_حقیقی_خیالی_الف}\quad
$\tfrac{1}{1+i}\,\text{خیالی}$\\
جواب:\quad
$-\tfrac{1}{2}$
\انتہا{سوال}
%===============================
\ابتدا{سوال}\quad
$\tfrac{1-i}{1+i}\,\text{حقیقی}$\\
جواب:\quad
$0$
\انتہا{سوال}
%===============================
\ابتدا{سوال}\quad
$z^2\,\text{خیالی}$\\
جواب:\quad
$2xy$
\انتہا{سوال}
%===============================
\ابتدا{سوال}\quad
$z^3\,\text{حقیقی}$\\
جواب:\quad
$x^3-3xy^2$
\انتہا{سوال}
%===============================
\ابتدا{سوال}\quad
$z^4\,\text{خیالی}$\\
جواب:\quad
$4xy(x^2-y^2)$
\انتہا{سوال}
%===============================
\ابتدا{سوال}\quad
$\tfrac{(-1+i)^2}{-5+i4}\,\text{حقیقی}$\\
جواب:\quad
$-\tfrac{8}{41}$
\انتہا{سوال}
%===============================
\ابتدا{سوال}\quad
$\tfrac{3-i7}{-5+i2}\,\text{خیالی}$\\
جواب:\quad
$1$
\انتہا{سوال}
%===============================
\ابتدا{سوال}\quad
$\tfrac{3-i7}{-5+i2}\,\text{حقیقی}$\\
جواب:\quad
$-1$
\انتہا{سوال}
%===============================
\ابتدا{سوال}\quad
$\tfrac{z}{\bar{z}}\,\text{خیالی}$\\
جواب:\quad
$\tfrac{2xy}{x^2+y^2}$
\انتہا{سوال}
%==============================
\ابتدا{سوال}\شناخت{سوال_مخلوط_حقیقی_خیالی_ب}\quad
$\tfrac{z}{\bar{z}}\,\text{حقیقی}$\\
جواب:\quad
$\tfrac{x^2-y^2}{x^2+y^2}$
\انتہا{سوال}
%===============================
\ابتدا{سوال}\quad
قانون تبادل ثابت کریں (مساوات \حوالہ{مساوات_مخلوط_قواعد})۔
\انتہا{سوال}
%====================
\ابتدا{سوال}\quad
قانون تلازم ثابت کریں (مساوات \حوالہ{مساوات_مخلوط_قواعد})۔
\انتہا{سوال}
%====================
\ابتدا{سوال}\quad
قانون جزئیتی تقسیم ثابت کریں (مساوات \حوالہ{مساوات_مخلوط_قواعد})۔
\انتہا{سوال}
%==================
\ابتدا{سوال}\quad
اگر دو مخلوط اعداد کا حاصل ضرب صفر کے برابر ہو تب ثابت کریں کہ ان میں سے کم از کم ایک مخلوط عدد صفر ہو گا۔ 
\انتہا{سوال}
%========================
سوال \حوالہ{سوال_مخلوط_ثبوت_الف} تا سوال \حوالہ{سوال_مخلوط_ثبوت_ب} میں ثبوت پیش درکار ہیں۔

%======================
\ابتدا{سوال}\شناخت{سوال_مخلوط_ثبوت_الف}\quad
کسی بھی  عدد کے جوڑی دار مخلوط کا جوڑی دار مخلوط اس عدد کے برابر ہو گا۔
\انتہا{سوال}
%========================
\ابتدا{سوال}\quad
$\overline{iz}=-i\bar{z}$
\انتہا{سوال}
%========================
\ابتدا{سوال}\quad
\عددی{z} صرف اور صرف اس صورت حقیقی ہو گا جب \عددی{\bar{z}=z} ہو۔
\انتہا{سوال}
%=========================
\ابتدا{سوال}\quad
\عددی{z} صرف اور صرف اس صورت خالص خیالی ہو گا جب \عددی{\bar{z}=-z} ہو۔
\انتہا{سوال}
%=========================
\ابتدا{سوال}\quad
\عددی{z} صرف اور صرف اس صورت حقیقی یا خالص خیالی ہو گا جب \عددی{(\bar{z})^2=z^2} ہو۔
\انتہا{سوال}
%=========================
\ابتدا{سوال}\quad
مساوات \حوالہ{مساوات_مخلوط_تعلقات} ثابت کریں۔
\انتہا{سوال}
%=========================
\ابتدا{سوال}\quad
$(iz)\,\text{حقیقی}=-z\,\text{خیالی}, \quad (iz)\,\text{خیالی}=z\,\text{حقیقی}$
\انتہا{سوال}
%===========================
\ابتدا{سوال}\شناخت{سوال_مخلوط_ثبوت_ب}\quad
$(\overline{iz})\,\text{حقیقی}=-z\,\text{خیالی}, \quad (\overline{iz})\,\text{خیالی}=-z\,\text{حقیقی}$
\انتہا{سوال}
%=============================

\حصہ{مخلوط اعداد کی قطبی صورت۔ تکونی عدم مساوات}
ہم مخلوط سطح میں درج ذیل قطبی محدد \عددی{r}، \عددی{\theta} متعارف کرتے ہیں۔
\begin{align}
x=r\cos\theta,\quad y=r\sin\theta
\end{align}
یوں کسی بھی مخلوط عدد \عددی{z=x+iy\ne 0} کو
\begin{align}
z=r\cos\theta+ir\sin\theta=r(\cos\theta+i\sin\theta)
\end{align} 
لکھا جا سکتا ہے جو مخلوط عدد کی \اصطلاح{قطبی روپ}\فرہنگ{قطبی!روپ}\حاشیہب{polar form}\فرہنگ{polar!form} یا \ترچھا{تکونیاتی روپ}\فرہنگ{تکونیاتی!روپ}\حاشیہب{trigonometric form}\فرہنگ{trigonometric!form} کہلاتی ہے۔\عددی{r} کو مخلوط عدد \عددی{z} کی \اصطلاح{حتمی قیمت}\فرہنگ{حتمی!قیمت}\حاشیہب{absolute value}\فرہنگ{absolute!value} یا \اصطلاح{معیار}\فرہنگ{معیار}\حاشیہب{modulus}\فرہنگ{modulus} کہتے ہیں جسے \عددی{\abs{z}} سے ظاہر  کیا جاتا ہے۔ یوں درج ذیل ہو گا (شکل \حوالہ{شکل_مخلوط_سطح_پر_نقطے}-الف)۔
\begin{align}
\abs{z}=r=\sqrt{x^2+y^2}=\sqrt{z\bar{z}}\quad (\ge 0)
\end{align}
مثبت \عددی{x} محور سے لکیر \عددی{MN} تک زاویہ کو \عددی{z} کی \اصطلاح{دلیل}\فرہنگ{دلیل}\حاشیہب{argument}\فرہنگ{argument} کہتے ہیں جس کو \عددی{\phase{z}} سے ظاہر کیا جاتا ہے۔زاویہ کو ریڈیئن میں ناپا جاتا ہے۔گھڑی کی سوئیوں کے گھومنے کی الٹ رخ چلتے ہوئے زاویہ بڑھتا ہے۔\عددی{z} کا زاویہ درج ذیل ہو گا۔
\begin{align}  
\phase{z}=\theta=\sin^{-1}\frac{y}{r}=\cos^{-1}\frac{x}{r}=\tan^{-1}\frac{y}{x}
\end{align}
دھیان رہے کہ \عددی{z=0} کے لئے زاویہ \عددی{\theta} غیر معین ہے۔اسی لئے اوپر شرط \عددی{z\ne 0} لاگو کی گئی ہے۔

جیومیٹریائی طور پر مبدا \عددی{M} سے نقطہ \عددی{z} تک فاصلہ \عددی{\abs{z}} ہے (شکل \حوالہ{شکل_مخلوط_سطح_پر_نقطے}-الف)۔یوں
\begin{align*}
\abs{z_1}>\abs{z_2}
\end{align*}
کا مطلب ہے کہ مبدا سے \عددی{z_1} کا فاصلہ، مبدا سے \عددی{z_2} کے فاصلے سے زیادہ ہے اور \عددی{\abs{z_1-z_2}} سے مراد \عددی{z_1} اور \عددی{z_2} کے درمیان فاصلہ ہے  (شکل \حوالہ{شکل_مخلوط_سطح_پر_نقطے}-ب)۔

یہاں بتلاتا چلوں کہ حقیقی محور سے ہٹ کر  مخلوط اعداد کے لئے عدم مساوات \عددی{z_1<z_2} یا \عددی{z_1\ge z_2} کوئی معنی نہیں رکھتی ہیں۔

مخلوط عدد کے زاویہ \عددی{\theta} کی وہ قیمت جو  وقفہ
\begin{align*}
-\pi <\theta \le \pi
\end{align*}
 میں پائی جاتی ہو  کو \عددی{z} کے زاویے کی \اصطلاح{صدر قیمت}\فرہنگ{صدر!قیمت}\فرہنگ{قیمت!صدر}\حاشیہب{principal value}\فرہنگ{principal!value} کہتے ہیں جس کے ساتھ \عددی{\mp n\pi} جمع  کرنے سے \عددی{z} کے زاویے کی دیگر قیمتیں حاصل ہوتی ہیں جہاں \عددی{n=0,1,2,\cdots}  ہے۔
%
\begin{figure}
\centering
\begin{subfigure}{0.35\textwidth}
\centering
\begin{tikzpicture}
\pgfmathsetmacro{\len}{2.5}
\pgfmathsetmacro{\ang}{30}
\pgfmathsetmacro{\kA}{\len*cos(\ang)}
\pgfmathsetmacro{\kB}{\len*sin(\ang)}
%
\draw(0,0)--(3,0)node[right,align=right]{حقیقی\\محور};
\draw(0,0)--(0,2)node[above]{\RL{خیالی محور}};
\draw(0,0)node[below]{$M$}--++(\ang:\len)coordinate(T)node[right]{$z=x+iy$}node[above,pos=0.4,rotate=30]{$\abs{z}=r$};
\draw[-stealth]([shift={(0:0.5)}]0,0) arc (0:\ang:0.5);
\draw(1/2*\ang:0.8)node{$\theta$};
\draw[dashed](T)--(\kA,0)node[below,solid]{$x$};
\draw[dashed](T)--(0,\kB)node[left,solid]{$y$};
\draw(T)node[ocirc]{}node[above]{$N$};
\end{tikzpicture}
\caption*{(الف) مخلوط سطح۔مخلوط عدد کی قطبی روپ}
\end{subfigure}%
\begin{subfigure}{0.35\textwidth}
\centering
\begin{tikzpicture}
\draw(0,0)--(3,0)node[below]{$x$};
\draw(0,0)--(0,2)node[left]{$y$};
\draw(0,0)--(20:2.75)coordinate(A)node[right]{$z_1$}node[pos=0.7,below,rotate=20]{$\abs{z_1}$};
\draw(0,0)--(60:2.2)coordinate(B)node[left]{$z_2$}node[pos=0.6,shift={(150:0.3)},rotate=60]{$\abs{z_2}$};
\draw(A)--(B)node[pos=0.65,shift={(45:0.4)},sloped]{$\abs{z_1-z_2}$};
\draw(A)node[ocirc]{};
\draw(B)node[ocirc]{};
\end{tikzpicture}
\caption*{(ب) دو نقطوں کے مابین فاصلہ}
\end{subfigure}%
\begin{subfigure}{0.3\textwidth}
\centering
\begin{tikzpicture}
\draw(0,0)--(2,0)node[below]{$x$};
\draw(0,0)--(0,2)node[left]{$y$};
\draw(0,0)--++(45:1.4142)node[ocirc]{}node[right]{$1+i$};
\draw(1,0)node[below]{$1$}--++(0,0.1);
\draw(0,1)node[left]{$1$}--++(0.1,0);
\draw[-stealth]([shift={(0:0.5)}]0,0) arc (0:45:0.5);
\draw(22.5:0.8)node{$\tfrac{\pi}{4}$};
\end{tikzpicture}
\caption*{(پ) صدر زاویہ (مثال \حوالہ{مثال_مخلوط_عدد_قطبی_روپ})}
\end{subfigure}%
\caption{مخلوط سطح اور اس پر مخلوط نقطے۔}
\label{شکل_مخلوط_سطح_پر_نقطے}
\end{figure}

%=================
\ابتدا{مثال}\شناخت{مثال_مخلوط_عدد_قطبی_روپ}\quad \موٹا{مخلوط اعداد کی قطبی روپ۔ صدر قیمت}\\
فرض کریں کہ \عددی{z=1+i} ہے۔ تب درج ذیل ہو گا۔
\begin{align*}
z=\sqrt{2}\big(\cos \frac{\pi}{4}+i\sin\frac{\pi}{4}\big), \quad \abs{z}=\sqrt{2},\quad \phase{z}=\frac{\pi}{4}\mp2n\pi \quad \quad (n=0,1,\cdots)
\end{align*}
\عددی{z} کے زاویہ کی صدر قیمت \عددی{\tfrac{\pi}{4}} ہے (شکل \حوالہ{شکل_مخلوط_سطح_پر_نقطے}-پ)۔
\انتہا{مثال}
%======================
\ابتدا{مثال}\quad \موٹا{مخلوط اعداد کی قطبی روپ۔ صدر قیمت}\\
فرض کریں کہ \عددی{z=-2+i2\sqrt{3}} ہے تب \عددی{z=4(\cos \tfrac{2\pi}{3}+i\sin\tfrac{2\pi}{3})} ہو گا۔\عددی{z} کی حتمی قیمت \عددی{\abs{z}=4} اور اس کا صدر زاویہ \عددی{\tfrac{2\pi}{3}} ہو گا۔
\انتہا{مثال}
%======================

مخلوط اعداد کی  ضرب یا تقسیم میں مخلوط اعداد کی قطبی روپ نہایت مفید ثابت ہوتی ہے۔فرض کریں کہ
\begin{align*}
z_1=r_1(\cos\theta_1+i\sin\theta_1)\quad \text{اور}\quad z_2=r_2(\cos \theta_2+i\sin\theta_2)
\end{align*} 
ہیں۔مساوات \حوالہ{مساوات_مخلوط_عمل_ضرب} کے تحت 
\begin{align*}
z_1z_2=r_1r_2[(\cos\theta_1\cos\theta_2-\sin\theta_1\sin\theta_2)_i(\sin\theta_1\cos\theta_2)+\cos\theta_1\sin\theta_2]
\end{align*}
یعنی
\begin{align}\label{مساوات_مخلوط_ضرب_نتائج_الف}
z_1z_2=r_1r_2[\cos(\theta_1+\theta_2)+i\sin(\theta_1+\theta_2)]
\end{align}
ہو گا جس سے درج ذیل اہم نتائج
\begin{align}\label{مساوات_مخلوط_ضرب_نتائج_ب}
\abs{z_1z_2}=\abs{z_1}\abs{z_2}
\end{align}
اور
\begin{align}\label{مساوات_مخلوط_ضرب_نتائج_پ}
\phase{z_1z_2}=\phase{z_1}+\phase{z_2}\quad \quad\quad (\mp 2n\pi, \quad n=0,1,\cdots)
\end{align}
حاصل ہوتے ہیں۔اسی طرح تقسیم کی تعریف سے 
\begin{align}\label{مساوات_مخلوط_تقسیم_نتائج_الف}
\abs{\frac{z_1}{z_2}}=\frac{\abs{z_1}}{\abs{z_2}}
\end{align}
اور
\begin{align}\label{مساوات_مخلوط_تقسیم_نتائج_ب}
\phase{\tfrac{z_1}{z_2}}=\phase{z_1}-\phase{z_2} \quad \quad\quad (\mp 2n\pi, \quad n=0,1,\cdots)
\end{align}
حاصل ہوتے ہیں۔

%======================
\ابتدا{مثال}\quad \موٹا{کلیات ڈی موے ور}\\ 
مساوات \حوالہ{مساوات_مخلوط_ضرب_نتائج_ب} اور مساوات \حوالہ{مساوات_مخلوط_ضرب_نتائج_پ} سے درج ذیل حاصل ہوتا ہے
\begin{align}
z^n=r^n(\cos \theta+i\sin\theta)^n=r^n(\cos n\theta+i\sin n\theta)
\end{align}
جس سے \اصطلاح{ کلیہ ڈی موے ور}\فرہنگ{ڈی موے ور!کلیہ}\حاشیہب{De Moivre formula}\فرہنگ{De Moivre!formula} 
\begin{align}
(\cos\theta+i\sin\theta)^n=\cos n\theta+i\sin n\theta
\end{align}
حاصل ہوتا ہے۔
\انتہا{مثال}
%======================
