\باب{ترتیب اور تسلسل}
اس باب میں مخلوط اور حقیقی ترتیب اور تسلسل کے بنیادی تصورات پیش کیے جائیں گے۔

\حصہ{ترتیب}
تسلسل، بالخصوص طاقتی تسلسل مخلوط تجزیہ میں کلیدی کردار ادا کرتے ہیں۔ان کو متعارف کرنے کی خاطر ہم پہلے ترتیب اور اس سے متعلقہ تصورات کی تعریف پیش کرتے ہیں۔ہم دیکھیں گے کہ مخلوط  ترتیب اور تسلسل کی زیادہ تر مسئلے اور تعریف، حقیقی ترتیب اور تسلسل کے مسائل اور تعریف کی مانند ہوں گے جنہیں حقیقی علم الاحصاء میں استعمال کیا جاتا ہے۔

اگر ہر مثبت عدد صحیح \عددی{n} کو عدد \عددی{z_n} مختص کی جائے تب  ہم کہتے ہیں کہ اعداد
\begin{align*}
z_1,\,z_2,\cdots,z_n,\cdots
\end{align*} 
\اصطلاح{لامتناہی ترتیب}\فرہنگ{ترتیب!لامتناہی}\حاشیہب{infinite sequence}\فرہنگ{sequence} یا،  مختصراً، \اصطلاح{ترتیب} بناتے ہیں۔ان  اعداد \عددی{z_n} کو ترتیب کے \اصطلاح{مقدار} یا  \اصطلاح{اجزاء}\فرہنگ{اجزاء}\حاشیہب{terms}\فرہنگ{terms} کہتے ہیں۔

حقیقی اجزاء پر مبنی ترتیب کو \اصطلاح{حقیقی ترتیب}\فرہنگ{ترتیب!حقیقی}\فرہنگ{حقیقی!ترتیب}\حاشیہب{real sequence}\فرہنگ{real!sequence}\فرہنگ{sequence!real} کہتے ہیں۔

بعض اوقات ہم ترتیب کے اجزاء کی گنتی \عددی{0} یا \عددی{2} یا کسی دیگر عدد صحیح سے شروع کرتے ہیں۔

ایک ترتیب \عددی{z_1,z_2,\cdots} اس صورت \ترچھا{مرکوز}\فرہنگ{مرکوز}  یا \ترچھا{مرتکز}\فرہنگ{مرتکز} ہو گا جب ایسا عدد \عددی{c} پایا جاتا ہو کہ کسی بھی مثبت  (غیر صفر) حقیقی عدد \عددی{\epsilon} (جو چاہے جتنا چھوٹا کیوں نہ ہو) کی صورت میں ہم ایسا عدد صحیح \عددی{N} تلاش کر سکتے ہوں کہ  تمام \عددی{n>N} کے لئے درج ذیل درست ہو۔
\begin{align}\label{مساوات_ترتیب_حد_الف}
\abs{z_n-c}<\epsilon \quad \quad  n>N
\end{align}
\عددی{c} کو ترتیب کی \اصطلاح{حد}\فرہنگ{حد}\حاشیہب{limit}\فرہنگ{limit} کہتے ہیں جس کو عموماً
\begin{align*}
z_n\to c\quad\quad (n\to \infty)
\end{align*}
لکھا جاتا ہے اور ہم کہتے ہیں کہ ترتیب \عددی{c} کو مرکوز ہے یا کہ ترتیب کی حد \عددی{c} ہے۔

ایسا ترتیب جو مرتکز نہ ہو \اصطلاح{منفرج}\فرہنگ{منفرج}\حاشیہب{divergent}\فرہنگ{divergent} کہلاتا ہے۔

مساوات \حوالہ{مساوات_ترتیب_حد_الف} کا  ایک سادہ جیومیٹریائی مطلب ہے۔یہ مساوات کہتی ہے  کہ \عددی{n>N} کی صورت میں ہر جزو \عددی{z_n} اس کھلے قرص میں پایا جاتا ہے جس کا رداس \عددی{\epsilon} اور مرکز \عددی{c} ہے (شکل \حوالہ{شکل_ترتیب_مخلوط_مرتکز})  جبکہ قرص کا رداس \عددی{\epsilon} کتنا ہی کم کیوں نہ کر دیا جائے  اس قرص کے باہر  اجزاء \عددی{z_n} کی زیادہ سے زیادہ تعداد محدود ہو گی۔ظاہر ہے کہ \عددی{N} کی قیمت عموماً \عددی{\epsilon} پر منحصر ہو گی۔
\begin{figure}
\centering
\begin{tikzpicture}
%\draw[thick] (0,0) grid (3,2);
%\draw[thin,gray,step=0.1](0,0) grid (3,2);
%
\draw(0,0)--(3,0)node[right]{$x$};
\draw(0,0)--(0,2)node[left]{$y$};
\foreach \x/\y in {0.1/2,0.2/1.9,0.4/1.8}{\draw[fill](\x,\y)circle (1pt);}
\foreach \x/\y in {0.55/1.85,0.7/1.7,0.9/1.5,1/1.5}{\draw[fill](\x,\y)circle (1pt);}
\foreach \x/\y in {1.05/1.45,1.2/1.35,1.3/1.3,1.3/1.2,1.35/1.2,1.4/1.2}{\draw[fill](\x,\y)circle (1pt);}
\foreach \x/\y in {1.4/1.15,1.5/1.05,1.45/1.15,1.5/1.1}{\draw[fill](\x,\y)circle (1pt);}
\draw(1.5,1)circle (0.75);
\draw[-latex](1.5,1)node[ocirc]{}node[below]{$c$}--++(30:0.75)node[pos=0.35,above]{$\epsilon$};
\end{tikzpicture}
\caption{مرتکز مخلوط ترتیب}
\label{شکل_ترتیب_مخلوط_مرتکز}
\end{figure} 

حقیقی ترتیب کی صورت میں مساوات \حوالہ{مساوات_ترتیب_حد_الف} جیومیٹریائی طور کہتی ہے کہ \عددی{n>N} کی صورت میں جزو \عددی{z_n} وقفہ \عددی{c-\epsilon} تا \عددی{c+\epsilon} پر پایا جائے گا (شکل \حوالہ{شکل_ترتیب_حقیقی_مرتکز}) اور اس وقفہ سے باہر اجزاء کی زیادہ سے زیادہ تعداد محدود ہو گی۔ 
\begin{figure}
\centering
\begin{tikzpicture}
\draw(-2.5,0)--(2.5,0)node[right]{$x$};
\draw(-1.5,0)node[below]{$c-\epsilon$}--++(0,0.1);
\draw(0,0)node[below]{$c$}--++(0,0.1);
\draw(1.5,0)node[below]{$c+\epsilon$}--++(0,0.1);
\end{tikzpicture}
\caption{حقیقی مرتکز ترتیب}
\label{شکل_ترتیب_حقیقی_مرتکز}
\end{figure}

%=======================
\ابتدا{مثال}\شناخت{مثال_ترتیب_مرتکز_اور_منفرج_ترتیب}\quad \موٹا{مرتکز اور منفرج ترتیب}\\
ترتیب \عددی{z_n=1+\tfrac{2}{n}} کے اجزاء 
$3,2,\tfrac{5}{3},\tfrac{6}{4},\tfrac{7}{5},\cdots$
ہیں۔یہ ترتیب  مرتکز ہے اور اس کی حد \عددی{c=1} ہے۔در حقیقت مساوات \حوالہ{مساوات_ترتیب_حد_الف} سے
\begin{align*}
z_n-c=1+\tfrac{2}{n}-1=\tfrac{2}{n}
\end{align*}
لکھا جا سکتا ہے۔یوں \عددی{\tfrac{2}{n}<\epsilon} اس صورت ہو گا جب \عددی{\tfrac{n}{2}>\tfrac{1}{\epsilon}} یا \عددی{n>\tfrac{2}{\epsilon}} ہو۔مثلاً \عددی{\epsilon=0.01} منتخب کرتے ہوئے \عددی{\tfrac{2}{n}<0.01} تب ہو گا جب \عددی{n>200} ہو۔

ترتیب
$1,2,3,\cdots$
اور
$\tfrac{1}{4},\tfrac{3}{4},\tfrac{1}{5},\tfrac{4}{5},\tfrac{1}{6},\tfrac{5}{6},\cdots$
منفرج ہیں۔

وہ ترتیب جس کے اجزاء 
\begin{align*}
z_n=2-\frac{1}{n}+i\big(1+\frac{2}{n}\big)
\end{align*}
یعنی
\begin{align*}
1+i,\quad \frac{3}{2}+i2,\quad \frac{7}{4}+i\frac{3}{2},\cdots
\end{align*}
ہیں کو شکل میں دکھایا گیا ہے۔یہ ترتیب مرتکز ہے اور اس کی حد \عددی{c=2+i} ہے۔مساوات \حوالہ{مساوات_ترتیب_حد_الف} سے
\begin{align*}
\abs{z_n-c}=\abs{\frac{2n-1}{n}+i\frac{n+2}{n}-(2+i)}=\abs{-\frac{1}{n}+i\frac{2}{n}}=\frac{\sqrt{5}}{n}
\end{align*}
لکھا جا سکتا ہے۔یوں \عددی{\tfrac{\sqrt{5}}{n}<\epsilon} تب ہو گا جب \عددی{\tfrac{n}{\sqrt{5}}>\tfrac{1}{\epsilon}} یعنی \عددی{n>\tfrac{\sqrt{5}}{\epsilon}} ہو۔مثال کے طور پر \عددی{\epsilon=\tfrac{1}{100}} منتخب کرتے ہوئے  \عددی{\abs{z_n-c}<\epsilon} تب ہو گا جب \عددی{n>223.6} یعنی
\عددی{n=224} یا \عددی{n=225}، وغیرہ ہو۔
\begin{figure}
\centering
\begin{tikzpicture}
\draw(0,0)--(3,0)node[right]{$x$};
\draw(0,0)--(0,3.5)node[right]{$y$};
\foreach \x in {1,2}{\draw(\x,0)node[below]{$\x$}--++(0,0.1);}
\foreach \y in {1,2,3}{\draw(0,\y)node[left]{$\y$}--++(0.1,0);}
\foreach \n in {1,2,3,4,5,6,7,8,9,10,11,12,16,20,25,30,40,60,80,90,100}{\draw[fill](2-1/\n,1+2/\n) circle (1pt);}
\draw(2,1)node[ocirc]{}node[right]{$c=2+i$};
\draw(0,0)node[below]{$0$}node[left]{$0$};
\end{tikzpicture}
\caption{مثال \حوالہ{مثال_ترتیب_مرتکز_اور_منفرج_ترتیب} میں آخری ترتیب}
\label{شکل_مثال_ترتیب_مرتکز_اور_منفرج_ترتیب}
\end{figure}
\انتہا{مثال}
%===========================
مخلوط ترتیب \عددی{z_1,z_2,z_3,\cdots} کی صورت میں \عددی{z_n=x_n+iy_n} لکھ کر ہم حقیقی حصوں کی ترتیب اور خیالی حصوں کی ترتیب
\begin{align*}
x_1,x_2,x_3,\cdots\quad \text{اور}\quad y_1,y_2,y_3,\cdots
\end{align*}
 پر علیحدہ علیحدہ غور کر سکتے ہیں۔مثلاً  مثال \حوالہ{مثال_ترتیب_مرتکز_اور_منفرج_ترتیب} کی آخری ترتیب کے دو علیحدہ علیحدہ ترتیب درج ذیل ہوں گی۔
\begin{align*}
1,\frac{3}{2},\frac{5}{3},\frac{7}{4},\cdots \quad \text{اور}\quad 3,2,\frac{5}{3},\frac{3}{2},\cdots
\end{align*}
ہم دیکھتے ہیں کہ حقیقی اور خیالی ترتیب کے حد  بالترتیب \عددی{2} اور \عددی{1} ہیں (شکل \حوالہ{شکل_مثال_ترتیب_مرتکز_اور_منفرج_ترتیب}) جو اصل مخلوط ترتیب کی حقیقی اور خیالی حصوں کی حد ہیں۔عموماً ایسا ہی ہوتا ہے جو درج ذیل کی ایک مثال ہے۔ 

%==================
\ابتدا{مسئلہ}\شناخت{مسئلہ_ترتیب_حقیقی_خیالی_اجزاء_ترتیب}\quad \موٹا{(حقیقی اور خیالی اجزاء کی ترتیب)}\\
مخلوط اعداد \عددی{z_n=x_n+iy_n\quad (n=1,2,\cdots)} کی ترتیب \عددی{z_1,z_2,\cdots,z_n,\cdots} صرف اور صرف اس صورت حد \عددی{c=a+ib} پر مرکوز ہو گا جب حقیقی حصوں کی ترتیب   \عددی{x_1,x_2,\cdots} نقطہ \عددی{a} پر مرتکز ہو اور خیالی حصوں کی ترتیب \عددی{y_1,y_2,\cdots} نقطہ \عددی{b} پر مرتکز ہو۔
\انتہا{مسئلہ}
%=======================
\ابتدا{ثبوت}\quad
اگر \عددی{\abs{z_n-c}<\epsilon} ہو تب  \عددی{z_n=x_n+iy_n} اس دائرہ کے اندر پایا جائے گا جس کا رداس \عددی{\epsilon} اور مرکز \عددی{c=a+ib} ہوں۔ یوں لازماً 
\begin{align*}
\abs{x_n-a}<\epsilon, \quad \abs{y_n-b}<\epsilon
\end{align*}
ہو گا (شکل \حوالہ{شکل_مسئلہ_ترتیب_حقیقی_خیالی_اجزاء_ترتیب}-الف)۔ یوں \عددی{n\to \infty} کی صورت میں مرکوزیت \عددی{z_n\to c} سے مراد مرکوزیت \عددی{x_n\to a} اور مرکوزیت \عددی{y_n\to b} ہے۔
\begin{figure}
\centering
\begin{subfigure}{0.5\textwidth}
\centering
\begin{tikzpicture}
\pgfmathsetmacro{\xlen}{1.5}
\pgfmathsetmacro{\ylen}{1.5}
\pgfmathsetmacro{\r}{1}
\draw(\xlen,\ylen)coordinate(kA);
%
\draw(0,0)--(3.5,0)node[right]{$x$};
\draw(0,0)--(0,3)node[right]{$y$};
%
\draw[dashed](kA)--(\xlen,0)node[below]{$a$};
\draw[dashed](kA)--(0,\ylen)node[left]{$b$};
\draw[dashed](\xlen,\ylen+\r)--(0,\ylen+\r)node[left]{$b+\epsilon$};
\draw[dashed](\xlen,\ylen-\r)--(0,\ylen-\r)node[left]{$b-\epsilon$};
\draw[dashed](\xlen-\r,\ylen)--(\xlen-\r,0)node[below]{$a-\epsilon$};
\draw[dashed](\xlen+\r,\ylen)--(\xlen+\r,0)node[below]{$a+\epsilon$};
\draw(kA)node[ocirc]{}node[right]{$c$} circle (\r);
\end{tikzpicture}
\caption*{(الف)}
\end{subfigure}%
\begin{subfigure}{0.5\textwidth}
\centering
\begin{tikzpicture}
\pgfmathsetmacro{\xlen}{1.5}
\pgfmathsetmacro{\ylen}{1.5}
\pgfmathsetmacro{\r}{1}
\draw(\xlen,\ylen)coordinate(kA);
%
\draw(0,0)--(3.5,0)node[right]{$x$};
\draw(0,0)--(0,3)node[right]{$y$};
%
\draw[dashed](kA)--(\xlen,0)node[below]{$a$};
\draw[dashed](kA)--(0,\ylen)node[left]{$b$};
\draw[dashed](\xlen-\r/2,\ylen+\r/2)--(0,\ylen+\r/2)node[left]{$b+\tfrac{\epsilon}{2}$};
\draw[dashed](\xlen-\r/2,\ylen-\r/2)--(0,\ylen-\r/2)node[left]{$b-\tfrac{\epsilon}{2}$};
\draw[dashed](\xlen-\r/2,\ylen-\r/2)--(\xlen-\r/2,0);
\draw[dashed](\xlen+\r/2,\ylen-\r/2)--(\xlen+\r/2,0);
\draw(kA)node[ocirc]{}node[right]{$c$} circle (\r);
\draw[thick](\xlen-\r/2,\ylen-\r/2) rectangle ++(\r,\r);
\draw[stealth-](\xlen-\r/2,0)++(0,-0.1) to [out=-90,in=0]++(-0.3,-0.3)node[left]{$a-\tfrac{\epsilon}{2}$};
\draw[stealth-](\xlen+\r/2,0)++(0,-0.1) to [out=-90,in=180]++(0.3,-0.3)node[right]{$a+\tfrac{\epsilon}{2}$};
\end{tikzpicture}
\caption*{(ب)}
\end{subfigure}%
\caption{مسئلہ \حوالہ{مسئلہ_ترتیب_حقیقی_خیالی_اجزاء_ترتیب} کا ثبوت}
\label{شکل_مسئلہ_ترتیب_حقیقی_خیالی_اجزاء_ترتیب}
\end{figure}

اس کی الٹ چلتے ہوئے، اگر \عددی{n\to \infty} کی صورت میں \عددی{x_n\to a} اور \عددی{y_n\to b} ہوں  تب کسی بھی دیے گئے \عددی{\epsilon>0} کی صورت میں ہم ایسا \عددی{N} اتنا بڑا منتخب کر سکتے ہیں کہ ہر \عددی{n>N} کے لئے 
\begin{align*}
\abs{x_n-a}<\frac{\epsilon}{2},\quad \abs{y_n-b}<\frac{\epsilon}{2}
\end{align*}
ہو۔ان دو عدم مساوات کہتی ہیں کہ \عددی{z_n=x_n+iy_n} اس چکور کے اندر پایا جائے گا جس کے اطراف کی لمبائی \عددی{\epsilon} اور مرکز \عددی{c} ہو (شکل \حوالہ{شکل_مسئلہ_ترتیب_حقیقی_خیالی_اجزاء_ترتیب}-ب)۔یوں ثبوت مکمل ہوتا ہے۔
\انتہا{ثبوت}
%=================

اس مسئلہ کی باعث حقیقی حصہ اور خیالی حصہ کی ترتیب پر غور کرتے ہوئے مخلوط ترتیب کی مرکوزیت کو حقیقی ترتیب سے حاصل کیا جا سکتا ہے۔

اگر ایسا مثبت عدد \عددی{K} پایا جاتا ہو کہ مرکز پر رداس \عددی{K} کے دائرے میں ترتیب \عددی{z_1,z_2,\cdots} کے تمام اجزاء  پائے جاتے ہوں  یعنی
\begin{align*}
\abs{z_n}<K \quad \quad \text{\RL{تمام $n$}}
\end{align*}
تب یہ ترتیب \اصطلاح{محدود}\فرہنگ{محدود}\حاشیہب{bounded}\فرہنگ{bounded} کہلاتا  ہے۔ایسا ترتیب جو محدود نہ ہو \اصطلاح{غیر محدود}\فرہنگ{غیر محدود}\حاشیہب{unbounded}\فرہنگ{unbounded} کہلاتا ہے۔

اس تصور کو استعمال کرتے ہوئے انفراج  کو عموماً درج ذیل سادہ مسئلہ  سے دریافت کیا جا سکتا ہے۔

%=========================
\ابتدا{مسئلہ}\شناخت{مسئلہ_ترتیب_مرتکز_ترتیب_محدود_ہو_گی}
ہر مرتکز ترتیب محدود ہو گی۔یوں اگر ایک ترتیب غیر محدود ہو تب یہ منفرج ہو گی۔
\انتہا{مسئلہ}
%===========================
\ابتدا{ثبوت}
فرض کریں کہ ترتیب \عددی{z_1,z_2,\cdots} مرکوز ہے اور اس کی حد \عددی{c} ہے۔ تب ہم \عددی{\epsilon>0} منتخب کرتے ہوئے ایسا مطابقتی \عددی{N} تلاش کر سکتے ہیں کہ  \عددی{n>N} کے لئے ہر \عددی{z_n} رداس \عددی{\epsilon} کے قرص، جس کا مرکز \عددی{c} ہو، میں پائے جائیں گے اور وہ  \عددی{z_n} جو اس قرص کے باہر ہوں کی زیادہ سے زیادہ تعداد محدود ہو گی۔اب ظاہر ہے کہ ہم مرکز پر اتنے بڑی  رداس \عددی{K} کا دائرہ منتخب کر سکتے ہیں کہ یہ قرص اور قرص کے باہر تمام \عددی{z_n} اس دائرے میں پائیں جاتے ہوں۔اس سے ثابت ہوتا ہے کہ یہ ترتیب محدود ہے۔
\انتہا{ثبوت}
%=========================

یہاں دہان رہے کہ محدود ہونا مرکوزیت کے لئے کافی نہیں ہے۔مثلاً ترتیب \عددی{1,0,1,0,\cdots} محدود  لیکن منفرج ہے۔ (کیوں؟) غیر محدود ترتیب کی مثالیں درج ذیل ہیں
\begin{align*}
1,2,3,4,\cdots\quad \text{اور}\quad \frac{1}{2},2,\frac{1}{3},3,\frac{1}{4},4,\cdots
\end{align*} 
جو مسئلہ \حوالہ{مسئلہ_ترتیب_مرتکز_ترتیب_محدود_ہو_گی} کے تحت منفرج ترتیب ہیں۔

%=====================
\حصہء{سوالات}
سوال \حوالہ{سوال_ترتیب_نقشہ_الف} تا سوال \حوالہ{سوال_ترتیب_نقشہ_ب} میں دیے ترتیب کے ابتدائی چند اجزاء لکھ کر ترسیم کریں۔

%======================
\ابتدا{سوال}\شناخت{سوال_ترتیب_نقشہ_الف}\quad
$\tfrac{n}{n+3}$\\
جواب:\quad
$\tfrac{1}{4},\tfrac{2}{5},\tfrac{1}{2},\tfrac{4}{7},\tfrac{5}{8},\cdots$
\انتہا{سوال}
%======================
\ابتدا{سوال}\quad
$\tfrac{2n}{n^2+1}$\\
جواب:\quad
$1,\tfrac{4}{5},\tfrac{3}{5},\tfrac{8}{17},\tfrac{5}{13},\cdots$
\انتہا{سوال}
%======================
\ابتدا{سوال}\quad
$\tfrac{i^n}{n^2}$\\
جواب:\quad
$i,-\tfrac{1}{4},-\tfrac{i}{9},\tfrac{1}{16},\tfrac{i}{25},\cdots$
\انتہا{سوال}
%======================
\ابتدا{سوال}\quad
$\tfrac{in}{n+1}$\\
جواب:\quad
$\tfrac{i}{2},\tfrac{i2}{3},\tfrac{i3}{4},\tfrac{i4}{5},\tfrac{i5}{6},\cdots$
\انتہا{سوال}
%======================
\ابتدا{سوال}\quad
$\tfrac{i^n n^2}{n+i}$\\
جواب:\quad
$\tfrac{1}{2}(1+i),\tfrac{4}{5}(-2+i),\tfrac{9}{10}(-1-i3),\tfrac{16}{17}(4-i),\tfrac{25}{26}(1+i5),\cdots$
\انتہا{سوال}
%======================
\ابتدا{سوال}\شناخت{سوال_ترتیب_نقشہ_ب}\quad
$(-1)^n+i2\pi n$\\
جواب:\quad
$-1+i2\pi, 1+i4\pi,-1+i6\pi, 1+i8\pi, -1+i10\pi,\cdots$
\انتہا{سوال}
%======================
\ابتدا{سوال}\quad
ترتیب \عددی{z_1=1}، \عددی{z_2=\tfrac{i}{2}}، \عددی{z_n=iz_{n-2}z_{n-1}\,\, (n=3,4,\cdots)} کے ابتدائی چند اجزاء لکھیں۔اس ترتیب  کی حد تلاش کریں۔\\
جواب:\quad
$1,\tfrac{i}{2}, -\tfrac{1}{2},\tfrac{1}{4}, -\tfrac{i}{8},\cdots$
\انتہا{سوال}
%=========================
سوال \حوالہ{سوال_ترتیب_محدود_مرکوز_حد_الف} تا سوال \حوالہ{سوال_ترتیب_محدود_مرکوز_حد_ب} میں دریافت کریں کہ آیا دی گئی ترتیب محدود ہے؟ کیا یہ ترتیب مرکوز ہے؟ مرکوزیت کی صورت میں ترتیب کی حد تلاش کریں۔
  
%================
\ابتدا{سوال}\شناخت{سوال_ترتیب_محدود_مرکوز_حد_الف}\quad
$z_n=i^n$\\
جواب:\quad
محدود، منفرج
\انتہا{سوال}
%======================
\ابتدا{سوال}\quad
$z_n=\tfrac{i^n}{n}$\\
جواب:\quad
محدود، مرکوز، حد $0$
\انتہا{سوال}
%======================
\ابتدا{سوال}\quad
$z_n=\tfrac{in}{n+1}$\\
جواب:\quad
محدود، مرکوز، حد $i$
\انتہا{سوال}
%======================
\ابتدا{سوال}\quad
$z_n=\tfrac{n^2}{n+i}$\\
جواب:\quad
غیر محدود، منفرج
\انتہا{سوال}
%======================
\ابتدا{سوال}\quad
$z_n=\tfrac{(-1)^n}{n^3}$\\
جواب:\quad
محدود، مرکوز، حد $0$
\انتہا{سوال}
%======================
\ابتدا{سوال}\شناخت{سوال_ترتیب_محدود_مرکوز_حد_ب}\quad
$z_n=e^{i\tfrac{n\pi}{4}}$\\
جواب:\quad
محدود، منفرج
\انتہا{سوال}
%======================
