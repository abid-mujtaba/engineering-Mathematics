
\حصہ{نقطہ فاصل کے جانچ کا اصول۔استحکام}
ہم مستقل عددی سر والے متجانس خطی نظام \حوالہ{مساوات_نظام_جانچ_نقطہ_فاصل_الف} پر گفتگو جاری رکھتے ہیں۔
\begin{gather}\label{مساوات_نظام_جانچ_نقطہ_فاصل_الف}
\begin{aligned}
\bM{y}'=\begin{bmatrix} a_{11} & a_{12} \\ a_{21} & a_{22} \end{bmatrix} \bM{y}
\end{aligned}, \quad \implies \quad 
\begin{aligned}
y_1'&=a_{11}y_1+a_{12}y_2\\
y_2'&=a_{21}y_1'+a_{22}y_2
\end{aligned}
\end{gather}
اب تک  حصہ \حوالہ{حصہ_نظام_مستقل_عددی_سر_نظام} میں ہم نے دیکھا کہ نسل حل \عددی{\bM{y}=[y_1(t)\quad y_2(t)]^T} کے خطوط کو \عددی{y_1 y_2} \اصطلاح{سطح حرکت} پر کھینچتے ہوئے عمومی جائزہ لیا جا سکتا ہے۔ اس سطح پر منحنی کو نظام \حوالہ{مساوات_نظام_جانچ_نقطہ_فاصل_الف} کا \اصطلاح{خط حرکت} کہتے ہیں۔تمام خط حرکت کو ملا کر \اصطلاح{پیکر مرحلہ} حاصل ہوتا ہے۔

ہم دیکھ چکے کہ \عددی{\bM{y}=\bM{x}e^{\lambda t}} کو حل تصور کرتے ہوئے مساوات \حوالہ{مساوات_نظام_جانچ_نقطہ_فاصل_الف} میں پر کرتے ہوئے
\begin{align*}
\bM{y}'=\lambda \bM{x}e^{\lambda t}=\bM{A}\bM{y}=\bM{A}\bM{x}e^{\lambda t}
\end{align*}
لکھا جا سکتا ہے جس کو \عددی{e^{\lambda t}} سے تقسیم کرتے ہوئے
\begin{align}\label{مساوات_نظام_جانچ_نقطہ_فاصل_ب}
\bM{A}\bM{x}=\lambda \bM{x}
\end{align}
ملتا ہے۔یوں \عددی{\lambda} قالب \عددی{\bM{A}}  کا آئگنی قدر اور \عددی{\bM{x}} نظیری آئگنی سمتیہ ہونے کی صورت میں  \عددی{\bM{y}(t)} مساوات \حوالہ{مساوات_نظام_جانچ_نقطہ_فاصل_الف} کا (غیر صفر) حل ہو گا۔

گزشتہ حصے کے مثالوں سے واضح ہے کہ پیکر مرحلہ کی صورت کا دارومدار بڑی حد تک نظام \حوالہ{مساوات_نظام_جانچ_نقطہ_فاصل_الف} کی \اصطلاح{نقطہ فاصل} کی قسم پر منحصر ہے جہاں نقطہ فاصل سے مراد ایسا نقطہ ہے جہاں \عددی{\tfrac{\dif y_1}{\dif y_2}} نا قابل معلوم قیمت \عددی{\tfrac{0}{0}} ہو۔[مساوات \حوالہ{مساوات_نظام_نقطہ_فاصل_الف} دیکھیں۔]
\begin{align}\label{مساوات_نظام_جانچ_نقطہ_فاصل_پ}
\frac{\dif y_2}{\dif y_1}=\frac{y_2' \dif t}{y_1'\dif t}=\frac{y_2'}{y_1'}=\frac{a_{21}y_1+a_{22}y_2}{a_{11}y_1+a_{12}y_2}
\end{align}
حصہ \حوالہ{حصہ_نظام_مستقل_عددی_سر_نظام}  سے ہم یہ بھی جانتے ہیں نقطہ فاصل کے کئی اقسام پائے جاتے ہیں۔

موجودہ حصے میں ہم دیکھیں گے کہ  نقطہ فاصل کی قسم  کا تعلق آئگنی قدر سے ہے جو امتیازی مساوات
\begin{align}\label{مساوات_نظام_جانچ_نقطہ_فاصل_ت}
\abs{\bM{A}-\lambda \bM{I}}=\begin{vmatrix} a_{11}-\lambda & a_{12} \\a_{21}&a_{22}-\lambda\end{vmatrix}=\lambda^2-(a_{11}+a_{22})\lambda+a_{11}a_{22}-a_{12}a_{21}=0
\end{align}
کے حل \عددی{\lambda_1} اور \عددی{\lambda_2} ہیں۔امتیازی مساوات دو درجی مساوات \عددی{\lambda^2-p\lambda+q=0} ہے جس کے عددی سر \عددی{p}،  \عددی{q} اور \اصطلاح{جدا کنندہ}\فرہنگ{جدا کنندہ}\حاشیہب{discriminant}\فرہنگ{discriminant} \عددی{\Delta} درج ذیل ہیں۔
\begin{align}\label{مساوات_نظام_جانچ_نقطہ_فاصل_ٹ}
p=a_{11}+a_{22}, \quad q=a_{11}a_{22}-a_{12}a_{21},\quad \Delta=p^2-4q
\end{align}
دو درجی مساوات کے حل الجبرا کی مدد سے \عددی{\lambda=\tfrac{1}{2}(p+\mp\sqrt{p^2-4q})} یعنی
\begin{align}\label{مساوات_نظام_جانچ_نقطہ_فاصل_ث}
\lambda_1=\frac{1}{2}(p+\sqrt{\Delta}),\quad \lambda_2=\frac{1}{2}(p-\sqrt{\Delta})
\end{align} 
لکھتے ہیں۔ان آئگنی قیمتوں کو استعمال کرتے ہوئے امتیازی مساوات کو اجزائے ضربی کی صورت 
\begin{align*}
(\lambda-\lambda_1)(\lambda-\lambda_2)=\lambda^2-(\lambda_1+\lambda_2)+\lambda_1\lambda_2=0
\end{align*}
میں لکھا جا سکتا ہے  جہاں سے ظاہر ہے کہ \عددی{p} آئگنی قیمتوں کا مجموعہ ہے جبکہ \عددی{q} ان کا حاصل ضرب ہے۔اسی طرح مساوات \حوالہ{مساوات_نظام_جانچ_نقطہ_فاصل_ث} کی مدد سے \عددی{\lambda_1-\lambda_2=\sqrt{\Delta}} لکھا جا سکتا ہے۔
\begin{align}
p=\lambda_1+\lambda_2,\quad q=\lambda_1\lambda_2,\quad \Delta=(\lambda_1-\lambda_2)^2
\end{align}
ان نتائج سے  نقطہ فاصل کی جانچ کے اصول طے کئے جا سکتے ہیں جنہیں جدول \حوالہ{جدول_نظام_نقطہ_فاصل_اصول_جانچ} میں پیش کیا گیا ہے۔ان اصولوں کو اسی حصے میں اخذ کیا جائے گا۔
\begin{table}
\caption{آئگنی قدر سے نقطہ فاصل کی درجہ بندی۔}
\label{جدول_نظام_نقطہ_فاصل_اصول_جانچ}
\centering
\begin{tabular}{rcccr}
نام& $p=\lambda_1+\lambda_2$&$q=\lambda_1\lambda_2$&$\Delta=(\lambda_1-\lambda_2)^2$&$\lambda_1$ اور $\lambda_2$  پر تبصرہ\\
\hline
(الف) جوڑ & & \عددی{q >0} & \عددی{\Delta \ge 0} & حقیقی۔یکساں علامتیں\\
(ب) نقطہ نیز & & \عددی{q<0}& & حقیقی۔آپس میں الٹ علامتیں\\
(پ) وسط & \عددی{p=0} & \عددی{q>0}&& خیالی عدد (حقیقی جزو صفر ہے)\\
(ت) نقطہ مرغولہ & \عددی{p \ne 0} & & \عددی{\Delta <0}& مخلوط عدد (حقیقی اور خیالی اجزاء غیر صفر ہیں)
\end{tabular}
\end{table}
%===================

\جزوحصہء{استحکام}
نقطہ فاصل کی درجہ بندی ان کی \اصطلاح{استحکام}\فرہنگ{استحکام}\حاشیہب{stability}\فرہنگ{stability} کی بنیاد پر بھی کی جا سکتی ہے۔انجینئری کے علاوہ دیگر شعبوں میں بھی  استحکام نہایت اہم تصور ہے۔مستحکم نظام میں کسی لمحے پر معمولی تبدیلی یا خلل سے بعد کے تمام لمحات پر معمولی خلل ہی پایا جاتا ہے۔ نقطہ فاصل کے لئے درج ذیل تصورات اہم ہیں۔
%============

\ابتدا{تعریف}\quad مستحکم، غیر مستحکم، مستحکم اور جاذب\\
اگر نظام \حوالہ{مساوات_نظام_جانچ_نقطہ_فاصل_الف} کے نقطہ فاصل \عددی{P_0} کے قریب تمام خط حرکت مستقبل میں بھی \عددی{P_0} کے قریب رہیں تب \عددی{P_0} \اصطلاح{مستحکم}\فرہنگ{مستحکم}\حاشیہب{stable}\فرہنگ{stable}  کہلائے گا۔ یوں اگر کسی بھی رداس \عددی{\epsilon} کی ٹکیا \عددی{D_\epsilon} کے لئے  رداس \عددی{\sigma} کی ایسی ٹکیا \عددی{D_\sigma} موجود ہو،  جہاں دونوں ٹکیوں کا مرکز \عددی{P_0} ہے، کہ  ٹکیا \عددی{D_\sigma} میں (لمحہ \عددی{t=t_1} کا نظیری) نقطہ \عددی{P_1} پر  پائے جانے والا، نظام \حوالہ{مساوات_نظام_جانچ_نقطہ_فاصل_الف} کا ہر خط حرکت، مستقبل میں ٹکیا \عددی{D_{\epsilon}} میں رہتا ہو، تب \عددی{P_0} کا نقطہ فاصل \اصطلاح{مستحکم}\فرہنگ{مستحکم}\حاشیہب{stable}\فرہنگ{stable}\حاشیہد{روسی ریاضی دان سکندر میکائل لیاپونو [1857-1918] کا مستحکم تفرقی مساوات پر کام بنیادی حیثیت رکھتا ہے۔استحکام کی یہ تعریف انہوں نے ہی پیش کی۔} کہلائے گا۔[شکل \حوالہ{شکل_نظام_نقطہ_فاصل_تعریف}-الف دیکھیں]

اگر \عددی{P_0} مستحکم نہ ہو تب یہ \اصطلاح{غیر مستحکم}\فرہنگ{غیر مستحکم}\حاشیہب{unstable}\فرہنگ{unstable} کہلاتا ہے۔ 

ایسا مستحکم \عددی{P_0} جہاں وہ تمام خط حرکت جن کا کوئی بھی نقطہ، \عددی{D_{\sigma}} پر پایا جاتا ہو، آخر کار (\عددیء{t \to \infty}) \عددی{P_0} کے قریب تر پہنچے  \اصطلاح{مستحکم اور جاذب}\فرہنگ{مستحکم اور جاذب}\حاشیہب{stable and attractive}\فرہنگ{stable and attractive} کہلاتا ہے۔[شکل \حوالہ{شکل_نظام_نقطہ_فاصل_تعریف}-ب دیکھیں۔]
\انتہا{تعریف}
%==============================

\begin{figure}
\centering
\begin{subfigure}{0.5\textwidth}
\centering
\begin{tikzpicture}
\draw(0,0) circle (1.25);
\draw(0,0) circle (2);
\draw[-stealth](0,0)--++(25:1.25)node[pos=0.5,fill=white]{$\sigma$};
\draw[-stealth](0,0)--++(125:2)node[pos=0.25,fill=white]{$\epsilon$};
\draw[fill=white](0,0) node [ocirc]{}node[below]{$P_0$};
%trajectory
\draw[-stealth](0,1)node[ocirc]{}node[below]{$P_1$} to [out=-10,in=-170]++(0.75,0) to [out=10,in=90]++(0.75,-1) to [out=-90,in=0]++(-1,-0.6) to [out=180,in=-15]++(-1.25,-.75) to [out=165,in=-90]++(-0.75,1);
\end{tikzpicture}
\caption*{(الف) مستحکم نقطہ فاصل \عددی{P_0} کی صورت میں خط حرکت \عددی{D_{\epsilon}} میں رہتی ہے۔}
\end{subfigure}%
\begin{subfigure}{0.5\textwidth}
\centering
\begin{tikzpicture}
\draw(0,0) circle (1.25);
\draw(0,0) circle (2);
\draw[-stealth](0,0)--++(25:1.25)node[pos=0.5,fill=white]{$\sigma$};
\draw[-stealth](0,0)--++(125:2)node[pos=0.35,fill=white]{$\epsilon$};
\draw[fill=white](0,0) node [ocirc]{}node[shift={(0,-0.4)}]{$P_0$};
%trajectory
\draw[stealth-] (0,0)node[ocirc](ka){};\draw[stealth-] (ka) to [out=100,in=-120]++(0.25,0.85)node[ocirc]{};
\draw[stealth-](ka) to [out=-135,in=-45]++(-0.75,-0.5) to [out=135,in=-170]++(0.3,0.7)node[ocirc]{};
\draw[stealth-] (ka) to [out=-30,in=-145]++(1,-0.5) to [out=35,in=-90]++(0.5,1) to [out=90,in=0] (0,1.75) to [out=180,in=90] (-1.75,0) to [out=-90,in=160] (-135:1.3) to [out=-20,in=-135]++(0.5,0)node[ocirc]{};
\end{tikzpicture}
\caption*{(ب) مستحکم اور جاذب نقطہ فاصل \عددی{P_0}۔}
\end{subfigure}%
\caption{نظام \حوالہ{مساوات_نظام_جانچ_نقطہ_فاصل_الف} کے نقطہ فاصل۔}
\label{شکل_نظام_نقطہ_فاصل_تعریف}
\end{figure}%
%=====================

استحکام کی بنیاد پر نقطہ فاصل کی درجہ بندی جدول \حوالہ{جدول_نظام_نقطہ_فاصل_بالمقابل_استحکام} میں دی گئی ہے۔
\begin{table}
\caption{استحکام کی بنیاد پر نقطہ فاصل کی درجہ بندی۔}
\label{جدول_نظام_نقطہ_فاصل_بالمقابل_استحکام}
\centering
\begin{tabular}{rlr}
استحکام کی قسم& \عددی{p=\lambda_1+\lambda_2}& \عددی{q=\lambda_1\lambda_2}\\
\hline
(الف) مستحکم اور جاذب& \عددی{p<0} & \عددی{q>0}\\
(ب) مستحکم & \عددی{p\le 0} & \عددی{q>0}\\
(پ) غیر مستحکم & \multicolumn{2}{c}{ \عددی{p>0} \,\,\, یا \,\,\, \عددی{q<0}}
\end{tabular}
\end{table}
%===================

آئیں جدول \حوالہ{جدول_نظام_نقطہ_فاصل_اصول_جانچ} اور جدول \حوالہ{جدول_نظام_نقطہ_فاصل_بالمقابل_استحکام} کو حاصل کریں۔اگر \عددی{q=\lambda_1\lambda_2>0} ہو تب دونوں آئگنی قدر مثبت ہوں گے یا دونوں آئگنی قدر منفی ہوں گے اور یا آئگنی قدر جوڑی دار مخلوط ہوں گے۔ اب اگر \عددی{p=\lambda_1+\lambda_2<0} ہو تب دونوں آئگنی قیمتیں منفی ہوں گے یا (مخلوط جوڑی دار صورت میں) ان کا حقیقی جزو منفی ہو گا لہٰذا \عددی{P_0} مستحکم اور جاذب ہو گا۔ جدول \حوالہ{جدول_نظام_نقطہ_فاصل_بالمقابل_استحکام} کے بقایا دو نتائج کو آپ خود اسی طرح اخذ کر سکتے ہیں۔

 \عددی{\Delta <0} کی صورت میں آئگنی قدر جوڑی دار مخلوط \عددی{\lambda_1=\alpha+i\beta} اور \عددی{\lambda_2=\alpha-i\beta} ہوں گے۔ اب اگر \عددی{p=\lambda_1+\lambda_2=2\alpha<0} ہو تب مستحکم، جاذب نقطہ مرغولہ حاصل ہو گا۔اس کے برعکس \عددی{p=2\alpha>0} کی صورت میں غیر مستحکم نقطہ مرغولہ حاصل ہو گا۔

\عددی{p=0} کی صورت میں \عددی{\lambda_2=-\lambda_1} ہو گا اور یوں \عددی{q=\lambda_1\lambda_2=-\lambda_1^2} ہو گا۔اب اگر \عددی{q>0} ہو تب  \عددی{\lambda_1^2=-q<0} ہو گا لہٰذا \عددی{\lambda_1} اور \عددی{\lambda_2} خیالی ہوں گے جن سے  \اصطلاح{دوری حل}\فرہنگ{حل!دوری}\حاشیہب{periodic solutions}\فرہنگ{solutions!periodic} حاصل ہو گا۔دوری حل کا خط حرکت ایسا بند دائرہ ہے جس کا مرکز \عددی{P_0} ہے۔
%=================

\ابتدا{مثال}\شناخت{مثال_نظام_جدول_کا_استعمال}\quad جدول \حوالہ{جدول_نظام_نقطہ_فاصل_اصول_جانچ} اور جدول \حوالہ{جدول_نظام_نقطہ_فاصل_بالمقابل_استحکام} کا عملی استعمال\\
گزشتہ حصے کے مثال \حوالہ{مثال_نظام_غیر_متناسب_جوڑ} میں نظام \حوالہ{مساوات_نظام_خط_حرکت_الف} یعنی 
 \عددی{\bM{y}'=\begin{bmatrix} -2&1\\1&-2 \end{bmatrix} \bM{y}} کی بات کی گئی جہاں \عددی{p=-4}، \عددی{q=3} اور \عددی{\Delta=4} ہیں۔یوں جدول \حوالہ{جدول_نظام_نقطہ_فاصل_اصول_جانچ}-الف کے تحت نقطہ فاصل ایک جوڑ ہو گا۔جدول \حوالہ{جدول_نظام_نقطہ_فاصل_بالمقابل_استحکام}-الف کے تحت یہ جوڑ مستحکم اور جاذب ہے۔
\انتہا{مثال}
%=====================

\ابتدا{مثال}\quad اسپرنگ اور کمیت کی آزادانہ حرکت\\
اسپرنگ اور کمیت [حصہ \حوالہ{حصہ_سادہ_اسپرنگ_کمیت} دیکھیں] کے نظام \عددی{my''+cy'+ky=0} کا نقطہ فاصل دریافت کریں۔

حل:تفرقی مساوات کو معیاری صورت میں لکھنے کی خاطر \عددی{m} سے تقسیم کرتے  ہوئے \عددی{y''+\tfrac{c}{m}y'+\tfrac{k}{m}y=0} لکھتے ہیں۔دو درجی مساوات سے تفرقی مساوات کا نظام حاصل کرنے کی خاطر [حصہ \حوالہ{حصہ_نظام_قالب} دیکھیں] ہم \عددی{y_1=y} اور \عددی{y_2=y'} لیتے ہیں۔یوں 
\عددی{y_2'=y''=-\tfrac{k}{m}y_1-\tfrac{c}{m}y_2} ہو گا۔اس طرح
\begin{align*}
\bM{y}'=\begin{bmatrix*}[r] 0&1\\-\frac{k}{m}&-\frac{c}{m}  \end{bmatrix*} \bM{y},\quad  \abs{\bM{A}-\lambda\bM{I}}=\begin{bmatrix} -\lambda&1\\-\frac{k}{m}&-\frac{c}{m}-\lambda \end{bmatrix}=\lambda^2+\frac{c}{m}\lambda+\frac{k}{m}=0
\end{align*}
لکھا جائے گا جس سے \عددی{p=-\tfrac{c}{m}}، \عددی{q=\tfrac{k}{m}} اور \عددی{\Delta=\tfrac{c^2}{m^2}-4\tfrac{k}{m}} ملتے ہیں جنہیں استعمال کرتے ہوئے جدول \حوالہ{جدول_نظام_نقطہ_فاصل_اصول_جانچ} اور جدول \حوالہ{جدول_نظام_نقطہ_فاصل_بالمقابل_استحکام} سے درج ذیل نتائج حاصل ہوتے ہیں جہاں \عددی{\Delta} اہم کردار ادا کرتا ہے۔
\begin{itemize}
\شے[بلا تقصیر] \عددی{c=0}، \عددی{p=0} اور \عددی{q>0}  وسط دیتا ہے۔
\شے[کم مقصور]  \عددی{c^2<4mk}، \عددی{p<0}، \عددی{q>0} اور \عددی{\Delta<0} مستحکم جاذب نقطہ مرغولہ دیتا ہے۔
\شے[فاصل تقصیر] \عددی{c^2=4mk}، \عددی{p<0}، \عددی{q>0} اور \عددی{\Delta=0}  مستحکم جاذب جوڑ دیتا ہے۔
\شے[زیادہ مقصور] \عددی{c^2>4mk}، \عددی{p<0}، \عددی{q>0} اور \عددی{\Delta>0} مستحکم جاذب جوڑ دیتا ہے۔
\end{itemize}
\انتہا{مثال}
%====================

\حصہء{سوالات}
سوال \حوالہ{سوال_نظام_نقطہ_فاصل_اور_حل_الف} تا سوال \حوالہ{سوال_نظام_نقطہ_فاصل_اور_حل_ب} کے نقطہ فاصل کی قسم جدول \حوالہ{جدول_نظام_نقطہ_فاصل_اصول_جانچ} اور جدول \حوالہ{جدول_نظام_نقطہ_فاصل_بالمقابل_استحکام} کی مدد سے  دریافت کریں۔ان کے حقیقی عمومی حل حاصل کریں اور ان کے خط حرکت کمپیوٹر کی مدد سے کھینچیں۔[پہلے چار جوابات کے خط حرکت دکھائے گئے ہیں۔]

%===========
\ابتدا{سوال}\شناخت{سوال_نظام_نقطہ_فاصل_اور_حل_الف}
\begin{align*}
y_1'&=y_1\\
y_2'&=3y_2
\end{align*}
جوابات:غیر مستحکم، غیر مناسب جوڑ۔\عددی{\bM{y}=c_1\begin{bmatrix} 1\\0  \end{bmatrix}e^t+c_2\begin{bmatrix} 0\\1 \end{bmatrix}e^{3t}} یعنی 
 \عددی{y_1=c_1e^t}، \عددی{y_2=c_2e^{3t}}؛ شکل \حوالہ{شکل_سوال_نظام_نقطہ_فاصل_اور_حل_الف}-الف۔
\begin{figure}
\centering
\begin{subfigure}{0.5\textwidth}
\centering
\begin{tikzpicture}
\begin{axis}[small,axis lines*=middle,xtick=\empty,ytick=\empty,xlabel={$y_1$},ylabel={$y_2$},xlabel style={at={(axis description cs:1.05,0.5)}},ylabel style ={rotate=-90},ylabel style={at={(axis description cs:0.5,1.05)}}]
\pgfmathsetmacro{\ca}{1}
\pgfmathsetmacro{\cb}{0}
\addplot[domain=0:3]({\ca*e^x},{\cb*e^(3*x)})node[pos=0.5](ka){}node[pos=0.501](kb){};
\draw[-stealth](ka)--(kb);
\pgfmathsetmacro{\ca}{-1}
\pgfmathsetmacro{\cb}{0}
\addplot[domain=0:3]({\ca*e^x},{\cb*e^(3*x)})node[pos=0.5](ka){}node[pos=0.501](kb){};
\draw[-stealth](ka)--(kb);
\pgfmathsetmacro{\ca}{0}
\pgfmathsetmacro{\cb}{1}
\addplot[domain=0:3]({\ca*e^x},{\cb*e^(3*x)})node[pos=0.5](ka){}node[pos=0.501](kb){};
\draw[-stealth](ka)--(kb);
\pgfmathsetmacro{\ca}{0}
\pgfmathsetmacro{\cb}{-1}
\addplot[domain=0:3]({\ca*e^x},{\cb*e^(3*x)})node[pos=0.5](ka){}node[pos=0.501](kb){};
\draw[-stealth](ka)--(kb);
%
\pgfmathsetmacro{\ca}{1}
\pgfmathsetmacro{\cb}{1}
\addplot[domain=0:3]({\ca*e^x},{\cb*e^(3*x)})node[pos=0.5](ka){}node[pos=0.501](kb){};
\draw[-stealth](ka)--(kb);
\pgfmathsetmacro{\ca}{1}
\pgfmathsetmacro{\cb}{-1}
\addplot[domain=0:3]({\ca*e^x},{\cb*e^(3*x)})node[pos=0.5](ka){}node[pos=0.501](kb){};
\draw[-stealth](ka)--(kb);
\pgfmathsetmacro{\ca}{-1}
\pgfmathsetmacro{\cb}{1}
\addplot[domain=0:3]({\ca*e^x},{\cb*e^(3*x)})node[pos=0.5](ka){}node[pos=0.501](kb){};
\draw[-stealth](ka)--(kb);
\pgfmathsetmacro{\ca}{-1}
\pgfmathsetmacro{\cb}{-1}
\addplot[domain=0:3]({\ca*e^x},{\cb*e^(3*x)})node[pos=0.5](ka){}node[pos=0.501](kb){};
\draw[-stealth](ka)--(kb);
\addplot[fill=white]plot coordinates {(0,0)}node[ocirc]{}; 
\end{axis}
\end{tikzpicture}
\caption*{(الف) سوال \حوالہ{سوال_نظام_نقطہ_فاصل_اور_حل_الف} غیر مستحکم، غیر مناسب جوڑ۔}
\end{subfigure}%
\begin{subfigure}{0.5\textwidth}
\centering
\begin{tikzpicture}
\begin{axis}[small,axis lines*=middle,xtick=\empty,ytick=\empty,xlabel={$y_1$},ylabel={$y_2$},xlabel style={at={(axis description cs:1.05,0.5)}},ylabel style ={rotate=-90},ylabel style={at={(axis description cs:0.5,1.05)}}]
\pgfmathsetmacro{\ca}{1}
\pgfmathsetmacro{\cb}{0}
\addplot[domain=0:3]({\ca*e^(-3*x)},{\cb*e^(-5*x)})node[pos=0.5](ka){}node[pos=0.501](kb){};
\draw[-stealth](ka)--(kb);
\pgfmathsetmacro{\ca}{-1}
\pgfmathsetmacro{\cb}{0}
\addplot[domain=0:3]({\ca*e^(-3*x)},{\cb*e^(-5*x)})node[pos=0.5](ka){}node[pos=0.501](kb){};
\draw[-stealth](ka)--(kb);
\pgfmathsetmacro{\ca}{0}
\pgfmathsetmacro{\cb}{1}
\addplot[domain=0:3]({\ca*e^(-3*x)},{\cb*e^(-5*x)})node[pos=0.5](ka){}node[pos=0.501](kb){};
\draw[-stealth](ka)--(kb);
\pgfmathsetmacro{\ca}{0}
\pgfmathsetmacro{\cb}{-1}
\addplot[domain=0:3]({\ca*e^(-3*x)},{\cb*e^(-5*x)})node[pos=0.5](ka){}node[pos=0.501](kb){};
\draw[-stealth](ka)--(kb);
%
\pgfmathsetmacro{\ca}{1}
\pgfmathsetmacro{\cb}{1}
\addplot[domain=0:3]({\ca*e^(-3*x)},{\cb*e^(-5*x)})node[pos=0.5](ka){}node[pos=0.501](kb){};
\draw[-stealth](ka)--(kb);
\pgfmathsetmacro{\ca}{1}
\pgfmathsetmacro{\cb}{-1}
\addplot[domain=0:3]({\ca*e^(-3*x)},{\cb*e^(-5*x)})node[pos=0.5](ka){}node[pos=0.501](kb){};
\draw[-stealth](ka)--(kb);
\pgfmathsetmacro{\ca}{-1}
\pgfmathsetmacro{\cb}{1}
\addplot[domain=0:3]({\ca*e^(-3*x)},{\cb*e^(-5*x)})node[pos=0.5](ka){}node[pos=0.501](kb){};
\draw[-stealth](ka)--(kb);
\pgfmathsetmacro{\ca}{-1}
\pgfmathsetmacro{\cb}{-1}
\addplot[domain=0:3]({\ca*e^(-3*x)},{\cb*e^(-5*x)})node[pos=0.5](ka){}node[pos=0.501](kb){};
\draw[-stealth](ka)--(kb);
\addplot[fill=white]plot coordinates {(0,0)}node[ocirc]{}; 
\end{axis}
\end{tikzpicture}
\caption*{(ب) سوال \حوالہ{سوال_نظام_نقطہ_فاصل_اور_حل_الف_ب}  مستحکم، جاذب، غیر مناسب جوڑ۔}
\end{subfigure}%
\caption{سوال \حوالہ{سوال_نظام_نقطہ_فاصل_اور_حل_الف} اور سوال \حوالہ{سوال_نظام_نقطہ_فاصل_اور_حل_الف_ب} کے اشکال۔}
\label{شکل_سوال_نظام_نقطہ_فاصل_اور_حل_الف}
\end{figure}
\انتہا{سوال}
%===========
\ابتدا{سوال}\شناخت{سوال_نظام_نقطہ_فاصل_اور_حل_الف_ب}
\begin{align*}
y_1'&=-3y_1\\
y_2'&=-5y_2
\end{align*}
جوابات:مستحکم، جاذب، غیر مناسب جوڑ۔  \عددیء{y_1=c_1e^{-3t}}، \عددی{y_2=c_2e^{-5t}}؛ شکل \حوالہ{شکل_سوال_نظام_نقطہ_فاصل_اور_حل_الف}-ب۔
\انتہا{سوال}
%=======================
\ابتدا{سوال}\شناخت{سوال_نظام_نقطہ_فاصل_اور_حل_الف_پ}
\begin{align*}
y_1'&=y_2\\
y_2'&=-16y_1
\end{align*}
جوابات:مستحکم وسط۔ \عددی{y_1=A\cos 4t+B\sin 4t}، \عددی{y_2=4B\cos 4t-4A\sin 4t}؛ شکل \حوالہ{شکل_سوال_نظام_نقطہ_فاصل_اور_حل_الف_پ}-الف۔
\begin{figure}
\centering
\begin{subfigure}{0.5\textwidth}
\centering
\begin{tikzpicture}
\begin{axis}[small,axis lines*=middle,xtick=\empty,ytick=\empty,xlabel={$y_1$},ylabel={$y_2$},xlabel style={at={(axis description cs:1.05,0.5)}},ylabel style ={rotate=-90},ylabel style={at={(axis description cs:0.5,1.05)}}]
\pgfmathsetmacro{\ca}{1}
\pgfmathsetmacro{\cb}{0}
\addplot[domain=-45:45,samples=100]({\ca*cos(4*x)+\cb*sin(4*x)},{4*\cb*cos(4*x)-4*\ca*sin(4*x)})node[pos=0.2](ka){}node[pos=0.201](kb){};
\draw[-stealth](ka)--(kb);
\pgfmathsetmacro{\ca}{1}
\pgfmathsetmacro{\cb}{1}
\addplot[domain=-45:45,samples=100]({\ca*cos(4*x)+\cb*sin(4*x)},{4*\cb*cos(4*x)-4*\ca*sin(4*x)})node[pos=0.1](ka){}node[pos=0.101](kb){};
\draw[-stealth](ka)--(kb);
%
\addplot[fill=white]plot coordinates {(0,0)}node[ocirc]{}; 
\end{axis}
\end{tikzpicture}
\caption*{(الف) سوال \حوالہ{سوال_نظام_نقطہ_فاصل_اور_حل_الف_پ}  مستحکم وسط۔}
\end{subfigure}%
\begin{subfigure}{0.5\textwidth}
\centering
\begin{tikzpicture}
\begin{axis}[small,axis lines*=middle,xtick=\empty,ytick=\empty,xlabel={$y_1$},ylabel={$y_2$},xlabel style={at={(axis description cs:1.05,0.5)}},ylabel style ={rotate=-90},ylabel style={at={(axis description cs:0.5,1.05)}}]
\pgfmathsetmacro{\lmt}{0.75}
%
\pgfmathsetmacro{\ca}{1}
\pgfmathsetmacro{\cb}{1}
\addplot[domain=0:\lmt]({\ca*e^(-3*x)+\cb*e^(3*x)},{-5*\ca*e^(-3*x)+\cb*e^(3*x)})node[pos=0.5](ka){}node[pos=0.501](kb){};
\draw[-stealth](ka)--(kb);
\pgfmathsetmacro{\ca}{-1}
\pgfmathsetmacro{\cb}{1}
\addplot[domain=0:\lmt]({\ca*e^(-3*x)+\cb*e^(3*x)},{-5*\ca*e^(-3*x)+\cb*e^(3*x)})node[pos=0.5](ka){}node[pos=0.501](kb){};
\draw[-stealth](ka)--(kb);
\pgfmathsetmacro{\ca}{1}
\pgfmathsetmacro{\cb}{-1}
\addplot[domain=0:\lmt]({\ca*e^(-3*x)+\cb*e^(3*x)},{-5*\ca*e^(-3*x)+\cb*e^(3*x)})node[pos=0.5](ka){}node[pos=0.501](kb){};
\draw[-stealth](ka)--(kb);
\pgfmathsetmacro{\ca}{-1}
\pgfmathsetmacro{\cb}{-1}
\addplot[domain=0:\lmt]({\ca*e^(-3*x)+\cb*e^(3*x)},{-5*\ca*e^(-3*x)+\cb*e^(3*x)})node[pos=0.5](ka){}node[pos=0.501](kb){};
\draw[-stealth](ka)--(kb);
\pgfmathsetmacro{\ca}{0}
\pgfmathsetmacro{\cb}{1}
\addplot[domain=0:\lmt]({\ca*e^(-3*x)+\cb*e^(3*x)},{-5*\ca*e^(-3*x)+\cb*e^(3*x)})node[pos=0.5](ka){}node[pos=0.501](kb){};
\draw[-stealth](ka)--(kb);
\pgfmathsetmacro{\ca}{0}
\pgfmathsetmacro{\cb}{-1}
\addplot[domain=0:\lmt]({\ca*e^(-3*x)+\cb*e^(3*x)},{-5*\ca*e^(-3*x)+\cb*e^(3*x)})node[pos=0.5](ka){}node[pos=0.501](kb){};
\draw[-stealth](ka)--(kb);
\pgfmathsetmacro{\ca}{1}
\pgfmathsetmacro{\cb}{0}
\addplot[domain=0:\lmt]({\ca*e^(-3*x)+\cb*e^(3*x)},{-5*\ca*e^(-3*x)+\cb*e^(3*x)})node[pos=0.5](ka){}node[pos=0.51](kb){};
\draw[-stealth](ka)--(kb);
\pgfmathsetmacro{\ca}{-1}
\pgfmathsetmacro{\cb}{0}
\addplot[domain=0:\lmt]({\ca*e^(-3*x)+\cb*e^(3*x)},{-5*\ca*e^(-3*x)+\cb*e^(3*x)})node[pos=0.5](ka){}node[pos=0.51](kb){};
\draw[-stealth](ka)--(kb);
%
\addplot[fill=white]plot coordinates {(0,0)}node[ocirc]{}; 
\end{axis}
\end{tikzpicture}
\caption*{(ب) سوال \حوالہ{سوال_نظام_نقطہ_فاصل_اور_حل_الف_ت} غیر مستحکم، نقطہ نیز۔}
\end{subfigure}%
\caption{سوال \حوالہ{سوال_نظام_نقطہ_فاصل_اور_حل_الف_پ} اور سوال \حوالہ{سوال_نظام_نقطہ_فاصل_اور_حل_الف_ت} کے اشکال۔}
\label{شکل_سوال_نظام_نقطہ_فاصل_اور_حل_الف_پ}
\end{figure}
\انتہا{سوال}
%========================
\ابتدا{سوال}\شناخت{سوال_نظام_نقطہ_فاصل_اور_حل_الف_ت}
\begin{align*}
y_1&=2y_1+y_2\\
y_2&=5y_1-2y_2
\end{align*}
جوابات: غیر مستحکم نقطہ نیز؛ \عددی{y_1=c_1e^{-3t}+c_2e^{3t}}، \عددی{y_2=-5c_1e^{-3t}+c_2e^{3t}}؛شکل \حوالہ{شکل_سوال_نظام_نقطہ_فاصل_اور_حل_الف_پ}-ب۔
\انتہا{سوال}
%================
\ابتدا{سوال}\شناخت{سوال_نظام_نقطہ_فاصل_اور_حل_الف_ٹ}
\begin{align*}
y_1&=-2y_1-2y_2\\
y_2&=2y_1-2y_2
\end{align*}
جوابات:مستحکم اور جاذب نقطہ مرغولہ؛ \عددی{y_1=e^{-2t}(A\cos 2t+B\sin 2t)}، \عددی{y_2=e^{-2t}(-B\cos 2t+A\sin 2t)}
\انتہا{سوال}
%================
\ابتدا{سوال}
\begin{align*}
y_1&=-10y_1+2y_2\\
y_2&=-15y_1+y_2
\end{align*}
جوابات:مستحکم اور جاذب جوڑ؛ \عددی{y_1=c_1e^{-5t}+c_2e^{-4t}}، \عددی{y_2=\tfrac{5}{2}c_1e^{-5t}+3c_2e^{-4t}}
\انتہا{سوال}
%================
\ابتدا{سوال}
\begin{align*}
y_1&=-y_1+y_2\\
y_2&=2y_2
\end{align*}
جوابات:غیر مستحکم نقطہ نیز؛ \عددی{y_1=c_1e^{-t}+c_2e^{2t}}، \عددی{y_2=3c_2e^{2t}}
\انتہا{سوال}
%================
\ابتدا{سوال}
\begin{align*}
y_1&=-y_1+2y_2\\
y_2&=6y_1+3y_2
\end{align*}
جوابات:غیر مستحکم نقطہ نیز؛ \عددی{y_1=c_1e^{-3t}+c_2e^{5t}}، \عددی{y_2=-c_1e^{-3t}+3c_2e^{5t}}
\انتہا{سوال}
%================
\ابتدا{سوال}
\begin{align*}
y_1&=13y_1-3y_2\\
y_2&=18y_1-2y_2
\end{align*}
جوابات:غیر مستحکم جوڑ؛ \عددی{y_1=c_1e^{7t}+c_2e^{4t}}، \عددی{y_2=2c_1e^{7t}+3c_2e^{4t}}
\انتہا{سوال}
%================
\ابتدا{سوال}\شناخت{سوال_نظام_نقطہ_فاصل_اور_حل_ب}
\begin{align*}
y_1&=y_2\\
y_2&=-5y_1-2y_2
\end{align*}
جوابات:مستحکم اور جاذب نقطہ مرغولہ؛ \عددی{y_1=e^{-t}(A\cos 2t+B\sin 2t)}، \\ \عددی{y_2=e^{-t}[-(A+2B)\cos 2t-(2A+B)\sin 2t]}
\انتہا{سوال}
%================

سوال \حوالہ{سوال_نظام_نقطہ_فاصل_دو_درجی_الف} تا سوال \حوالہ{سوال_نظام_نقطہ_فاصل_دو_درجی_الف} خط حرکت، دو درجی سادہ تفرقی مساوات اور نقطہ فاصل کے بارے میں ہیں۔

%==============
\ابتدا{سوال}\شناخت{سوال_نظام_نقطہ_فاصل_دو_درجی_الف} \quad قصری ارتعاش\\
\عددی{y''+4y'+5y=0} کو حل کریں۔امتیازی مساوات سے خط حرکت کی قسم دریافت کریں؟

جواب:\عددی{y=e^{-2t}(A\cos t+B\sin t)}؛ مستحکم اور جاذب نقطہ مرغولہ۔
\انتہا{سوال}
%====================
\ابتدا{سوال} \quad ہارمونی ارتعاش\\
\عددی{y''+4y=0=0} کو حل کریں۔امتیازی مساوات سے خط حرکت کی قسم دریافت کریں؟

جواب:\عددی{y=A\cos 2t+B\sin 2t}؛ مستحکم وسط۔
\انتہا{سوال}
%====================
\ابتدا{سوال} \quad مقدار معلوم کا تبادلہ\\
مثال \حوالہ{مثال_نظام_جدول_کا_استعمال} میں متغیرہ \عددی{\tau=-t} متعارف کرنے سے نقطہ فاصل پر کیا اثر پڑے گا؟

جواب: اب \عددی{\bM{A}=\begin{bmatrix} 2&-1\\-1&2  \end{bmatrix}} ہو گا لہٰذا غیر مستحکم جوڑ پایا جائے گا۔
\انتہا{سوال}
%====================
\ابتدا{سوال} \quad وسط میں خلل\\
سوال \حوالہ{سوال_نظام_نقطہ_فاصل_اور_حل_الف_پ} میں \عددی{\bM{A}} کو تبدیل کرتے ہوئے \عددی{\bM{A}-0.12\bM{I}} کرنے سے نقطہ فاصل پر کیا اثر پیدا ہو گا؟ \عددی{\bM{I}}  اکائی قالب ہے۔ 

جواب: اب \عددی{p=-0.2=\ne 0}، \عددی{q >0} اور \عددی{\Delta <0} ہیں لہٰذا غیر مستحکم نقطہ مرغولہ پایا جائے گا۔
\انتہا{سوال}
%====================
\ابتدا{سوال} \quad وسط میں خلل\\
سوال \حوالہ{سوال_نظام_نقطہ_فاصل_اور_حل_الف_پ} میں تمام \عددی{a_{jk}} کی جگہ \عددی{a_{jk}+b} پر کریں۔ (الف) \عددی{b} کی ایسی قیمت دریافت کریں کہ نقطہ زین حاصل ہو۔اسی طرح \عددی{b} کی ایسی قیمتیں دریافت کریں جن پر (ب) مستحکم اور جاذب جوڑ، (پ) مستحکم اور جاذب نقطہ مرغولہ اور (ت) غیر مستحکم نقطہ مرغولہ پایا جائے۔

جواب:مثلاً (الف) \عددی{b=-2}، (ب) \عددی{b=-1}، (پ) \عددی{b=-0.2}، (ت) \عددی{b=15}
\انتہا{سوال}
%====================

\حصہ{کیفی تراکیب برائے غیر خطی نظام}
\اصطلاح{کیفی تراکیب}\فرہنگ{کیفی!تراکیب}\فرہنگ{ترکیب!کیفی}\حاشیہب{qualitative methods}\فرہنگ{qualitative methods} سے مسئلے کو حل کئے بغیر حل کے بارے میں کیفی معلومات حاصل کی جاتی ہیں۔ایسے مسائل جن کا \اصطلاح{تحلیلی حل} مشکل یا نا قابل حصول ہو، کے لئے یہ ترکیب خاص طور پر کار آمد ہے۔عملاً اہم کئی غیر خطی نظام
\begin{gather}\label{مساوات_نظام_غیر_خطی_ترکیب_مرحلہ_الف}
\begin{aligned}
\bM{y}'=\bM{f}(\bM{y})
\end{aligned}\quad \implies \quad 
\begin{aligned}
y_1&=f_1(y_1,y_2)\\
y_2&=f_2(y_1,y_2)
\end{aligned}
\end{gather}
کے لئے یہ درست ہے۔

گزشتہ حصے میں \اصطلاح{سطح مرحلہ کی ترکیب}\فرہنگ{سطح مرحلہ!ترکیب} خطی نظام کے لئے استعمال کیا گیا۔اس حصے میں اس ترکیب کو وسعت دے کر غیر خطی نظام کے لئے استعمال کیا جائے گا۔ ہم فرض کرتے ہیں کہ مساوات \حوالہ{مساوات_نظام_غیر_خطی_ترکیب_مرحلہ_الف} \اصطلاح{خود مختار}\فرہنگ{خود مختار}\حاشیہب{autonomous}\فرہنگ{autonomous} ہے یعنی اس میں غیر تابع متغیرہ \عددی{t} \ترچھا{صریحاً} نہیں پایا جاتا۔(اس حصے میں تمام مثال خود مختار ہیں۔) ہم یہاں بھی حل کی نسل پیش کریں گے۔اعدادی ترکیب سے ایک وقت میں صرف ایک (تقریباً درست) حل حاصل ہوتا ہے۔ اس لحاض سے سطح مرحلہ کی ترکیب زیادہ مفید ثابت ہوتی ہے۔

گزشتہ حصے کے چند تصورات اس حصے میں بھی درکار ہیں۔ان میں \اصطلاح{سطح حرکت} (\عددیء{y_1 y_2} سطح)، \اصطلاح{خط حرکت} (مساوات \حوالہ{مساوات_نظام_غیر_خطی_ترکیب_مرحلہ_الف} کا \عددی{y_1 y_2} سطح پر حل)، مساوات \حوالہ{مساوات_نظام_غیر_خطی_ترکیب_مرحلہ_الف} کا \اصطلاح{پیکر مرحلہ} (تمام خط حرکت کا مجموعہ)،  اور مساوات \حوالہ{مساوات_نظام_غیر_خطی_ترکیب_مرحلہ_الف} کا \اصطلاح{نقطہ فاصل} (ایسا نقطہ \عددی{(y_1,y_2)} جہاں \عددیء{f_1(y_1,y_2)} اور \عددیء{f_2(y_1,y_2)} دونوں صفر کے برابر ہوں۔) کے تصورات شامل ہیں۔

مساوات \حوالہ{مساوات_نظام_غیر_خطی_ترکیب_مرحلہ_الف} کے کئی نقطہ فاصل ہو سکتے ہیں۔ ان پر باری باری بات کی جائے گی۔مرکز سے ہٹ کر پائے جانے والے نقطہ فاصل پر غور کرنے سے پہلے، تکنیکی آسانی کی خاطر، ایسے نقطہ فاصل کو گھمائے بغیر  مرکز پر منتقل کیا جائے گا۔یوں مرکز \عددی{(0,0)} سے ہٹ کر پائے جانے والے نقطہ فاصل \عددی{P_0:(a,b)} کو گھمائے بغیر  مرکز \عددی{(0,0)} پر درج ذیل عمل سے منتقل کیا جائے گا۔
\begin{align*}
\tilde{y}_1=y_1-a,\quad \tilde{y}_2=y_2-b
\end{align*}
اس عمل کے بعد نقطہ فاصل \عددی{P_0} مرکز \عددی{(0,0)} پر پایا جائے گا۔یوں ہم فرض کرتے ہیں کہ یہاں دیے گئے تمام مثالوں میں نقطہ  فاصل مرکز پر پایا جاتا ہے لہٰذا \عددی{\tilde{y}_1}، \عددی{\tilde{y}_2} کی جگہ ہم \عددی{y_1} اور \عددی{y_2} ہی لکھیں گے۔ہم یہ بھی فرض کرتے ہیں کہ نقطہ فاصل  \اصطلاح{تنہا}\فرہنگ{تنہا}\حاشیہب{isolated}\فرہنگ{isolated} ہے یعنی ایسے کسی بھی معقول حد تک چھوٹی ٹکیا جس کا وسط مرکز پر پایا جاتا ہو میں مساوات  \حوالہ{مساوات_نظام_غیر_خطی_ترکیب_مرحلہ_الف} کا صرف  ایک عدد نقطہ فاصل پایا جاتا ہے۔ اگر مساوات \حوالہ{مساوات_نظام_غیر_خطی_ترکیب_مرحلہ_الف} کے محدود تعداد میں نقطہ فاصل پائے جاتے ہوں تب ایسے تمام نقطہ فاصل خود بخود تنہا ہوں گے۔
%=============

\جزوحصہء{غیر خطی نظام کو خطی بنانا}

