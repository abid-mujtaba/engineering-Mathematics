
\حصہ{نقطہ فاصل کے جانچ کا اصول۔استحکام}
ہم مستقل عددی سر والے متجانس خطی نظام \حوالہ{مساوات_نظام_جانچ_نقطہ_فاصل_الف} پر گفتگو جاری رکھتے ہیں۔
\begin{gather}\label{مساوات_نظام_جانچ_نقطہ_فاصل_الف}
\begin{aligned}
\bM{y}'=\begin{bmatrix} a_{11} & a_{12} \\ a_{21} & a_{22} \end{bmatrix} \bM{y}
\end{aligned}, \quad \implies \quad 
\begin{aligned}
y_1'&=a_{11}y_1+a_{12}y_2\\
y_2'&=a_{21}y_1'+a_{22}y_2
\end{aligned}
\end{gather}
اب تک  حصہ \حوالہ{حصہ_نظام_مستقل_عددی_سر_نظام} میں ہم نے دیکھا کہ نسل حل \عددی{\bM{y}=[y_1(t)\quad y_2(t)]^T} کے خطوط کو \عددی{y_1 y_2} \اصطلاح{سطح حرکت} پر کھینچتے ہوئے عمومی جائزہ لیا جا سکتا ہے۔ اس سطح پر منحنی کو نظام \حوالہ{مساوات_نظام_جانچ_نقطہ_فاصل_الف} کا \اصطلاح{خط حرکت} کہتے ہیں۔تمام خط حرکت کو ملا کر \اصطلاح{پیکر مرحلہ} حاصل ہوتا ہے۔

ہم دیکھ چکے کہ \عددی{\bM{y}=\bM{x}e^{\lambda t}} کو حل تصور کرتے ہوئے مساوات \حوالہ{مساوات_نظام_جانچ_نقطہ_فاصل_الف} میں پر کرتے ہوئے
\begin{align*}
\bM{y}'=\lambda \bM{x}e^{\lambda t}=\bM{A}\bM{y}=\bM{A}\bM{x}e^{\lambda t}
\end{align*}
لکھا جا سکتا ہے جس کو \عددی{e^{\lambda t}} سے تقسیم کرتے ہوئے
\begin{align}\label{مساوات_نظام_جانچ_نقطہ_فاصل_ب}
\bM{A}\bM{x}=\lambda \bM{x}
\end{align}
ملتا ہے۔یوں \عددی{\lambda} قالب \عددی{\bM{A}}  کا آئگنی قدر اور \عددی{\bM{x}} نظیری آئگنی سمتیہ ہونے کی صورت میں  \عددی{\bM{y}(t)} مساوات \حوالہ{مساوات_نظام_جانچ_نقطہ_فاصل_الف} کا (غیر صفر) حل ہو گا۔

گزشتہ حصے کے مثالوں سے واضح ہے کہ پیکر مرحلہ کی صورت کا دارومدار بڑی حد تک نظام \حوالہ{مساوات_نظام_جانچ_نقطہ_فاصل_الف} کی \اصطلاح{نقطہ فاصل} کی قسم پر منحصر ہے جہاں نقطہ فاصل سے مراد ایسا نقطہ ہے جہاں \عددی{\tfrac{\dif y_1}{\dif y_2}} نا قابل معلوم قیمت \عددی{\tfrac{0}{0}} ہو۔[مساوات \حوالہ{مساوات_نظام_نقطہ_فاصل_الف} دیکھیں۔]
\begin{align}\label{مساوات_نظام_جانچ_نقطہ_فاصل_پ}
\frac{\dif y_2}{\dif y_1}=\frac{y_2' \dif t}{y_1'\dif t}=\frac{y_2'}{y_1'}=\frac{a_{21}y_1+a_{22}y_2}{a_{11}y_1+a_{12}y_2}
\end{align}
حصہ \حوالہ{حصہ_نظام_مستقل_عددی_سر_نظام}  سے ہم یہ بھی جانتے ہیں نقطہ فاصل کے کئی اقسام پائے جاتے ہیں۔

موجودہ حصے میں ہم دیکھیں گے کہ  نقطہ فاصل کی قسم  کا تعلق آئگنی قدر سے ہے جو امتیازی مساوات
\begin{align}\label{مساوات_نظام_جانچ_نقطہ_فاصل_ت}
\abs{\bM{A}-\lambda \bM{I}}=\begin{vmatrix} a_{11}-\lambda & a_{12} \\a_{21}&a_{22}-\lambda\end{vmatrix}=\lambda^2-(a_{11}+a_{22})\lambda+a_{11}a_{22}-a_{12}a_{21}=0
\end{align}
کے حل \عددی{\lambda_1} اور \عددی{\lambda_2} ہیں۔امتیازی مساوات دو درجی مساوات \عددی{\lambda^2-p\lambda+q=0} ہے جس کے عددی سر \عددی{p}،  \عددی{q} اور \اصطلاح{جدا کنندہ}\فرہنگ{جدا کنندہ}\حاشیہب{discriminant}\فرہنگ{discriminant} \عددی{\Delta} درج ذیل ہیں۔
\begin{align}\label{مساوات_نظام_جانچ_نقطہ_فاصل_ٹ}
p=a_{11}+a_{22}, \quad q=a_{11}a_{22}-a_{12}a_{21},\quad \Delta=p^2-4q
\end{align}
دو درجی مساوات کے حل الجبرا کی مدد سے \عددی{\lambda=\tfrac{1}{2}(p+\mp\sqrt{p^2-4q})} یعنی
\begin{align}\label{مساوات_نظام_جانچ_نقطہ_فاصل_ث}
\lambda_1=\frac{1}{2}(p+\sqrt{\Delta}),\quad \lambda_2=\frac{1}{2}(p-\sqrt{\Delta})
\end{align} 
لکھتے ہیں۔ان آئگنی قیمتوں کو استعمال کرتے ہوئے امتیازی مساوات کو اجزائے ضربی کی صورت 
\begin{align*}
(\lambda-\lambda_1)(\lambda-\lambda_2)=\lambda^2-(\lambda_1+\lambda_2)+\lambda_1\lambda_2=0
\end{align*}
میں لکھا جا سکتا ہے  جہاں سے ظاہر ہے کہ \عددی{p} آئگنی قیمتوں کا مجموعہ ہے جبکہ \عددی{q} ان کا حاصل ضرب ہے۔اسی طرح مساوات \حوالہ{مساوات_نظام_جانچ_نقطہ_فاصل_ث} کی مدد سے \عددی{\lambda_1-\lambda_2=\sqrt{\Delta}} لکھا جا سکتا ہے۔
\begin{align}
p=\lambda_1+\lambda_2,\quad q=\lambda_1\lambda_2,\quad \Delta=(\lambda_1-\lambda_2)^2
\end{align}
ان نتائج سے  نقطہ فاصل کی جانچ کے اصول طے کئے جا سکتے ہیں جنہیں جدول \حوالہ{جدول_نظام_نقطہ_فاصل_اصول_جانچ} میں پیش کیا گیا ہے۔ان اصولوں کو اسی حصے میں اخذ کیا جائے گا۔
\begin{table}
\caption{آئگنی قدر سے نقطہ فاصل کی درجہ بندی۔}
\label{جدول_نظام_نقطہ_فاصل_اصول_جانچ}
\centering
\begin{tabular}{rcccr}
نام& $p=\lambda_1+\lambda_2$&$q=\lambda_1\lambda_2$&$\Delta=(\lambda_1-\lambda_2)^2$&$\lambda_1$ اور $\lambda_2$  پر تبصرہ\\
\hline
(الف) جوڑ & & \عددی{q >0} & \عددی{\Delta \ge 0} & حقیقی۔یکساں علامتیں\\
(ب) نقطہ نیز & & \عددی{q<0}& & حقیقی۔آپس میں الٹ علامتیں\\
(پ) وسط & \عددی{p=0} & \عددی{q>0}&& خیالی عدد (حقیقی جزو صفر ہے)\\
(ت) نقطہ مرغولہ & \عددی{p \ne 0} & & \عددی{\Delta <0}& مخلوط عدد (حقیقی اور خیالی اجزاء غیر صفر ہیں)
\end{tabular}
\end{table}
%===================

\جزوحصہء{استحکام}
نقطہ فاصل کی درجہ بندی ان کی \اصطلاح{استحکام}\فرہنگ{استحکام}\حاشیہب{stability}\فرہنگ{stability} کی بنیاد پر بھی کی جا سکتی ہے۔انجینئری کے علاوہ دیگر شعبوں میں بھی  استحکام نہایت اہم تصور ہے۔مستحکم نظام میں کسی لمحے پر معمولی تبدیلی یا خلل سے بعد کے تمام لمحات پر معمولی خلل ہی پایا جاتا ہے۔ نقطہ فاصل کے لئے درج ذیل تصورات اہم ہیں۔
%============

\ابتدا{تعریف}\quad مستحکم، غیر مستحکم، مستحکم اور جاذب\\
اگر نظام \حوالہ{مساوات_نظام_جانچ_نقطہ_فاصل_الف} کے نقطہ فاصل \عددی{P_0} کے قریب تمام خط حرکت مستقبل میں بھی \عددی{P_0} کے قریب رہیں تب \عددی{P_0} \اصطلاح{مستحکم}\فرہنگ{مستحکم}\حاشیہب{stable}\فرہنگ{stable}  کہلائے گا۔ یوں اگر کسی بھی رداس \عددی{\epsilon} کی ٹکیا \عددی{D_\epsilon} کے لئے  رداس \عددی{\sigma} کی ایسی ٹکیا \عددی{D_\sigma} موجود ہو،  جہاں دونوں ٹکیوں کا مرکز \عددی{P_0} ہے، کہ  ٹکیا \عددی{D_\sigma} میں (لمحہ \عددی{t=t_1} کا نظیری) نقطہ \عددی{P_1} پر  پائے جانے والا، نظام \حوالہ{مساوات_نظام_جانچ_نقطہ_فاصل_الف} کا ہر خط حرکت، مستقبل میں ٹکیا \عددی{D_{\epsilon}} میں رہتا ہو، تب \عددی{P_0} کا نقطہ فاصل \اصطلاح{مستحکم}\فرہنگ{مستحکم}\حاشیہب{stable}\فرہنگ{stable}\حاشیہد{روسی ریاضی دان سکندر میکائل لیاپونو [1857-1918] کا مستحکم تفرقی مساوات پر کام بنیادی حیثیت رکھتا ہے۔استحکام کی یہ تعریف انہوں نے ہی پیش کی۔} کہلائے گا۔[شکل \حوالہ{شکل_نظام_نقطہ_فاصل_تعریف}-الف دیکھیں]

اگر \عددی{P_0} مستحکم نہ ہو تب یہ \اصطلاح{غیر مستحکم}\فرہنگ{غیر مستحکم}\حاشیہب{unstable}\فرہنگ{unstable} کہلاتا ہے۔ 

ایسا مستحکم \عددی{P_0} جہاں وہ تمام خط حرکت جن کا کوئی بھی نقطہ، \عددی{D_{\sigma}} پر پایا جاتا ہو، آخر کار (\عددیء{t \to \infty}) \عددی{P_0} کے قریب تر پہنچے  \اصطلاح{مستحکم اور جاذب}\فرہنگ{مستحکم اور جاذب}\حاشیہب{stable and attractive}\فرہنگ{stable and attractive} کہلاتا ہے۔[شکل \حوالہ{شکل_نظام_نقطہ_فاصل_تعریف}-ب دیکھیں۔]
\انتہا{تعریف}
%==============================

\begin{figure}
\centering
\begin{subfigure}{0.5\textwidth}
\centering
\begin{tikzpicture}
\draw(0,0) circle (1.25);
\draw(0,0) circle (2);
\draw[-stealth](0,0)--++(25:1.25)node[pos=0.5,fill=white]{$\sigma$};
\draw[-stealth](0,0)--++(125:2)node[pos=0.25,fill=white]{$\epsilon$};
\draw[fill=white](0,0) node [ocirc]{}node[below]{$P_0$};
%trajectory
\draw[-stealth](0,1)node[ocirc]{}node[below]{$P_1$} to [out=-10,in=-170]++(0.75,0) to [out=10,in=90]++(0.75,-1) to [out=-90,in=0]++(-1,-0.6) to [out=180,in=-15]++(-1.25,-.75) to [out=165,in=-90]++(-0.75,1);
\end{tikzpicture}
\caption*{(الف) مستحکم نقطہ فاصل \عددی{P_0} کی صورت میں خط حرکت \عددی{D_{\epsilon}} میں رہتی ہے۔}
\end{subfigure}%
\begin{subfigure}{0.5\textwidth}
\centering
\begin{tikzpicture}
\draw(0,0) circle (1.25);
\draw(0,0) circle (2);
\draw[-stealth](0,0)--++(25:1.25)node[pos=0.5,fill=white]{$\sigma$};
\draw[-stealth](0,0)--++(125:2)node[pos=0.35,fill=white]{$\epsilon$};
\draw[fill=white](0,0) node [ocirc]{}node[shift={(0,-0.4)}]{$P_0$};
%trajectory
\draw[stealth-] (0,0)node[ocirc](ka){};\draw[stealth-] (ka) to [out=100,in=-120]++(0.25,0.85)node[ocirc]{};
\draw[stealth-](ka) to [out=-135,in=-45]++(-0.75,-0.5) to [out=135,in=-170]++(0.3,0.7)node[ocirc]{};
\draw[stealth-] (ka) to [out=-30,in=-145]++(1,-0.5) to [out=35,in=-90]++(0.5,1) to [out=90,in=0] (0,1.75) to [out=180,in=90] (-1.75,0) to [out=-90,in=160] (-135:1.3) to [out=-20,in=-135]++(0.5,0)node[ocirc]{};
\end{tikzpicture}
\caption*{(ب) مستحکم اور جاذب نقطہ فاصل \عددی{P_0}۔}
\end{subfigure}%
\caption{نظام \حوالہ{مساوات_نظام_جانچ_نقطہ_فاصل_الف} کے نقطہ فاصل۔}
\label{شکل_نظام_نقطہ_فاصل_تعریف}
\end{figure}%
%=====================

استحکام کی بنیاد پر نقطہ فاصل کی درجہ بندی جدول \حوالہ{جدول_نظام_نقطہ_فاصل_بالمقابل_استحکام} میں دی گئی ہے۔
\begin{table}
\caption{استحکام کی بنیاد پر نقطہ فاصل کی درجہ بندی۔}
\label{جدول_نظام_نقطہ_فاصل_بالمقابل_استحکام}
\centering
\begin{tabular}{rlr}
استحکام کی قسم& \عددی{p=\lambda_1+\lambda_2}& \عددی{q=\lambda_1\lambda_2}\\
\hline
(الف) مستحکم اور جاذب& \عددی{p<0} & \عددی{q>0}\\
(ب) مستحکم & \عددی{p\le 0} & \عددی{q>0}\\
(پ) غیر مستحکم & \multicolumn{2}{c}{ \عددی{p>0} \,\,\, یا \,\,\, \عددی{q<0}}
\end{tabular}
\end{table}
