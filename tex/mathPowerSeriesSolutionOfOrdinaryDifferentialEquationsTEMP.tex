\باب{قائمہ الزاویہ تفاعل کا سلسلہ}
لیژانڈر تفاعل (حصہ \حوالہ{حصہ_لیژانڈر_تفاعل}) اور بیسل تفاعل کی ایک خاصیت جسے \اصطلاح{قائمیت}\فرہنگ{قائمیت}\حاشیہب{orthogonality}\فرہنگ{orthogonality} کہتے ہیں انجینئری حساب میں نمایاں کردار ادا کرتی ہے۔اس حصے میں قائمیت سے وابستہ تصورات اور  علامت نویسی سیکھتے ہیں۔اگلے حصے میں  ایسی سرحدی قیمت مسائل (سٹیورم لیوویل مسائل) پر غور کیا جائے گا جن کے حل قائمہ الزاویہ تفاعل کا سلسلہ دیتے ہیں۔ان مسائل  پر غور کے دوران حاصل نتائج کو استعمال کرتے ہوئے لیژانڈر تفاعل اور بیسل تفاعل پر غور کیا جائے گا۔

آئیں پہلے تفاعل کی قائمیت کی تعریف پیش کرتے ہیں۔فرض کریں کہ وقفہ \عددی{a\le x\le b} پر حقیقی قیمت تفاعل \عددی{g_m(x)} اور \عددی{g_n(x)} معین ہیں اور اس وقفے پر ان تفاعل کے حاصل ضرب \عددی{g_m(x)g_n(x)} کا تکمل موجود ہے۔اس تکمل کو روایتی طور پر \عددی{(g_m,g_n)} لکھا جاتا ہے۔
\begin{align}
(g_m,g_n)=\int_{a}^{b} g_m(x)g_n(x)\dif x
\end{align}
اگر درج بالا تکمل صفر کے برابر ہو تب تفاعل \عددی{g_m(x)} اور \عددی{g_n(x)} وقفہ \عددی{a\le x\le b} پر  \اصطلاح{قائمہ الزاویہ}\فرہنگ{قائمہ الزاویہ}\حاشیہب{orthogonal}\فرہنگ{orthogonal} کہلاتے ہیں۔
\begin{align}
(g_m,g_n)=\int_{a}^{b} g_m(x)g_n(x)\dif x=0\quad \quad (m \ne n)
\end{align}
حقیقی قیمت تفاعل کا سلسلہ \عددی{g_1(x)}، \عددی{g_2(x)}، \عددی{g_3(x)}، \نقطے اس صورت وقفہ  \عددی{a\le x\le b} پر \اصطلاح{قائمہ الزاویہ سلسلہ}\فرہنگ{قائمہ الزاویہ!سلسلہ}\حاشیہب{orthogonal set}\فرہنگ{orthogonal set}  کہلائے گا جب اس وقفے پر یہ تمام تفاعل معین اور تمام تکمل \عددی{(g_m,g_n)} موجود ہوں اور اس سلسلے میں تمام ممکنہ منفرد جوڑیوں کے یہ تکمل صفر کے برابر ہوں۔ 

\عددی{(g_m,g_m)} کے غیر صفر جذر کو \عددی{g_m} کا \اصطلاح{معیار}\فرہنگ{معیار}\حاشیہب{norm}\فرہنگ{norm} کہتے ہیں جسے عموماً \عددی{\norm{g_m}} سے ظاہر کیا جاتا ہے۔
\begin{align}
\norm{g_m}=\sqrt{(g_m,g_m)}=\sqrt{\int_{a}^{b} g_m^2(x)\dif x}
\end{align}
ہم پوری بحث کے دوران درج ذیل فرض کریں گے۔\\
\موٹا{عمومی مفروضہ:} \quad تمام تفاعل جن پر غور کیا جا رہا ہو محدود ہیں، جن تکمل پر غور کیا جا رہا ہو وہ موجود ہیں اور معیار غیر صفر ہیں۔

 ظاہر ہے کہ وقفہ \عددی{a\le x\le b} پر ایسے قائمہ الزاویہ سلسلہ \عددی{g_1}، \عددی{g_2}، \نقطے جن  میں ہر تفاعل کا معیار اکائی \عددی{(1)} ہو درج ذیل تعلقات پر پورا اترتے ہیں۔
\begin{align}
(g_m,g_n)=\int_{a}^{b}g_m(x) g_n(x)\dif x=
\begin{cases}
0 & m\ne n \quad (m=1,2,\cdots)\\
1& m=n \quad (n=1,2,\cdots)
\end{cases}
\end{align}
ایسے سلسلے کو وقفہ \عددی{a\le x\le b} پر \اصطلاح{معیاری قائمہ الزاویہ سلسلہ}\فرہنگ{معیاری قائمہ الزاویہ سلسلہ}\فرہنگ{قائمہ الزاویہ!معیاری سلسلہ}\حاشیہب{orthonormal set}\فرہنگ{orthonormal!set} کہتے ہیں۔

کسی بھی قائمہ الزاویہ سلسلے کے ہر تفاعل کو،زیر غور وقفے پر، اس تفاعل کی  معیار سے تقسیم کرتے ہوئے معیاری قائمہ الزاویہ سلسلہ حاصل کیا جا سکتا ہے۔ 

%====================
\ابتدا{مثال}\شناخت{مثال_طاقتی_اضافی_سائن_عمودیت_الف}
تفاعل  \عددی{g_m(x)=\sin mx} جہاں \عددی{m=1,2,\cdots} کا سلسلہ وقفہ \عددی{-\pi\le x\le \pi} پر قائمہ الزاویہ ہے کیونکہ ان تفاعل کے لئے درج ذیل لکھا جا سکتا ہے (ضمیمہ \حوالہ{ضمیمہ_مفید_معلومات} میں مساوات \حوالہ{مساوات_ضمیمہ_مفید_گیارہ})۔
\begin{gather}
\begin{aligned}
(g_m,g_n)&=\int_{-\pi}^{\pi}\sin mx\sin nx\dif x\quad (m\ne n)\\
&=\frac{1}{2}\int_{-\pi}^{\pi}\cos(m-n)x\dif x-\frac{1}{2}\int_{-\pi}^{\pi}\cos(m+n)\dif x=0
\end{aligned}
\end{gather}
ان تفاعل کا معیار \عددی{\norm{g_m}=\sqrt{\pi}}  ہے۔
\begin{align*}
\norm{g_m}^2=\int_{-\pi}^{\pi} \sin^2 mx\dif x=\pi\quad \quad (m=1,2,\cdots)
\end{align*} 
یوں اس سلسلے سے درج ذیل معیاری قائمہ الزاویہ سلسلہ حاصل ہوتا ہے۔
\begin{align*}
\frac{\sin x}{\sqrt{\pi}},\quad \frac{\sin 2x}{\sqrt{\pi}},\quad \frac{\sin 3x}{\sqrt{\pi}}
\end{align*} 
\انتہا{مثال}
%=====================
\ابتدا{مثال}\شناخت{مثال_طاقتی_اضافی_سائن_عمودیت_ب}
کوسائن تفاعل \عددی{\cos mx} کے سلسلے کو بھی مثال \حوالہ{مثال_طاقتی_اضافی_سائن_عمودیت_الف} کی طرح قائمہ الزاویہ ثابت کیا جا سکتا ہے۔مزید تمام \عددی{m,n=0,1,\cdots} کے لئے درج ذیل لکھا جا سکتا ہے۔
\begin{align*}
\int_{-\pi}^{\pi} \cos mx \sin nx \dif x=\frac{1}{2}\int_{-\pi}^{\pi}\sin(m+n)x\dif x-\frac{1}{2}\int_{-\pi}^{\pi}\sin(m-n)x \dif x=0
\end{align*}
یوں ظاہر ہے کہ درج ذیل سلسلہ  وقفہ \عددی{-\pi\le x\le \pi} پر قائمہ الزاویہ ہے
\begin{align*}
1,\quad \cos x,\quad \sin x,\quad \cos 2x,\quad \sin 2x,\quad \cdots
\end{align*}
جس سے درج ذیل معیاری قائمہ الزاویہ سلسلہ حاصل ہوتا ہے۔
\begin{align*}
\frac{1}{\sqrt{2\pi}},\quad \frac{\cos x}{\sqrt{\pi}},\quad \frac{\sin x}{\sqrt{\pi}},\quad \frac{\cos 2x}{\sqrt{\pi}},\quad \frac{\sin 2x}{\sqrt{\pi}},\quad \cdots
\end{align*}
\انتہا{مثال}
%=======================

قائمہ الزاویہ سلسلہ استعمال کرتے ہوئے مختلف تفاعل کو تسلسل کی صورت میں لکھا جا سکتا ہے۔فرض کریں کہ وقفہ \عددی{1\le x\le b} پر \عددی{g_1(x)}، \عددی{g_2(x)}، \نقطے کوئی بھی قائمہ الزاویہ سلسلہ ہے۔اب فرض کریں کہ \عددی{f(x)} کوئی بھی تفاعل ہے جس کو ان \عددی{g(x)} کی ایسی تسلسل
\begin{align}\label{مساوات_طاقتی_اضافی_فوریئر_عمومی_الف}
f(x)=\sum_{n=1}^{\infty} c_n g_n(x)=c_1g_1(x)+c_2g_2(x)+\cdots
\end{align}
لکھنا ممکن ہو جو \اصطلاح{مرکوز} ہو۔اس تسلسل کو \عددی{f(x)} کی \اصطلاح{عمومی فوریئر تسلسل}\فرہنگ{فوریئر تسلسل!عمومی}\حاشیہب{generalized Fourier series}\فرہنگ{Fourier series!generalized} کہتے ہیں جبکہ \عددی{c_1}، \عددی{c_2}، \نقطے کو ان قائمہ الزاویہ سلسلے کے لحاض سے تسلسل کے \اصطلاح{فوریئر مستقل}\فرہنگ{فوریئر!مستقل}\حاشیہب{Fourier constants}\فرہنگ{Fourier constants} کہتے ہیں۔ 

قائمیت کی بنا ان مستقل کو نہایت آسانی سے حاصل کیا جا سکتا ہے۔مساوات \حوالہ{مساوات_طاقتی_اضافی_فوریئر_عمومی_الف} کے دونوں اطراف کو \عددی{g_m(x)} (معین \عددی{m} ) سے ضرب دیتے ہوئے وقفہ \عددی{a\le x\le b} پر تکمل لینے سے درج ذیل ملتا ہے جہاں فرض کیا گیا ہے کہ جزو در جزو تکمل لیا جا سکتا ہے۔ 
\begin{align*}
(f,g_m)=\int_{a}^{b}fg_m\dif x=\sum_{n=1}^{\infty} c_n(g_n,g_m)=\sum_{n=1}^{\infty} c_n\int_{a}^{b} g_n g_m \dif x
\end{align*}
بائیں ہاتھ جن تکملات میں \عددی{n=m} ہو، وہ  \عددی{(g_n,g_m)=\norm{g_m}^2} کے برابر ہوں گے جبکہ قائمیت کی بنا  باقی تمام تکملات صفر کے برابر ہوں گے لہٰذا
\begin{align}\label{مساوات_طاقتی_اضافی_فوریئر_عمومی_ب}
(f,g_m)=c_m\norm{g_m}^2
\end{align}
ہو گا اور یوں فوریئر مستقل کا درج ذیل کلیہ حاصل ہوتا ہے۔
\begin{align}\label{مساوات_طاقتی_اضافی_فوریئر_عمومی_پ}
c_m=\frac{(f,g_m)}{\norm{g_m}^2}=\frac{1}{\norm{g_m}^2}\int_{a}^{b} f(x)g_m(x)\dif x \quad \quad (m=1,2,\cdots)
\end{align}
%===============
\ابتدا{مثال}\شناخت{مثال_طاقتی_فوریئر_تسلسل_عمومی_الف}\quad فوریئر تسلسل\\
مساوات \حوالہ{مساوات_طاقتی_اضافی_فوریئر_عمومی_الف} کو مثال \حوالہ{مثال_طاقتی_اضافی_سائن_عمودیت_ب} کے معیاری قائمہ الزاویہ سلسلہ کی صورت درج ذیل لکھا جا سکتا ہے
\begin{align}\label{مساوات_طاقتی_اضافی_فوریئر_عمومی_ت}
f(x)=a_0+\sum_{n=1}^{\infty}(a_n\cos nx+b_n\sin nx)
\end{align}
اور مساوات \حوالہ{مساوات_طاقتی_اضافی_فوریئر_عمومی_پ} اب درج ذیل دے گا۔
\begin{gather}
\begin{aligned}\label{مساوات_طاقتی_اضافی_فوریئر_عمومی_ٹ}
a_0&=\frac{1}{2\pi}\int_{-\pi}^{\pi} f(x)\dif x\\
a_n&=\frac{1}{\pi}\int_{-\pi}^{\pi} f(x)\cos nx \dif x\\
b_n&=\frac{1}{\pi}\int_{-\pi}^{\pi}f(x)\sin nx \dif x\quad \quad (n=1,2,\cdots)
\end{aligned}
\end{gather} 
اب اگر تسلسل \حوالہ{مساوات_طاقتی_اضافی_فوریئر_عمومی_ت} مرکوز ہو تب یہ \عددی{f(x)} کی \اصطلاح{فوریئر تسلسل} کہلائے گا اور \عددی{a_0}، \عددی{a_n}، \عددی{b_n} اس کے \اصطلاح{فوریئر عددی سر}\فرہنگ{فوریئر!عددی سر}\فرہنگ{عددی سر!فوریئر}\حاشیہب{Fourier coefficients}\فرہنگ{Fourier!coefficients} کہلائیں گے۔کلیات \حوالہ{مساوات_طاقتی_اضافی_فوریئر_عمومی_ٹ} کو ان عددی سر کے \اصطلاح{یولر کلیات}\فرہنگ{یولر!کلیات}\حاشیہب{Euler formulae}\فرہنگ{Euler!formulae} کہتے ہیں۔
\انتہا{مثال}
%============================

ایسے کئی اہم سلسلے پائے جاتے ہیں جو از خود قائمہ الزاویہ نہیں ہیں البتہ ان کے حقیقی تفاعل \عددی{g_1}، \عددی{g_2}، \نقطے  درج ذیل پر پورا اترتے ہیں جہاں \عددی{p(x)} کوئی غیر صفر تفاعل ہے۔
\begin{align}\label{مساوات_طاقتی_اضافی_عمودی_تفاعل_قدر_الف}
\int_{a}^{b} p(x) g_m(x)g_n(x)\dif x=0\quad \quad (m\ne n)
\end{align}  
ہم کہتے ہیں کہ ایسا سلسلہ وقفہ \عددی{a\le x\le b} پر \اصطلاح{تفاعل قدر}\فرہنگ{تفاعل قدر}\حاشیہب{weight function}\فرہنگ{weight function} \عددی{p(x)} کے لحاض سے قائمہ الزاویہ ہے۔\عددی{g_m} کے \اصطلاح{معیار} کی تعریف اب درج ذیل ہے۔
\begin{align}\label{مساوات_طاقتی_اضافی_معیار_قدر_تفاعل}
\norm{g_m}=\sqrt{\int_{a}^{b}p(x)g_m^2\dif x}
\end{align}
اگر سلسلے کے ہر تفاعل \عددی{g_m} کا معیار اکائی \عددی{(1)} ہو تب وقفہ \عددی{a\le x\le b} پر تفاعل قدر \عددی{p(x)} کے لحاض سے یہ سلسلہ معیاری قائمہ الزاویہ کہلائے گا۔

ہم \عددی{h_m=\sqrt{p}g_m} اور \عددی{h_n=\sqrt{p}g_n} لکھ کر مساوات \حوالہ{مساوات_طاقتی_اضافی_عمودی_تفاعل_قدر_الف} کو درج ذیل لکھ سکتے ہیں
\begin{align}
\int_{a}^{b}h_m(x)h_n(x)\dif x=0 \quad \quad (m\ne n)
\end{align}
اور یوں ظاہر ہے کہ \عددی{h_m} تفاعل قائمہ الزاویہ ہیں۔

اگر تفاعل قدر \عددی{p(x)} کے لحاض سے، وقفہ \عددی{a\le x\le b} پر تفاعل \عددی{g_1(x)}، \عددی{g_2(x)}، \نقطے  قائمہ الزاویہ ہوں اور اگر کسی تفاعل \عددی{f(x)} کو درج ذیل عمومی فوریئر تسلسل کی صورت میں لکھنا ممکن ہو (مساوات \حوالہ{مساوات_طاقتی_اضافی_فوریئر_عمومی_الف} دیکھیں)
\begin{align}\label{مساوات_ضمیمہ_اضافی_تفاعل_قدر_تسلسل_الف}
f(x)=c_1g_1(x)+c_2g_2(x)+\cdots
\end{align} 
 تب اس سلسلے کے لحاض سے فوریئر مستقل \عددی{c_1}، \عددی{c_2}،\نقطے کو بھی پہلی کی طرح حاصل کیا جا سکتا ہے بس فرق اتنا ہے کہ اب مساوات
 \حوالہ{مساوات_ضمیمہ_اضافی_تفاعل_قدر_تسلسل_الف} کے دونوں اطراف کو (\عددی{g_m} کی بجائے) \عددی{pg_m} سے ضرب دے کر آگے بڑھا جائے گا۔باقی سب کچھ پہلے کی طرح حل کرتے ہوئے درج ذیل ملتا ہے جہاں تفاعل کا معیار اب مساوات \حوالہ{مساوات_طاقتی_اضافی_معیار_قدر_تفاعل} دے گا۔ 
\begin{align}
c_m=\frac{1}{\norm{g_m}^2}\int_{a}^{b}p(x)f(x)g_m(x)\dif x\quad \quad (m=1,2,\cdots)
\end{align} 
%====================
\حصہء{سوالات}
سوال \حوالہ{سوال_طاقتی_اضافی_فوریئر_الف} تا سوال \حوالہ{سوال_طاقتی_اضافی_فوریئر_ت} میں ثابت کریں کہ دیے گئے وقفے پر دیا گیا سلسلہ قائمہ الزاویہ ہے۔معیاری قائمہ الزاویہ سلسلہ بھی دریافت کریں۔

%=============
\ابتدا{سوال}\شناخت{سوال_طاقتی_اضافی_فوریئر_الف}\quad
$1,\, \cos x,\, \cos 2x,\, \cos 3x,\, \cdots,\quad \quad 0\le x\le 2\pi$\\
جوابات:
$\frac{1}{\sqrt{2\pi}},\, \frac{\cos x}{\sqrt{\pi}},\, \frac{\cos 2x}{\sqrt{\pi}},\, \frac{\cos 3x}{\sqrt{\pi}}$
\انتہا{سوال}
%=====================
\ابتدا{سوال}\quad
$\sin x, \sin 2x, \sin 3x,\, \cdots,\quad \quad 0\le x\le \pi$\\
جوابات:
$\sqrt{\frac{2}{\pi}}\sin x, \sqrt{\frac{2}{\pi}}\sin 2x, \sqrt{\frac{2}{\pi}}\sin 3x,\, \cdots$
\انتہا{سوال}
%======================
\ابتدا{سوال}\quad
$\sin \pi x, \sin 2\pi x, \sin 3\pi x,\, \cdots,\quad \quad -1\le x\le 1$\\
جوابات:
$\sin \pi x, \sin 2\pi x, \sin 3\pi x,\, \cdots$
\انتہا{سوال}
%======================
\ابتدا{سوال}\quad
$1,\, \cos 2x,\, \cos 4x,\, \cos 6x,\, \cdots,\quad \quad 0\le x\le \pi$\\
جوابات:
$\frac{1}{\sqrt{\pi}},\, \sqrt{\frac{2}{\pi}}\cos 2x,\, \sqrt{\frac{2}{\pi}}\cos 4x,\, \sqrt{\frac{2}{\pi}}\cos 6x,\, \cdots$
\انتہا{سوال}
%=====================
\ابتدا{سوال}\شناخت{سوال_طاقتی_اضافی_فوریئر_ب}\quad
$1,\, \cos \frac{2n\pi}{T}x, \quad (n=1,2,\cdots),\quad \quad 0\le x\le T$\\
جوابات:
$\frac{1}{\sqrt{T}},\, \sqrt{\frac{2}{T}}\cos \frac{2n\pi}{T}x, $
\انتہا{سوال}
%======================
\ابتدا{سوال}\quad
$\sin \frac{2n\pi}{T}x, \quad (n=1,2,\cdots),\quad \quad 0\le x\le T$\\
جوابات:
$\sqrt{\frac{2}{T}}\sin \frac{2n\pi}{T}x, $
\انتہا{سوال}
%======================
\ابتدا{سوال}\quad (حصہ \حوالہ{حصہ_لیژانڈر_تفاعل} کے لیژانڈر تفاعل)
$P_0(x),\, P_1(x),\, P_2(x),\, \cdots,\quad \quad -1\le x\le 1$\\
جوابات:
$\frac{P_0}{\sqrt{2}},\, \sqrt{\frac{3}{2}}P_1,\,  \sqrt{\frac{5}{2}}P_2,\, \sqrt{\frac{7}{2}}P_3$
\انتہا{سوال}
%============================
\ابتدا{سوال}\quad ایسے \عددی{a_0}، \عددی{b_0}، \نقطے،\عددی{c_2} دریافت کریں کہ  وقفہ \عددی{-1\le x\le 1} پر  \عددی{g_1}، \عددی{g_2} اور \عددی{g_3} معیاری قائمہ الزاویہ ہوں۔ حاصل جواب کا لیژانڈر تفاعل کے ساتھ موازنہ کریں۔ \quad 
$g_1=a_0,\, g_2=b_0+b_1x,\, g_3=c_0+c_1x+c_2x^2$\\
\انتہا{سوال}
%===========================
\ابتدا{سوال}\شناخت{سوال_طاقتی_اضافی_فوریئر_پ}
ثابت کریں کہ اگر وقفہ \عددی{a\le x\le b} پر تفاعل \عددی{g_1(x)}، \عددی{g_2(x)}، \نقطے قائمہ الزاویہ ہوں تب وقفہ \عددی{\tfrac{a-k}{c} \le t \le \tfrac{b-k}{c}} پر تفاعل \عددی{g_1(ct+k)}، \عددی{g_2(ct+k)}، \نقطے قائمہ الزاویہ ہوں گے۔
\انتہا{سوال}
%=====================
\ابتدا{سوال}\شناخت{سوال_طاقتی_اضافی_فوریئر_ت}
سوال \حوالہ{سوال_طاقتی_اضافی_فوریئر_پ} کے نتیجے کو استعمال کرتے ہوئے سوال \حوالہ{سوال_طاقتی_اضافی_فوریئر_الف} سے سوال \حوالہ{سوال_طاقتی_اضافی_فوریئر_ب} کا نتیجہ حاصل کریں۔
\انتہا{سوال}
%=========================

\حصہ{مسئلہ سٹیورم لیوویل}\شناخت{مسئلہ_طاقتی_سٹیورم_لیوویل}
انجینئری حساب میں کئی اہم قائمہ الزاویہ سلسلوں کے تفاعل وقفہ \عددی{a\le x \le b} پر بطور درج ذیل دو درجی تفرقی مساوات کے حل سامنے آتے ہیں
\begin{align}\label{مساوات_طاقتی_سٹیورم_لیوویل_مساوات_الف}
[r(x)y']'+[q(x)+\lambda p(x)]y=0
\end{align}
جو درج ذیل شرائط پر پورا اترتے ہیں۔
\begin{gather}
\begin{aligned}\label{مساوات_طاقتی_سٹیورم_لیوویل_مساوات_ب}
\text{(الف)}\quad k_1y(a)+k_2y'(a)&=0\quad \quad (\text{\RL{\عددی{k_1} اور \عددی{k_2} بیکوقت صفر نہیں ہو سکتے ہیں}})\\
\text{(ب)}\quad l_1y(b)+l_2y'(b)&=0\quad \quad (\text{\RL{\عددی{l_1} اور \عددی{l_2} بیکوقت صفر نہیں ہو سکتے ہیں}})
\end{aligned}
\end{gather}
یہاں \عددی{\lambda} مقدار معلوم ہے جبکہ \عددی{k_1}، \عددی{k_2}، \عددی{l_1} اور \عددی{l_2} حقیقی مستقل ہیں۔

مساوات \حوالہ{مساوات_طاقتی_سٹیورم_لیوویل_مساوات_الف} کو \اصطلاح{مساوات سٹیورم لیوویل}\فرہنگ{سٹیورم لیوویل مساوات}\حاشیہب{Sturm-Liouville equation}\فرہنگ{Sturm-Liouville equation} کہتے ہیں۔ مساوات \حوالہ{مساوات_طاقتی_سٹیورم_لیوویل_مساوات_ب} وقفے کے آخری سروں \عددی{a} اور \عددی{b} سے تعلق رکھتے ہیں لہٰذا انہیں   \اصطلاح{سرحدی شرائط}\فرہنگ{سٹیورم لیوویل!سرحدی شرائط}\فرہنگ{Sturm-Liouville!boundary conditions} کہتے ہیں۔ آپ دیکھیں گے کہ لیژانڈر، بیسل اور دیگر مساوات کو مساوات \حوالہ{مساوات_طاقتی_سٹیورم_لیوویل_مساوات_الف} کی صورت میں لکھا جا سکتا ہے۔تفرقی مساوات اور سرحدی شرائط مل کر \اصطلاح{سرحدی مسئلہ}\فرہنگ{سرحدی مسئلہ}\حاشیہب{boundary problem}\فرہنگ{boundary problem} دیتے ہیں۔مساوات \حوالہ{مساوات_طاقتی_سٹیورم_لیوویل_مساوات_الف} اور  مساوات \حوالہ{مساوات_طاقتی_سٹیورم_لیوویل_مساوات_ب} کے سرحدی مسئلے کو \اصطلاح{سٹیورم لیوویل مسئلہ}\فرہنگ{سٹیورم لیوویل مسئلہ}\حاشیہد{سوئزرلینڈ کے ریاضی دان جیکویس چارلس فرانکوئس سٹیوورم [1803-1882] اور فرانسیسی ریاضی دان یوسف لیوویل [1809-1882]}\فرہنگ{Sturm-Liouville problem} کہتے ہیں۔

آپ دیکھ سکتے ہیں کہ \عددی{\lambda} کی کسی بھی قیمت کے لئے سٹیورم لیوویل مسئلے کا غیر اہم صفر حل \عددی{y\equiv 0} پایا جاتا ہے جو پورے وقفے پر \عددی{y(x)=0} دیتا ہے۔اگر غیر صفر اہم حل \عددی{y \not \equiv 0}  موجود ہوں تو انہیں \اصطلاح{امتیازی تفاعل} یا \اصطلاح{آئگنی تفاعل}\فرہنگ{آئگنی تفاعل}\حاشیہب{eigenfunctions}\فرہنگ{eigenfunctions} کہتے ہیں اور \عددی{\lambda} کی ان قیمتوں جن کے لئے مسئلے کا حل موجود ہو کو \اصطلاح{امتیازی قیمت} یا \اصطلاح{آئگنی قدر}\فرہنگ{آئگنی قدر}\حاشیہب{eigenvalue}\فرہنگ{eigenvalue} کہتے ہیں۔
%======================
\ابتدا{مثال}\شناخت{مثال_طاقتی_تسلسل_مثال_سٹیورم_الف}
درج ذیل سٹیورم لیوویل مسئلے کے آئگنی قدر اور آئگنی تفاعل دریافت کریں۔
\begin{align*}
y''+\lambda y=0,\quad y(0)=0,\, y(\pi)=0
\end{align*}
حل:\عددی{\lambda} کی منفی قیمتوں \عددی{\lambda=-v^2} کے لئے تفرقی مساوات کا عمومی حل درج ذیل ہے۔
\begin{align*}
y(x)=c_1e^{vx}+c_2e^{-vx}
\end{align*}
دیے گئے سرحدی شرائط استعمال کرتے ہوئے \عددی{c_1=c_2=0} اور \عددی{y\equiv 0} ملتا ہے جو آئگنی تفاعل نہیں ہے۔\عددی{\lambda=0} کی صورت میں بھی یہی صورت حال پائی جاتی ہے۔مثبت \عددی{\lambda=v^2} کے لئے تفرقی مساوات کا عمومی حل درج ذیل ہے۔
\begin{align*}
y(x)=A\cos vx+B\sin vx
\end{align*}
پہلی سرحدی شرط سے \عددی{y(0)=A=0}  ملتا ہے۔دوسری سرحدی شرط سے درج ذیل ملتا ہے۔
\begin{align*}
y(\pi)=B\sin v \pi=0 \quad \implies \quad v=0, \mp1, \mp2,\cdots
\end{align*}
\عددی{v=0} سے \عددی{y\equiv 0} ملتا ہے جبکہ \عددی{B=1} لیتے ہوئے \عددی{\lambda=v^2=1,4,9,\cdots} کے لئے 
\begin{align*}
y(x)=\sin vx\quad \quad v=1,2,3,\cdots
\end{align*}
ملتا ہے۔ یوں اس مسئلے کے آئگنی اقدار \عددی{\lambda=v^2} ہیں جہاں \عددی{v=1,2,\cdots} ہیں اور ان کے مطابقتی آئگنی تفاعل \عددی{y(x)=\sin vx} ہیں۔
 
\انتہا{مثال}
%=========================

سٹیورم لیوویل مسئلہ  درج ذیل قائمیت کی خاصیت رکھتا ہے۔ 

%===============
\ابتدا{مسئلہ}\شناخت{مسئلہ_طاقتی_تسلسل_قائمیت_سٹیورم}\quad آئگنی تفاعل کی قائمیت\\
فرض کریں کہ مساوات \حوالہ{مساوات_طاقتی_سٹیورم_لیوویل_مساوات_الف} میں دیے گئے سٹیورم لیوویل مسئلے  میں  \عددی{p}، \عددی{q}، \عددی{r} اور \عددی{r'}  حقیقی قیمت تفاعل ہیں جو وقفہ \عددی{a\le x\le b} پر استمراری ہیں۔فرض کریں کہ دو منفرد آئگنی قدر \عددی{\lambda_m} اور \عددی{\lambda_n} کے لئے مساوات \حوالہ{مساوات_طاقتی_سٹیورم_لیوویل_مساوات_الف} میں دیے گئے سٹیورم لیوویل مسئلے  کے مطابقتی حل   \عددی{y_m(x)} اور \عددی{y_n(x)}  ہیں۔ اس وقفے پر تفاعل قدر \عددی{p} کے لحاض سے \عددی{y_m} اور \عددی{y_n} قائمہ الزاویہ ہوں گے۔

اگر \عددی{r(a)=0} ہو تب مساوات \حوالہ{مساوات_طاقتی_سٹیورم_لیوویل_مساوات_ب}-الف کی ضرورت نہیں ہو گی لہٰذا اس کو مسئلے سے نکالا جا سکتا ہے۔ اسی طرح اگر \عددی{r(b)=0} تب مساوات \حوالہ{مساوات_طاقتی_سٹیورم_لیوویل_مساوات_ب}-ب کی ضرورت نہیں ہو گی لہٰذا اس کو مسئلے سے نکالا جا سکتا ہے۔اگر \عددی{r(a)=r(b)} ہو تب مساوات \حوالہ{مساوات_طاقتی_سٹیورم_لیوویل_مساوات_ب} کی جگہ درج ذیل شرط لکھی جا سکتی ہے۔
\begin{align}\label{مساوات_طاقتی_تسلسل_ثبوت_عمودیت_الف}
y(a)=y(b),\quad y'(a)=y'(b)
\end{align}
\انتہا{مسئلہ}
%====================

\ابتدا{ثبوت}
چونکہ \عددی{y_m} اور \عددی{y_n} اس مسئلے کے حل ہیں لہٰذا یہ مساوات \حوالہ{مساوات_طاقتی_سٹیورم_لیوویل_مساوات_الف} پر پورا اترتے ہیں اور یوں درج ذیل لکھا جا سکتا ہے۔
\begin{align*}
(ry_m')'+(q+\lambda_m p)y_m&=0\\
(ry_n')'+(q+\lambda_n p)y_n&=0
\end{align*}
پہلی مساوات کو \عددی{y_n} اور دوسری مساوات کو \عددی{-y_m} سے ضرب دے کر ان کا مجموعے لیتے ہیں۔
\begin{align*}
(\lambda_m-\lambda_n)p y_my_n&=y_m(ry_n')'-y_n(ry_m')'\\
&=[(ry_n')y_m-(ry_m')y_n]'
\end{align*}
آپ آخری مساوات \عددی{[(ry_n')y_m-(ry_m')y_n]'} کو کھول کر پہلی مساوات حاصل کرتے ہوئے اس کی درستگی ثابت کر سکتے ہیں۔چونکہ قیاس کے تحت \عددی{r} اور \عددی{r'} استمراری ہیں جبکہ \عددی{y_m} اور \عددی{y_n} مسئلے کے حل ہیں لہٰذا \عددی{[(ry_n')y_m-(ry_m')y_n]'} استمراری ہے۔وقفہ \عددی{a\le x\le b} پر اس کا تکمل لیتے ہیں
\begin{align}\label{مساوات_طاقتی_تسلسل_ثبوت_عمودیت_ب}
(\lambda_m-\lambda_n)\int_a^b py_my_n\dif x=\left[r(y_n'y_m-y_m'y_n)\right]_a^b
\end{align}
جہاں دایاں ہاتھ درج ذیل کے برابر ہے۔
\begin{align}\label{مساوات_طاقتی_تسلسل_ثبوت_عمودیت_پ}
r(b)[y_n'(b)y_m(b)-y_m'(b)y_n(b)]-r(a)[y_n'(a)y_m(a)-y_m'(a)y_n(a)]
\end{align}

پہلی صورت: اگر \عددی{r(a)=0} اور \عددی{r(b)=0} ہوں تب  مساوات \حوالہ{مساوات_طاقتی_تسلسل_ثبوت_عمودیت_پ} صفر کے برابر ہو گا لہٰذا مساوات \حوالہ{مساوات_طاقتی_تسلسل_ثبوت_عمودیت_ب} کا بایاں ہاتھ بھی صفر ہو گا اور چونکہ \عددی{y_m} اور \عددی{y_n} منفرد ہیں ہمیں مساوات \حوالہ{مساوات_طاقتی_سٹیورم_لیوویل_مساوات_ب} میں دیے گئے سرحدی شرائط کے استعمال کے بغیر درج ذیل قائمیت ملتی ہے۔
\begin{align}\label{مساوات_طاقتی_تسلسل_ثبوت_عمودیت_ت}
\int_a^b py_my_n\dif x=0\quad \quad (m\ne n)
\end{align} 

دوسری صورت:اگر \عددی{r(b)=0} لیکن \عددی{r(a)\ne 0} ہو تب مساوات \حوالہ{مساوات_طاقتی_تسلسل_ثبوت_عمودیت_پ} کا بایاں حصہ صفر کے برابر ہو گا۔آئیں مساوات \حوالہ{مساوات_طاقتی_تسلسل_ثبوت_عمودیت_پ} کے دائیں حصے پر غور کرتے ہیں۔مساوات \حوالہ{مساوات_طاقتی_سٹیورم_لیوویل_مساوات_ب}-الف کے تحت
\begin{align*}
k_1y_n(a)+k_2y_n'(a)&=0\\
k_1y_m(a)+k_2y_m'(a)&=0
\end{align*}
ہو گا۔فرض کریں کہ \عددی{k_2 \ne 0} ہے۔یوں پہلی مساوات کو \عددی{y_m(a)} اور دوسری مساوات کو \عددی{-y_n(a)} سے ضرب دے کر ان کا مجموعہ لیتے ہیں۔
\begin{align*}
k_2[y_n'(a)y_m(a)-y_m'(a)y_n(a)]=0
\end{align*} 
اب چونکہ \عددی{k_2 \ne 0} ہے لہٰذا قوسین میں بند تفاعل صفر کے برابر ہو گا۔اب قوسین میں بند تفاعل عین مساوات \حوالہ{مساوات_طاقتی_تسلسل_ثبوت_عمودیت_پ} کے دائیں حصے میں قوسین میں بند حصہ ہے لہٰذا مساوات \حوالہ{مساوات_طاقتی_تسلسل_ثبوت_عمودیت_پ} صفر کے برابر ہو گا اور یوں مساوات \حوالہ{مساوات_طاقتی_تسلسل_ثبوت_عمودیت_ب} سے  مساوات \حوالہ{مساوات_طاقتی_تسلسل_ثبوت_عمودیت_ت} ملتی ہے۔

تیسری صورت: اگر \عددی{r(a)=0} لیکن \عددی{r(b)\ne 0} ہو تب بالکل دوسری صورت کی طرح مساوات \حوالہ{مساوات_طاقتی_تسلسل_ثبوت_عمودیت_ت} حاصل کی جا سکتی ہے۔

چوتھی صورت: اگر \عددی{r(a)\ne 0} اور \عددی{r(b)\ne 0} ہوں تب مساوات \حوالہ{مساوات_طاقتی_سٹیورم_لیوویل_مساوات_ب} کے دونوں شرائط استعمال کرتے ہوئے  دوسری اور تیسری صورت کی طرز پر  مساوات \حوالہ{مساوات_طاقتی_تسلسل_ثبوت_عمودیت_ت} حاصل ہو گی۔

پانچویں صورت: اگر \عددی{r(a)=r(b)} ہو تب مساوات \حوالہ{مساوات_طاقتی_تسلسل_ثبوت_عمودیت_پ}  درج ذیل صورت اختیار کرے گی
\begin{align*}
r(b)[y_n'(b)y_m(b)-y_m'(b)y_n(b)-y_n'(a)y_m(a)+y_m'(a)y_n(a)]
\end{align*}
جو پہلی کی طرح  مساوات \حوالہ{مساوات_طاقتی_سٹیورم_لیوویل_مساوات_ب} کے استعمال سے صفر کے برابر ثابت ہوتا ہے۔یہاں ہم دیکھ سکتے ہیں کہ مساوات \حوالہ{مساوات_طاقتی_تسلسل_ثبوت_عمودیت_الف} کی مدد سے بھی درج بالا صفر کے برابر ثابت ہوتی ہے لہٰذا ہم مساوات \حوالہ{مساوات_طاقتی_سٹیورم_لیوویل_مساوات_ب} کی جگہ مساوات \حوالہ{مساوات_طاقتی_تسلسل_ثبوت_عمودیت_الف} کی شرط استعمال کر سکتے ہیں۔یوں مساوات \حوالہ{مساوات_طاقتی_تسلسل_ثبوت_عمودیت_ب} سے مساوات \حوالہ{مساوات_طاقتی_تسلسل_ثبوت_عمودیت_ت} ملتی ہے اور مسئلے کا ثبوت مکمل ہوتا ہے۔
\انتہا{ثبوت}
%======================
\ابتدا{مثال}\شناخت{مثال_طاقتی_تسلسل_سٹیورم_سادہ_ترین}
مثال \حوالہ{مثال_طاقتی_تسلسل_مثال_سٹیورم_الف} کے تفرقی مساوات کو مساوات \حوالہ{مساوات_طاقتی_سٹیورم_لیوویل_مساوات_الف} کے طرز پر لکھتے ہوئے \عددی{r=1}، \عددی{q=0} اور \عددی{p=1} ملتے ہیں۔مسئلہ \حوالہ{مسئلہ_طاقتی_تسلسل_قائمیت_سٹیورم} کے تحت وقفہ \عددی{0\le x\le \pi} پر اس کے آئگنی تفاعل قائمہ الزاویہ ہوں گے۔
\انتہا{مثال}
%========================
\ابتدا{مثال}\شناخت{مثال_طاقتی_تسلسل_فوریئر_عمومی}\quad فوریئر تسلسل\\
آپ ثابت کر سکتے ہیں کہ مثال \حوالہ{مثال_طاقتی_فوریئر_تسلسل_عمومی_الف} میں پائے جانے والے درج ذیل تفاعل 
\begin{align*}
1,\, \cos x , \, \sin x,\, \cos 2x,\, \sin 2x, \cdots
\end{align*}
درج ذیل سٹیورم لیوویل مسئلے کے آئگنی تفاعل ہیں
\begin{align*}
y''+\lambda y=0, \quad y(\pi)=y(-\pi), \quad y'(\pi)=y'(-\pi)
\end{align*}
 لہٰذا مسئلہ \حوالہ{مسئلہ_طاقتی_تسلسل_قائمیت_سٹیورم} کے تحت وقفہ \عددی{-\pi \le x \le \pi} پر  یہ آپس میں قائمہ الزاویہ سلسلہ دیتے ہیں۔اس مثال کے سرحدی شرائط مساوات \حوالہ{مساوات_طاقتی_تسلسل_ثبوت_عمودیت_الف} کی طرز کے ہیں۔
\انتہا{مثال}
%==========================

ایسی عمومی فوریئر تسلسل جس میں (قائمہ الزاویہ) آئگنی تفاعل کا سلسلہ استعمال ہو \اصطلاح{آئگنی تفاعل پھیلاو}\فرہنگ{آئگنی تفاعل پھیلاو}\حاشیہب{eigenfunction expansion}\فرہنگ{eigenfunction expansion} کہلاتی ہے۔ 

%=====================================
\ابتدا{مسئلہ}\شناخت{مسئلہ_طاقتی_حقیقی_آئگنی_قیمت}\quad حقیقی آئگنی اقدار\\
اگر سٹیورم لیوویل مسئلہ  جسے مساوات \حوالہ{مساوات_طاقتی_سٹیورم_لیوویل_مساوات_الف} اور مساوات \حوالہ{مساوات_طاقتی_سٹیورم_لیوویل_مساوات_ب} میں پیش کیا گیا ہے، مسئلہ \حوالہ{مسئلہ_طاقتی_تسلسل_قائمیت_سٹیورم} کے شرائط پر پورا اترتا ہو اور پورے وقفہ \عددی{a\le x \le b} پر \عددی{p} مثبت ہو (یا اس پورے وقفے پر \عددی{p}  منفی ہو) تب اس سٹیورم لیوویل مسئلے کے تمام آئگنی اقدار حقیقی ہوں گی۔
\انتہا{مسئلہ}
%===============================

\ابتدا{ثبوت}
فرض کریں کہ اس سٹیورم لیوویل مسئلے کا \عددی{\lambda=\alpha+i\beta}  آئگنی قدر ہے جس کا مطابقتی آئگنی تفاعل درج ذیل ہے جہاں \عددی{\alpha}، \عددی{\beta}، \عددی{u} اور \عددی{v} حقیقی ہیں۔
\begin{align}
y(x)=u(x)+iv(x)
\end{align} 
اس کو مساوات \حوالہ{مساوات_طاقتی_سٹیورم_لیوویل_مساوات_الف} میں پر کرتے ہوئے
\begin{align*}
(ru'+irv')'+(q+\alpha p+i\beta p)(u+iv)=0
\end{align*}
ملتا ہے جس کے حقیقی اور  خیالی حصوں کو علیحدہ علیحدہ کرتے ہوئے درج ذیل دو مساوات ملتے ہیں۔
\begin{align*}
(ru')'+(q+\alpha p)u-\beta p v&=0\\
(rv')'+(q+\alpha p)v-\beta p u&=0
\end{align*}
پہلی مساوات کو \عددی{v} اور دوسری مساوات کو \عددی{-u} سے ضرب دے کر مجموعہ لیتے ہیں
\begin{align*}
-\beta(u^2+v^2)p&=u(rv')'-v(ru')'\\
&=[(rv')u-(ru')v]'
\end{align*}
جس کا \عددی{x=a} تا \عددی{x=b} تکمل درج ذیل ہے۔
\begin{align*}
-\beta \int_a^b (u^2+v^2)p\dif x=\left[r(uv'-u'v)\right]_a^b
\end{align*}
مسئلہ \حوالہ{مسئلہ_طاقتی_تسلسل_قائمیت_سٹیورم} کی ثبوت کی طرز پر، سرحدی شرائط استعمال کرتے ہوئے دایاں ہاتھ صفر کے برابر ملتا ہے۔چونکہ \عددی{y} آئگنی تفاعل ہے لہٰذا \عددی{u^2+v^2 \not \equiv 0} ہو گا۔اب \عددی{y} اور \عددی{p} استمراری ہیں اور پورے وقفے پر \عددی{p>0}  ہے (یا پورے وقفے پر \عددی{p<0}  ہے) لہٰذا تکمل کا بایاں ہاتھ صفر نہیں ہو سکتا ہے۔یوں \عددی{\beta =0} ہو گا لہٰذا  \عددی{\lambda=\alpha} حقیقی ہو گا۔یوں مسئلے کا ثبوت پورا ہوتا ہے۔
\انتہا{ثبوت}
%=============================

مثال \حوالہ{مثال_طاقتی_تسلسل_سٹیورم_سادہ_ترین} اور مثال \حوالہ{مثال_طاقتی_تسلسل_فوریئر_عمومی} کے آئگنی اقدار مسئلہ  \حوالہ{مسئلہ_طاقتی_حقیقی_آئگنی_قیمت}کے تحت حقیقی ہیں۔

%=================
\حصہء{سوالات}

%==========
\ابتدا{سوال}
مثال \حوالہ{مثال_طاقتی_تسلسل_مثال_سٹیورم_الف} کے لئے مسئلہ  \حوالہ{مسئلہ_طاقتی_تسلسل_قائمیت_سٹیورم} ثابت کریں۔
\انتہا{سوال}
%==========================
\ابتدا{سوال}
مسئلہ  \حوالہ{مسئلہ_طاقتی_تسلسل_قائمیت_سٹیورم} میں تیسری اور چوتھی صورت  کا ثبوت مکمل کریں۔
\انتہا{سوال}
%===========================
\ابتدا{سوال}
اگر مساوات \حوالہ{مساوات_طاقتی_سٹیورم_لیوویل_مساوات_الف} اور مساوات \حوالہ{مساوات_طاقتی_سٹیورم_لیوویل_مساوات_ب} میں دیے گئے مسئلے کی آئگنی قدر \عددی{\lambda_0} اور مطابقتی آئگنی تفاعل \عددی{y=y_0} ہوں تب ثابت کریں کہ \عددی{\lambda_0} کا مطابقتی آئگنی تفاعل \عددی{y=\alpha y_0} بھی ہو گا جہاں \عددی{\alpha} غیر صفر اختیاری مستقل ہے۔(اس خاصیت کو استعمال کرتے ہوئے ایسے آئگنی تفاعل دریافت کئے جا سکتے ہیں جن کا معیار اکائی ہو۔)
\انتہا{سوال}
%=========================== 
سوال \حوالہ{سوال_طاقتی_آئگنی_تفاعل_الف} تا سوال \حوالہ{سوال_طاقتی_آئگنی_تفاعل_ب} میں دیے گئے سٹیورم لیوویل مسئلوں کے آئگنی قدر اور آئگنی تفاعل دریافت کریں۔

%===============
\ابتدا{سوال}\شناخت{سوال_طاقتی_آئگنی_تفاعل_الف}\quad
$y''+\lambda y=0, \quad y(0)=0,\,y(l)=0$\\
جوابات:\عددی{y_n=\sin \tfrac{n\pi x}{l}}، \عددی{\lambda=\tfrac{n^2\pi^2}{l^2}} جہاں \عددی{n=1,2,\cdots} ہیں۔چونکہ \عددی{n=0} سے \عددی{y=0} ملتا ہے جو آئگنی تفاعل نہیں ہے لہٰذا \عددی{n=0} جواب میں شامل نہیں کیا جائے گا۔
\انتہا{سوال}
%========================
\ابتدا{سوال}\quad
$y''+\lambda y=0, \quad y(0)=0,\,y'(l)=0$\\
جوابات:\عددی{y_n=\sin \tfrac{(2n+1)\pi x}{2l}}، \عددی{\lambda=\left[\tfrac{(2n+1)\pi}{2l}\right]^2} جہاں \عددی{n=0,1,\cdots} ہیں۔

\انتہا{سوال}
%=======================
\ابتدا{سوال}\quad
$y''+\lambda y=0, \quad y'(0)=0,\,y(l)=0$\\
جوابات:\عددی{y_n=\cos \tfrac{(2n+1)\pi x}{2l}}، \عددی{\lambda=\left[\tfrac{(2n+1)\pi}{2l}\right]^2} جہاں \عددی{n=0,1,\cdots} ہیں۔

\انتہا{سوال}
%=======================
\ابتدا{سوال}\quad
$y''+\lambda y=0, \quad y'(0)=0,\,y'(l)=0$\\
جوابات:\عددی{y_n=\cos \tfrac{n\pi x}{l}}، \عددی{\lambda=\left[\tfrac{n(2n+1)\pi}{l}\right]^2} جہاں \عددی{n=0,1,\cdots} ہیں۔

\انتہا{سوال}
%=======================
\ابتدا{سوال}\quad
$y''+\lambda y=0, \quad y(0)=y(2\pi),\,y'(0)=y'(2\pi)$\\
جوابات:\عددی{y_n=\cos nx}، \عددی{\lambda=n^2} جہاں \عددی{n=0,1,\cdots} ہیں۔

\انتہا{سوال}
%=======================
\ابتدا{سوال}\quad
$(xy')'+\lambda x^{-1}y=0,\quad y(1)=0,\, y(e)=0$\\
جوابات:\عددی{y_n=\sin(n\pi\ln \abs{x})}، \عددی{\lambda=n^2\pi^2} جہاں \عددی{n=1,2,\cdots} ہے۔
\انتہا{سوال}
%==========================
\ابتدا{سوال}\quad
$(e^{2x}y')'+e^{2x}(\lambda+1)y=0,\quad y(0)=0,\, y(\pi)=0$\\
جوابات:\عددی{y_n=e^{-x}\sin nx}، \عددی{\lambda=n^2} جہاں \عددی{n=1,2,\cdots} ہے۔
\انتہا{سوال}
%======================
\ابتدا{سوال}\شناخت{سوال_طاقتی_آئگنی_تفاعل_ب}
ثابت کریں کہ مسئلہ سٹیورم لیوویل
\begin{align*}
y''+\lambda y=0, \quad y(0)=0,\, y(1)+y'(1)=0
\end{align*} 
کے حل مساوات \عددی{\sin \sqrt{\lambda}+\sqrt{\lambda}\cos \sqrt{\lambda}=0} سے حاصل کیے جاتے ہیں۔اس مساوات کے کتنے حل ممکن ہیں۔

جواب: لاتعداد
\انتہا{سوال}
%=======================
\ابتدا{سوال}
ایسا سٹیورم لیوویل مسئلہ دریافت کریں جس کے آئگنی تفاعل درج ذیل ہوں۔
\begin{align*}
1,\, \cos x,\, \cos 2x,\, \cos 3x\, \cdots
\end{align*}

جواب:
$y''+\lambda y=0, \quad y'(0)=0, \, y'(\pi)=0$
\انتہا{سوال}
%============================

\حصہ{قائمیت لیژانڈر کثیر رکنی اور بیسل تفاعل}
لیژانڈر  مساوات (مساوات \حوالہ{مساوات_بیسل_لیژانڈر_الف}) کو درج ذیل لکھا جا سکتا ہے
\begin{align}
[(1-x^2)y']'+\lambda y=0,\quad \lambda=n(n+1)
\end{align}
لہٰذا یہ  مساوات سٹیورم لیوویل (حصہ \حوالہ{مسئلہ_طاقتی_سٹیورم_لیوویل}) ہے جہاں \عددی{r=1-x^2}، \عددی{q=0} اور \عددی{p=1} ہیں۔ چونکہ \عددی{x=\mp 1} پر \عددی{r=0} ہے لہٰذا  سرحدی شرائط کے بغیر وقفہ \عددی{-1\le x \le 1} پر اس سے مسئلہ سٹیورم لیوویل حاصل ہوتا ہے۔  ہم جانتے ہیں کہ \عددی{n=0,1,2,\cdots} پر اس مسئلے کے حل لیژانڈر کثیر رکنی \عددی{P_n(x)} ہیں لہٰذا یہ آئگنی تفاعل ہیں جو مسئلہ \حوالہ{مسئلہ_طاقتی_تسلسل_قائمیت_سٹیورم} کے تحت  قائمہ الزاویہ ہوں گے یعنی
\begin{align}
\int_{-1}^{1}P_m(x)P_n(x)\dif x=0\quad \quad (m \ne n)
\end{align}
اور ان آئگنی تفاعل کا معیار مساوات \حوالہ{مساوات_طاقتی_لیژانڈر_معیار} دیتی ہے جسے یہاں دوبارہ پیش کرتے ہیں۔
\begin{align}
\norm{P_m}=\sqrt{\int_{-1}^{1} P_m^2(x) \dif x}=\sqrt{\frac{2}{2m+1}}\quad \quad m=0,1,\cdots
\end{align}

بیسل تفاعل (حصہ \حوالہ{حصہ_طاقتی_بیسل_تفاعل})  جو مساوات بیسل (مساوات  \حوالہ{مساوات_بیسل_الف}) پر پورا اترتے ہیں کے  اہم انجینئری استعمال پائے جاتے ہیں مثلاً دائری سطح کی ارتعاش جس پر اس کتاب میں غور کیا جائے گا۔ مساوات بیسل کو یہاں دوبارہ پیش کرتے ہیں
\begin{align*}
s^2\ddot{J}_n+s\dot{J}_n+(s^2-n^2)J_n=0
\end{align*}
جہاں تفاعل کا \عددی{s} کے ساتھ تفرق کو نقطہ ظاہر کرتا ہے۔ہم فرض کرتے ہیں کہ \عددی{n} غیر منفی عدد صحیح ہے۔ \عددی{s=\lambda x} لیتے ہوئے جہاں \عددی{\lambda} غیر صفر مستقل ہے ہم \عددی{\tfrac{\dif x}{\dif s}=\tfrac{1}{\lambda}} اور زنجیری تفرق سے درج ذیل لکھ سکتے ہیں جہاں \عددی{'} سے مراد \عددی{x} کے ساتھ تفرق ہے۔
\begin{align*}
\dot{J}_n=\frac{J_n'}{\lambda}, \quad \ddot{J}_n=\frac{J_n''}{\lambda^2}
\end{align*}
انہیں مساوات بیسل میں پر کر کے
\begin{align*}
x^2J_n''(\lambda x)+xJ_n'(\lambda x)+(\lambda^2 x^2-n^2)J_n(\lambda x)=0
\end{align*}
ملتا ہے جس کو \عددی{x} سے تقسیم کر کے درج ذیل لکھا جا سکتا ہے
\begin{align}\label{مساوات_طاقتی_بیسل_سٹیورم_الف}
[xJ_n'(\lambda x)]'+\left(-\frac{n^2}{x}+\lambda^2 x\right)J_n(\lambda x)=0
\end{align}
جو \عددی{n} کی ہر معین قیمت کے لئے ایک مساوات سٹیورم لیوویل دیتا ہے جہاں مقدار معلوم کو \عددی{\lambda} کی بجائے \عددی{\lambda^2} لکھا گیا ہے اور 
\begin{align*}
p(x)=x,\quad q(x)=-\frac{n^2}{x},\quad r(x)=x
\end{align*}
ہیں۔چونکہ \عددی{x=0} پر \عددی{r(x)=0} ہے لہٰذا مسئلہ \حوالہ{مسئلہ_طاقتی_تسلسل_قائمیت_سٹیورم} کے تحت وقفہ \عددی{0\le x \le R} پر مساوات \حوالہ{مساوات_طاقتی_بیسل_سٹیورم_الف} کے وہ حل جو درج ذیل سرحدی شرط پر پورا اترتے ہوں تفاعل قدر \عددی{p(x)=x} کے لحاض سے  قائمہ الزاویہ سلسلہ دیں گے۔(یہاں دھیان رہے کہ \عددی{n \ne 0} کی صورت میں تفاعل \عددی{q} نقطہ \عددی{x=0} پر غیر استمراری ہے البتہ اس کا مسئلہ \حوالہ{مسئلہ_طاقتی_تسلسل_قائمیت_سٹیورم} کے ثبوت پر کوئی اثر نہیں پایا جاتا ہے۔)
\begin{align}\label{مساوات_طاقتی_بیسل_سٹیورم_ب}
J_n(\lambda R)=0
\end{align}
یہ ثابت کیا جا سکتا ہے کہ \عددی{J_n(s)} کے لامحدود تعداد کے حقیقی صفر پائے جاتے ہیں۔ \عددی{J_n(s)} کے مثبت صفروں کو \عددی{\alpha_{1n}<\alpha_{2n}<\alpha_{3n}\cdots}، سے ظاہر کرتے ہیں۔یوں مساوات \حوالہ{مساوات_طاقتی_بیسل_سٹیورم_ب} کی شرط تب پوری ہو گی جب
\begin{align}
\lambda R=\alpha_{mn} \quad \implies \quad \lambda=\lambda_{mn}=\frac{\alpha_{mn}}{R} \quad \quad (m=1,2,\cdots)
\end{align}
ہو جس سے درج ذیل مسئلہ ملتا ہے۔

%=================
\ابتدا{مسئلہ}\quad بیسل تفاعل کی قائمیت\\

\انتہا{مسئلہ}
%==============================
