\باب{مخلوط تکملات}
مخلوط تکملات دو وجوہات کی بنا اہم ہیں۔عملی وجہ یہ ہے کہ حقیقی تکملات حل کرنے کی تراکیب سے کئی حقیقی تکملات حل کرنا ناممکن ہے جبکہ ان کو مخلوط تکملات کی ترکیب سے حل کیا جا سکتا ہے۔دوسری وجہ نظریاتی ہے۔ جہاں مخلوط تکملات کی ترکیب سے تحلیلی تفاعل کی چند بنیادی خصوصیات دریافت ہوتی ہیں (بالخصوص بلند درجی تفرق کی موجودگی) جن کا ثبوت  تکمل استعمال کیے بغیر انتہائی مشکل ہو گا۔یہ صورت حال حقیقی اور مخلوط علم الاحصاء میں بنیادی فرق کی نشاندہی کرتی ہے۔

اس باب میں ہم پہلے مخلوط تکملات کی تعریف پیش کرتے ہیں۔سب سے بنیادی نتیجہ  کوشی مخلوط تکمل کا مسئلہ حاصل ہو گا جس سے  سے کوشی تکمل کی کلیات حاصل ہوں گی جو بہت اہم  ہیں۔ ہم ثابت کریں گے کہ اگر کوئی تفاعل تحلیلی ہو تب اس کے ہر درجہ کے تفرق موجود ہوں گے۔اس نقطہ نظر سے مخلوط تحلیلی تفاعل حقیقی متغیر کی حقیقی تفاعل سے زیادہ سادہ رویہ رکھتے ہیں۔

%=====================
\حصہ{مخلوط مستوی میں خطی تکمل}\شناخت{حصہ_مخلوط_تکمل_مخلوط_مستوی_میں_خطی_تکمل}
حقیقی علم الاحصاء کی طرح ہم قطعی تکمل اور غیر قطعی تکمل میں تمیز کرتے ہیں۔ایک غیر قطعی تکمل ایسا تفاعل ہوتا ہے جس کا تفرق خطے میں دیا گیا تحلیلی تفاعل ہو گا۔تفاعل کی تفرق کو الٹ لکھتے ہوئے ہم کئی غیر قطعی تکمل دریافت کر سکتے ہیں۔

آئیں اب مخلوط تفاعل \عددی{f(z)}، جہاں \عددی{z=x+iy} ہے، کی قطعی تکمل یا خطی تکمل کی تعریف پیش کرتے ہیں۔ہم دیکھیں گے کہ حقیقی قطعی تکمل کی تصور کو وسعت دیتے ہوئے  مخلوط قطعی تکمل کا تصور پیدا ہوتا ہے۔یوں موجودہ بحث عین حصہ \حوالہ{حصہ_سمتی_تکمل_خطی_تکمل} کی طرح ہو گی۔قطعی تکمل کی صورت میں حقیقی محور پر کوئی وقفہ تکمل کی راہ  ہو گی۔مخلوط قطعی تکمل کی صورت میں ہم مخلوط مستوی پر کسی منحنی\حاشیہد{درحقیقت منحنی کے کسی حصے یا قوس پر تکمل لیا جائے گا۔اپنی آسانی کی خاطر ہم  "منحنی" کی اصطلاح کو  پوری منحنی کے لئے اور منحنی کے چھوٹے حصہ کے لئے بھی استعمال کریں گے۔} پر چلتے ہوئے تکمل حاصل کریں گے۔

فرض کریں کہ  مخلوط \عددی{z} مستوی میں \عددی{C} ایک ہموار منحنی (حصہ \حوالہ{حصہ_نقش_محافظ_زاویہ_نقش}) ہے۔تب ہم \عددی{C} کو درج ذیل روپ میں لکھ سکتے ہیں
\begin{align}\label{مساوات_مخلوط_تکمل_راہ_الف}
z(t)=x(t)+iy(t)\quad \quad \quad (a\le t\le b)
\end{align}
جہاں تمام \عددی{t} کے لئے \عددی{z(t)} کا استمراری تفرق \عددی{\dot{z}(t)\ne 0} پایا جاتا ہے، اور یوں \عددی{C} قابل تصحیح (حصہ \حوالہ{حصہ_الاحصاء_لمبائی_قوس}) ہو گی جس کا ہر نقطہ پر یکتا مماس ہو گا۔آپ کو یاد ہو گا کہ \عددی{C} پر مثبت رخ سے مراد  \عددی{t} کی بڑھتی قیمت کا مطابقتی رخ  ہے۔

فرض کریں کہ  \عددی{f(z)} ایک استمراری تفاعل ہے جو (کم از کم)  \عددی{C} کی ہر نقطہ پر معین ہے۔ہم  مساوات \حوالہ{مساوات_مخلوط_تکمل_راہ_الف} میں دیے گئے وقفہ \عددی{a\le t\le b} کو درج ذیل ٹکڑوں میں تقسیم کرتے ہیں
\begin{align*}
t_0(=a),t_1,\cdots,t_{n-1},t_n(=b)
\end{align*} 
جہاں \عددی{t_0<t_1<\cdots<t_n} ہے۔اس کے مطابق \عددی{C} کے ٹکڑے (شکل \حوالہ{شکل_مخلوط_تکمل_ٹکڑے_راہ})
\begin{align*}
z_0,z_1,\cdots,z_{n-1},z_n()=Z
\end{align*}
 پائے جاتے ہیں جہاں \عددی{z_j=z(t_j)} ہے۔
\begin{figure}
\centering
\begin{tikzpicture}
%\draw[thick](0,0) grid (3,2);
%\draw[thin,gray,step=0.1] (0,0) grid (3,2);
\draw(0,0) node[ocirc]{}node[right]{$z_0$} to [out=135,in=-135] coordinate[pos=0.3](kA)coordinate[pos=0.5](kB) coordinate[pos=0.7](kC)coordinate[pos=0.9](kD)  (0.5,1.5) to [out=45,in=80] coordinate[pos=0.2](kE)coordinate[pos=0.5](kF)coordinate[pos=0.7](kG) coordinate[pos=0.8](kH)coordinate[pos=0.9](kI)coordinate[pos=1,ocirc](kJ)node[pos=1,below]{$Z$}(3,1);
\draw(kA)++(0:0.1)--++(0:-0.2)node[left]{$z_1$};
\draw(kB)++(-10:0.1)--++(-10:-0.2)node[left]{$z_2$};
\draw(kC)++(-40:0.1)--++(-40:-0.2)++(-40:-0.25)coordinate(kkA);
\draw(kD)++(-40:0.1)--++(-40:-0.2)++(-40:-0.25)coordinate(kkB);
\draw($(kkA)!0.5!(kkB)$)node[rotate=50]{$\cdots$};
\draw(kE)++(-60:0.1)--++(-60:-0.2)node[above]{$z_{m-1}$};
\draw(kF)node[ocirc]{}node[above]{$\zeta_m$};
\draw(kG)++(-115:0.1)--++(-115:-0.2)node[above]{$z_m$};
\draw(kH)++(-130:0.1)--++(-130:-0.2);
\draw(kI)++(-170:0.1)--++(-170:-0.2);
\draw(kE)--(kG)node[pos=0.5,below]{$\abs{\Delta z_m}$};
\end{tikzpicture}
\caption{مخلوط خطی تکمل}
\label{شکل_مخلوط_تکمل_ٹکڑے_راہ}
\end{figure}
ہم \عددی{C} کے ہر ٹکڑے پر کوئی اختیاری نقطہ منتخب کرتے ہیں، مثلاً ہم \عددی{z_0} اور \عددی{z_1} کے درمیان نقطہ \عددی{\zeta_1} منتخب کرتے ہیں 
(یعنی \عددی{\zeta_1=z(t)} جہاں \عددی{t_0\le t\le t_1} ہے) اور \عددی{z_1} اور \عددی{z_2} کے درمیان نقطہ \عددی{\zeta_2} منتخب کرتے ہیں، وغیرہ وغیرہ۔ہم اب مجموعہ
\begin{align}\label{مساوات_مخلوط_تکمل_راہ_ب}
S_n=\sum_{m=1}^{n}f(\zeta_m)\Delta z_m
\end{align}
لیتے ہیں جہاں
\begin{align*}
\Delta z_m=z_m-z_{m-1}
\end{align*}
ہے۔ہم ایسے مجموعے \عددی{n=2,3,\cdots} کے لئے مکمل بے قاعدگی سے حاصل کرتے ہیں پس اتنا دھیان رکھتے ہیں کہ جب \عددی{n} لامتناہی کے قریب پہنچے تب \عددی{\abs{\Delta z_m}} کی زیادہ سے زیادہ قیمت صفر کے قریب پہنچتی ہو۔یوں ہمیں مخلوط قیمتوں کا سلسلہ \عددی{S_2,S_3,\cdots} ملتا ہے۔اس سلسلے کی حد، راہ \عددی{C} پر  \عددی{f(z)} کا \اصطلاح{خطی تکمل}\فرہنگ{تکمل!خطی}\حاشیہب{line integral}\فرہنگ{integral!line} (یا صرف \ترچھا{تکمل}) کہلاتا ہے جس کو درج ذیل لکھا جاتا ہے۔
\begin{align}\label{مساوات_مخلوط_تکمل_راہ_پ}
\int\limits_C f(z)\dif z
\end{align}
منحنی \عددی{C} کو تکمل کی راہ کہتے ہیں۔

ہم درج ذیل پوری بحث میں فرض کرتے ہیں کہ مخلوط خطی تکمل کی تمام راہ \موٹا{ٹکڑوں میں ہموار} ہیں یعنی ہر راہ محدود تعداد کی ہموار منحنیات پر مشتمل ہے۔

ہمارے مفروضوں کی مد نظر  خطی تکمل مساوات \حوالہ{مساوات_مخلوط_تکمل_راہ_پ} موجود ہو گا، بلکہ \عددی{f(z)=u(x,y)+iv(x,y)} لکھتے ہوئے اور
\begin{align*}
\zeta_m=\xi_m+i\eta_m\quad \text{اور}\quad \Delta z_m=\Delta x_m+i\Delta y_m
\end{align*}
لیتے ہوئے مساوات \حوالہ{مساوات_مخلوط_تکمل_راہ_ب} کو 
\begin{align}\label{مساوات_مخلوط_تکمل_راہ_ت}
S_n=\sum(u+iv)(\Delta x_m+i\Delta y_m)
\end{align}
لکھا جا سکتا ہے جہاں \عددی{u=u(\zeta_m,\xi_m)} اور \عددی{v=v(\zeta_m,\xi_m)} ہیں اور ہم \عددی{m} کو \عددی{1} تا \عددی{n} لیتے ہوئے مجموعہ حاصل کرتے ہیں۔ہم اب \عددی{S_n} کو چار مجموعوں میں تقسیم کر سکتے ہیں۔
\begin{align*}
S_n=\sum u\Delta x_m-\sum v\Delta y_m+i[\sum u\Delta y_m+\sum v\Delta x_m]
\end{align*}
یہ مجموعے حقیقی ہیں۔چونکہ \عددی{f} استمراری ہے لہٰذا \عددی{u} اور \عددی{v} بھی استمراری ہوں گے۔یوں اگر ہم \عددی{n} کی قیمت کو متذکرہ بالا طریقے سے بڑھا کر لامتناہی کے قریب کریں تب \عددی{\Delta x_m} اور \عددی{\Delta y_m} کی زیادہ سے زیادہ قیمت صفر کے قریب ہو  گی اور دائیں ہاتھ ہر مجموعہ حقیقی تکمل کی صورت اختیار کرے گا:
\begin{align}\label{مساوات_مخلوط_تکمل_راہ_ٹ}
\lim_{n\to \infty} S_n =\int\limits_C f(z)\dif z=\int\limits_C u\dif x-\int\limits_C  v\dif y+i\big[\int\limits_C u\dif y+\int\limits_C v\dif x\big]
\end{align} 
اس سے ظاہر ہوتا ہے کہ خطی تکمل مساوات \حوالہ{مساوات_مخلوط_تکمل_راہ_پ} موجود ہو گا اور اس تکمل کی قیمت پر راہ ٹکڑے کرنے کی ترکیب اور ہر ٹکڑے کے بیچ نقطہ \عددی{\zeta_m} کی انتخاب کا کوئی اثر نہیں ہو گا۔

مزید، حصہ \حوالہ{حصہ_خطی_تکمل_کا_حل} کی طرح، منحنی \عددی{C} کی مساوات \حوالہ{مساوات_مخلوط_تکمل_راہ_الف} استعمال کرتے ہوئے ہم ان میں سے ہر حقیقی تکمل کو قطعی تکمل میں تبدیل کر سکتے ہیں:
\begin{align}\label{مساوات_مخلوط_تکمل_راہ_ث}
\int\limits_C f(z)\dif z=\int\limits_a^b u \dot{x}\dif t-\int\limits_a^b v\dot{y}\dif t+i\big[\int\limits_a^b u \dot{y} \dif t+\int\limits_a^b v\dot{x}\dif t\big]
\end{align}
جہاں \عددی{u=u[x(t),y(t)]}، \عددی{v=v[x(t),y(t)]} ہیں جبکہ \عددی{t} کے ساتھ تفرق کو نقطہ سے ظاہر کیا گیا ہے۔

ہم اس کو عموماً
\begin{align*}
\int\limits_C f(z)\dif z=\int_a^b (u+iv)(\dot{x}+i\dot{y})\dif t
\end{align*}
یا مختصراً
\begin{align}\label{مساوات_مخلوط_تکمل_راہ_ج}
\int\limits_C f(z)\dif z=\int\limits_a^b f[z(t)]\dot{z}(t)\dif t
\end{align}
لکھتے ہیں۔

آئیں چند سادہ مثالیں دیکھیں۔

%============================
\ابتدا{مثال}\شناخت{مثال_مخلوط_تکمل_دائرے_پر_تکمل}\quad \موٹا{اکائی دائرے پر \عددی{\tfrac{1}{z}} کا تکمل}\\
اکائی دائرہ \عددی{C} پر گھڑی کی الٹ رخ \عددی{z=1} سے شروع کر کے ایک چکر لگاتے ہوئے \عددی{f(z)=\tfrac{1}{z}} کا تکمل حاصل کریں۔ہم \عددی{C} کو درج ذیل روپ میں لکھ سکتے ہیں۔
\begin{align}\label{مساوات_مخلوط_تکمل_اکائی_رداس_الف}
z(t)=\cos t+i\sin t\quad \quad \quad (0\le t\le 2\pi)
\end{align}
یوں
\begin{align*}
\dot{z}(t)=-\sin t+i\cos t
\end{align*}
ہو گا لہٰذا مساوات \حوالہ{مساوات_مخلوط_تکمل_راہ_ج} کے تحت درکار تکمل
\begin{align*}
\int_C\frac{\dif z}{z}=\int_0^{2\pi} \frac{1}{\cos t+i\sin t} (-\sin t+i\cos t)\dif t=i\int_0^{2\pi} \dif t=i2\pi
\end{align*}
ہو گا۔یہ بنیادی نتیجہ ہے جو ہم بار بار استعمال کریں گے۔

ظاہر ہے کہ ہم مساوات \حوالہ{مساوات_مخلوط_تکمل_اکائی_رداس_الف} کو مختصراً
\begin{align}\label{مساوات_مخلوط_تکمل_اکائی_رداس_ب}
z(t)=e^{it}\quad \quad \quad (0\le t\le 2\pi)
\end{align}
لکھ سکتے ہیں۔یوں تفرق لیتے ہوئے
\begin{align*}
\dot{z}(t)=ie^{it},\quad \dif z=ie^{it}\dif t
\end{align*}
لکھ کر یہی نتیجہ
\begin{align}
\int_C \frac{\dif z}{z}=\int_0^{2\pi} \frac{1}{e^{it}}ie^{it}\dif t=i\int_0^{2\pi}\dif t=i2\pi
\end{align}
دوبارہ حاصل کرتے ہیں۔ 
\انتہا{مثال}
%=============================
\ابتدا{مثال}\شناخت{مثال_مخلوط_تکمل_غیر_تحلیلی}\quad \موٹا{غیر تحلیلی تفاعل کا تکمل}\\
سیدھی راہ \عددی{C_1} پر \عددی{z_0=0} تا \عددی{z=1+i} تفاعل \عددی{f(z)=\text{\RL{حقیقی $z\,$}}=x} کا تکمل تلاش کریں 
(شکل \حوالہ{شکل_مثال_مخلوط_تکمل_غیر_تحلیلی}-الف)۔\\
اس راہ کو درج ذیل روپ میں لکھا جا سکتا ہے۔
\begin{align*}
z(t)=x(t)+iy(t)=(1+i)t\quad \quad (0\le t\le 1)
\end{align*} 
یوں 
\begin{align*}
f[z(t)]=\text{\RL{حقیقی $z\,$}}=x(t)=t,\quad \dif z=(1+i)\dif t
\end{align*}
ہو گا جس سے ہم تکمل حاصل کرتے ہیں:
\begin{align*}
\int_{C_1} \text{\RL{حقیقی $z\,$}}\dif z=\int_0^1 t(1+i)\dif t=(1+i)\int_0^1 t\dif t=\frac{1}{2}(1+i)
\end{align*}

آئیں اب حقیقی محور پر \عددی{0} تا \عددی{1} چل کر، یہاں سے خیالی محور کے متوازی چلتے ہوئے  \عددی{1+i} تک اسی تفاعل \عددی{f(z)=\text{\RL{حقیقی $z\,$}}=x} کا تکمل حاصل کرتے ہیں (شکل \حوالہ{شکل_مثال_مخلوط_تکمل_غیر_تحلیلی}-الف میں راہ \عددی{C_2})۔ ہم اس راہ  کے پہلے حصے کو
\begin{align*}
z=z(t)=t\quad \quad\quad (0\le t\le 1)
\end{align*}
اور دوسرے حصے کو
\begin{align*}
z(t)=1+i(t-1)\quad \quad \quad (1\le t\le 2)
\end{align*}
لکھ سکتے ہیں۔یوں پوری راہ وقفہ \عددی{0\le t\le 2} کی مطابقتی ہو گی۔پہلے حصے پر \عددی{\text{\RL{حقیقی $z\,$}}=t,\,\dif z=\dif t} اور دوسرے حصے پر \عددی{\text{\RL{حقیقی $z\,$}}=1,\,\dif z=i\dif t} ہو گا۔یوں پورا تکمل دو ٹکڑوں میں حاصل ہو گا:
\begin{align*}
\int_{C_2} \text{\RL{}}\dif z=\int_0^1t\dif t+\int_1^2 i\dif t=\frac{1}{2}+i
\end{align*}
آپ دیکھ سکتے ہیں کہ راہ کے دوسرے حصے کو درج ذیل بھی لکھا جا سکتا ہے
\begin{align*}
z(t)=1+it\quad\quad\quad(0\le t\le 1)
\end{align*}
جس کو استعمال کرتے ہوئے تکمل کے حدود \عددی{0} اور \عددی{1} ہوں گے اور تکمل کی قیمت وہی رہے گی۔

اس مثال سے آپ دیکھ سکتے ہیں کہ غیر تحلیلی تفاعل کے تکمل کی قیمت  نا صرف راہ کی آخری حدود بلکہ راہ کی جیومیٹریائی شکل  پر بھی منحصر ہوتی ہے۔
\begin{figure}
\centering
\begin{subfigure}{0.5\textwidth}
\centering
\begin{tikzpicture}
\draw(0,0)--(2.5,0)node[below]{$x$};
\draw(0,0)--(0,1.75)node[left]{$y$};
\draw[thick,dashed,->-=0.5](0,0)--(1.5,1.5)node[pos=0.6,above left]{$C_1$};
\draw[thick,->-=0.5](0,0)--(1.5,0)node[below]{$1$}node[pos=0.5,below]{$C_2$};
\draw[thick,->-=0.5](1.5,0)--(1.5,1.5)node[ocirc]{}node[right]{$z=1+i$};
\end{tikzpicture}
\caption*{(الف) مثال \حوالہ{مثال_مخلوط_تکمل_غیر_تحلیلی} میں تکمل کی راہ}
\end{subfigure}%
\begin{subfigure}{0.5\textwidth}
\centering
\begin{tikzpicture}
\draw(0,0)--(2,0)node[below]{$x$};
\draw(0,0)--(0,1.75)node[left]{$y$};
\draw[->-=0.125](1,0.825) circle (0.75);
\draw[-latex](1,0.825)node[ocirc]{}node[below]{$z_0$}--++(135:0.75)node[pos=0.4,shift={(45:0.25)}]{$\rho$};
\draw(1,0.825)++(42:0.95)node[]{$C$};
\end{tikzpicture}
\caption*{(ب) مثال \حوالہ{مثال_مخلوط_تکمل_طاقت} میں تکمل کی راہ}
\end{subfigure}
\caption{تکملات کی راہ}
\label{شکل_مثال_مخلوط_تکمل_غیر_تحلیلی}
\end{figure}
\انتہا{مثال}
%==========================
\ابتدا{مثال}\شناخت{مثال_مخلوط_تکمل_طاقت}\quad \موٹا{عدد صحیح طاقت کے تکمل}\\
مان لیں کہ \عددی{f(z)=(z-z_0)^m} ہے جہاں \عددی{m} عدد صحیح اور \عددی{z_0} مستقل ہیں۔گھڑی کی الٹ رخ رداس \عددی{\rho} کے دائرہ \عددی{C} پر تکمل حاصل کریں۔دائرے کا مرکز \عددی{z_0} ہے (شکل \حوالہ{شکل_مثال_مخلوط_تکمل_غیر_تحلیلی}-ب)۔ \\
ہم \عددی{C} کو 
\begin{align*}
z(t)=z_0+\rho(\cos t+i\sin t)=z_0+\rho e^{it} \quad \quad (0\le t\le 2\pi)
\end{align*}
لکھ سکتے ہیں۔یوں 
\begin{align*}
(z-z_0)^m=\rho^m e^{imt},\quad \dif z=i\rho e^{it}\dif t
\end{align*}
ہو گا لہٰذا تکمل درج ذیل ہو گا۔
\begin{align*}
\int_C (z-z_0)^m\dif z=\int_0^{2\pi} \rho^m e^{imt} i\rho e^{it}\dif t=i\rho^{m+1}\int_0^{2\pi} e^{i(m+1)t} \dif t
\end{align*}
\عددی{m=-1} کی صورت مثال \حوالہ{مثال_مخلوط_تکمل_دائرے_پر_تکمل} میں دیکھی گئی ہے جبکہ \عددی{m\ne -1} کی صورت میں درج ذیل  ہو گا (مساوات \حوالہ{مساوات_تحلیلی_قوت_نمائی_خیالی_دوری_عرصہ} دیکھیں):
\begin{align*}
\int_0^{2\pi} e^{i(m+1)t}=\big[\frac{e^{i(m+1)t}}{i(m+1)}\big]_0^{2\pi}=0\quad \quad (m\ne -1, \text{\RL{عدد صحیح}})
\end{align*}
یوں تکمل کا حل درج ذیل ہو گا۔
\begin{align}
\int_C (z-z_0)^m\dif z=
\begin{cases}
i2\pi& (m=-1)\\
0& (m\ne -1, \text{\RL{عدد صحیح}})
\end{cases}
\end{align}
\انتہا{مثال}
%============================
\ابتدا{مثال}\شناخت{مثال_مخلوط_تکمل_تعریف_استعمال_الف}\quad \موٹا{تکمل کی تعریف کی عملی استعمال}\\
مان لیں کہ \عددی{f(z)=\text{مستقل}=k} ہے جبکہ ابتدائی نقطہ \عددی{z_0} اور اختتامی نقطہ \عددی{Z} کے درمیان \عددی{C} کوئی راہ ہے۔اس صورت میں ہم تکمل کی تعریف، یعنی مساوات \حوالہ{مساوات_مخلوط_تکمل_راہ_ب} میں دیے گئے مجموعہ \عددی{S_n} کی حد، استعمال کرتے ہیں۔یوں
\begin{align*}
S_n=\sum_{m=1}^{n} k\Delta z_m=k[(z_1-z_0) +(z_2-z_1)+\cdots+(Z-z_{n-1})]=k(Z-z_0)
\end{align*}
ہو گا جس سے تکمل کی قیمت درج ذیل حاصل ہو گی۔
\begin{align*}
\int_C k\dif z=\lim_{n\to \infty}S_n=k(Z-z_0)
\end{align*}
ہم دیکھتے ہیں کہ اس تکمل کی قیمت صرف ابتدائی اور اختتامی نقطوں \عددی{z_0} اور \عددی{Z} پر منحصر ہے نا ان نقطوں کے مابین راہ پر۔بالخصوص اگر راہ \عددی{C} بند ہو  تب \عددی{Z=z_0} ہو گا لہٰذا تکمل کی قیمت صفر ہو گی۔
\انتہا{مثال}
%=========================
\ابتدا{مثال}\شناخت{مثال_مخلوط_تکمل_تعریف_استعمال_ب}\quad \موٹا{تکمل کی تعریف کی دوسری مثال}\\
فرض کریں کہ \عددی{f(z)=z} ہے جبکہ  ابتدائی نقطہ \عددی{z_0} اور اختتامی نقطہ \عددی{Z} کے مابین \عددی{C} کوئی راہ ہے۔ہم دوبارہ  مساوات \حوالہ{مساوات_مخلوط_تکمل_راہ_ب}  استعمال کرتے ہیں۔\عددی{\zeta_m=z_m} لیتے ہوئے
\begin{align*}
S_n=\sum_{m=1}^n z_m\Delta z_m=z_1(z_1-z_0)+z_2(z_2-z_1)+\cdots+Z(Z-z_{n-1})
\end{align*}
حاصل ہو گا۔اسی طرح \عددی{\zeta_m=z_{m-1}} لیتے ہوئے
\begin{align*}
S^*_n=\sum_{m=1}^n z_{m-1} \Delta z_m=z_0(z_1-z_0)+z_1(z_2-z_1)+\cdots+z_{n-1}(Z-z_{n-1})
\end{align*}
حاصل ہو گا۔ان دونوں کو جمع کرتے ہوئے \عددی{S_n+S^*_n=Z^2-z_0^2} ملتا ہے۔یوں
\begin{align*}
\lim_{n\to \infty} (S_n+S^*_n)=2\int_{z_0}^Z z\dif z=Z^2-z_0^2
\end{align*}
ہو گا  جس سے ان نقطوں کے مابین ہر راہ پر تکمل کی قیمت
\begin{align*}
\int_{z_0}^Z z\dif z=\frac{1}{2} (Z^2-z_0^2)
\end{align*}
حاصل ہوتی ہے۔بالخصوص اگر \عددی{C} بند راہ ہو تب \عددی{Z=z_0} ہو گا لہٰذا
\begin{align}
\oint_C z\dif z=0
\end{align}
ہو گا۔یہی نتیجہ مسئلہ \حوالہ{خطی_تکمل_سطحی_مسئلہ_گرین} سے  مساوات \حوالہ{مساوات_مخلوط_تکمل_راہ_ث} میں دیے گئے کلیہ کی مدد سے  بھی حاصل کیا جا سکتا ہے۔
\انتہا{مثال}
%============================

\حصہء{سوالات}
سوال \حوالہ{سوال_مخلوط_تکمل_راہ_کی_روپ_الف} تا سوال \حوالہ{سوال_مخلوط_تکمل_راہ_کی_روپ_ب} میں \عددی{A} تا \عددی{B} قطع کو \عددی{z=z(t)} روپ میں لکھیں۔

%================
\ابتدا{سوال}\شناخت{سوال_مخلوط_تکمل_راہ_کی_روپ_الف}\quad
$A:z=0,\quad B:z=2-i3$\\
جواب:\quad
$(2-i3)t,\quad 0\le t\le 1$
\انتہا{سوال}
%=======================
\ابتدا{سوال}\quad
$A:z=0,\quad B:z=1+i$\\
جواب:\quad
$(1+i)t,\quad 0\le t\le 1$
\انتہا{سوال}
%=======================
\ابتدا{سوال}\quad
$A:z=1-i,\quad B:z=-1+i$\\
جواب:\quad
$1-i+(-1+i)t,\quad 0\le t\le 2$
\انتہا{سوال}
%=======================
\ابتدا{سوال}\quad
$A:z=-2-i,\quad B:z=0$\\
جواب:\quad
$-2-i+(2+i)t,\quad 0\le t\le 1$
\انتہا{سوال}
%=======================
\ابتدا{سوال}\quad
$A:z=i3,\quad B:z=3$\\
جواب:\quad
$i3+(1-i)t,\quad 0\le t\le 3$
\انتہا{سوال}
%=======================
\ابتدا{سوال}\شناخت{سوال_مخلوط_تکمل_راہ_کی_روپ_ب}\quad
$A:z=3i,\quad B:z=-2i$\\
جواب:\quad
$i3-it,\quad 0\le t\le 5$
\انتہا{سوال}
%=======================
سوال \حوالہ{سوال_مخلوط_تکمل_منحنی_روپ_درکار_الف} تا سوال \حوالہ{سوال_مخلوط_تکمل_منحنی_روپ_درکار_ب} میں دی گئی منحنیات کو \عددی{z=z(t)} روپ میں لکھیں۔

%============================
\ابتدا{سوال}\شناخت{سوال_مخلوط_تکمل_منحنی_روپ_درکار_الف}\quad
$\abs{z-2+i3}=5$\\
جواب:\quad
$z=2-i3+5e^{it},\quad 0\le t\le 2\pi$
\انتہا{سوال}
%===========================
\ابتدا{سوال}\quad
$y=x,\quad \text{\RL{$(0,0)$ تا $(4,4)$}}$\\
جواب:\quad
$z=(1+i)t,\quad 0\le t\le 4$
\انتہا{سوال}
%===========================
\ابتدا{سوال}\quad
$y=x^2,\quad \text{\RL{$(0,0)$ تا $(3,9)$}}$\\
جواب:\quad
$z=t+it^2,\quad 0\le t\le 3$
\انتہا{سوال}
%===========================
\ابتدا{سوال}\quad
$x^2+4y^2=4$\\
جواب:\quad
$z=2\cos t+i\sin t,\quad 0\le t\le 2\pi$
\انتہا{سوال}
%===========================
\ابتدا{سوال}\quad
$4x^2+9y^2=36$\\
جواب:\quad
$z=3\cos t+i2\sin t,\quad 0\le t\le 2\pi$
\انتہا{سوال}
%===========================
\ابتدا{سوال}\quad
$4(x-2)^2+9(y+3)^2=36$\\
جواب:\quad
$z=(2+3\cos t)+i(-3+2\sin t),\quad 0\le t\le 2\pi$
\انتہا{سوال}
%===========================
\ابتدا{سوال}\quad
$y=\sqrt{x},\quad \text{\RL{$(1,1)$ تا $(9,3)$}}$\\
جواب:\quad
$\,z=t+i\sqrt{t},\quad 1\le t\le 9$
یا
$z=t^2+it,\quad 1\le t\le 3\,$
\انتہا{سوال}
%===========================
\ابتدا{سوال}\quad
$y=\tfrac{1}{x},\quad \text{\RL{$(1,1)$ تا $(5,\tfrac{1}{5})$}}$\\
جواب:\quad
$z=t+\tfrac{i}{t},\quad 1\le t\le 5$

\انتہا{سوال}
%===========================
\ابتدا{سوال}\شناخت{سوال_مخلوط_تکمل_منحنی_روپ_درکار_ب}\quad
$y=5+2x-3x^2,\quad \text{\RL{$(0,5)$ تا $(2,-3)$}}$\\
جواب:\quad
$z=t+i(5+2t-3t^2),\quad 0\le t\le 2$

\انتہا{سوال}
%===========================
سوال \حوالہ{سوال_مخلوط_تکمل_منحنیات_کون_سی_الف} تا سوال \حوالہ{سوال_مخلوط_تکمل_منحنیات_کون_سی_ب} میں دیے تفاعل کن منحنیات کو ظاہر کرتے ہیں۔ 

%======================
\ابتدا{سوال}\شناخت{سوال_مخلوط_تکمل_منحنیات_کون_سی_الف}\quad
$-1+(2+i)t,\quad 0\le t\le 1$\\
جواب:\quad 
سیدھی لکیر \عددی{x-2y+1=0} پر \عددی{-1} تا \عددی{1+i}
\انتہا{سوال}
%==============================
\ابتدا{سوال}\quad
$i+t+i2t^2,\quad -2\le t\le 1$\\
جواب:\quad 
منحنی \عددی{y=2x^2+1} پر \عددی{(-2+i9)} تا \عددی{(1+i3)}
\انتہا{سوال}
%==============================
\ابتدا{سوال}\quad
$2-i3+5e^{it},\quad 0\le t\le \pi$\\
جواب:\quad 
بالائی نصف دائرہ 
$(x+2)^2+(y-3)^2=25\,$
\انتہا{سوال}
%==============================
\ابتدا{سوال}\quad
$1+2e^{it},\quad -\pi\le t\le 0$\\
جواب:\quad 
نچلا نصف دائرہ 
$(x-1)^2+y^2=4\,$
\انتہا{سوال}
%==============================
\ابتدا{سوال}\quad
$t+i2t^3,\quad -1\le t\le 0$\\
جواب:\quad 
منحنی \عددی{y=2x^3} پر \عددی{(-1,-i2)} تا مبدا
\انتہا{سوال}
%==============================
\ابتدا{سوال}\شناخت{سوال_مخلوط_تکمل_منحنیات_کون_سی_ب}\quad
$i-t+it^3,\quad 0\le t\le 3$\\
جواب:\quad
منحنی \عددی{y=1-x^3} پر \عددی{i} تا \عددی{(-3+28i)}
\انتہا{سوال}
%==========================
سوال \حوالہ{سوال_مخلوط_تکمل_تلاش-الف} تا سوال \حوالہ{سوال_مخلوط_تکمل_تلاش-ب} میں \عددی{z^2} کا تکمل دی گئی قطع پر تلاش کریں۔

%=========================
\ابتدا{سوال}\شناخت{سوال_مخلوط_تکمل_تلاش-الف}\quad
\عددی{0} تا \عددی{i3}\\
جواب:\quad
$-i9$
\انتہا{سوال}
%=====================
\ابتدا{سوال}\quad
\عددی{0} تا \عددی{3+i3}\\
جواب:\quad
$-18+i18$
\انتہا{سوال}
%=====================
\ابتدا{سوال}\quad
\عددی{1+i} تا \عددی{2-i}\\
جواب:\quad
$\tfrac{1}{3}(4-i13)$
\انتہا{سوال}
%=====================
\ابتدا{سوال}\شناخت{سوال_مخلوط_تکمل_تلاش-ب}\quad
$\,-1+i$
تا 
$1-i\,$\\
جواب:\quad
$-\tfrac{4}{3}(1+i)$
\انتہا{سوال}
%=====================
\ابتدا{سوال}\quad
تفاعل \عددی{z^2} کا، گھڑی کی الٹ رخ، تکون کے گرد ایک مرتبہ تکمل حاصل کریں۔ تکون کے کونے \عددی{0}، \عددی{1} اور \عددی{i} ہیں۔ \\
جواب:\quad
$0$
\انتہا{سوال}
%===========================
\ابتدا{سوال}\quad
\عددی{z+\tfrac{1}{z}} کا اکائی رداس کے گرد گھڑی کی رخ تکمل تلاش کریں۔\\
جواب:\quad
$-i2\pi$
\انتہا{سوال}
%========================
\ابتدا{سوال}\quad
\عددی{z} کا \عددی{1} سے انتصابی \عددی{1+i} تک اور یہاں سے افقی \عددی{-1+i} تک تکمل تلاش کریں۔ \\
جواب:\quad
$-\tfrac{1}{2}-i$
\انتہا{سوال}
%======================
\ابتدا{سوال}\quad
\عددی{az+b} کا \عددی{0} تا \عددی{2+i3} قطع پر تکمل تلاش کریں۔\\
جواب:\quad
$2b-2.5a+i(3b+6a)$
\انتہا{سوال}
%==============================
\ابتدا{سوال}\quad
تکمل
$\int_C (z-1)^{-1}\dif z \,$
کو گھڑی کی رخ \عددی{C:\abs{z-1}=2} پر تلاش کریں۔یہی تکمل گھڑی کی الٹ رخ تلاش کریں۔\\
جواب:\quad
گھڑی کی الٹ رخ \عددی{i2\pi} جبکہ گھڑی کی رخ \عددی{-i2\pi} ہے۔
\انتہا{سوال}
%===============================
\ابتدا{سوال}\quad
تکمل
$\, \int_C z\text{حقیقی} \dif z \,$
گھڑی کی الٹ رخ دائرہ \عددی{\abs{z}=r} کے گرد  حاصل کریں۔\\
جواب:\quad
$i\pi r^2$
\انتہا{سوال}
%==================
\ابتدا{سوال}\quad
تکمل
$\, \int_C \abs{z} \dif z \,$
کو \عددی{A:z=-i} تا \عددی{B:z=i}  (الف)  قطع \عددی{AB}  پر تلاش کریں، (ب) بائیں نصف مستوی میں اکائی دائرہ پر تلاش کریں، (پ) دائیں نصف مستوی میں اکائی دائرہ پر تلاش کریں۔\\
جواب:\quad
(الف) \عددی{i}، (ب) \عددی{i2}، (پ) \عددی{i2}
\انتہا{سوال}
%==================

\حصہ{مخلوط خطی تکمل کی خواص}
مجموعہ کی حد، مخلوط خطی تکمل کی تعریف ہے۔اس سے درج ذیل خواص اخذ ہوتے  ہیں۔

\ترچھا{اگر ہم راہ \عددی{C} کو دو ٹکڑوں \عددی{C_1} اور \عددی{C_2} میں تقسیم کریں (شکل \حوالہ{شکل_مخلوط_تکمل_راہ_کی_ٹکڑے}) تب درج ذیل ہو گا:} 
\begin{align}\label{مساوات-مخلوط_تکمل_ٹکڑے}
\int_C f(z)\dif z=\int_{C_1} f(z)\dif z+\int_{C_2}f(z)\dif z
\end{align}
%
\begin{figure}
\centering
\begin{tikzpicture}
\draw[->-=0.25,->-=0.75](0,0)node[ocirc]{}node[left]{$z_0$} to [out=20,in=-160] coordinate[pos=0.5](kA)coordinate[pos=1](kB)++(1.5,1) to [out=20,in=-120] coordinate[pos=0.5](kC)++(1.5,1)node[ocirc]{}node[right]{$Z$};
\draw(kB)++(20+90:0.1)--++(20+90:-0.2);
\draw(kA)node[above left]{$C_1$};
\draw(kC) node[above left]{$C_2$};
\end{tikzpicture}
\caption{تکمل کی راہ کے ٹکڑے}
\label{شکل_مخلوط_تکمل_راہ_کی_ٹکڑے}
\end{figure}

\ترچھا{اگر ہم تکمل لیتے ہوئے راہ پر الٹ رخ چلیں تب تکمل کی قیمت منفی اکائی سے ضرب ہو گی}
\begin{align}\label{مساوات_مخلوط_تکمل_راہ_الٹ_کرنے_سے_تکمل_منفی_اکائی}
\int_{z_0}^{Z} f(z)\dif z=-\int_{Z}^{z_0} f(z)\dif z
\end{align}
جہاں \عددی{z_0} اور \عددی{Z} راہ \عددی{C} کے سر ہیں؛ بائیں ہاتھ تکمل کو \عددی{z_0} تا \عددی{Z} حاصل کیا گیا ہے جبکہ دائیں ہاتھ تکمل کو \عددی{Z} تا \عددی{z_0} حاصل کیا گیا ہے۔ 

\ترچھا{دو یا دو سے زیادہ تفاعل کے مجموعہ کا تکمل جزو در جزو حاصل کیا جا سکتا ہے، اور  مشترک مستقل جزو ضربی کو تکمل کے باہر منتقل کیا جا سکتا ہے، یعنی:}
\begin{align}\label{مساوات_مخلوط_تکمل_مستقل_جزو_ضربی_باہر}
\int_C[k_1f_1(z)+k_2f_2(z)]\dif z=k_1\int_C f_1(z)\dif z+k_2\int_C f_2(z)\dif z
\end{align} 

ہمیں مخلوط خطی تکمل کی حتمی قیمت کا تخمینہ بار بار درکار ہو گا جس کی حصول کا بنیادی کلیہ درج ذیل ہے
\begin{align}\label{مساوات_مخلوط_تکمل_حتمی_قیمت_تخمینہ}
\abs{\int_C f(z)\dif z} \le Ml
\end{align}
جہاں \عددی{l} راہ \عددی{C} کی لمبائی ہے جبکہ \عددی{M} ایسا حقیقی مستقل ہے کہ پوری \عددی{C} پر \عددی{\abs{f(z)}\le M} ہو۔

مساوات \حوالہ{مساوات_مخلوط_تکمل_حتمی_قیمت_تخمینہ} کو ثابت کرنے کی خاطر ہم مساوات \حوالہ{مساوات_مخلوط_عدم_مساوات_ب} کو مساوات \حوالہ{مساوات_مخلوط_تکمل_راہ_ب} کے ساتھ ملا کر 
\begin{align*}
\abs{S_n}=\abs{\sum_{m=1}^n f(\zeta_m)\Delta z_m}\le \sum_{m=1}^n \abs{f(\zeta_m)}\abs{\Delta z_m}\le M\sum_{m=1}^n \abs{\Delta z_m}
\end{align*}
لکھتے ہیں۔اب \عددی{\Delta z_m} اس قطع کی لمبائی ہے جس کے سر \عددی{z_{m-1}} اور \عددی{z_m} ہیں (شکل \حوالہ{شکل_مخلوط_تکمل_ٹکڑے_راہ} دیکھیں)۔یوں دائیں ہاتھ مجموعہ درحقیقت ان سیدھی قطعات کی لمبائیوں کا مجموعہ \عددی{L} ہے جن کے سر \عددی{z_0,z_1,\cdots,z_n(=Z)} ہیں۔  اگر \عددی{n} کی قیمت اس طرح لامتناہی کے قریب پہنچتی ہو کہ \عددی{\Delta z_m} کی زیادہ سے زیادہ لمبائی صفر کے قریب پہنچتی ہو تب \عددی{L} کی قیمت،  لمبائی قوس کی تعریف (حصہ \حوالہ{حصہ_الاحصاء_لمبائی_قوس}) کے تحت،  قوس  \عددی{C} کی لمبائی \عددی{l} کے قریب پہنچے گی۔ یوں \حوالہ{مساوات_مخلوط_تکمل_حتمی_قیمت_تخمینہ}  میں دیا گیا کلیہ ثابت ہوتا ہے۔

%=====================
\ابتدا{مثال}\quad \موٹا{مساوات \حوالہ{مساوات_مخلوط_تکمل_حتمی_قیمت_تخمینہ} کی استعمال}\\
تفاعل \عددی{f(z)=\tfrac{1}{z}} کا دائرہ \عددی{\abs{z}=\rho} کے گرد ایک مرتبہ تکمل تلاش کریں۔\\
حل:\quad یوں \عددی{l=2\pi \rho}  ہے اور دائرے پر \عددی{\abs{f(z)}=\tfrac{1}{\rho}} ہے۔اس طرح مساوات \حوالہ{مساوات_مخلوط_تکمل_حتمی_قیمت_تخمینہ} سے درج ذیل ملتا ہے۔
\begin{align*}
\abs{\int_C \frac{\dif z}{z}}\le \frac{1}{\rho} 2\pi \rho=2\pi\quad \quad \quad \text{\RL{(مثال \حوالہ{مثال_مخلوط_تکمل_دائرے_پر_تکمل} دیکھیں)}}
\end{align*}
\انتہا{مثال}
%========================
\ابتدا{مثال}\موٹا{ایک اور تکمل کی قیمت کا تخمینہ}\\
مثال \حوالہ{مثال_مخلوط_تکمل_غیر_تحلیلی} میں راہ \عددی{C_2} کی لمبائی \عددی{l=2} ہے اور \عددی{C_2} پر \عددی{\abs{\text{\RL{حقیقی $z\,$}}}\le 1} ہے۔یوں مساوات \حوالہ{مساوات_مخلوط_تکمل_حتمی_قیمت_تخمینہ} درج ذیل دے گی۔
\begin{align*}
\abs{\int_{C_2} \text{\RL{حقیقی $z\,$}} \dif z} \le 2
\end{align*}
\انتہا{مثال}
%========================

\حصہء{سوالات}
%====================
\ابتدا{سوال}\quad
مساوات \حوالہ{مساوات-مخلوط_تکمل_ٹکڑے} کی تصدیق کریں جہاں \عددی{C} اکائی دائرہ ہے جبکہ \عددی{C_1}اس کا  بالائی نصف حصہ  اور \عددی{C_2} اس کا نچلا نصف حصہ ہے۔
\انتہا{سوال}
%============================
\ابتدا{سوال}\quad
تفاعل \عددی{f(z)=z^2} کے لئے مساوات \حوالہ{مساوات_مخلوط_تکمل_راہ_الٹ_کرنے_سے_تکمل_منفی_اکائی} کی تصدیق کریں جہاں \عددی{C} نقطہ \عددی{-1-i} سے \عددی{1+i} تک ہے۔
\انتہا{سوال}
%=========================
\ابتدا{سوال}\quad
تفاعل \عددی{k_1f_1+k_2f_2=3z-z^2} کے لئے مساوات \حوالہ{مساوات_مخلوط_تکمل_مستقل_جزو_ضربی_باہر} کی تصدیق کریں جہاں \عددی{C} اکائی دائرے کا بالائی نصف حصہ \عددی{1} تا \عددی{-1} ہے۔ 
\انتہا{سوال}
%============================
\ابتدا{سوال}\quad
تفاعل \عددی{f(z)=\tfrac{1}{z}} کے لئے مساوات \حوالہ{مساوات_مخلوط_تکمل_حتمی_قیمت_تخمینہ} کی تصدیق کریں جہاں \عددی{C} اکائی دائرے کا بالائی نصف حصہ \عددی{1} تا \عددی{-1} ہے۔ 
\انتہا{سوال}
%============================
سوال \حوالہ{سوال_مخلوط_تکمل_کی_تلاش_الف} تا سوال \حوالہ{سوال_مخلوط_تکمل_کی_تلاش_ب} میں \عددی{\int_C f(z)\dif z} کی قیمت تلاش کریں۔

%=============
\ابتدا{سوال}\شناخت{سوال_مخلوط_تکمل_کی_تلاش_الف}\quad
سیدھی قطع  \عددی{-1-i} تا \عددی{2+i} پر تفاعل \عددی{f(z)=az+b} ہے۔\\
جواب:\quad
$\tfrac{(a+2b)(3+i2)}{2}$
\انتہا{سوال}
%====================
\ابتدا{سوال}\quad
گھڑی کی الٹ رخ اکائی دائرے پر \عددی{f(z)=z^2+\frac{2}{z}}\\
جواب:\quad
$i4\pi$
\انتہا{سوال}
%============================
\ابتدا{سوال}\quad
 \عددی{1} تا \عددی{-1} اکائی دائرے کی بالائی نصف  پر \عددی{f(z)=z^2+\frac{3}{z^4}}\\
جواب:\quad
$\tfrac{4}{3}$
\انتہا{سوال}
%============================
\ابتدا{سوال}\quad
گھڑی کی الٹ رخ اکائی دائرے   پر \عددی{f(z)=2z+\tfrac{1}{z}+\tfrac{2}{z^2}}\\
جواب:\quad
$i2\pi$
\انتہا{سوال}
%============================
\ابتدا{سوال}\quad
\عددی{0} تا \عددی{1+i\tfrac{\pi}{2}} سیدھی قطع پر \عددی{f(z)=e^z}\\
جواب:\quad
$-1+ie$
\انتہا{سوال}
%============================
\ابتدا{سوال}\quad
گھڑی کی رخ دائرہ \عددی{\abs{z-1}=4} پر \عددی{f(z)=\tfrac{1}{z-1}+\tfrac{2}{(z-1)^2}}\\
جواب:\quad
$-i2\pi$
\انتہا{سوال}
%=======================
\ابتدا{سوال}\quad
قطع \عددی{i\pi} تا \عددی{i2\pi} پر \عددی{f(z)=\cos z}\\
جواب:\quad
$i(\sinh 2\pi -\sinh \pi)$
\انتہا{سوال}
%======================
\ابتدا{سوال}\quad
قطع \عددی{i\pi} تا \عددی{i2\pi} پر \عددی{f(z)=\sin z}\\
جواب:\quad
$\cosh \pi-\cosh 2\pi$
\انتہا{سوال}
%======================
\ابتدا{سوال}\quad
قطع \عددی{-i} تا \عددی{i} پر \عددی{f(z)=\sin z}\\
جواب:\quad
$0$
\انتہا{سوال}
%======================
\ابتدا{سوال}\quad
قطع \عددی{0} تا \عددی{i} پر \عددی{f(z)=\sin z}\\
جواب:\quad
$1-\cosh 1$
\انتہا{سوال}
%======================
\ابتدا{سوال}\quad
قطع \عددی{0} تا \عددی{i} پر \عددی{f(z)=\sinh z}\\
جواب:\quad
$\cos 1-1$
\انتہا{سوال}
%======================
\ابتدا{سوال}\شناخت{سوال_مخلوط_تکمل_کی_تلاش_ب}\quad
قطع \عددی{0} تا \عددی{i} پر \عددی{f(z)=\cosh z}\\
جواب:\quad
$i\sin 1$
\انتہا{سوال}
%======================
سوال \حوالہ{سوال_مخلوط_تکمل_تخمینہ_الف} تا سوال \حوالہ{سوال_مخلوط_تکمل_تخمینہ_ب} میں مساوات \حوالہ{مساوات_مخلوط_تکمل_حتمی_قیمت_تخمینہ} کی مدد سے \عددی{0} تا \عددی{3+i4} راہ پر  تکمل کی زیادہ سے زیادہ قیمت کا تخمینہ لگائیں۔

%===============
\ابتدا{سوال}\شناخت{سوال_مخلوط_تکمل_تخمینہ_الف}\quad
$\int_C z\dif z$\\
جواب:\quad
$25$
\انتہا{سوال}
%====================
\ابتدا{سوال}\quad
$\int_C e^z\dif z$\\
جواب:\quad
$5e^3$
\انتہا{سوال}
%====================
\ابتدا{سوال}\quad
$\int_C \Ln (z+1)\dif z$
\انتہا{سوال}
%====================
\ابتدا{سوال}\شناخت{سوال_مخلوط_تکمل_تخمینہ_ب}\quad
$\int_C \tfrac{1}{z+1}\dif z$\\
جواب:\quad
$5$
\انتہا{سوال}
%====================
\ابتدا{سوال}\quad
سوال \حوالہ{سوال_مخلوط_تکمل_تخمینہ_الف} میں راہ کو دو ٹکڑوں میں تقسیم کرتے ہوئے تکمل کی زیادہ سے زیادہ قیمت کا بہتر تخمینہ لگائیں۔ 
\انتہا{سوال}
%=====================

\حصہ{کوشی کا مسئلہ تکمل}\شناخت{حصہ_مخلوط_تکمل_کوشی_مسئلہ_تکمل}
 مخلوط تجزیہ میں کوشی کا مسئلہ تکمل اہم کردار ادا کرتا ہے۔ اس کے علاوہ اس مسئلے کے  دیگر نظریاتی اور عملی اثرات بھی مرتب ہوتے ہیں۔اس مسئلہ کو بیان کرنے کی خاطر درج ذیل تصورات کی ضرورت پیش آئی گی۔

مخلوط مستوی میں دائرہ کار \عددی{D} اس صورت \اصطلاح{سادہ تعلق خطہ}\فرہنگ{سادہ تعلق خطہ}\فرہنگ{خطہ!سادہ تعلق}\حاشیہب{simply connected domain}\فرہنگ{domain!simply connected} کہلاتا ہے جب اس میں ہر سادہ بند منحنی (یعنی \عددی{D} میں  اپنے آپ کو غیر منقطع کرتا ہوا بند منحنی)  صرف \عددی{D} کے نقطوں کو گھیرتی ہو۔ایسا دائرہ کار جو سادہ تعلق  نہ ہو \اصطلاح{مضرب تعلق خطہ}\فرہنگ{خطہ!مضرب تعلق}\حاشیہب{multiply connected}\فرہنگ{domain!multiply connected} کہلاتا ہے۔

مثال کے طور پر ایک دائرے کی اندرون (دائری قرص)، قطع مکافی کی اندرون اور چکور کی اندرون سادہ تعلق ہیں۔ بلکہ کسی بھی سادہ بند منحنی کی اندرون سادہ تعلق ہو گی۔ دائری جھلی  (حصہ \حوالہ{حصہ_مخلوط_سطح_منحنیات_خطہ}) مضرب تعلق (زیادہ درست اصطلاح دوہرا تعلق ہو گی) ہے۔

مزید، ایسا دائرہ کار \عددی{D} جو مکمل طور پر مبدا کے گرد کسی دائرے میں پایا جاتا ہو \اصطلاح{محدود}\فرہنگ{محدود}\حاشیہب{bounded}\فرہنگ{bounded} کہلاتا ہے ورنہ  اس  کو \اصطلاح{غیر محدود}\فرہنگ{غیر محدود}\فرہنگ{محدود!غیر}\حاشیہب{unbounded}\فرہنگ{unbounded} کہتے ہیں۔

%===================
\ابتدا{مسئلہ}\quad \موٹا{کوشی مسئلہ تکمل}\\
سادہ تعلق محدود دائرہ کار \عددی{D} میں تحلیلی\حاشیہد{یاد رہے کہ تفاعل کی تعریف کے تحت تفاعل واحد قیمت تعلق ہوتا ہے۔} \عددی{f(z)} کی صورت میں \عددی{D} میں ہر سادہ  بند منحنی \عددی{C} پر درج ذیل ہو گا۔
\begin{align}\label{مساوات_مخلوط_تکمل_مسئلہ_کوشی_تکمل}
\int\limits_C f(z)\dif z=0
\end{align}

\begin{figure}
\centering
\begin{tikzpicture}
\draw(0,0.5) to [out=-90,in=180](1.5,0) to [out=0,in=-90] (4,1) to [out=90,in=0] (3,2) to [out=180,in=90] (0,0.5);
\draw[->-=0.75,rotate around={40:(2.5,1.25)}](2.5,1.25) circle (0.75cm and 0.5cm);
\draw(2.5,1.25)++(-70:0.75)node{$C$};
\draw(0.5,0.5)node{$D$};
\end{tikzpicture}
\caption{کوشی کا مسئلہ تکمل}
\label{شکل_مخلوط_تکمل_مسئلہ_کوشی_دائرہ_کار}
\end{figure}
\انتہا{مسئلہ}
%=======================
\ابتدا{ثبوت}\quad \موٹا{کوشی کا ثبوت}\\
مساوات \حوالہ{مساوات_مخلوط_تکمل_راہ_ٹ} سے 
\begin{align}\label{مساوت_مخلوط_تکمل_کوشی_تکمل_ثبوت_الف}
\int_C f(z)\dif z=\int_C(u\dif x-v\dif y)+i\int_C(u\dif y+v\dif x)
\end{align}
ملتا ہے۔ \عددی{f(z)} تحلیلی ہے لہٰذا \عددی{f'(z)} موجود ہے۔کوشی نے اضافی فرض کیا کہ \عددی{f'(z)} استمراری ہے۔تب مساوات \حوالہ{مساوات_مخلوط_کوشی_ریمان_ثبوت_پ} اور مساوات \حوالہ{مساوات_مخلوط_کوشی_ریمان_ثبوت_ت}  کے تحت \عددی{D} میں \عددی{u} اور \عددی{v} کے استمراری جزوی تفرق پائے جائیں گے۔مسئلہ \حوالہ{خطی_تکمل_سطحی_مسئلہ_گرین} قابل اطلاق ہو گا (جس میں \عددی{f} اور \عددی{g} کی جگہ بالترتیب \عددی{u} اور \عددی{v} پر کرتے ہیں) لہٰذا
\begin{align*}
\int_C (u\dif x-v\dif y)=\iint\limits_R \big(-\frac{\partial v}{\partial x}-\frac{\partial u}{\partial y}\big)\dif x\dif y
\end{align*}
لکھا جا سکتا ہے جہاں \عددی{R} کی سرحد \عددی{C} ہے۔مساوات \حوالہ{مساوات_مخلوط_کوشی_ریمان} کے تحت دائیں ہاتھ تکمل مکمل صفر کے برابر ہے لہٰذا بائیں ہاتھ کا تکمل صفر ہو گا۔اسی طرح  مساوات \حوالہ{مساوات_مخلوط_کوشی_ریمان} کے تحت مساوات \حوالہ{مساوت_مخلوط_تکمل_کوشی_تکمل_ثبوت_الف} کا آخری تکمل بھی صفر ہو گا۔یوں کوشی کا ثبوت مکمل ہوتا ہے۔
\انتہا{ثبوت}
%=======================
\ابتدا{ثبوت}\quad \موٹا{گرسا کا ثبوت}\\
\موٹا{گرسا}\حاشیہد{فرانسیسی ریاضی دان ایڈورڈ گرسا [1858-1936]} نے مسئلہ کوشی کو \عددی{f'(z)} کی استمراری ہونے کی شرط کے بغیر ثابت کیا جو بہت اہم حقیقت ہے۔ہم شروع ایسی صورت سے کرتے ہیں جہاں \عددی{C} ایک تکون کی سرحد ہے۔ہم اس تکون کو گھڑی کی الٹ رخ سمت بند کرتے ہیں۔تکون کی اطراف کی درمیانے نقطوں کو آپس میں ملاتے ہوئے تکون کو چار مماثل تکونوں میں تقسیم کرتے ہیں (شکل \حوالہ{شکل_مخلوط_تکمل_مسئلہ_کوشی_ثبوت})۔ یوں 
\begin{align*}
\int\limits_{C}f\dif z=\int\limits_{C_a}f\dif z+\int\limits_{C_b}f\dif z+\int\limits_{C_c}f\dif z+\int\limits_{C_d}f\dif z
\end{align*}
ہو گا جہاں \عددی{C_a}، \نقطے، \عددی{C_d} ان چار تکون کی سرحد ہیں۔اب دائیں ہاتھ میں ہم تقسیم کی ہر قطع پر تکمل دو مرتبہ آپس میں الٹ رخ حاصل کرتے ہیں۔ایک ہی قطع پر  آپس میں الٹ رخ کی جوڑی تکمل ایک دوسرے کو حذف  کرتے ہیں لہٰذا دائیں ہاتھ چار تکملوں کا مجموعہ بائیں ہاتھ کی تکمل کے برابر ہو گا۔دائیں ہاتھ کے تکملوں میں سے ایک تکمل، جس کی سرحد کو ہم \عددی{C_1} کہیں گے، ایسا ہو گا جس کے لئے درج ذیل لکھنا ممکن ہو گا۔
\begin{align*}
\abs{\int_C f\dif z}\le 4\abs{\int_{C_1}f\dif z}
\end{align*}
ہم درج بالا اس لئے لکھ سکتے ہیں کہ چاروں تکمل میں سے ہر ایک کی حتمی قیمت چاروں کے مجموعے کی حتمی قیمت سے چار گنا کم نہیں ہو سکتی ہے۔یہ مساوات \حوالہ{مساوات_مخلوط_عدم_مساوات_ب} سے اخذ کیا جا سکتا ہے۔
\begin{figure}
\centering
\begin{tikzpicture}
\draw[->-=0.25,->-=0.75](0,0)coordinate(kA)--++(20:4)coordinate(kB)coordinate[pos=0.5](kAA);
\draw[->-=0.25,->-=0.75](kB)--++(135:3)coordinate(kC)coordinate[pos=0.5](kBB);
\draw[->-=0.25,->-=0.75](kC)--(0,0)coordinate[pos=0.5](kCC);
\draw(kAA)--(kBB)--(kCC)--(kAA);
%
\path(kAA)--(kBB)coordinate[pos=0.25,shift={(-40:0.1)}](kABa)coordinate[pos=0.25,shift={(-40:-0.1)}](kkABa)coordinate[pos=0.75,shift={(-40:0.1)}](kABb)coordinate[pos=0.75,shift={(-40:-0.1)}](kkABb);
\draw[latex-](kABa)--(kABb);
\draw[-latex](kkABa)--(kkABb);
\path(kBB)--(kCC)coordinate[pos=0.25,shift={(-80:0.1)}](kABa)coordinate[pos=0.25,shift={(-80:-0.1)}](kkABa)coordinate[pos=0.75,shift={(-80:0.1)}](kABb)coordinate[pos=0.75,shift={(-80:-0.1)}](kkABb);
\draw[-latex](kABa)--(kABb);
\draw[latex-](kkABa)--(kkABb);
\path(kCC)--(kAA)coordinate[pos=0.25,shift={(40:0.1)}](kABa)coordinate[pos=0.25,shift={(40:-0.1)}](kkABa)coordinate[pos=0.75,shift={(40:0.1)}](kABb)coordinate[pos=0.75,shift={(40:-0.1)}](kkABb);
\draw[-latex](kABa)--(kABb);
\draw[latex-](kkABa)--(kkABb);
\end{tikzpicture}
\caption{مسئلہ کوشی کا ثبوت}
\label{شکل_مخلوط_تکمل_مسئلہ_کوشی_ثبوت}
\end{figure} 

ہم اس تکون جس کی سرحد \عددی{C_1} ہے کو اسی طرح چار تکونوں میں تقسیم کرتے ہیں اور ان میں سے ایسی تکون، جس کی سرحد کو ہم \عددی{C_2} کہیں گے، منتخب کرتے ہیں جس کے لئے 
\begin{align*}
\abs{\int_{C_1}f\dif z}\le 4\abs{\int_{C_2}f\dif z} \quad \implies \quad \abs{\int_C f\dif z}\le 4^2 \abs{\int_{C_2}f\dif z}
\end{align*}
لکھنا ممکن ہو۔ 

اسی طرح بڑھتے ہوئے ہمیں تکونوں کا ایک سلسلہ \عددی{T_1}،\عددی{T_2}، \نقطے حاصل ہو گا جن کی سرحدیں بالترتیب \عددی{C_1}، \عددی{C_2}، \نقطے  ہوں گی۔یہ تکون متشابہ ہوں گے اور \عددی{n>m} کی صورت میں  تکون \عددی{T_n} تکون \عددی{T_m} کے اندر پایا جائے گا۔مزید
\begin{align}\label{مساوات_مخلوط_تکمل_حتمی_حد}
\abs{\int_C f\dif z}\le 4^n\abs{\int_{C_n} f\dif z},\quad \quad \quad n=1,2,\cdots
\end{align}
لکھا جا سکتا ہے۔

فرض کریں کہ \عددی{z_0} ان تمام تکونوں کے اندر ایک نقطہ ہے۔چونکہ \عددی{f} نقطہ \عددی{z_0} پر قابل تفرق ہے لہٰذا \عددی{f'(z_0)} موجود ہو گا لہٰذا ہم
\begin{align}\label{مساوات_مخلوط_تکمل_تفاعل_کی_تسلسل}
f(z)=f(z_0)+(z-z_0)f'(z_0)+h(z)(z-z_0)
\end{align}
لکھ سکتے ہیں جس کا تکون \عددی{T_n} کی سرحد \عددی{C_n} پر تکمل حاصل کرتے ہوئے 
\begin{align}
\int_{C_n} f(z)\dif z=\int_{C_1} f(z_0)\dif z+\int_{C_1} (z-z_0) f'(z_0)\dif z+\int_{C_n} h(z)(z-z_0)\dif z
\end{align}
لکھتے ہیں۔چونکہ \عددی{f(z_0)} اور \عددی{f'(z_0)} مستقل ہیں لہٰذا مثال \حوالہ{مثال_مخلوط_تکمل_تعریف_استعمال_الف} اور مثال \حوالہ{مثال_مخلوط_تکمل_تعریف_استعمال_ب} کے نتیجہ کے تحت بائیں ہاتھ پہلے دو تکمل صفر کے برابر ہوں گے۔یوں
\begin{align}
\int_{C_n} f(z)\dif z=\int_{C_n} h(z)(z-z_0)\dif z
\end{align}
رہ جاتا ہے۔مساوات \حوالہ{مساوات_مخلوط_تکمل_تفاعل_کی_تسلسل} کو \عددی{z-z_0} سے تقسیم کر کے دو اجزاء کو بائیں ہاتھ منتقل کرتے ہوئے حتمی قیمت لے کر درج ذیل لکھا جا سکتا ہے۔
\begin{align*}
\abs{\frac{f(z)-f(z_0)}{z-z_0}-f'(z_0)}=\abs{h(z)}
\end{align*}
اس کا مساوات \حوالہ{مساوات_تحلیلی_نقطہ_کے_قریب_تر_شرط} کے ساتھ موازنہ کرتے ہوئے ہم دیکھتے ہیں کہ دیے گئے مثبت عدد \عددی{\epsilon} کی صورت میں ہم ایسا مثبت عدد \عددی{\sigma} تلاش کر سکتے ہیں جو درج ذیل شرط کو مطمئن کرے گا۔
\begin{align*}
\text{\RL{ہو گا۔}}\quad \abs{h(z)}\le \epsilon \quad \text{\RL{ہو تب}}\quad \abs{z-z_0} <\sigma \quad \text{جب}
\end{align*}
 
ہم اب عدد \عددی{n} اتنا بڑا لیتے ہیں کہ تکون \عددی{T_n} دائرہ \عددی{\abs{z-z_0}<\sigma} میں پایا جائے۔ہم \عددی{C_n} کی لمبائی کو \عددی{l_n} لکھتے ہیں۔یوں \عددی{T_n} میں \عددی{z_0} اور  \عددی{C_n} پر تمام \عددی{z} کے لئے  \عددی{\abs{z-z_0}\le \tfrac{l_n}{2}} ہو گا۔مساوات \حوالہ{مساوات_مخلوط_تکمل_حتمی_قیمت_تخمینہ} کی اطلاق سے ہم 
\begin{align}\label{مساوات_مخلوط_تکمل_قریبی_حد}
\abs{\int_{C_n} f(z)\dif z}=\abs{\int_{C_n} h(z)(z-z_0)\dif z}<\epsilon \frac{l_n}{2}l_n=\frac{\epsilon}{2}l_n^2 
\end{align}
لکھ سکتے ہیں۔فرض کریں کہ \عددی{C} کی لمبائی \عددی{l} ہو۔ تب راہ \عددی{C_1} کی لمبائی \عددی{l_1=\tfrac{l}{2}} ہو گی، راہ \عددی{C_2} کی لمبائی 
\عددی{l_2=\tfrac{l_1}{2}=\tfrac{l}{4}} ہو گی، اور اسی طرح \عددی{C_n} کی لمبائی 
\begin{align*}
l_n=\frac{l}{2^n}
\end{align*}
ہو گی۔مساوات \حوالہ{مساوات_مخلوط_تکمل_قریبی_حد} اور مساوات \حوالہ{مساوات_مخلوط_تکمل_حتمی_حد} سے درج ذیل لکھا جا سکتا ہے۔
\begin{align*}
\abs{\int_C f\dif z}\le 4^n \abs{\int_{C_n} f\dif z}< 4^n \frac{\epsilon}{2}l_n^2=4^n \frac{\epsilon}{2} \frac{l^2}{4^n}=\frac{\epsilon}{2}l^2
\end{align*}
اب \عددی{\epsilon\,(>0)} کی قیمت کو کافی چھوٹا  کرتے ہوئے ہم دائیں ہاتھ کو جتنا چاہیں چھوٹا بنا سکتے ہیں جبکہ دایاں ہاتھ (تکمل کی) ایک مستقل قیمت ہے۔اس سے ہم اخذ کرتے ہیں کہ اس تکمل کی قیمت صفر ہو گی۔یوں ثبوت مکمل ہوتا ہے۔

آئیں اب کثیر الاضلاع کے لئے اس مسئلے کو ثابت کریں۔ہم کثیر الاضلاع کو تکونوں میں تقسیم کرتے ہیں (شکل \حوالہ{شکل_مخلوط_تکمل_کثیر_رکنی_کی_سرحد})۔ایسی ہر تکون کا تکمل صفر ہو گا۔ ہر تکون کی سرحد پر گھڑی کی الٹ رخ تکمل حاصل کیا جاتا ہے لہٰذا ہر دو تکونوں کے درمیان تقسیمی قطع پر تکمل دو مرتبہ آپس میں الٹ رخ حاصل ہو گا۔ایسی ہر جوڑی تکملات کا مجموعہ صفر ہو گا۔یوں تمام تکونوں کی سرحد پر تکملوں کا مجموعہ کثیر رکنی کی سرحد پر تکمل کے برابر ہو گا۔اب چونکہ ہر تکون پر تکمل صفر کے برابر  ہے لہٰذا ان کا مجموعہ بھی صفر کے برابر ہو گا۔یوں کثیر رکنی کی سرحد پر تکمل صفر ہو گا۔
\begin{figure}
\centering
\begin{tikzpicture}
\draw[->-=0.5](0,0)coordinate(kA)--(4,0.5)coordinate(kB);
\draw[->-=0.5](4,0.5)--(3.5,2)coordinate(kC);
\draw[->-=0.5](3.5,2)--(2,1)coordinate(kD);
\draw[->-=0.5](2,1)--(1,1.5)coordinate(kE);
\draw[->-=0.5](1,1.5)--(0,0);
\draw[dashed](kA)--(kD)--(kB);
%
\path($(kA)!0.3!(kD)$)++(130:0.15)coordinate(kADa)++(130:-0.3)coordinate(kkADa) ($(kA)!0.75!(kD)$)++(130:0.15)coordinate(kADb)++(130:-0.3)coordinate(kkADb); 
\draw[-latex](kADa)--(kADb);
\draw[latex-](kkADa)--(kkADb);
\path($(kB)!0.3!(kD)$)++(70:0.15)coordinate(kBDa)++(70:-0.3)coordinate(kkBDa) ($(kB)!0.75!(kD)$)++(70:0.15)coordinate(kBDb)++(70:-0.3)coordinate(kkBDb); 
\draw[latex-](kBDa)--(kBDb);
\draw[-latex](kkBDa)--(kkBDb);
\end{tikzpicture}
\caption{کثیر رکنے کے لئے کوشی مسئلہ تکمل کا ثبوت}
\label{شکل_مخلوط_تکمل_کثیر_رکنی_کی_سرحد}
\end{figure}

کسی بھی بند راہ \عددی{C} کے  لئے اب ثبوت پیش کرتے ہیں۔کسی بھی بند راہ کے اندر اتنے اطراف کی کثیر رکنی \عددی{P} نقش کریں کہ \عددی{C} اور کثیر رکنی میں فرق قابل نظر انداز ہو۔ہم بغیر ثبوت پیش کیے (چونکہ یہ ثبوت پیچیدہ ہے) کہنا چاہیں گے کہ  \عددی{C} کے تکمل کی قیمت اور \عددی{P} کے تکمل کی قیمت میں فرق  \عددی{\epsilon} کو ہم جتنا چاہیں کم کر سکتے ہیں جہاں \عددی{\epsilon} ایک مثبت عدد ہے۔ چونکہ کثیر رکنی کے لئے ہم اس مسئلے کو ثابت کر چکے ہیں لہٰذا کسی بھی بند راہ کے لئے بھی مسئلہ ثابت ہوا۔ 
\انتہا{ثبوت}
%==========================

\ابتدا{مثال}\quad
\begin{align*}
\int_C e^z\dif z=0
\end{align*}
چونکہ ہر \عددی{z} پر \عددی{e^z} تحلیلی ہے لہٰذا درج بالا ہو گا۔
\انتہا{مثال}
%=======================
\ابتدا{مثال}\quad
\begin{align*}
\int_C\frac{\dif z}{z^2}=0
\end{align*}
جہاں \عددی{C} اکائی دائرہ ہے (حصہ \حوالہ{حصہ_مخلوط_تکمل_مخلوط_مستوی_میں_خطی_تکمل})۔چونکہ \عددی{z=0} پر \عددی{\tfrac{1}{z^2}} تحلیلی نہیں ہے لہٰذا یہ نتیجہ مسئلہ کوشی سے اخذ نہیں ہو گا۔یوں \عددی{D} میں \عددی{f} کی تحلیلی ہونے کی شرط، مساوات \حوالہ{مساوات_مخلوط_تکمل_مسئلہ_کوشی_تکمل} کی درست ہونے کے لئے،  کافی ہے نا کہ لازمی۔
\انتہا{مثال}
%===================
\ابتدا{مثال}\quad
\begin{align*}
\int_C\frac{\dif }{z}=i2\pi
\end{align*}
جہاں تکمل اکائی دائرے پر گھڑی کی الٹ رخ حاصل کیا گیا ہے (حصہ \حوالہ{حصہ_مخلوط_تکمل_مخلوط_مستوی_میں_خطی_تکمل})۔ راہ \عددی{C} جھلی \عددی{\tfrac{1}{2}<\abs{z}<\tfrac{3}{2}} میں پائی جاتی ہے جہاں \عددی{\tfrac{1}{z}} تحلیلی ہے لیکن یہ راہ سادہ تعلق نہیں رکھتی لہٰذا مسئلہ کوشی قابل اطلاق نہیں ہو گا۔یوں دائرہ کار \عددی{D} کی سادہ تعلق ہونے کی شرط انتہائی اہم ہے۔

\انتہا{مثال}
%=====================

مسئلہ کوشی میں راہ \عددی{C} کو دو ٹکڑوں \عددی{C_1} اور \عددی{C^*_2} میں تقسیم (شکل \حوالہ{شکل_مخلوط_تکمل_مختلف_راہ}-الف) کرنے سے مساوات \حوالہ{مساوات_مخلوط_تکمل_مسئلہ_کوشی_تکمل} درج ذیل روپ اختیار کرتی ہے۔
\begin{align*}
\int_C f\dif z=\int_{C_1} f\dif z+\int_{C^*_2}f\dif z=0
\end{align*}
%
\begin{figure}
\centering
\begin{subfigure}{0.5\textwidth}
\centering
\begin{tikzpicture}
\draw[->-=0.25,->-=0.75,rotate around={-20:(0,0)}] (0,0) circle (1.5cm and 1cm);
\draw(-110:1.3)node[]{$C^*_2$};
\draw(70:1.3)node[]{$C_1$};
\draw(-20:1.5)node[ocirc]{}node[right]{$z_1$};
\draw(-20:-1.5)node[ocirc]{}node[left]{$z_2$};
\end{tikzpicture}
\caption*{(الف) مساوات \حوالہ{مساوات_مخلوط_تکمل_بند_راہ_ٹکڑے_الف}}
\end{subfigure}%
\begin{subfigure}{0.5\textwidth}
\centering
\begin{tikzpicture}
\draw[->-=0.25,-<-=0.75,rotate around={-20:(0,0)}] (0,0) circle (1.5cm and 1cm);
\draw(-110:1.3)node[]{$C_2$};
\draw(70:1.3)node[]{$C_1$};
\draw(-20:1.5)node[ocirc]{}node[right]{$z_1$};
\draw(-20:-1.5)node[ocirc]{}node[left]{$z_2$};
\end{tikzpicture}
\caption*{(ب مساوات \حوالہ{مساوات_مخلوط_تکمل_بند_راہ_ٹکڑے_ب}}
\end{subfigure}%
\caption{دو نقطوں کے درمیان دو مختلف راہ}
\label{شکل_مخلوط_تکمل_مختلف_راہ}
\end{figure}
یوں
\begin{align}\label{مساوات_مخلوط_تکمل_بند_راہ_ٹکڑے_الف}
-\int_{C^*_2}f\dif z=\int_{C_1} f\dif z
\end{align}
ہو گا۔\عددی{C^*_2} پر تکمل کی سمت الٹ کرنے سے تکمل کی قیمت \عددی{-1} سے ضرب ہو گی۔یوں
\begin{align}\label{مساوات_مخلوط_تکمل_بند_راہ_ٹکڑے_ب}
\int_{C_2}f\dif z=\int_{C_1} f\dif z
\end{align}
حاصل ہو گا (شکل \حوالہ{شکل_مخلوط_تکمل_مختلف_راہ}-ب)۔اس طرح اگر \عددی{D} میں \عددی{f} تحلیلی ہو اور \عددی{D} میں دو نقطوں کے درمیان \عددی{C_1} اور \عددی{C_2} کوئی بھی راہ ہوں جن پر کوئی نقطہ مشترک نہ ہو تب ان راہ پر مساوات \حوالہ{مساوات_مخلوط_تکمل_بند_راہ_ٹکڑے_ب} درست ہو گی۔

اگر ان راہ \عددی{C_1} اور \عددی{C_2} میں محدود تعداد کے نقطے مشترک ہوں (شکل \حوالہ{شکل_مخلوط_تکمل_مختلف_راہ_تبدیل}-الف) تب ہر قریبی مشترک نقطوں کی جوڑی کے مابین چونکہ مساوات \حوالہ{مساوات_مخلوط_تکمل_بند_راہ_ٹکڑے_ب} قابل اطلاق ہے لہٰذا ان پوری راہ \عددی{C_1} اور \عددی{C_2} کے لئے  بھی مساوات \حوالہ{مساوات_مخلوط_تکمل_بند_راہ_ٹکڑے_ب} درست ہو گی۔
\begin{figure}
\centering
\begin{subfigure}{0.5\textwidth}
\centering
\begin{tikzpicture}
\draw[->-=0.5](0,0) to [out=30,in=135](1,0.5) to [out=-45,in=-110] (2,1.5) to [out=70,in=170](3,2);
\draw[dashed,->-=0.5](0,0)node[ocirc,solid]{}node[left]{$z_1$} to [out=-10,in=-120](1,0.5) to [out=70,in=160] (2,1.5) to [out=-20,in=-100](3,2)node[ocirc,solid]{}node[right]{$z_2$};
\draw(1.5,0.75)node[below right]{$C_1$};
\draw(1.4,1.1)node[above left]{$C_2$};
\end{tikzpicture}
\caption*{(الف) دونوں راہ محدود نقطوں پر ایک دوسرے کو قطع کرتی ہیں}
\end{subfigure}%
\begin{subfigure}{0.5\textwidth}
\centering
\begin{tikzpicture}
\draw[->-=0.5](0,0) to [out=-20,in=-100]node[pos=0.5,below]{$C_2$}(3,2);
\draw[->-=0.5,dashed](0,0) to [out=0,in=-130](3,2);
\draw[->-=0.5,dashed](0,0) to [out=15,in=180](3,2);
\draw[->-=0.5,dashed](0,0) to [out=30,in=135]node[pos=0.5,left,solid]{$C_1$}(3,2);
\draw(0,0)node[ocirc]{}node[left]{$z_1$};
\draw(3,2)node[ocirc]{}node[right]{$z_2$};
\end{tikzpicture}
\caption*{(ب) راہ کی مسلسل تبدیلی}
\end{subfigure}%
\caption{دو نقطوں کے مابین مختلف طرز کی راہ}
\label{شکل_مخلوط_تکمل_مختلف_راہ_تبدیل}
\end{figure}

درحقیقت کسی بھی دو نقطوں \عددی{z_1} اور \عددی{z_2} کے درمیان کسی بھی دو راہ، جو مکمل طور  پر سادہ تعلق دائرہ کار \عددی{D} میں ہوں جہاں \عددی{f(z)} تحلیلی ہے،  کے لئے مساوات \حوالہ{مساوات_مخلوط_تکمل_بند_راہ_ٹکڑے_ب} درست ہو گا۔ایسی صورت میں ہم کہتے ہیں کہ \عددی{z_1} تا \عددی{z_2} تفاعل \عددی{f(z)} کے تکمل کی قیمت \عددی{D} میں \موٹا{راہ کی غیر تابع}\فرہنگ{تکمل!راہ کے غیر تابع} (یا راہ سے آزاد) ہے ۔(ظاہر ہے کہ ایسی تکمل کی قیمت \عددی{z_1} اور \عددی{z_2} پر منحصر ہو گی۔)

ہم تصور کر سکتے ہیں کہ راہ \عددی{C_1} کو مسلسل تبدیل کرتے ہوئے راہ \عددی{C_2} حاصل کی گئی ہے (شکل \حوالہ{شکل_مخلوط_تکمل_مختلف_راہ_تبدیل}-ب)۔یوں ایسے تکمل میں، راہ کے سر  \عددی{z_1} اور \عددی{z_2} تبدیل کیے بغیر، تکمل کی راہ یوں مسلسل تبدیل کرنے سے کہ ایسے نقطہ سے نہ گزرا جائے  جہاں \عددی{f(z)} غیر تحلیلی ہو، تکمل کی قیمت تبدیل نہیں ہو گی۔    اس حقیقت کو \اصطلاح{تبدیلی راہ کا اصول}\فرہنگ{اصول!تبدیلی راہ}\فرہنگ{راہ!تبدیلی کا اصول}\حاشیہب{principle of deformation of path}\فرہنگ{path!principle of deformation} کہتے ہیں۔

مضرب تعلق دائرہ کار \عددی{D^*} کو یوں کاٹا جا سکتا ہے کہ حاصل دائرہ کار (یعنی \عددی{D^*} ماسوائے  ان نقطوں کے جو ایک کٹ یا ایک سے زیادہ کٹ پر ہوں) سادہ تعلق دائرہ کار ہو۔دوہرا تعلق دائرہ کار \عددی{D^*} پر ہمیں ایک عدد کٹ \عددی{\tilde{C}} درکار ہو گا (شکل \حوالہ{شکل_مخلوط_تکمل_دوہرا_تعلق_دائرہ_کار_کٹ}-الف)۔اگر  دائرہ کار \عددی{D^*}،  راہ \عددی{C_1} اور \عددی{C_2} پر \عددی{f(z)} تحلیلی ہو تب چونکہ \عددی{C_1}، \عددی{C_2} اور \عددی{\tilde{C}} سادہ تعلق دائرہ کار کو گھیرتے ہیں لہٰذا مسئلہ کوشی کے تحت راہ \عددی{C_1}، \عددی{C_2} اور \عددی{\tilde{C}} پر،  شکل \حوالہ{شکل_مخلوط_تکمل_دوہرا_تعلق_دائرہ_کار_کٹ}-الف میں تیر کی نشان سے دکھائے گئے رخ، \عددی{f(z)} کے تکمل کی قیمت صفر ہو گی۔ چونکہ ہم \عددی{\tilde{C}} پر دونوں رخ تکمل لیتے ہیں جن کا مجموعہ صفر ہو گا لہٰذا ہمیں
\begin{align}\label{مساوات_مخلوط_تکمل_دوہرا_تعلق_دائرہ_کار_مسئلہ_کوشی_الف}
\int_{C_1}f(z)\dif z+\int_{C_2}f(z)\dif z=0
\end{align}
حاصل ہوتا ہے جہاں ایک بند راہ پر گھڑی کی رخ اور دوسری راہ پر گھڑی کی الٹ رخ تکمل حاصل کیا جاتا ہے۔
%
\begin{figure}
\centering
\begin{subfigure}{0.5\textwidth}
\centering
\begin{tikzpicture}
%\draw[thick](-2,-2) grid (2,2);
%\draw[thin,gray,step=0.1](-2,-2) grid (2,2);
%
\fill[gray!20!white,rotate around={20:(0,0)}](0,0) circle (1.5cm and 1cm);
\fill[white,rotate around={20:(0,0)}] (0,0) circle (0.75cm and 0.5cm);
\draw[-<-=0.4,-<-=0.9,rotate around={20:(0,0)}] (0,0) circle (0.75cm and 0.5cm); 
\draw[->-=0.1,->-=0.4,->-=0.9,rotate around={20:(0,0)}] (0,0) circle (1.5cm and 1cm); 
\draw(0.3,0)node[]{$C_2$};
\draw(1.5,-0.5)node[]{$C_1$};
\draw[very thick] (-0.1,1)node[above]{$\tilde{C}$}--(0,0.5);
\draw[-latex] (-0.3,0.5) to [out=20,in=-70](-0.15,0.75) to [out=110,in=20] (-0.5,0.8);
\draw[-latex](0.2,1) to [out=-160,in=110] (0.05,0.75) to [out=-70,in=160](0.3,0.6);
\end{tikzpicture}
\caption*{(الف) دوہرا تعلق دائرہ کار}
\end{subfigure}%
\begin{subfigure}{0.5\textwidth}
\centering
\begin{tikzpicture}
\fill[gray!20!white,rotate around={20:(0,0)}](0,0) circle (1.5cm and 1cm);
\fill[white,rotate around={20:(0,0)}] (0,0) circle (0.75cm and 0.5cm);
\draw[->-=0.75,rotate around={20:(0,0)}](0,0) circle (0.75cm and 0.5cm);
\draw[->-=0.75,rotate around={20:(0,0)}](0,0) circle (1.5cm and 1cm);
\draw(-80:1)node[below]{$C_1$};
\draw(-80:0.5)node[above]{$C_2$};
\end{tikzpicture}
\caption*{(ب) مساوات \حوالہ{مساوات_مخلوط_تکمل_دوہرا_تعلق_دائرہ_کار_مسئلہ_کوشی_ب} میں تکمل کی راہ}
\end{subfigure}%
\caption{دوہرا تعلق دائرہ کار}
\label{شکل_مخلوط_تکمل_دوہرا_تعلق_دائرہ_کار_کٹ}
\end{figure}

مساوات \حوالہ{مساوات_مخلوط_تکمل_دوہرا_تعلق_دائرہ_کار_مسئلہ_کوشی_الف} کو درج ذیل بھی لکھا جا سکتا ہے
\begin{align}\label{مساوات_مخلوط_تکمل_دوہرا_تعلق_دائرہ_کار_مسئلہ_کوشی_ب}
\int_{C_1} f(z)\dif z=\int_{C_2}f(z)\dif z
\end{align}
جہاں دونوں بند راہ پر تکمل گھڑی کی ایک ہی رخ حاصل کیا جاتا ہے (شکل \حوالہ{شکل_مخلوط_تکمل_دوہرا_تعلق_دائرہ_کار_کٹ}-ب)۔ یاد رہے کہ مساوات \حوالہ{مساوات_مخلوط_تکمل_دوہرا_تعلق_دائرہ_کار_مسئلہ_کوشی_ب} اس صورت درست ہو گا جب \عددی{C_1} اور \عددی{C_2} کی ہر نقطہ پر اور ان کی گھیرے ہوئے دائرہ کار پر \عددی{f(z)} تحلیلی ہو۔ 

زیادہ پیچیدہ دائرہ کار میں ایک سے زیادہ کٹ درکار ہوں گے۔ان کٹ کو لگانے کا بنیادی اصول وہی رہے گا۔مثلاً تین تعلقی دائرہ کار (شکل \حوالہ{شکل_مخلوط_تکمل_تین_تعلق_دائرہ_کار}) کے لئے
\begin{align*}
\int_{C_1}f(z)\dif z+\int_{C_2}f(z)\dif z+\int_{C_3}f(z)\dif z=0
\end{align*}
لکھا جا سکتا ہے جہاں \عددی{C_2} اور \عددی{C_3} پر تکمل ایک ہی رخ حاصل کیا جائے گا جبکہ \عددی{C_1} پر تکمل ان کی الٹ رخ حاصل کیا جائے گا۔
\begin{figure}
\centering
\begin{tikzpicture}
%\draw[thick](-2,-2) grid (2,2);
%\draw[thin,gray,step=0.1](-2,-2) grid (2,2);
%
\fill[gray!20!white,rotate around={20:(0,0)}](0,0) circle (2cm and 1.25cm);
\fill[white,rotate around={20:(20:1)}](20:1) circle (0.75cm and 0.5cm);
\fill[white,rotate around={20:(20:-1)}](20:-1) circle (0.75cm and 0.5cm);
\draw[->-=0.75,rotate around={20:(0,0)}] (0,0) circle (2cm and 1.25cm);
\draw[-<-=0.75,rotate around={20:(20:1)}] (20:1) circle (0.75cm and 0.5cm);
\draw[-<-=0.75,rotate around={20:(20:-1)}] (20:-1) circle (0.75cm and 0.5cm);
\draw(-90:1.25)node[below right]{$C_1$};
\draw(20:1)++(-90:0.5)node[below]{$C_2$};
\draw(20:-1)++(-90:0.5)node[below right]{$C_3$};
\draw[very thick] (1.1,1.3) to [out=-110,in=110]node[pos=0.5,left]{$\tilde{C}_1$} (1,0.85);
\draw[very thick] (-1.5,0.5) to [out=-70,in=90]node[pos=0.2,right]{$\tilde{C}_2$} (-1.35,0);
\end{tikzpicture}
\caption{تین تعلقی دائرہ کار}
\label{شکل_مخلوط_تکمل_تین_تعلق_دائرہ_کار}
\end{figure}

سادہ بند راہ کو کبھی کبھار \اصطلاح{خط ارتفاع}\فرہنگ{خط!ارتفاع}\حاشیہب{contour}\فرہنگ{contour} بھی کہتے ہیں اور ایسی راہ پر تکمل کو \اصطلاح{ارتفاعی تکمل}\فرہنگ{ارتفاعی تکمل}\فرہنگ{تکمل!ارتفاعی}\حاشیہب{contour integral}\فرہنگ{integral!contour}\فرہنگ{contour!integral} کہتے ہیں۔

%=====================
\ابتدا{مثال}\quad
فرض کریں کہ \عددی{C_1} اکائی دائرہ \عددی{\abs{z}=1} ہے جبکہ \عددی{C_2} دائرہ \عددی{\abs{z}=\tfrac{1}{2}} ہے۔تب مساوات \حوالہ{مساوات_مخلوط_تکمل_دوہرا_تعلق_دائرہ_کار_مسئلہ_کوشی_ب} سے
\begin{align*}
\int_{C_2}\frac{\dif z}{z}=\int_{C_1}\frac{\dif z}{z}=i2\pi\quad \quad \quad \text{\RL{(مثال \حوالہ{مثال_مخلوط_تکمل_دائرے_پر_تکمل})}}
\end{align*}
ہو گا جہاں دونوں دائروں پر تکمل گھڑی کی الٹ رخ حاصل کیا گیا ہے۔
\انتہا{مثال}
%===================
\ابتدا{مثال}\شناخت{مثال_مخلوط_تکمل_طاقت_منفی_اکائی}
مثال \حوالہ{مثال_مخلوط_تکمل_طاقت} کا نتیجہ استعمال کرتے ہوئے تبدیلی راہ کے اصول کے تحت
\begin{align*}
\int_C (z-z_0)^m\dif z=
\begin{cases}
i2\pi & (m=-1)\\
0&(m\ne -1, \text{\RL{عدد صحیح}})
\end{cases}
\end{align*}
ہو گا جہاں \عددی{C} ایسا کوئی بھی خط ارتفاع ہو سکتا ہے جو نقطہ \عددی{z_0} کو گھیرتا ہو اور تکمل کو \عددی{C} پر گھڑی کی الٹ رخ ایک مرتبہ حاصل کیا جاتا ہے۔ 
\انتہا{مثال}
%=======================

\حصہء{سوالات}

%====================
\ابتدا{سوال}\quad
مسئلہ کوشی کی تصدیق \عددی{\int_C z^2\dif z} کے لئے کریں جہاں \عددی{C} وہ تکون ہے جس کے کونے \عددی{0}، \عددی{2} اور \عددی{i2} ہیں۔
\انتہا{سوال}
%=======================
\ابتدا{سوال}\quad
دکھائیں کہ اکائی دائرے کے گرد \عددی{\tfrac{1}{z^3}} کا تکمل صفر کے برابر ہے۔کیا یہ نتیجہ مسئلہ کوشی سے اخذ کیا جا سکتا ہے؟\\
جواب:\quad چونکہ \عددی{z=0} پر تفاعل \عددی{\tfrac{1}{z^3}} غیر تحلیلی ہے اور اکائی دائرہ اس نقطے کو گھیرتی ہے لہٰذا یہ نتیجہ مسئلہ کوشی سے اخذ نہیں کیا جا سکتا ہے۔
\انتہا{سوال}
%=======================
\ابتدا{سوال}\quad
کس سادہ بند راہ پر \عددی{\tfrac{1}{z}} کا تکمل صفر کے برابر ہو گا؟\\
جواب:\quad
وہ راہ جو \عددی{z=0} کو نہ گھیرتی ہو۔
\انتہا{سوال}
%====================
سوال \حوالہ{سوال_مخلوط_تکمل_مسئلہ_کوشی_الف} تا سوال \حوالہ{سوال_مخلوط_تکمل_مسئلہ_کوشی_ب} اکائی دائرے کے گرد ایک مرتبہ گھڑی کی الٹ رخ دیے گئے تفاعل کے تکمل کی قیمت حاصل  کریں۔ہر سوال میں بتلائیں کہ آیا اس سوال میں مسئلہ کوشی کا اطلاق ہو گا؟ 

%===================
\ابتدا{سوال}\شناخت{سوال_مخلوط_تکمل_مسئلہ_کوشی_الف}\quad
$f(z)=\tfrac{1}{z^4}$\\
جواب:\quad
\عددی{0} ؛ جی نہیں
\انتہا{سوال}
%======================
\ابتدا{سوال}\quad
$f(z)=e^{-z}$\\
جواب:\quad
\عددی{0} ؛ جی ہاں
\انتہا{سوال}
%======================
\ابتدا{سوال}\quad
$f(z)=\abs{z}$\\
جواب:\quad
\عددی{0}؛ جی نہیں
\انتہا{سوال}
%======================
\ابتدا{سوال}\شناخت{سوال_مخلوط_تکمل_مسئلہ_کوشی_پ}\quad
$f(z)=\text{\RL{خیالی $z$}}$\\
جواب:\quad
\عددی{-\pi}؛ جی نہیں
\انتہا{سوال}
%======================
\ابتدا{سوال}\شناخت{سوال_مخلوط_تکمل_مسئلہ_کوشی_ت}\quad
$f(z)=\text{\RL{حقیقی $z$}}$\\
جواب:\quad
\عددی{i\pi}؛ جی نہیں
\انتہا{سوال}
%======================
\ابتدا{سوال}\quad
$f(z)=\tanh z$\\
جواب:\quad
\عددی{0}؛ جی ہاں
\انتہا{سوال}
%======================
\ابتدا{سوال}\quad
$f(z)=\tfrac{1}{z^2+2}$\\
جواب:\quad
\عددی{0}؛ جی ہاں
\انتہا{سوال}
%======================
\ابتدا{سوال}\quad
$f(z)=\tfrac{1}{\bar{z}}$\\
جواب:\quad
\عددی{0}؛ جی  نہیں
\انتہا{سوال}
%======================
\ابتدا{سوال}\شناخت{سوال_مخلوط_تکمل_مسئلہ_کوشی_ب}\quad
$f(z)=z^2\sec z$\\
جواب:\quad
\عددی{0}؛ جی  ہاں
\انتہا{سوال}
%======================
\ابتدا{سوال}\quad
مسئلہ کوشی کا اطلاق \عددی{f(z)=z} پر کرتے ہوئے سوال \حوالہ{سوال_مخلوط_تکمل_مسئلہ_کوشی_پ} سے سوال \حوالہ{سوال_مخلوط_تکمل_مسئلہ_کوشی_ت} کا جواب حاصل کریں۔
\انتہا{سوال}
%=======================
\ابتدا{سوال}\شناخت{سوال_مخلوط_تکمل_قطع_مکافی}\quad
تبدیلی راہ کا اصول اور 
\begin{align*}
\frac{2z-1}{z^2-z}=\frac{1}{z}+\frac{1}{z-1}
\end{align*}
 استعمال کرتے ہوئے درج ذیل حاصل کریں جہاں \عددی{C} کو شکل \حوالہ{سوال_مخلوط_تکمل_قطع_مکافی} میں دکھایا گیا ہے۔
\begin{align*}
\int_C \frac{2z-1}{z^2-z}\dif z=\int_C \frac{\dif z}{z}+\int_C \frac{\dif z}{z-1}=i4\pi
\end{align*}
%
\begin{figure}
\centering
\begin{tikzpicture}
\draw(-0.75,0)--(2,0)node[right]{$x$};
\draw(0,-0.75)--(0,0.75)node[left]{$y$};
\draw[->-=0.125](0.5,0) circle (1cm and 0.5cm);
\draw(1.2,0.6)node{$C$};
\draw(0,0)node[ocirc]{};
\draw(1,0)node[ocirc]{}node[below]{$1$};
\end{tikzpicture}
\caption{شکل برائے سوال \حوالہ{سوال_مخلوط_تکمل_قطع_مکافی}}
\label{شکل_سوال_مخلوط_تکمل_قطع_مکافی}
\end{figure}
\انتہا{سوال}
%====================
\ابتدا{سوال}\quad
تفاعل \عددی{f(z)=\tfrac{\bar{z}}{\abs{z}}} کا تکمل  گھڑی کی الٹ رخ (الف) دائرہ \عددی{\abs{z}=2} اور (ب) دائرہ \عددی{\abs{z}=4} پر حاصل کریں۔کیا (الف) کے جواب سے تبدیلی راہ کے قاعدہ کی مدد سے (ب) کا جواب حاصل کیا جا سکتا ہے؟\\
جواب:\quad (الف) \عددی{i2\pi}، (ب) \عددی{i2\pi}؛ جی نہیں
\انتہا{سوال}
%=======================
سوال \حوالہ{سوال_مخلوط_تکمل_جزوی_کسر_الف} تا سوال \حوالہ{سوال_مخلوط_تکمل_جزوی_کسر_ب} میں تکمل کی قیمت تلاش کریں۔جہاں ضرورت ہو وہاں تفاعل کو جزوی کسر کی صورت میں لکھیں۔

%==============
\ابتدا{سوال}\شناخت{سوال_مخلوط_تکمل_جزوی_کسر_الف}\quad
$\int_C \tfrac{\dif z}{z},\quad C:\abs{z-2}=1\quad \text{\RL{گھڑی کی رخ}}$\\
جواب:\quad
$0$
\انتہا{سوال}
%===================
\ابتدا{سوال}\quad
$\int_C \tfrac{z^2-z-1}{z^3-z^2}\dif z,\quad \text{(الف)}\,C:\abs{z}=2,\quad \text{(ب)}\,C:\abs{z}=\tfrac{1}{2}\,\,\text{\RL{گھڑی کی الٹ رخ}}$\\
جواب:\quad
(الف) $i2\pi$، (ب) $0$
\انتہا{سوال}
%=================
\ابتدا{سوال}\quad
$\int_C \tfrac{\dif z}{z^2-1},\quad \text{(الف)}\, C:\abs{z}=2,\,\, \text{(ب)}\,C:\abs{z-1}=1,\,\text{\RL{گھڑی کی رخ}}$\\
جواب:\quad
(الف) $0$، (ب) $-i\pi$
\انتہا{سوال}
%====================
\ابتدا{سوال}\quad
$\int_C \tfrac{z}{z^2+1}\dif z,\quad \text{(الف)}\, C:\abs{z}=2,\,\, \text{(ب)}\,C:\abs{z+i}=1,\,\text{\RL{گھڑی کی الٹ رخ}}$\\
جواب:\quad
(الف) $i2\pi$، (ب) $i\pi$
\انتہا{سوال}
%====================
\ابتدا{سوال}\quad
$\int_C \tfrac{\dif z}{z^2+1},\quad \text{(الف)}\, C:\abs{z+i}=1,\,\, \text{(ب)}\,C:\abs{z-i}=1,\,\text{\RL{گھڑی کی الٹ رخ}}$\\
جواب:\quad
(الف) $-\pi$، (ب) $\pi$
\انتہا{سوال}
%====================
\ابتدا{سوال}\quad
$\int_C \tfrac{e^z}{z}\dif z,\quad \text{(الف)}\, C:\abs{z}=2,\,\, \text{(ب)}\,C:\abs{z}=1,\,\text{\RL{گھڑی کی الٹ رخ}}$\\
جواب:\quad
(الف) $0$، (ب) $0$
\انتہا{سوال}
%====================
\ابتدا{سوال}\quad
$\int_C \tfrac{\cos z}{z^2}\dif z,\quad \text{(الف)}\, C:\abs{z-i2}=1,\,\text{\RL{گھڑی کی الٹ رخ}}$\\
جواب:\quad
$0$
\انتہا{سوال}
%====================
\ابتدا{سوال}\quad
$\int_C \tfrac{3z+1}{z^3-z}\dif z,\quad \text{(الف)}\, C:\abs{z}=\tfrac{1}{2},\,\, \text{(ب)}\,C:\abs{z}=2,\,\text{\RL{گھڑی کی الٹ رخ}}$\\
جواب:\quad
(الف) $-i2\pi$، (ب) $-i2\pi$
\انتہا{سوال}
%====================
\ابتدا{سوال}\شناخت{سوال_مخلوط_تکمل_جزوی_کسر_ب}\quad
$\int_C \tfrac{\dif z}{z^4+4z^2},\quad \text{(الف)}\, C:\abs{z}=\tfrac{3}{2},\,\, \text{(ب)}\,C:\abs{z}=1,\,\text{\RL{گھڑی کی الٹ رخ}}$\\
جواب:\quad
(الف) $i2\pi$، (ب) $0$
\انتہا{سوال}
%====================

\حصہ{خطی تکمل کی قیمت کا حصول بذریعہ غیر قطعی تکمل}\شناخت{حصہ_مخلوط_تکمل_خطی_تکمل_کا_حصول_بذریعہ_غیر_قطعی_تکمل}
کوشی مسئلہ تکمل استعمال کرتے ہوئے ہم دکھانا چاہتے ہیں کہ بہت سارے مخلوط خطی تکمل کو ایک سادہ طریقہ کار، یعنی غیر قطعی تکمل،  سے حاصل کیا جا سکتا ہے۔

فرض کریں کہ سادہ تعلق دائرہ کار \عددی{D} میں \عددی{f(z)} تحلیلی ہے اور \عددی{D} میں \عددی{z_0} ایک مقررہ نقطہ ہے۔تب \عددی{z_0} اور \عددی{z} کے درمیان  \عددی{D} میں تمام راہ پر تکمل 
\begin{align*}
\int_{z_0}^z f(z^*)\dif z^*
\end{align*}
\عددی{z} کا تفاعل ہو گا لہٰذا ہم 
\begin{align}\label{مساوات_مخلوط_تکمل_بذریعہ_غیر_قطعی_تکمل_الف}
F(z)=\int_{z_0}^z f(z^*)\dif z^*
\end{align}
لکھ سکتے ہیں۔

آئیں ثابت کرتے ہیں کہ \عددی{D} میں \عددی{F(z)} متغیرہ \عددی{z} کا تحلیلی تفاعل ہے اور \عددی{F'(z)=f(z)} ہے۔

ہم \عددی{z} کو مقررہ رکھتے ہیں۔چونکہ \عددی{D}دائرہ کار ہے لہٰذا \عددی{z} کی پڑوس \عددی{N} بھی \عددی{D} کا حصہ ہو گی۔ ہم \عددی{N} میں نقطہ \عددی{z+\Delta z} یوں منتخب کرتے ہیں کہ وہ قطع جس کے سر \عددی{z} اور \عددی{z+\Delta z} ہوں از خود \عددی{N} کا اور یوں \عددی{D} کا حصہ ہو۔ مساوات \حوالہ{مساوات_مخلوط_تکمل_بذریعہ_غیر_قطعی_تکمل_الف} سے ہم درج ذیل حاصل کرتے ہیں
\begin{align*}
F(z+\Delta z)-F(z)=\int_{z_0}^{z+\Delta z}f(z^*)\dif z^*-\int_{z_0}^z f(z^*)\dif z^*=\int_z^{z+\Delta z} f(z^*)\dif z^*
\end{align*}
جہاں \عددی{z} تا \عددی{z+\Delta z} تکمل کو اس قطع پر حاصل کیا جا سکتا ہے (شکل \حوالہ{شکل_مخلوط_تکمل_بذریعہ_غیر_قطعی})۔ 
\begin{figure}
\centering
\begin{tikzpicture}
\draw(0,0.5) to [out=-90,in=180](1,0) to [out=0,in=-135](2,0.5) to [out=45,in=-90](4,1.5) to [out=90,in=0] (3,2) to [out=180,in=30](1,1.5) to [out=-150,in=90] (0,0.5);
\draw(1,1.25)node[]{$D$};
\draw(1,0.5)node[ocirc]{}node[below]{$z_0$} to [out=50,in=-170](1.5,1) to [out=10,in=-70] (2,1.5)node[ocirc]{}node[left]{$z$}--(3.25,1.75)node[ocirc]{}node[below]{$z+\Delta z$};
\end{tikzpicture}
\caption{تکمل کی راہ}
\label{شکل_مخلوط_تکمل_بذریعہ_غیر_قطعی}
\end{figure}
چونکہ \عددی{z} مقررہ ہے لہٰذا 
\begin{align*}
\frac{F(z+\Delta z)-F(z)}{\Delta z}-f(z)=\frac{1}{\Delta z}\int_{z}^{z+\Delta z} [f(z^*)-f(z)]\dif z^*
\end{align*}
لکھا جا سکتا ہے جہاں
\begin{align*}
-\frac{1}{\Delta z}\int_z^{z+\Delta z} f(z)\dif z^*=-\frac{f(z)}{\Delta z}\int_z^{z+\Delta z} \dif z^*=-f(z)
\end{align*}
ہو گا۔ اب \عددی{f(z)} استمراری ہے لہٰذا  کسی بھی دیے گئے \عددی{\epsilon>0} کے لئے ہم ایسا \عددی{\sigma>0} حاصل کر سکتے ہیں کہ
\begin{align*}
\text{\RL{ہو گا۔}}\quad \abs{f(z^*)-f(z)}<\epsilon, \quad \text{\RL{ہو تب}}\quad \abs{z^*-z}<\sigma \quad \text{جب}
\end{align*}
نتیجتاً اگر \عددی{\abs{\Delta z}<\sigma} ہو تب
\begin{align*}
\abs{\frac{F(z+\Delta z)-F(z)}{\Delta z}-f(z)}&=\frac{1}{\abs{\Delta z}}\abs{\int_z^{z+\Delta z}[f(z^*)-f(z)]\dif z^*}\\
&<\frac{\epsilon}{\abs{\Delta z}} \abs{\int_z^{z+\Delta z} \dif z^*}=\epsilon
\end{align*}
ہو گا لہٰذا درج ذیل لکھا جا سکتا ہے۔
\begin{align}\label{مساوات_مخلوط_تکمل_بذریعہ_غیر_قطعی_تکمل_ب}
F'(z)=\lim_{\Delta z\to 0} \frac{F(z+\Delta z)-F(z)}{\Delta z}=f(z)
\end{align}
اب مساوات \حوالہ{مساوات_مخلوط_تکمل_بذریعہ_غیر_قطعی_تکمل_الف} سے ہم دیکھتے ہیں کہ \عددی{z_0} کی جگہ کوئی دوسرا مقررہ نقطہ منتخب کرنے سے  تفاعل \عددی{F(z)} کے ساتھ ایک مستقل جمع ہو گا۔مساوات \حوالہ{مساوات_مخلوط_تکمل_بذریعہ_غیر_قطعی_تکمل_ب} سے ہم دیکھتے ہیں کہ \عددی{F(z)} تفاعل \عددی{f(z)} کا تفرق یا غیر قطعی تکمل ہے، جس کو درج ذیل لکھا جاتا ہے،
\begin{align*}
F(z)=\int f(z)\dif z
\end{align*}
یعنی  \عددی{D} میں \عددی{F(z)} تحلیلی تفاعل ہے جس کا تفرق \عددی{f(z)} ہے۔

اگر \عددی{F'(z)=f(z)} اور \عددی{G'(z)=f(z)} ہوں تب \عددی{D} میں \عددی{F'(z)-G'(z)\equiv 0} ہو گا۔یوں تفاعل \عددی{F(z)-G(z)} ایک مستقل (سوال \حوالہ{سوال_تحلیلی_تفرق_صفر_مطلب_مستقل_تفاعل}) ہو گا۔یوں غیر قطعی تکمل \عددی{F(z)} اور \عددی{G(z)} میں صرف ایک مستقل کا فرق ہو سکتا ہے۔اب مساوات \حوالہ{مساوات_مخلوط_تکمل_بذریعہ_غیر_قطعی_تکمل_الف} کو مد نظر رکھتے ہوئے \عددی{D} میں نقطہ \عددی{a} اور \عددی{b} اور   \عددی{D} میں \عددی{a} تا \عددی{b}  کسی بھی راہ کے لئے حقیقی قطعی تکمل کی طرح
\begin{align*}
\int_a^b f(z)\dif z=\int_{z_0}^b f(z)\dif z-\int_{z_0}^a f(z)\dif z=F(b)-F(a)
\end{align*}
لکھا جا سکتا ہے، پس اتنا ضروری ہے کہ تکمل کی راہ سادہ تعلق دائرہ کار \عددی{D} میں پائی جاتی ہو جہاں \عددی{f(z)} تحلیلی ہے۔ 

ہم متذکرہ بالا نتیجہ کو درج ذیل مسئلہ میں بیان کرتے ہیں۔

%==================
\ابتدا{مسئلہ}\quad \موٹا{تکمل کا حصول بذریعہ غیر قطعی تکمل}\\
اگر سادہ تعلق دائرہ کار \عددی{D} میں \عددی{f(z)} تحلیلی ہو تب \عددی{D} میں \عددی{f(z)} کا غیر قطعی تکمل موجود ہو گا، یعنی ایسا تحلیلی تفاعل \عددی{F(z)} کہ \عددی{D} میں \عددی{F'(z)=f(z)} ہو، اور \عددی{D} میں نقطہ \عددی{a} اور نقطہ \عددی{b} کے درمیان \عددی{D} میں ہر راہ کے لئے درج ذیل ہو۔
\begin{align}\label{مساوات_مخلوط_تکمل_بذریعہ_غیر_قطعی_تکمل}
\int_a^b f(z)\dif z=F(b)-F(a)
\end{align}
\انتہا{مسئلہ}
%============================

یہ مسئلہ مخلوط خطی تکمل کا حصول بذریعہ غیر قطعی تکمل ممکن بناتا ہے۔یاد رہے کہ چونکہ \عددی{F(z)} ایک مستقل جمعی جزو کے علاوہ یکتا ہے لہٰذا مساوات \حوالہ{مساوات_مخلوط_تکمل_بذریعہ_غیر_قطعی_تکمل} میں ہم \عددی{D} میں  \عددی{f(z)} کا کوئی بھی غیر قطعی تکمل \عددی{F(z)}  لے سکتے ہیں۔

%=============
\ابتدا{مثال}\quad
\begin{align*}
\int_i^{1+i4} z^2\dif z=\big[\frac{z^3}{3}\big]_i^{1+i4}=\frac{1}{3}[(1+i4)^3-i^3]=-\frac{47}{3}-i17
\end{align*}
\انتہا{مثال}
%=======================
\ابتدا{مثال}
\begin{align*}
\int_i^{\frac{\pi}{2}} \cos z\dif z= \sin z \big|_{i}^{\frac{\pi}{2}}=\sin \frac{\pi}{2}-\sin i=1-i\sinh 1
\end{align*}
\انتہا{مثال}
%========================

\حصہء{سوالات}
سوال \حوالہ{سوال_مخلوط_تکمل_بذریعہ_غیر_قطعی_تکمل_الف} تا سوال \حوالہ{سوال_مخلوط_تکمل_بذریعہ_غیر_قطعی_تکمل_ب} میں تکمل کی قیمت تلاش کریں۔

%===============
\ابتدا{سوال}\شناخت{سوال_مخلوط_تکمل_بذریعہ_غیر_قطعی_تکمل_الف}\quad
$\int_{i}^{1+i2} z\dif z$\\
جواب:\quad
$-1+i2$
\انتہا{سوال}
%==========================
\ابتدا{سوال}\quad
$\int_{i}^{2} z^2\dif z$\\
جواب:\quad
$\tfrac{1}{3}(8+i)$
\انتہا{سوال}
%==========================
\ابتدا{سوال}\quad
$\int_{i}^{1} (z-1)^2\dif z$\\
جواب:\quad
$-\tfrac{2}{3}(1+i)$
\انتہا{سوال}
%==========================
\ابتدا{سوال}\quad
$\int_{1+i}^{1-i} z^3\dif z$\\
جواب:\quad
$0$
\انتہا{سوال}
%==========================
\ابتدا{سوال}\quad
$\int_{1}^{1+i\pi} e^z\dif z$\\
جواب:\quad
$-2e$
\انتہا{سوال}
%==========================
\ابتدا{سوال}\quad
$\int_{i\pi}^{i2\pi} e^{3z}\dif z$\\
جواب:\quad
$\tfrac{2}{3}$
\انتہا{سوال}
%==========================
\ابتدا{سوال}\quad
$\int_{-i}^{i} ze^{z^2}\dif z$\\
جواب:\quad
$0$
\انتہا{سوال}
%==========================
\ابتدا{سوال}\quad
$\int_{1-i\pi}^{1+i\pi} e^{\tfrac{z}{2}}\dif z$\\
جواب:\quad
$i4\sqrt{e}$
\انتہا{سوال}
%==========================
\ابتدا{سوال}\quad
$\int_{0}^{i\pi} \cos z\dif z$\\
جواب:\quad
$i\sinh \pi$
\انتہا{سوال}
%==========================
\ابتدا{سوال}\quad
$\int_{0}^{i\frac{\pi}{2}} \sin z\dif z$\\
جواب:\quad
$1-\cosh \frac{\pi}{2}$
\انتہا{سوال}
%==========================
\ابتدا{سوال}\quad
$\int_{0}^{i\frac{\pi}{2}} z\sin z^2\dif z$\\
جواب:\quad
$\tfrac{1}{2}(1-\cos \frac{\pi^2}{4})$
\انتہا{سوال}
%==========================
\ابتدا{سوال}\quad
$\int_{0}^{i\frac{\pi}{2}}16 z\sin z\dif z$\\
جواب:\quad
$-ie^{-\tfrac{\pi}{2}}\big[2\pi^2e^{\tfrac{\pi}{2}}\sinh \tfrac{\pi}{2}+(-\pi^2+4\pi-8)e^{\pi}+\pi^2+4\pi+8\big]$
\انتہا{سوال}
%==========================
\ابتدا{سوال}\quad
$\int_{-i\pi}^{i\pi}\sin^2 z\dif z$\\
جواب:\quad
$i(\pi-\tfrac{1}{2}\sinh 2\pi)$
\انتہا{سوال}
%==========================
\ابتدا{سوال}\quad
$\int_{1-i}^{1+i}\cos z\dif z$\\
جواب:\quad
$\sin(i+1)+\sin(i-1)$
\انتہا{سوال}
%==========================
\ابتدا{سوال}\quad
$\int_{0}^{i3}\cosh z\dif z$\\
جواب:\quad
$i\sin 3$
\انتہا{سوال}
%==========================
\ابتدا{سوال}\quad
$\int_{i}^{1+i3}\sinh z\dif z$\\
جواب:\quad
$\cosh(1+i3)-\cos 1$
\انتہا{سوال}
%==========================
\ابتدا{سوال}\quad
$\int_{0}^{i3}\sinh z\dif z$\\
جواب:\quad
$\cos 3-1$
\انتہا{سوال}
%==========================
\ابتدا{سوال}\quad
$\int_{i}^{i3}z\sinh z^2\dif z$\\
جواب:\quad
$\tfrac{1}{2}(\cosh 9-\cosh 1)$
\انتہا{سوال}
%==========================
\ابتدا{سوال}\quad
$\int_{-1}^{1}z\cosh z^2\dif z$\\
جواب:\quad
$0$
\انتہا{سوال}
%==========================
\ابتدا{سوال}\شناخت{سوال_مخلوط_تکمل_بذریعہ_غیر_قطعی_تکمل_ب}\quad
$\int_{-i\pi}^{i\pi}z\cosh z\dif z$\\
جواب:\quad
$0$
\انتہا{سوال}
%==========================

\حصہ{کوشی کا کلیہ تکمل}
کوشی کے مسئلہ تکمل کا اہم ترین نتیجہ کوشی کا کلیہ تکمل ہے۔یہ کلیہ اور اس کے کے لازمی شرائط درج ذیل مسئلہ میں پیش کیے گئے ہیں۔

%====================
\ابتدا{مسئلہ}\quad \موٹا{کوشی کا کلیہ تکمل}\فرہنگ{کوشی!کلیہ تکمل}\حاشیہب{Cauchy's integral formula}\فرہنگ{Cauchy!integral formula}\\
فرض کریں کہ سادہ تعلق دائرہ کار \عددی{D} میں \عددی{f(z)} تحلیلی ہے۔تب \عددی{D} میں کسی بھی نقطہ \عددی{z_0} اور \عددی{D} میں کسی بھی بند راہ \عددی{C} جو \عددی{z_0} کو گھیرتا (شکل \حوالہ{شکل_مخلوط_تکمل_کوشی_کلیہ_تکمل}) ہو درج ذیل ہو گا
\begin{align}\label{مساوات_مخلوط_تکمل_کوشی_کلیہ_تکمل_الف}
\int\limits_C \frac{f(z)}{z-z_0}\dif z=i2\pi f(z_0)\quad \quad \text{\RL{کوشی کا کلیہ تکمل}}
\end{align}
جہاں تکمل کو \عددی{C} پر  گھڑی کی الٹ رخ حاصل کیا جاتا ہے۔
\begin{figure}
\centering
\begin{tikzpicture}
\draw(0,0.5) to [out=-90,in=180](1.5,-0.25) to [out=0,in=-135](2.5,0.5) to [out=45,in=-90](4,1.5) to [out=90,in=0] (3,2) to [out=180,in=30](1,1.5) to [out=-150,in=90] (0,0.5);
\draw(3,1.25)node[]{$D$};
\draw[->-=0.75,rotate around={20:(1.5,1)}] (1.5,1) circle (1cm and 0.5 cm);
\draw(1.5,1)++(-90:0.5)node[below]{$C$};
\draw(1.5,1)node[ocirc]{}node[right]{$z_0$};
\end{tikzpicture}
\caption{کوشی کا کلیہ تکمل}
\label{شکل_مخلوط_تکمل_کوشی_کلیہ_تکمل}
\end{figure}
\انتہا{مسئلہ}
%======================== 
\ابتدا{ثبوت}
\عددی{f(z)=f(z_0)+[f(z)-f(z_0)]} لکھ کر  یاد رکھتے ہوئے کہ مستقل کو تکمل سے باہر نکالا جا سکتا ہے ہم درج ذیل لکھ سکتے ہیں۔
\begin{align}\label{مساوات_مخلوط_تکمل_کوشی_کلیہ_تکمل_ب}
\int\limits_C \frac{f(z)}{z-z_0}\dif z=f(z_0)\int\limits_C \frac{\dif z}{z-z_0}+\int\limits_C\frac{f(z)-f(z_0)}{z-z_0}\dif z
\end{align}
مثال \حوالہ{مثال_مخلوط_تکمل_طاقت_منفی_اکائی} کے تحت دائیں ہاتھ پہلا تکمل \عددی{i2\pi f(z_0)} کے برابر ہے۔ یوں مساوات \حوالہ{مساوات_مخلوط_تکمل_کوشی_کلیہ_تکمل_ب} میں دائیں ہاتھ دوسرا تکمل صفر ہونے کی صورت میں مساوات \حوالہ{مساوات_مخلوط_تکمل_کوشی_کلیہ_تکمل_الف} درست ثابت ہو گی۔اب اس تکمل لا متکمل ماسوائے نقطہ \عددی{z_0} کے \عددی{D} میں تحلیلی ہے۔یوں ہم  تکمل کی قیمت بغیر تبدیل کیے \عددی{C} کی جگہ \عددی{z_0} کے گرد ایک چھوٹے دائرے پر تکمل حاصل کر سکتے ہیں (شکل \حوالہ{شکل_مخلوط_تکمل_کوشی_کلیہ_ثبوت})۔چونکہ \عددی{f(z)} تحلیلی ہے لہٰذا یہ استمراری ہے۔یوں کسی بھی دیے گئے \عددی{\epsilon>0} کی صورت میں ہم ایسا \عددی{\sigma>0} تلاش کر سکتے ہیں کہ
\begin{align*}
\text{\RL{ہو گا۔}}\quad \abs{f(z)-f(z_0)}<\epsilon\quad \text{\RL{میں ہر \عددی{z} کے لئے }} \quad \abs{z-z_0}<\sigma \quad \text{\RL{قرص}}
\end{align*}
قرص کا رداس \عددی{\rho} چھوٹے سے چھوٹا کرتے ہوئے \عددی{K} کو \عددی{\sigma} سے کم بنایا جا سکتا ہے۔یوں \عددی{K} پر ہر نقطہ کے لئے درج ذیل لکھا جا سکتا ہے۔
\begin{align*}
\abs{\frac{f(z)-f(z_0)}{z-z_0}}<\frac{\epsilon}{\rho}
\end{align*}
\عددی{K} کی لمبائی \عددی{2\pi \rho} ہے۔یوں مساوات \حوالہ{مساوات_مخلوط_تکمل_حتمی_قیمت_تخمینہ} کے تحت 
\begin{align*}
\abs{\int\limits_K \frac{f(z)-f(z_0)}{z-z_0}\dif z}<\frac{\epsilon}{\rho}2\pi\rho=2\pi\epsilon
\end{align*}
ہو گا۔چونکہ \عددی{\epsilon} کو ہم جتنا چاہیں چھوٹا کر سکتے ہیں لہٰذا مساوات \حوالہ{مساوات_مخلوط_تکمل_کوشی_کلیہ_تکمل_ب} میں آخری تکمل صفر ہو گا۔یوں مسئلے کا ثبوت مکمل ہوتا ہے۔
%
\begin{figure}
\centering
\begin{tikzpicture}
\draw[->-=0.5,rotate around={20:(0,0)}] (0,0) circle (1.5cm and 1cm);
\draw[->-=0.5](0.5,0.2) circle (0.75);
\draw[-latex](0.5,0.2)node[ocirc]{}node[left]{$z_0$}--++(45:0.75)node[pos=0.75,below]{$\rho$};
\draw(-1.7,-0.5)node[]{$C$};
\draw(0.5,0.2)++(-0.95,0)node[]{$K$};
\end{tikzpicture}
\caption{کوشی کے کلیہ تکمل کا ثبوت}
\label{شکل_مخلوط_تکمل_کوشی_کلیہ_ثبوت}
\end{figure}
\انتہا{ثبوت}
%========================
\ابتدا{مثال}\quad \موٹا{مختلف راہوں پر تکمل کی قیمت}\\
درج ذیل تکمل گھڑی کی الٹ رخ اکائی دائرے پر حاصل کریں۔ دائرے کا مرکز (الف) \عددی{z=1}، (ب) \عددی{z=\tfrac{1}{2}}، (پ) \عددی{z=-1} اور (ت) \عددی{z=i} پر لیں۔
\begin{align*}
\int_C\frac{z^2+1}{z^2-1}\dif z
\end{align*}
جواب:\quad
(الف) \quad  اس تکمل کو
\begin{align*}
\int_C\frac{z^2+1}{z^2-1}\dif z=\int_C \frac{z^2+1}{z+1}\frac{\dif z}{z-1}
\end{align*}
لکھا جا سکتا ہے۔دائیں ہاتھ کا مساوات \حوالہ{مساوات_مخلوط_تکمل_کوشی_کلیہ_تکمل_الف} کے ساتھ موازنہ کرتے ہوئے
\begin{align*}
f(z)=\frac{z^2+1}{z+1}
\end{align*}
 لکھا جا سکتا ہے۔چونکہ نقطہ \عددی{z_0=1} دائرہ \عددی{C} کے اندر پایا جاتا ہے اور \عددی{f(z)} راہ \عددی{C} پر اور اس کے اندر ہر نقطہ پر تحلیلی ہے (نقطہ \عددی{z=-1} جہاں \عددی{f(z)} غیر تحلیلی ہے \عددی{C} کے باہر پایا جاتا ہے۔) لہٰذا کوشی کے کلیہ تکمل کے تحت درج ذیل ہو گا۔
\begin{align*}
\int_C \frac{z^2+1}{z^2-1}\dif z=\int_C \frac{z^2+1}{z+1}\frac{\dif z}{z-1}=i2\pi\big[\frac{z^2+1}{z+1}\big]_{z=1}=i2\pi
\end{align*}
(ب) \quad ہمیں یہی نتیجہ دوبارہ ملتا ہے چونکہ دیا گیا تفاعل نقطہ \عددی{z=1} اور نقطہ \عددی{z=-1} پر غیر تحلیلی ہے اور ہم (الف) میں استعمال ہوئے دائرے کو، بغیر کسی غیر تحلیلی نقطہ سے گزرتے ہوئے، مسلسل تبدیل کرتے ہوئے یہاں درکار دائرہ حاصل کر سکتا ہے۔\\
(پ) \quad ہم اب درج ذیل لکھ سکتے ہیں۔
\begin{align*}
\int_C \frac{z^2+1}{z^2-1}\dif z=\int_C \frac{z^2+1}{z-1}\frac{\dif z}{z+1}=i2\pi\big[\frac{z^2}{z-1}\big]_{z=-1}=-i2\pi
\end{align*}
(ت)\quad چونکہ دیا گیا تفاعل دائرے پر اور دائرے کے اندر ہر نقطہ پر تحلیلی ہے لہٰذا کوشی کے کلیہ تکمل کے تحت یہ تکمل صفر کے برابر ہو گا۔ 
\انتہا{مثال}
%======================


مضرب تعلق دائرہ کار میں ہم حصہ \حوالہ{حصہ_مخلوط_تکمل_کوشی_مسئلہ_تکمل} کی طرح ہی بڑھتے ہیں۔مثلاً اگر \عددی{C_1} اور \عددی{C_2}  کے درمیان دائرہ کار (شکل \حوالہ{شکل_مساوات_مخلوط_تکمل_کوشی_کلیہ_تکمل_پ}) میں \عددی{f(z)} تحلیلی ہو اور \عددی{C_1} اور \عددی{C_2} پر بھی \عددی{f(z)} تحلیلی ہو اور اس دائرہ کار میں \عددی{z_0} کوئی نقطہ ہو تب
\begin{align}\label{مساوات_مخلوط_تکمل_کوشی_کلیہ_تکمل_پ}
f(z_0)=\frac{1}{i2\pi}\int\limits_{C_1}\frac{f(z)}{z-z_0}\dif z-\frac{1}{i2\pi}\int\limits_{C_2} \frac{f(z)}{z-z_0}\dif z
\end{align}
ہو گا جہاں دونوں تکمل گھڑی کی الٹ رخ حاصل کیے جائیں گے۔
\begin{figure}
\centering
\begin{tikzpicture}
\fill[gray!20!white,rotate around={20:(0,0)}](0,0) circle (1.5cm and 1cm);
\fill[white,rotate around={20:(0,0)}](0,0) circle (0.75cm and 0.5cm);
\draw[->-=0.75,rotate around={20:(0,0)}](0,0) circle (1.5cm and 1cm);
\draw[->-=0.75,rotate around={20:(0,0)}](0,0) circle (0.75cm and 0.5cm);
\draw(-90+20:0.5)node[above]{$C_2$};
\draw(-90+20:1)node[below]{$C_1$};
\draw(20:1.25)node[ocirc]{}node[below]{$z_0$};
\end{tikzpicture}
\caption{شکل برائے مساوات \حوالہ{مساوات_مخلوط_تکمل_کوشی_کلیہ_تکمل_پ}}
\label{شکل_مساوات_مخلوط_تکمل_کوشی_کلیہ_تکمل_پ}
\end{figure}

%=====================
\حصہء{سوالات}
سوال \حوالہ{سوال_مخلوط_تکمل_مختلف_دائرے_الف} تا سوال \حوالہ{سوال_مخلوط_تکمل_مختلف_دائرے_ب} میں دیے دائرے پر گھڑی کی الٹ رخ \عددی{\tfrac{z^2}{z^2+1}} کے تکمل کی قیمت تلاش کریں۔

%===============
\ابتدا{سوال}\شناخت{سوال_مخلوط_تکمل_مختلف_دائرے_الف}\quad
$\abs{z+i}=1$\\
جواب:\quad
$\pi$
\انتہا{سوال}
%====================
\ابتدا{سوال}\quad
$\abs{z-i}=\tfrac{2}{3}$\\
جواب:\quad
$-\pi$
\انتہا{سوال}
%====================
\ابتدا{سوال}\quad
$\abs{z}=3$\\
جواب:\quad
$0$
\انتہا{سوال}
%====================
\ابتدا{سوال}\شناخت{سوال_مخلوط_تکمل_مختلف_دائرے_ب}\quad
$\abs{z}=\tfrac{1}{3}$\\
جواب:\quad
$0$
\انتہا{سوال}
%====================
سوال \حوالہ{سوال_مخلوط_تکمل_دوسرا_تفاعل_الف} تا سوال \حوالہ{سوال_مخلوط_تکمل_دوسرا_تفاعل_ب} میں گھڑی کی الٹ رخ دیے گئے  دائرے پر \عددی{\tfrac{z^2}{z^4-1}} کے تکمل کی قیمت تلاش کریں۔ 

%====================
\ابتدا{سوال}\شناخت{سوال_مخلوط_تکمل_دوسرا_تفاعل_الف}\quad
$\abs{z-1}=1$\\
جواب:\quad
$i\tfrac{\pi}{2}$
\انتہا{سوال}
%=======================
\ابتدا{سوال}\quad
$\abs{z+i}=1$\\
جواب:\quad
$-\tfrac{\pi}{2}$
\انتہا{سوال}
%=======================
\ابتدا{سوال}\quad
$\abs{z-i}=1$\\
جواب:\quad
$\tfrac{\pi}{2}$
\انتہا{سوال}
%=======================
\ابتدا{سوال}\شناخت{سوال_مخلوط_تکمل_دوسرا_تفاعل_ب}\quad
$\abs{z}=3$\\
جواب:\quad
$0$
\انتہا{سوال}
%=======================
سوال \حوالہ{سوال_مخلوط_تکمل_اکائی_دائرے_پر_الف} تا سوال \حوالہ{سوال_مخلوط_تکمل_اکائی_دائرے_پر_ب} میں دیے تفاعل کی اکائی دائرے پر گھڑی کی الٹ رخ تکمل حاصل کریں۔

%=================
\ابتدا{سوال}\شناخت{سوال_مخلوط_تکمل_اکائی_دائرے_پر_الف}\quad
$\tfrac{1}{z}$\\
جواب:\quad
$i2\pi$
\انتہا{سوال}
%====================
\ابتدا{سوال}\quad
$\tfrac{1}{z^2+9}$\\
جواب:\quad
$0$
\انتہا{سوال}
%====================
\ابتدا{سوال}\quad
$\tfrac{1}{3z+1}$\\
جواب:\quad
$i\tfrac{2\pi}{3}$
\انتہا{سوال}
%====================
\ابتدا{سوال}\quad
$\tfrac{e^z}{z}$\\
جواب:\quad
$i2\pi$
\انتہا{سوال}
%====================
\ابتدا{سوال}\quad
$\tfrac{e^{3z}}{z+i3}$\\
جواب:\quad
$0$
\انتہا{سوال}
%====================
\ابتدا{سوال}\quad
$\tfrac{e^{3z}}{3z+i}$\\
جواب:\quad
$\tfrac{i2\pi e^{-i}}{3}$
\انتہا{سوال}
%====================
\ابتدا{سوال}\quad
$\tfrac{\cos z}{z}$\\
جواب:\quad
$i2\pi$
\انتہا{سوال}
%====================
\ابتدا{سوال}\quad
$\tfrac{\sin z}{z}$\\
جواب:\quad
$0$
\انتہا{سوال}
%====================
\ابتدا{سوال}\quad
$\tfrac{e^z-1}{z}$\\
جواب:\quad
$0$
\انتہا{سوال}
%====================
\ابتدا{سوال}\quad
$\tfrac{\sinh z}{z}$\\
جواب:\quad
$0$
\انتہا{سوال}
%====================
\ابتدا{سوال}\quad
$\tfrac{\cosh z}{z}$\\
جواب:\quad
$i2\pi$
\انتہا{سوال}
%====================
\ابتدا{سوال}\شناخت{سوال_مخلوط_تکمل_اکائی_دائرے_پر_ب}\quad
$\tfrac{\cosh z}{z-2}$\\
جواب:\quad
$0$
\انتہا{سوال}
%====================

\حصہ{تحلیلی تفاعل کے تفرق}\شناخت{حصہ_مخلوط_تکمل_تحلیل_تفاعل_کے_تفرق}
یہ جانتے ہوئے کہ ایک حقیقی تفاعل ایک مرتبہ قابل تفرق ہے سے یہ جاننا ممکن نہیں ہے کہ  اس کے بلند درجی تفرق موجود ہوں گے یا نہیں۔ہم اب دیکھیں گے کہ یہ جانتے ہوئے کہ ایک مخلوط تفاعل کا دائرہ کار \عددی{D} میں  ایک درجی تفرق موجود ہے سے ہم کہہ سکتے ہیں کہ \عددی{D} میں اس تفاعل کے ہر درجے کا تفرق موجود ہو گا۔اس لحاظ سے مخلوط تفاعل ایک مرتبہ قابل تفرق حقیقی تفاعل سے زیادہ سادہ رویہ رکھتے ہیں۔

%================
\ابتدا{مسئلہ}\شناخت{مسئلہ_مخلوط_تکمل_تحلیلی_تفاعل_کے_تفرق}\quad \موٹا{تحلیلی تفاعل کے تفرق}\\
دائرہ کار \عددی{D} میں تحلیلی تفاعل \عددی{f(z)} کا \عددی{D} میں ہر درجے کا تفرق موجود ہے اور ایسا تفرق از خود \عددی{D} میں تحلیلی ہو گا۔ \عددی{D} میں نقطہ \عددی{z_0} پر ان تفاعل کے تفرق درج ذیل کلیات
\begin{align}
f'(z_0)&=\frac{1}{i2\pi}\int\limits_C \frac{f(z)}{(z-z_0)^2}\dif z\label{مساوات_مخلوط_تکمل_تحلیلی_تفاعل_تفرق_الف}\\
f''(z_0)&=\frac{2!}{i2\pi}\int\limits_C \frac{f(z)}{(z-z_0)^3}\dif z\label{مساوات_مخلوط_تکمل_تحلیلی_تفاعل_تفرق_ب}
\end{align}
اور عمومی کلیہ
\begin{align}\label{مساوات_مخلوط_تکمل_تحلیلی_تفاعل_کے_تفرق}
f^{(n)}(z_0)=\frac{n!}{i2\pi}\int\limits_C \frac{f(z)}{(z-z_0)^{n+1}}\dif z\quad \quad (n=1,2,\cdots)
\end{align}
سے حاصل ہوں گے جہاں  دائرہ کار \عددی{D} میں \عددی{C} کوئی بھی ایسی سادہ بند راہ ہے جو \عددی{z_0} کو گھیرتی ہو اور جس کی مکمل اندرون \عددی{D} میں پائی جاتی ہو؛ تکمل گھڑی کی الٹ رخ \عددی{C} پر حاصل کیا جاتا ہے (شکل \حوالہ{شکل_مسئلہ_مخلوط_تکمل_تحلیلی_تفاعل_کے_تفرق})۔ 
\begin{figure}
\centering
\begin{tikzpicture}
\draw[fill=gray!20!white](1,0.75) to [out=-140,in=90] (0,0) to [out=-90,in=160](1,-0.5) to [out=-20,in=180](2,-1) to [out=0,in=-90](4,0) to [out=90,in=0](2,1) to [out=180,in=40](1,0.75);
\draw[fill=white,rotate around={-30:(1,0)}](1,0)circle (0.4cm and 0.2cm);
\draw[->-=0.75,rotate around={-15:(2.5,0.25)}]  (2.5,0.25) circle (0.75cm and 0.5cm);
\draw(2.5,0.25)++(-90:0.5)node[below left]{$C$};
\draw[-latex] (2.75,0)++(75:0.2)node[ocirc]{}node[below]{$z_0$}--++(45:0.4)node[pos=0.5,shift={(135:0.2)}]{$d$};
\end{tikzpicture}
\caption{شکل برائے مسئلہ \حوالہ{مسئلہ_مخلوط_تکمل_تحلیلی_تفاعل_کے_تفرق}}
\label{شکل_مسئلہ_مخلوط_تکمل_تحلیلی_تفاعل_کے_تفرق}
\end{figure}
\انتہا{مسئلہ}
%====================

\موٹا{رائے۔} مساوات \حوالہ{مساوات_مخلوط_تکمل_کوشی_کلیہ_تکمل_الف} میں تکمل کی نشان کے اندر \عددی{z_0} کے لحاظ سے تفرق لینے سے مساوات  \حوالہ{مساوات_مخلوط_تکمل_تحلیلی_تفاعل_کے_تفرق} کو با ضابطہ طور پر حاصل کیا جاتا ہے۔ مساوات  \حوالہ{مساوات_مخلوط_تکمل_تحلیلی_تفاعل_کے_تفرق} کو یاد رکھنے کا یہی بہترین طریقہ ہے۔

%================
\ابتدا{ثبوت}\quad
ہم مساوات \حوالہ{مساوات_مخلوط_تکمل_تحلیلی_تفاعل_تفرق_الف} کو ثابت کرتے ہیں۔ تفرق کی تعریف کے تحت
\begin{align*}
f'(z_0)=\lim_{\Delta z\to 0} \frac{f(z_0+\Delta z)-f(z_0)}{\Delta z}
\end{align*}
ہو گا۔یوں کوشی کے کلیہ تکمل مساوات \حوالہ{مساوات_مخلوط_تکمل_کوشی_کلیہ_تکمل_الف} سے
\begin{align}\label{مساوات_مخلوط_تکمل_تحلیلی_تفاعل_تفرق_پ}
f'(z_0)=\lim_{\Delta z \to 0} \frac{1}{i2\pi \Delta z}\big[\int\limits_C\frac{f(z)}{z-(z_0+\Delta z)}\dif z-\int\limits_C\frac{f(z)}{z-z_0}\dif z  \big]   
\end{align}
لکھا جا سکتا ہے۔اب آپ تسلی کر سکتے ہیں کہ درج ذیل لکھا جا سکتا ہے۔
\begin{align*}
\frac{1}{\Delta z}\big[\frac{1}{z-(z_0+\Delta z)}-\frac{1}{z-z_0}\big]=\frac{1}{(z-z_0)^2}+\frac{\Delta z}{(z-z_0-\Delta z)(z-z_0)^2}
\end{align*}
یوں مساوات \حوالہ{مساوات_مخلوط_تکمل_تحلیلی_تفاعل_تفرق_پ} کو
\begin{align*}
f'(z_0)=\frac{1}{i2\pi}\int\limits_C \frac{f(z)}{(z-z_0)^2}\dif z+\lim_{\Delta z\to 0}\frac{\Delta z}{i2\pi}\int\limits_C\frac{f(z)}{(z-z_0-\Delta z)(z-z_0)^2}\dif z
\end{align*}

لکھا جا سکتا ہے۔اگر ہم ثابت کر سکیں کہ دائیں ہاتھ آخری جزو صفر کے برابر ہے تب ہم مساوات \حوالہ{مساوات_مخلوط_تکمل_تحلیلی_تفاعل_تفرق_الف} کو ثابت کر پائیں گے۔آئیں ایسا ہی کرتے ہیں۔

\عددی{C} پر تفاعل \عددی{f(z)} استمراری ہے۔یوں \عددی{C} پر \عددی{f(z)} کی حتمی قیمت محدود ہو گی مثلاً \عددی{\abs{f(z)}<M} جہاں \عددی{M} حقیقی عدد ہے۔فرض کریں کہ \عددی{z_0} سے \عددی{C} کا قریب ترین نقطہ یا نقطوں کا فاصلہ \عددی{d} ہے۔تب \عددی{C} پر تمام \عددی{z} کے لئے
\begin{align*}
\frac{1}{\abs{z-z_0}}\le \frac{1}{d}\quad \text{اور}\quad \abs{z-z_0}\ge d
\end{align*}
ہوں گے۔ مزید اگر \عددی{\abs{\Delta z}\le \tfrac{d}{2}} ہو تب \عددی{C} پر تمام \عددی{z} کے لئے
\begin{align*}
\frac{1}{\abs{z-z_0-\Delta z}}\le \frac{2}{d}\quad \text{اور}\quad \abs{z-z_0-\Delta z}\ge \frac{d}{2}
\end{align*}
ہوں گے۔یوں \عددی{C} کی لمبائی کو \عددی{L} سے ظاہر کرتے ہوئے مساوات \حوالہ{مساوات_مخلوط_تکمل_حتمی_قیمت_تخمینہ} سے درج ذیل ملتا ہے۔
\begin{align*}
\abs{\frac{\Delta z}{i2\pi} \int\limits_C \frac{f(z)}{(z-z_0-\Delta z)(z-z_0)^2}\dif z}<\frac{\abs{\Delta z}}{2\pi}\frac{M}{\frac{1}{2}dd^2}L\quad \quad \big(\abs{\Delta z}\le \frac{d}{2}\big)
\end{align*}
\عددی{\Delta z} صفر کے قریب پہنچنے سے دایاں ہاتھ بھی صفر کے قریب پہنچتا ہے۔ یوں مساوات \حوالہ{مساوات_مخلوط_تکمل_تحلیلی_تفاعل_تفرق_الف} ثابت ہوتا ہے۔ یاد رہے کہ ہم نے یہاں کوشی کا کلیہ تکمل مساوات \حوالہ{مساوات_مخلوط_تکمل_کوشی_کلیہ_تکمل_الف}  استعمال کیا لیکن اگر ہمیں صرف اتنا معلوم ہوتا کہ \عددی{f(z_0)} کو مساوات \حوالہ{مساوات_مخلوط_تکمل_کوشی_کلیہ_تکمل_الف} سے ظاہر کیا جا سکتا ہے تب ہمارے متذکرہ بالا دلائل اس حقیقت کو ثابت کر پاتے کہ \عددی{f(z)} کا تفرق \عددی{f'(z_0)} موجود ہے۔اس سے آپ اخذ کر سکتے ہیں کہ اسی طرح کے دلائل مساوات \حوالہ{مساوات_مخلوط_تکمل_کوشی_کلیہ_تکمل_ب} کو ثابت کر پائیں گے۔اسی طرح الکراجی ماخوذ سے ہم عمومی تفرق کی مساوات \حوالہ{مساوات_مخلوط_تکمل_تحلیلی_تفاعل_کے_تفرق} کو بھی ثابت کر پائیں گے۔ 
\انتہا{ثبوت}
%======================

مسئلہ \حوالہ{مسئلہ_مخلوط_تکمل_تحلیلی_تفاعل_کے_تفرق} کی استعمال سے مسئلہ کوشی کا الٹ ثابت کرتے ہیں۔

%=====================
\ابتدا{مسئلہ}\quad \موٹا{مسئلہ موریرا}\حاشیہد{اطالوی ریاضی دان جاچینتو موریرا [1856-1909]}\\
اگر سادہ تعلق دائرہ کار \عددی{D} میں \عددی{f(z)} استمراری ہو اور  \عددی{D} میں ہر بند راہ پر
\begin{align}
\int_C f(z)\dif z=0
\end{align}
ہو تب \عددی{D} میں \عددی{f(z)} تحلیلی ہو گا۔
\انتہا{مسئلہ}
%=============================
\ابتدا{ثبوت}
حصہ \حوالہ{حصہ_مخلوط_تکمل_خطی_تکمل_کا_حصول_بذریعہ_غیر_قطعی_تکمل} میں دکھایا گیا کہ \عددی{D} میں \عددی{f(z)} تحلیلی ہونے کی صورت میں \عددی{D} میں
\begin{align*}
F(z)=\int_{z_0}^z f(z^*)\dif z^*
\end{align*}
تحلیلی ہو گا  اور \عددی{F'(z)=f(z)} ہو گا۔ ایسا ثابت کرتے ہوئے ہم نے صرف \عددی{f(z)} کی استمرار اور اس حقیقت کو استعمال کیا کہ \عددی{D} میں ہر بند راہ پر \عددی{f(z)} کا تکمل صفر ہے؛ ان مفروضوں سے ہم نے اخذ کیا کہ \عددی{F(z)} تحلیلی ہے۔ مسئلہ \حوالہ{مسئلہ_مخلوط_تکمل_تحلیلی_تفاعل_کے_تفرق}  کے تحت \عددی{F(z)} کا تفرق تحلیلی ہے یعنی \عددی{D} میں \عددی{f(z)} تحلیلی ہے۔یوں مسئلہ موریرا ثابت ہوا۔
\انتہا{ثبوت}
%======================

ہم اب ایک اہم عدم مساوات دریافت کرتے ہیں۔ مساوات \حوالہ{مساوات_مخلوط_تکمل_تحلیلی_تفاعل_کے_تفرق} میں فرض کریں کہ \عددی{C} رداس \عددی{r} کا ایک دائرہ ہے جس کا مرکز \عددی{z_0} پر ہے اور \عددی{C} پر \عددی{\abs{f(z)}} کی زیادہ سے زیادہ قیمت \عددی{M} ہے۔تب مساوات \حوالہ{مساوات_مخلوط_تکمل_حتمی_قیمت_تخمینہ} کو مساوات \حوالہ{مساوات_مخلوط_تکمل_تحلیلی_تفاعل_کے_تفرق} پر لاگو کرتے ہوئے
\begin{align*}
\abs{f^{(n)}(z_0)}=\frac{n!}{2\pi} \abs{\int_C \frac{f(z)}{(z-z_0)^{n+1}}\dif z}\le \frac{n!}{2\pi}M\frac{1}{r^{n+1}}2\pi r
\end{align*}
ملتا ہے جس سے \اصطلاح{کوشی عدم مساوات}\فرہنگ{کوشی!عدم مساوات}\حاشیہب{Cauchy's inequality}\فرہنگ{Cauchy!inequality} 
\begin{align}\label{مساوات_مخلوط_تکمل_کوشی_عدم_مساوات}
\abs{f^{(n)}(z_0)}\le \frac{n!M}{r^n}
\end{align}
حاصل ہوتی ہے۔

آئیں مساوات \حوالہ{مساوات_مخلوط_تکمل_کوشی_عدم_مساوات} سے درج ذیل اہم اور بنیادی نتیجہ حاصل کرتے ہیں۔

%===================
\ابتدا{مسئلہ}\شناخت{مسئلہ_مخلوط_تکمل_لییویل}\quad \موٹا{مسئلہ لییویل}\\
محدود مخلوط مستوی (حصہ \حوالہ{حصہ_محافظ_زاویہ_خطی_کسی_تبادل}) میں تمام \عددی{z} کے لئے تحلیلی \عددی{f(z)} اور محدود \عددی{\abs{f(z)}} کی صورت میں \عددی{f(z)} مستقل ہو گا۔
\انتہا{مسئلہ}
%=========================
\ابتدا{ثبوت}\quad
ہم فرض کر چکے ہیں کہ تمام \عددی{z} کے لئے  \عددی{\abs{f(z)}} محدود ہے مثلاً \عددی{\abs{f(z)}<K} جہاں \عددی{K} حقیقی عددی ہے۔ مساوات \حوالہ{مساوات_مخلوط_تکمل_کوشی_عدم_مساوات} استعمال کرتے ہوئے ہم دیکھتے ہیں کہ \عددی{\abs{f'(z_0)}<\tfrac{K}{r}} ہو گا۔چونکہ یہ ہر \عددی{r} کے لئے درست ہے لہٰذا ہم \عددی{r} کو جتنا چاہیں بڑا لے سکتے ہیں جس سے \عددی{f'(z_0)=0} حاصل ہوتا ہے۔چونکہ \عددی{z_0} اختیاری ہے اور تمام محدود \عددی{z} کے لئے \عددی{f'(z)=0} ہے لہٰذا \عددی{f(z)} مستقل (سوال \حوالہ{سوال_تحلیلی_تفرق_صفر_مطلب_مستقل_تفاعل}) ہو گا۔یوں ثبوت مکمل ہوتا ہے۔
\انتہا{ثبوت}
%===========================

\حصہء{سوالات}
سوال \حوالہ{سوال_مخلوط_تکمل_تحلیل_تفاعل_کا_تفرق_الف} تا سوال \حوالہ{سوال_مخلوط_تکمل_تحلیل_تفاعل_کا_تفرق_ب} میں دیے  تفاعل کا گھڑی کی الٹ رخ اکائی دائرے پر تکمل تلاش کریں۔

%====================
\ابتدا{سوال}\شناخت{سوال_مخلوط_تکمل_تحلیل_تفاعل_کا_تفرق_الف}\quad
$\tfrac{z^2}{(3z-1)^2}$\\
جواب:\quad
$i\tfrac{4\pi}{27}$
\انتہا{سوال}
%=======================
\ابتدا{سوال}\quad
$\tfrac{z^2}{(3z-1)^4}$\\
جواب:\quad
$0$
\انتہا{سوال}
%=======================
\ابتدا{سوال}\quad
$\tfrac{z^2}{(2z-i)^3}$\\
جواب:\quad
$-\tfrac{3\pi}{8}$
\انتہا{سوال}
%=======================
\ابتدا{سوال}\quad
$\tfrac{z^4}{(z+i)^2}$\\
جواب:\quad
$-8\pi$
\انتہا{سوال}
%=======================
\ابتدا{سوال}\quad
$\tfrac{z}{(5z+i)^2}$\\
جواب:\quad
$i\tfrac{2\pi}{25}$
\انتہا{سوال}
%=======================
\ابتدا{سوال}\quad
$\tfrac{e^z}{z^2}$\\
جواب:\quad
$i2\pi$
\انتہا{سوال}
%=======================
\ابتدا{سوال}\quad
$\tfrac{e^z}{z^4}$\\
جواب:\quad
$i\tfrac{\pi}{3}$
\انتہا{سوال}
%=======================
\ابتدا{سوال}\quad
$\tfrac{e^z}{z^n}$\\
جواب:\quad
$i\tfrac{2\pi}{(n-1)!}$
\انتہا{سوال}
%=======================
\ابتدا{سوال}\quad
$\tfrac{ze^z}{(z+i\pi)^2}$\\
جواب:\quad
$i2\pi(i\pi-1)$
\انتہا{سوال}
%=======================
\ابتدا{سوال}\quad
$z^{-2}\cos z$\\
جواب:\quad
$0$
\انتہا{سوال}
%=======================
\ابتدا{سوال}\quad
$z^{-2}\sin z$\\
جواب:\quad
$i2\pi$
\انتہا{سوال}
%=======================
\ابتدا{سوال}\quad
$z^{-2n-1}\cos z$\\
جواب:\quad
$i\tfrac{2\pi (-1)^n}{(2n)!}$
\انتہا{سوال}
%=======================
\ابتدا{سوال}\quad
$\tfrac{e^{z^2}}{z^3}$\\
جواب:\quad
$i2\pi$
\انتہا{سوال}
%=======================
\ابتدا{سوال}\quad
$z^{-2}e^z\sin z$\\
جواب:\quad
$i2\pi$
\انتہا{سوال}
%=======================
\ابتدا{سوال}\شناخت{سوال_مخلوط_تکمل_تحلیل_تفاعل_کا_تفرق_ب}\quad
$z^{-3}e^{z^3}$\\
جواب:\quad
$0$
\انتہا{سوال}
%=======================
\ابتدا{سوال}\شناخت{سوال_مخلوط_تکمل_خواص_الف}\quad
اگر \عددی{f(z)} غیر مستقل ہو اور تمام (محدود) \عددی{z} کے لئے تحلیلی ہو، اور \عددی{M} اور \عددی{R} کوئی مثبت حقیقی اعداد ہیں (جو جتنا چاہیں بڑے ہو سکتے ہیں) تب دکھائیں کہ \عددی{z} کی ایسی قیمتیں موجود ہوں گی جن کے لئے \عددی{\abs{z}>R} اور \عددی{\abs{f(z)}>M} ہو گا۔اشارہ۔ مسئلہ لییویل استعمال کریں۔
\انتہا{سوال}
%=========================
\ابتدا{سوال}\شناخت{سوال_مخلوط_تکمل_خواص_ب}\quad
اگر \عددی{f(z)} درجہ \عددی{n>0} کا کثیر رکنی ہو اور \عددی{M} (جتنا چاہیں بڑا) اختیاری مثبت حقیقی عدد ہو تب دکھائیں کہ ایسا حقیقی مثبت عدد \عددی{R} موجود ہو گا کہ تمام \عددی{\abs{z}>R} کے لئے \عددی{\abs{f(z)}>M} ہو گا۔
\انتہا{سوال}
%==========================
\ابتدا{سوال}\شناخت{سوال_مخلوط_تکمل_خواص_پ}\quad
دکھائیں کہ \عددی{f(z)=e^z} سوال \حوالہ{سوال_مخلوط_تکمل_خواص_الف} میں بیان کی گئی خاصیت رکھتا ہے جبکہ سوال \حوالہ{سوال_مخلوط_تکمل_خواص_ب} میں بیان کہ گئی خاصیت نہیں رکھتا ہے۔
\انتہا{سوال}
%========================
\ابتدا{سوال}\quad
\اصطلاح{الجبرا کا بنیادی مسئلہ}\فرہنگ{بنیادی مسئلہ!الجبرا}\حاشیہب{Fundamental theorem of algebra}\فرہنگ{algebra!Fundamental theorem} کہتا ہے کہ  اگر غیر مستقل تفاعل \عددی{f(z)} متغیرہ \عددی{z} کا کثیر رکنی ہو تب \عددی{z} کی  کم از کم ایک قیمت کے لئے \عددی{f(z)=0} ہو گا۔اس مسئلے کو ثابت کریں۔ اشارہ۔ یہ فرض کرتے ہوئے  کہ تمام \عددی{z} کے لئے  \عددی{f(z)\ne 0} ہے سوال  \حوالہ{سوال_مخلوط_تکمل_خواص_الف} کا نتیجہ \عددی{g=\tfrac{1}{f}} پر لاگو کریں۔
\انتہا{سوال}
%========================
