\باب{مخلوط تکملات}
مخلوط تکملات دو وجوحات کی بنا اہم ہیں۔عملی وجہ یہ ہے کہ حقیقی تکملات حل کرنے کی تراکیب سے کئی حقیقی تکملات حل کرنا ناممکن ہے جبکہ ان کو مخلوط تکملات کی ترکیب سے حل کیا جا سکتا ہے۔دوسری وجہ نظریاتی ہے۔ جہاں مخلوط تکملات کی ترکیب سے تحلیلی تفاعل کی چند بنیادی خصوصیات دریافت ہوتی ہیں (بالخصوص بلند درجی تفرق کی موجودگی) جن کا ثبوت  تکمل استعمال کیے بغیر انتہائی مشکل ہو گا۔یہ صورت حال حقیقی اور مخلوط علم الاحصاء میں بنیادی فرق کی نشاندہی کرتی ہے۔

اس باب میں ہم پہلے مخلوط تکملات کی تعریف پیش کرتے ہیں۔سب سے بنیادی نتیجہ  کوشی مخلوط تکمل کا مسئلہ حاصل ہو گا جس سے  سے کوشی تکمل کی کلیات حاصل ہوں گی جو بہت اہم  ہیں۔ ہم ثابت کریں گے کہ اگر کوئی تفاعل تحلیلی ہو تب اس کے ہر درجہ کے تفرق موجود ہوں گے۔اس نقطہ نظر سے مخلوط تحلیلی تفاعل حقیقی متغیر کی حقیقی تفاعل سے زیادہ سادہ رویہ رکھتے ہیں۔

%=====================
\حصہ{مخلوط مستوی میں خطی تکمل}
حقیقی علم الاحصاء کی طرح ہم قطعی تکمل اور غیر قطعی تکمل میں تمیز کرتے ہیں۔ایک غیر قطعی تکمل ایسا تفاعل ہوتا ہے جس کا تفرق خطے میں دیا گیا تحلیلی تفاعل ہو گا۔تفاعل کی تفرق کو الٹ لکھتے ہوئے ہم کئی غیر قطعی تکمل دریافت کر سکتے ہیں۔

آئیں اب مخلوط تفاعل \عددی{f(z)}، جہاں \عددی{z=x+iy} ہے، کی قطعی تکمل یا خطی تکمل کی تعریف پیش کرتے ہیں۔ہم دیکھیں گے کہ حقیقی قطعی تکمل کی تصور کو وسعت دیتے ہوئے  مخلوط قطعی تکمل کا تصور پیدا ہوتا ہے۔یوں موجودہ بحث عین حصہ \حوالہ{حصہ_سمتی_تکمل_خطی_تکمل} کی طرح ہو گی۔قطعی تکمل کی صورت میں حقیقی محور پر کوئی وقفہ تکمل کی راہ  ہو گی۔مخلوط قطعی تکمل کی صورت میں ہم مخلوط مستوی پر کسی منحنی\حاشیہد{درحقیقت منحنی کے کسی حصے یا قوس پر تکمل لیا جائے گا۔اپنی آسانی کی خاطر ہم  "منحنی" کی اصطلاح کو  پوری منحنی کے لئے اور منحنی کے چھوٹے حصہ کے لئے بھی استعمال کریں گے۔} پر چلتے ہوئے تکمل حاصل کریں گے۔

فرض کریں کہ  مخلوط \عددی{z} مستوی میں \عددی{C} ایک ہموار منحنی (حصہ \حوالہ{حصہ_نقش_محافظ_زاویہ_نقش}) ہے۔تب ہم \عددی{C} کو درج ذیل روپ میں لکھ سکتے ہیں
\begin{align}\label{مساوات_مخلوط_تکمل_راہ_الف}
z(t)=x(t)+iy(t)\quad \quad \quad (a\le t\le b)
\end{align}
جہاں تمام \عددی{t} کے لئے \عددی{z(t)} کا استمراری تفرق \عددی{\dot{z}(t)\ne 0} پایا جاتا ہے، اور یوں \عددی{C} قابل تصحیح (حصہ \حوالہ{حصہ_الاحصاء_لمبائی_قوس}) ہو گی جس کا ہر نقطہ پر یکتا مماس ہو گا۔آپ کو یاد ہو گا کہ \عددی{C} پر مثبت رخ سے مراد  \عددی{t} کی بڑھتی قیمت کا مطابقتی رخ  ہے۔

فرض کریں کہ  \عددی{f(z)} ایک استمراری تفاعل ہے جو (کم از کم)  \عددی{C} کی ہر نقطہ پر معین ہے۔ہم  مساوات \حوالہ{مساوات_مخلوط_تکمل_راہ_الف} میں دیے گئے وقفہ \عددی{a\le t\le b} کو درج ذیل ٹکڑوں میں تقسیم کرتے ہیں
\begin{align*}
t_0(=a),t_1,\cdots,t_{n-1},t_n(=b)
\end{align*} 
جہاں \عددی{t_0<t_1<\cdots<t_n} ہے۔اس کے مطابق \عددی{C} کے ٹکڑے (شکل \حوالہ{شکل_مخلوط_تکمل_ٹکڑے_راہ})
\begin{align*}
z_0,z_1,\cdots,z_{n-1},z_n()=Z
\end{align*}
 پائے جاتے ہیں جہاں \عددی{z_j=z(t_j)} ہے۔
\begin{figure}
\centering
\begin{tikzpicture}
%\draw[thick](0,0) grid (3,2);
%\draw[thin,gray,step=0.1] (0,0) grid (3,2);
\draw(0,0) node[ocirc]{}node[right]{$z_0$} to [out=135,in=-135] coordinate[pos=0.3](kA)coordinate[pos=0.5](kB) coordinate[pos=0.7](kC)coordinate[pos=0.9](kD)  (0.5,1.5) to [out=45,in=80] coordinate[pos=0.2](kE)coordinate[pos=0.5](kF)coordinate[pos=0.7](kG) coordinate[pos=0.8](kH)coordinate[pos=0.9](kI)coordinate[pos=1,ocirc](kJ)node[pos=1,below]{$Z$}(3,1);
\draw(kA)++(0:0.1)--++(0:-0.2)node[left]{$z_1$};
\draw(kB)++(-10:0.1)--++(-10:-0.2)node[left]{$z_2$};
\draw(kC)++(-40:0.1)--++(-40:-0.2)++(-40:-0.25)coordinate(kkA);
\draw(kD)++(-40:0.1)--++(-40:-0.2)++(-40:-0.25)coordinate(kkB);
\draw($(kkA)!0.5!(kkB)$)node[rotate=50]{$\cdots$};
\draw(kE)++(-60:0.1)--++(-60:-0.2)node[above]{$z_{m-1}$};
\draw(kF)node[ocirc]{}node[above]{$\zeta_m$};
\draw(kG)++(-115:0.1)--++(-115:-0.2)node[above]{$z_m$};
\draw(kH)++(-130:0.1)--++(-130:-0.2);
\draw(kI)++(-170:0.1)--++(-170:-0.2);
\draw(kE)--(kG)node[pos=0.5,below]{$\Delta z_m$};
\end{tikzpicture}
\caption{مخلوط خطی تکمل}
\label{شکل_مخلوط_تکمل_ٹکڑے_راہ}
\end{figure}
ہم \عددی{C} کے ہر ٹکڑے پر کوئی اختیاری نقطہ منتخب کرتے ہیں، مثلاً ہم \عددی{z_0} اور \عددی{z_1} کے درماین نقطہ \عددی{\zeta_1} منتخب کرتے ہیں 
(یعنی \عددی{\zeta_1=z(t)} جہاں \عددی{t_0\le t\le t_1} ہے) اور \عددی{z_1} اور \عددی{z_2} کے درمیان نقطہ \عددی{\zeta_2} منتخب کرتے ہیں، وغیرہ وغیرہ۔ہم اب مجموعہ
\begin{align}
S_n=\sum_{m=1}^{n}f(\zeta-m)\Delta z_m
\end{align}
لیتے ہیں جہاں
\begin{align*}
\Delta z_m=z_m-z_{m-1}
\end{align*}
ہے۔ہم ایسے مجموعے \عددی{n=2,3,\cdots} کے لئے مکمل بے قاعدگی سے حاصل کرتے ہیں پس اتنا دھیان رکھتے ہیں کہ جب \عددی{n} لامتناہی کے قریب پہنچے تب \عددی{\abs{\Delta z_m}} کی زیادہ سے زیادہ قیمت صفر کے قریب پہنچتی ہو۔یوں ہمیں مخلوط قیمتوں کا سلسلہ \عددی{S_2,S_3,\cdots} ملتا ہے۔اس سلسلے کی حد، راہ \عددی{C} پر  \عددی{f(z)} کا \اصطلاح{خطی تکمل}\فرہنگ{تکمل!خطی}\حاشیہب{line integral}\فرہنگ{integral!line} (یا صرف \ترچھا{تکمل}) کہلاتا ہے۔
